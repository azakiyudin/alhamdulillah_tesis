\section{Percobaan Numerik}
Pada bagian ini, dilakukan percobaan numerik untuk menguji laju konvergensi dari skema iterasi Sabri dibanding dengan skema iterasi lainnya. Untuk percobaan ini, terlebih dahulu dikenalkan skema iterasi JK yang dikembangkan pada ruang $CAT_p(0)$ oleh Salisu sebagai berikut. 
\begin{defn}[\textbf{Skema Iterasi JK}]\cite{Salisu2022}\label{defn:jk}
% Diberikan $(X,d,G)$ adalah ruang $CAT _p(0)$ dan $W$ adalah himpunan bagian tak kosong dari $X$ yang konveks. 
Untuk suatu pemetaan $T:W\to W$, $x_0\in W$, dan $n\in\mathbb{N}\cup \{0\} $ didefinisikan skema iterasi JK pada ruang $CAT_p(0)$ sebagai berikut:
    \begin{align}\label{eq:jkcat}
       \begin{cases}
            q_n &= (1-c_n)x_n \oplus c_nTx_n,\\
        y_n&= Tq_n,\\
        x_{n+1} &= T\left((1-a_n)Tq_n\oplus a_n Ty_n\right),
       \end{cases}
    \end{align}
    dengan $\{a_n\}\subseteq [0,1]$ dan $\{c_n\}\subseteq [0,1]$.
\end{defn}
Selain itu, berikut ini dikenalkan pula skema iterasi Thakur, Abbas, dan Agarwal pada ruang $CAT_p(0)$. 
\begin{defn}[\textbf{Skema Iterasi Thakur}]\label{defn:thakur}
% Diberikan $(X,d,G)$ adalah ruang $CAT _p(0)$ dan $W$ adalah himpunan bagian tak kosong dari $X$ yang konveks. 
Untuk suatu pemetaan $T:W\to W$, $x_0\in W$, dan $n\in\mathbb{N}\cup \{0\} $ didefinisikan skema iterasi Thakur pada ruang $CAT_p(0)$ sebagai berikut:
    \begin{align}\label{eq:thakurcat}
       \begin{cases}
            q_n&= (1-c_n)x_n \oplus c_nTx_n,\\
        y_n&= T((1-a_n)x_n\oplus a_nq_n),\\
        x_{n+1} &= Ty_n,
       \end{cases}
    \end{align}
    dengan $\{a_n\}\subseteq [0,1]$ dan $\{c_n\}\subseteq [0,1]$.
\end{defn}
\begin{defn}[\textbf{Skema Iterasi Abbas}]\label{defn:abbas}
% Diberikan $(X,d,G)$ adalah ruang $CAT _p(0)$ dan $W$ adalah himpunan bagian tak kosong dari $X$ yang konveks. 
Untuk suatu pemetaan $T:W\to W$, $x_0\in W$, dan $n\in\mathbb{N}\cup \{0\} $ didefinisikan skema iterasi Abbas pada ruang $CAT_p(0)$ sebagai berikut:
    \begin{align}\label{eq:abbascat}
       \begin{cases}
            q_n&= (1-c_n)x_n \oplus c_nTx_n,\\
        y_n&= (1-b_n)Tx_n\oplus b_nTq_n,\\
        x_{n+1} &= (1-a_n)Ty_n\oplus a_n Tq_n,
       \end{cases}
    \end{align}
    dengan $\{a_n\}\subseteq [0,1]$, $\{b_n\}\subseteq [0,1]$, dan $\{c_n\}\subseteq [0,1]$.
\end{defn}
\begin{defn}[\textbf{Skema Iterasi Agarwal}]\label{defn:agarwal}
% Diberikan $(X,d,G)$ adalah ruang $CAT _p(0)$ dan $W$ adalah himpunan bagian tak kosong dari $X$ yang konveks. 
Untuk suatu pemetaan $T:W\to W$, $x_0\in W$, dan $n\in\mathbb{N}\cup \{0\} $ didefinisikan skema iterasi Agarwal pada ruang $CAT_p(0)$ sebagai berikut:
    \begin{align}\label{eq:agarwalcat}
       \begin{cases}
        y_n&= (1-c_n)x_n\oplus a_n Tx_n,\\
        x_{n+1} &= (1-a_n)Tx_n\oplus a_n Ty_n,
       \end{cases}
    \end{align}
    dengan $\{a_n\}\subseteq [0,1]$ dan $\{c_n\}\subseteq [0,1]$.
\end{defn}

Percobaan numerik dilakukan dengan perangkat lunak Google Colab yang menggunakan bahasa pemrograman Python. Digunakan beberapa variasi parameter, baik dari nilai awal $x_0$ maupun parameter di skema iterasinya, yaitu barisan $\{a_n\},\{b_n\},\{c_n\}$.
Pemetaan dan ruang yang digunakan dalam percobaan ini diambil dari Contoh \ref{con:abcnon} yang memiliki titik tetap $x^*=(1,1,0,0,\dots)$. 

Untuk hasil numerik tiap iterasi, digunakan batas galat $d(x_{n+1},x_n)<10^{-6}$, sedangkan untuk jumlah iterasi yang diperlukan untuk konvergen dari masing-masing skema iterasi, digunakan batas galat $d(x_{n+1},x_n)<10^{-16}$. Dalam percobaan ini, laju konvergensi skema iterasi Sabri \eqref{eq:sabricat} dibandingkan dengan Skema iterasi JK \eqref{eq:jkcat}, Thakur \eqref{eq:thakurcat}, Abbas \ref{eq:abbascat}, dan Agarwal \eqref{eq:agarwalcat}. Kode dari percobaan ini dapat dilihat pada Lampiran A.1.

Pada \ref{tab:sabjk}, \ref{tab:sabthkur}, \ref{tab:sababb}, dan \ref{tab:sabag} disajikan hasil numerik dari tiap iterasi dengan nilai awal $(2,3,0,0,\dots)$ dan parameter $a_n=0.92, b_n=0.83, c_n=0.81$ untuk setiap $n\in\mathbb{N}$ hingga batas galat $d(x_{n+1},x_n)<10^{-6}$.

\begin{table}[H]
    \centering
    \caption{Hasil numerik skema iterasi Sabri dan JK}
    \begin{tabular}{ccc}
\hline
\textbf{Iterasi} & \multicolumn{2}{c}{\textbf{Nilai $x_n$ hingga $d(x_{n+1},x_n)<10^{-6}$}} \\
\hline
\textbf{n} & \textbf{Sabri} & \textbf{JK} \\
\hline
0 & (2.000000, 3.000000, 0, 0, \dots) & (2.000000, 3.000000, 0, 0, \dots) \\
1 & (1.000843, 1.002532, 0, 0, \dots) & (1.007605, 1.022988, 0, 0, \dots) \\
2 & (1.000001, 1.000002, 0, 0, \dots) & (1.000058, 1.000174, 0, 0, \dots) \\
3 & (1.000000, 1.000000, 0, 0, \dots) & (1.000000, 1.000001, 0, 0, \dots) \\
4 & (1, 1, 0, 0, \dots) & (1.000000, 1.000000, 0, 0, \dots) \\
\hline
\end{tabular}
    \label{tab:sabjk}
\end{table}

\begin{table}[H]
    \centering
        \caption{Hasil numerik skema iterasi Sabri dan Thakur}
    \begin{tabular}{ccc}
\hline
\textbf{Iterasi} & \multicolumn{2}{c}{\textbf{Nilai $x_n$ hingga $d(x_{n+1},x_n)<10^{-6}$}} \\
\hline
\textbf{n} & \textbf{Sabri} & \textbf{Thakur} \\
\hline
0 & (2.000000, 3.000000, 0, 0, \dots) & (2.000000, 3.000000, 0, 0, \dots) \\
1 & (1.000843, 1.002532, 0, 0, \dots) & (1.027569, 1.085007, 0, 0, \dots) \\
2 & (1.000001, 1.000002, 0, 0, \dots) & (1.000760, 1.002282, 0, 0, \dots) \\
3 & (1.000000, 1.000000, 0, 0, \dots) & (1.000021, 1.000063, 0, 0, \dots) \\
4 & (1, 1, 0, 0, \dots) & (1.000001, 1.000002, 0, 0, \dots) \\
5 & (1, 1, 0, 0, \dots) & (1.000000, 1.000000, 0, 0, \dots) \\
\hline
\end{tabular}
    \label{tab:sabthkur}
\end{table}

\begin{table}[H]
    \centering
        \caption{Hasil numerik skema iterasi Sabri dan Abbas}
    \begin{tabular}{ccc}
\hline
\textbf{Iterasi} & \multicolumn{2}{c}{\textbf{Nilai $x_n$ hingga $d(x_{n+1},x_n)<10^{-6}$}} \\
\hline
\textbf{n} & \textbf{Sabri} & \textbf{Abbas} \\
\hline
0 & (2.000000, 3.000000, 0, 0, \dots) & (2.000000, 3.000000, 0, 0, \dots) \\
1 & (1.000843, 1.002532, 0, 0, \dots) & (1.092754, 1.304869, 0, 0, \dots) \\
2 & (1.000001, 1.000002, 0, 0, \dots) & (1.008603, 1.026033, 0, 0, \dots) \\
3 & (1.000000, 1.000000, 0, 0, \dots) & (1.000798, 1.002396, 0, 0, \dots) \\
4 & (1, 1, 0, 0, \dots) & (1.000074, 1.000222, 0, 0, \dots) \\
5 & (1, 1, 0, 0, \dots) & (1.000007, 1.000021, 0, 0, \dots) \\
6 & (1, 1, 0, 0, \dots) & (1.000001, 1.000002, 0, 0, \dots) \\
7 & (1, 1, 0, 0, \dots) & (1.000000, 1.000000, 0, 0, \dots) \\
\hline
\end{tabular}
    \label{tab:sababb}
\end{table}

% \begin{longtable}{ccc}
% \caption{\centering Hasil numerik skema iterasi Sabri dan Abbas}\\
% \hline
% \textbf{Iterasi} & \multicolumn{2}{c}{\textbf{Nilai $x_n$ hingga $d(x_{n+1},x_n)<10^{-6}$}} \\
% \hline
% \textbf{n} & \textbf{Sabri} & \textbf{Abbas} \\
% \hline
% \endfirsthead 
% \multicolumn{3}{c}{ \thetable{} -- Lanjutan dari halaman sebelumnya} \\
% \hline
% \textbf{n} & \textbf{Sabri} & \textbf{Abbas} \\
% \hline
% \endhead 
% \hline
% 0 & (2.000000, 3.000000, 0, 0, \dots) & (2.000000, 3.000000, 0, 0, \dots) \\
% 1 & (1.007270, 1.021970, 0, 0, \dots) & (1.109842, 1.367048, 0, 0, \dots) \\
% 2 & (1.000053, 1.000159, 0, 0, \dots) & (1.012065, 1.036634, 0, 0, \dots) \\
% 3 & (1.000000, 1.000001, 0, 0, \dots) & (1.001325, 1.003981, 0, 0, \dots) \\
% 4 & (1, 1, 0, 0, \dots) & (1.000146, 1.000437, 0, 0, \dots) \\
% 5 & (1, 1, 0, 0, \dots) & (1.000016, 1.000048, 0, 0, \dots) \\
% 6 & (1, 1, 0, 0, \dots) & (1.000002, 1.000005, 0, 0, \dots) \\
% 7 & (1, 1, 0, 0, \dots) & (1.000000, 1.000001, 0, 0, \dots) \\
% \hline
% \label{tab:sababb}
% \end{longtable}


\begin{table}[H]
    \centering
        \caption{Hasil numerik skema iterasi Sabri dan Agarwal}
    \begin{tabular}{ccc}
\hline
\textbf{Iterasi} & \multicolumn{2}{c}{\textbf{Nilai $x_n$ hingga $d(x_{n+1},x_n)<10^{-6}$}} \\
\hline
\textbf{n} & \textbf{Sabri} & \textbf{Agarwal} \\
\hline
0 & (2.000000, 3.000000, 0, 0, \dots) & (2.000000, 3.000000, 0, 0, \dots) \\
1 & (1.000843, 1.002532, 0, 0, \dots) & (1.110275, 1.368648, 0, 0, \dots) \\
2 & (1.000001, 1.000002, 0, 0, \dots) & (1.012161, 1.036927, 0, 0, \dots) \\
3 & (1.000000, 1.000000, 0, 0, \dots) & (1.001341, 1.004028, 0, 0, \dots) \\
4 & (1, 1, 0, 0, \dots) & (1.000148, 1.000444, 0, 0, \dots) \\
5 & (1, 1, 0, 0, \dots) & (1.000016, 1.000049, 0, 0, \dots) \\
6 & (1, 1, 0, 0, \dots) & (1.000002, 1.000005, 0, 0, \dots) \\
7 & (1, 1, 0, 0, \dots) & (1.000000, 1.000001, 0, 0, \dots) \\
8 & (1, 1, 0, 0, \dots) & (1.000000, 1.000000, 0, 0, \dots) \\
\hline
\end{tabular}
    \label{tab:sabag}
\end{table}

\begin{longtable}{cccccc}
\caption{\centering Galat $d(x_n, x^*)$}\label{tab:galat}\\
\hline
% \textbf{Iterasi} & \multicolumn{5}{c}{\textbf{Nilai Galat $d(x_n, x^*)$}} \\
% \hline
$\mathbf{n}$ & \textbf{Sabri} & \textbf{JK} & \textbf{Thakur} & \textbf{Abbas} & \textbf{Agarwal} \\
\hline
\endfirsthead 
\multicolumn{6}{c}{ \thetable{} -- Lanjutan dari halaman sebelumnya} \\
\hline
$\mathbf{n}$ & \textbf{Sabri} & \textbf{JK} & \textbf{Thakur} & \textbf{Abbas} & \textbf{Agarwal} \\
\hline
\endhead 
\hline
% \multicolumn{3}{|c|}{{Lanjut pada halaman berikutnya}} \\
\endfoot 
\hline
\endlastfoot
\hline
1 & $5.0\times10^{0}$ & $5.0\times10^{0}$ & $5.0\times10^{0}$ & $5.0\times10^{0}$ & $5.0\times10^{0}$ \\
2 & $8.4\times10^{-4}$ & $7.5\times10^{-3}$ & $2.6\times10^{-2}$ & $8.4\times10^{-2}$ & $9.8\times10^{-2}$ \\
3 & $7.1\times10^{-7}$ & $5.7\times10^{-5}$ & $7.3\times10^{-4}$ & $7.8\times10^{-3}$ & $1.0\times10^{-2}$ \\
4 & $5.9\times10^{-10}$ & $4.3\times10^{-7}$ & $2.0\times10^{-5}$ & $7.2\times10^{-4}$ & $1.1\times10^{-3}$ \\
5 & $5.0\times10^{-13}$ & $3.3\times10^{-9}$ & $5.6\times10^{-7}$ & $6.7\times10^{-5}$ & $1.3\times10^{-4}$ \\
6 & $4.4\times10^{-16}$ & $2.5\times10^{-11}$ & $1.5\times10^{-8}$ & $6.2\times10^{-6}$ & $1.4\times10^{-5}$ \\
7 & 0 & $1.9\times10^{-13}$ & $4.2\times10^{-10}$ & $5.7\times10^{-7}$ & $1.5\times10^{-6}$ \\
8 & 0 & $1.5\times10^{-15}$ & $1.1\times10^{-11}$ & $5.3\times10^{-8}$ & $1.7\times10^{-7}$ \\
9 & 0 & 0 & $3.2\times10^{-13}$ & $4.9\times10^{-9}$ & $1.9\times10^{-8}$ \\
10 & 0 & 0 & $9.1\times10^{-15}$ & $4.6\times10^{-10}$ & $2.1\times10^{-9}$ \\
11 & 0 & 0 & $2.2\times10^{-16}$ & $4.2\times10^{-11}$ & $2.3\times10^{-10}$ \\
12 & 0 & 0 & 0 & $3.9\times10^{-12}$ & $2.6\times10^{-11}$ \\
13 & 0 & 0 & 0 & $3.6\times10^{-13}$ & $2.8\times10^{-12}$ \\
14 & 0 & 0 & 0 & $3.4\times10^{-14}$ & $3.1\times10^{-13}$ \\
15 & 0 & 0 & 0 & $2.8\times10^{-15}$ & $3.5\times10^{-14}$ \\
16 & 0 & 0 & 0 & $4.4\times10^{-16}$ & $3.7\times10^{-15}$ \\
17 & 0 & 0 & 0 & 0 & $4.4\times10^{-16}$ \\
\hline
\end{longtable}
% \begin{table}[H]
%     \centering
%         \caption{Galat $d(x_n, x^*)$}
%     \label{tab:galat}
%     \begin{tabular}{cccccc}
%     \hline
%     $\mathbf{n}$ & \textbf{Sabri} & \textbf{JK} & \textbf{Thakur} & \textbf{Abbas} & \textbf{Agarwal} \\
%     \hline
%     1 & 0.007270 & 0.032859 & 0.056397 & 0.109842 & 0.225588 \\
%     2& 0.000053 & 0.001080 & 0.003181 & 0.012065 & 0.050890 \\
% 3 & 0.000000 & 0.000035 & 0.000179 & 0.001325 & 0.011480 \\
% 4 & NaN & 0.000001 & 0.000010 & 0.000146 & 0.002590 \\
% 5 & NaN & 0.000000 & 0.000001 & 0.000016 & 0.000584 \\
% 6 & NaN & NaN & NaN & 0.000002 & 0.000132 \\
% 7 & NaN & NaN & NaN & 0.000000 & 0.000030 \\
% 8 & NaN & NaN & NaN & NaN & 0.000007 \\
% 9 & NaN & NaN & NaN & NaN & 0.000002 \\
% 10 & NaN & NaN & NaN & NaN & 0.000000 \\
% \hline
%     \end{tabular}
% \end{table}
\begin{figure}[H]
    \centering
    \includegraphics[width=\linewidth]{Bab_4/galatpercobaan.png}
    \caption{Galat $d(x_{n+1},x_n)$ (dalam log) vs iterasi}
    \label{fig:galatperc}
\end{figure}
Dengan galat kurang dari $10^{-6}$, skema iterasi Sabri hanya membutuhkan 3 iterasi dibanding dengan JK (4 iterasi), Thakur (5 iterasi), Abbas (7 iterasi), dan Agarwal (8 iterasi). Pada \ref{tab:galat}, diberikan pula nilai galat $d(x_n,x^*)$ dari tiap iterasi. \ref{fig:galatperc} juga memberikan gambaran mengenai penurunan galatnya. 


Selanjutnya, pada \ref{tab:paramkonstan}, \ref{tab:paramturun}, \ref{tab:paramnaik}, \ref{tab:paramnaikturun}, dan \ref{tab:paramturunnaik}, disajikan jumlah iterasi yang diperlukan sampai batas galat $d(x_{n+1},x^*)<10^{-16}$ dari masing-masing skema iterasi dengan nilai awal dan parameter yang berbeda. Nilai awal dipilih secara acak dengan batasan nilainya berada pada domain pemetaan yang digunakan. Pada tabel-tabel tersebut parameter barisan $\{a_n\}$ dan $\{c_n\}$ berturut-turut dipilih dengan kondisi konstan-konstan, turun-turun, naik-naik, naik-turun, dan turun-naik. 
% \begin{longtable}{cccccc}
% \caption{\centering Iterasi dengan parameter $a_n = 0.42, b_n=0.83, c_n=0.31$}\\
% \hline
% $\mathbf{n}$ & \textbf{Sabri} & \textbf{JK} & \textbf{Thakur} & \textbf{Abbas} & \textbf{Agarwal} \\
% \hline
% \endfirsthead 
% \multicolumn{6}{c}{ \thetable{} -- Lanjutan dari halaman sebelumnya} \\
% \hline
% $\mathbf{n}$ & \textbf{Sabri} & \textbf{JK} & \textbf{Thakur} & \textbf{Abbas} & \textbf{Agarwal} \\
% \hline
% \endhead 
% \hline
% (2.0, 3.0, 0, 0, \dots) & 9 & 12 & 14 & 18 & 26 \\
% (3.0, 4.0, 0, 0, \dots) & 9 & 12 & 14 & 19 & 26 \\
% (4.0, 1.0, 0, 0, \dots) & 9 & 12 & 14 & 19 & 26 \\
% (1.5, 4.0, 0, 0, \dots) & 9 & 12 & 14 & 18 & 26 \\
% (2.7, 4.2, 0, 0, \dots) & 9 & 12 & 14 & 19 & 26 \\
% (5.0, 1.8, 0, 0, \dots) & 9 & 12 & 15 & 19 & 27 \\
% (3.1, 3.5, 0, 0, \dots) & 9 & 12 & 14 & 19 & 26 \\
% \hline
% \label{tab:paramkonstan}
% \end{longtable}
\begin{table}[H]
\caption{Iterasi dengan parameter $a_n = 0.92, b_n=0.83, c_n=0.81$}
    \centering
    \begin{tabular}{cccccc}
\hline
\textbf{Nilai Awal} & \multicolumn{5}{c}{\textbf{Jumlah Iterasi}} \\
\hline
$x_0$ & Sabri & JK & Thakur & Abbas & Agarwal \\
\hline
(2.0, 3.0, 0, 0, \dots) & 8 & 10 & 13 & 18 & 19 \\
(3.0, 4.0, 0, 0, \dots) & 8 & 10 & 13 & 18 & 19 \\
(4.0, 1.0, 0, 0, \dots) & 8 & 10 & 13 & 18 & 19 \\
(1.5, 4.0, 0, 0, \dots) & 8 & 10 & 12 & 18 & 19 \\
(2.7, 4.2, 0, 0, \dots) & 8 & 10 & 13 & 18 & 19 \\
(5.0, 1.8, 0, 0, \dots) & 8 & 10 & 13 & 19 & 20 \\
(3.1, 3.5, 0, 0, \dots) & 8 & 10 & 13 & 18 & 19 \\
\hline
\end{tabular}
    \label{tab:paramkonstan}
\end{table}
\begin{table}[H]
\caption{Iterasi dengan parameter $a_n =\frac{n^2}{n^3+1}, b_n=\frac{2}{n+3},c_n=\frac{4n+2}{{7n+4}}.$}
    \centering
    \begin{tabular}{cccccc}
\hline
\textbf{Nilai Awal} & \multicolumn{5}{c}{\textbf{Jumlah Iterasi}} \\
\hline
$x_0$ & Sabri & JK & Thakur & Abbas & Agarwal \\
\hline
(2.0, 3.0, 0, 0, \dots) & 10 & 13 & 15 & 16 & 28 \\
(3.0, 4.0, 0, 0, \dots) & 10 & 13 & 15 & 16 & 28 \\
(4.0, 1.0, 0, 0, \dots) & 10 & 13 & 16 & 17 & 28 \\
(1.5, 4.0, 0, 0, \dots) & 10 & 13 & 15 & 16 & 27 \\
(2.7, 4.2, 0, 0, \dots) & 10 & 13 & 15 & 17 & 28 \\
(5.0, 1.8, 0, 0, \dots) & 10 & 13 & 16 & 17 & 29 \\
(3.1, 3.5, 0, 0, \dots) & 10 & 13 & 15 & 16 & 28 \\
\hline
\end{tabular}
    \label{tab:paramturun}
\end{table}
\begin{table}[H]
\caption{Iterasi dengan parameter $a_n =1- \frac{\sqrt{4n+9}}{2n+13}, b_n=0.8,c_n=1-\frac{n^2}{\sqrt{n^7+3}}.$}
    \centering
    \begin{tabular}{cccccc}
\hline
\textbf{Nilai Awal} & \multicolumn{5}{c}{\textbf{Jumlah Iterasi}} \\
\hline
$x_0$ & Sabri & JK & Thakur & Abbas & Agarwal \\
\hline
(2.0, 3.0, 0, 0, \dots) & 8 & 10 & 13 & 16 & 19 \\
(3.0, 4.0, 0, 0, \dots) & 8 & 10 & 13 & 16 & 19 \\
(4.0, 1.0, 0, 0, \dots) & 8 & 11 & 13 & 16 & 19 \\
(1.5, 4.0, 0, 0, \dots) & 8 & 10 & 13 & 16 & 19 \\
(2.7, 4.2, 0, 0, \dots) & 8 & 11 & 13 & 16 & 19 \\
(5.0, 1.8, 0, 0, \dots) & 8 & 11 & 13 & 17 & 20 \\
(3.1, 3.5, 0, 0, \dots) & 8 & 10 & 13 & 16 & 19 \\
\hline
\end{tabular}
    \label{tab:paramnaik}
\end{table}
\begin{table}[H]
\caption{Iterasi dengan parameter $a_n = 1-\frac{n^3}{n^5+1}, b_n=\frac{1}{n+1},c_n=\frac{3n+4}{5n^2+4}.$}
    \centering
    \begin{tabular}{cccccc}
\hline
\textbf{Nilai Awal} & \multicolumn{5}{c}{\textbf{Jumlah Iterasi}} \\
\hline
$x_0$ & Sabri & JK & Thakur & Abbas & Agarwal \\
\hline
(2.0, 3.0, 0, 0, \dots) & 8 & 11 & 15 & 25 & 27 \\
(3.0, 4.0, 0, 0, \dots) & 8 & 11 & 15 & 25 & 27 \\
(4.0, 1.0, 0, 0, \dots) & 8 & 11 & 15 & 26 & 27 \\
(1.5, 4.0, 0, 0, \dots) & 8 & 11 & 14 & 25 & 26 \\
(2.7, 4.2, 0, 0, \dots) & 8 & 11 & 15 & 26 & 27 \\
(5.0, 1.8, 0, 0, \dots) & 8 & 11 & 15 & 26 & 27 \\
(3.1, 3.5, 0, 0, \dots) & 8 & 11 & 15 & 26 & 27 \\
\hline
\end{tabular}
    \label{tab:paramnaikturun}
\end{table}
\begin{table}[H]
\caption{Iterasi dengan parameter $a_n = \frac{n^5+1}{5n^7+9}, b_n=1-\frac{3}{n+5},c_n=1-\frac{\sqrt{n^3+8}}{5n^5+9}.$}
    \centering
    \begin{tabular}{cccccc}
\hline
\textbf{Nilai Awal} & \multicolumn{5}{c}{\textbf{Jumlah Iterasi}} \\
\hline
$x_0$ & Sabri & JK & Thakur & Abbas & Agarwal \\
\hline
(2.0, 3.0, 0, 0, \dots) & 9 & 11 & 16 & 13 & 29 \\
(3.0, 4.0, 0, 0, \dots) & 9 & 12 & 16 & 13 & 29 \\
(4.0, 1.0, 0, 0, \dots) & 9 & 12 & 16 & 14 & 29 \\
(1.5, 4.0, 0, 0, \dots) & 9 & 11 & 15 & 13 & 28 \\
(2.7, 4.2, 0, 0, \dots) & 9 & 12 & 16 & 13 & 29 \\
(5.0, 1.8, 0, 0, \dots) & 9 & 12 & 16 & 14 & 29 \\
(3.1, 3.5, 0, 0, \dots) & 9 & 12 & 16 & 13 & 29 \\
\hline
\end{tabular}
    \label{tab:paramturunnaik}
\end{table}
Hasil tersebut menunjukkan bahwa skema iterasi Sabri memiliki laju konvergensi yang lebih baik daripada yang lainnya secara numerik. Dengan parameter tertentu, skema tersebut hanya membutuhkan 8 iterasi untuk konvergen ke titik tetapnya dengan galat kurang dari $10^{-16}$. Secara analitik, skema iterasi Sabri lebih cepat karena nilai galat asimtotiknya, yaitu $\lambda\leq \frac{\alpha^3}{(1-\gamma)^3}$, sedangkan untuk JK dan Thakur adalah $\lambda\leq \frac{\alpha^2}{(1-\gamma)^2}$ yang ditunjukkan dalam teorema berikut ini. 
\begin{thm}
Diberikan $(X,d,G)$ adalah ruang $CAT _p(0)$ yang lengkap dan $W$ adalah himpunan bagian tak kosong dari $X$ yang tertutup, konveks, dan kompak. Jika $T:W\to W$ adalah pemetaan $(\alpha,\beta,\gamma)$-nonekspansif dengan $\alpha+\gamma<1$, $Fix(T)\neq \emptyset$, serta $\{x_n\}$ adalah barisan yang dikonstruksi melalui skema iterasi JK \eqref{eq:jkcat} dengan $\{a_n\},\{c_n\}\subset (0,1)$, maka $\{x_n\}$ memiliki laju konvergensi linear dengan galat asimtotik $\lambda\leq \frac{\alpha^2}{(1-\gamma)^2}$.
\end{thm}
\begin{bukti}
    Berdasarkan Lema \ref{lema:d,d^p} dan Lema \ref{lemma:tutx}, diperoleh bahwa untuk setiap $x^*\in Fix(T)$ berlaku
    \begin{align*}
        d(q_n,x^*) &= d\qty((1-c_n)x_n\oplus c_n Tx_n,Tx^*)\\
        &\leq (1-c_n)d(x_n,x^*)+c_n d(Tx_n,x^*)\\
        &\leq (1-c_n)d(x_n,x^*)+\frac{\alpha c_n}{1-\gamma}d(x_n,x^*)\\
        &= \frac{(1-\gamma)(1-c_n)+\alpha c_n}{(1-\gamma)}d(x_n,x^*),\\
        d(y_n,x^*) &= d\qty(Tq_n,x^*)\\
        &\leq \frac{\alpha}{1-\gamma}d(q_n,x^*),\\
        d(x_{n+1},x^*) &= d\qty(T\qty((1-a_n)Tq_n\oplus a_n Ty_n),Tx^*)\\
        &\leq \frac{\alpha}{1-\gamma}d\qty((1-a_n)Tq_n\oplus a_n Ty_n,x^*)\\
        &\leq \frac{\alpha}{1-\gamma}\qty[(1-a_n)d(Tq_n,x^*)+a_n d(Ty_n,x^*)]\\
        &\leq \qty(\frac{\alpha}{1-\gamma})^2\qty[(1-a_n)d(q_n,x^*)+a_n d(y_n,x^*)]\\
        &\leq \qty(\frac{\alpha}{1-\gamma})^2\qty[(1-a_n)d(q_n,x^*)+a_n \frac{\alpha}{1-\gamma}d(q_n,x^*)]\\
        &= \qty(\frac{\alpha}{1-\gamma})^2\qty[\frac{(1-a_n)(1-\gamma)+a_n\alpha}{(1-\gamma)}d(q_n,x^*)]\\
        &\leq \frac{\alpha^2}{(1-\gamma)^3} \qty((1-a_n)(1-\gamma)+a_n\alpha)\times \frac{(1-\gamma)(1-c_n)+\alpha c_n}{(1-\gamma)}d(x_n,x^*)\\
        &= \frac{\alpha^2}{(1-\gamma)^4}\qty((1-a_n)(1-\gamma)+a_n\alpha)\qty((1-c_n)(1-\gamma)+c_n \alpha)d(x_n,x^*).
    \end{align*}
    Karena $\alpha< 1-\gamma$, didapat 
    \begin{align}
        d(x_{n+1},x^*) &\leq \frac{\alpha^2}{(1-\gamma)^4}\qty(1-\gamma)\qty(1-\gamma)d(x_n,x^*) = \frac{\alpha^2}{(1-\gamma)^2}d(x_n,x^*),
    \end{align}
    sehingga 
    \begin{align*}
        \lim_{n\to\infty} \dfrac{d(x_{n+1},x^*)}{d(x_n,x^*)} \leq \frac{\alpha^2}{(1-\gamma)^2}<+\infty
    \end{align*}
    Hal ini berarti $\{x_n\}$ memiliki laju konvergensi linear dengan galat asimtotik $\lambda\leq \frac{\alpha^2}{(1-\gamma)^2}$.
\end{bukti}
\begin{thm}
Diberikan $(X,d,G)$ adalah ruang $CAT _p(0)$ yang lengkap dan $W$ adalah himpunan bagian tak kosong dari $X$ yang tertutup, konveks, dan kompak. Jika $T:W\to W$ adalah pemetaan $(\alpha,\beta,\gamma)$-nonekspansif dengan $\alpha+\gamma<1$, $Fix(T)\neq \emptyset$, serta $\{x_n\}$ adalah barisan yang dikonstruksi melalui skema iterasi Thakur \eqref{eq:thakurcat} dengan $\{a_n\},\{c_n\}\subset (0,1)$, maka $\{x_n\}$ memiliki laju konvergensi linear dengan galat asimtotik $\lambda\leq \frac{\alpha^2}{(1-\gamma)^2}$.
\end{thm}
\begin{bukti}
    Berdasarkan Lema \ref{lema:d,d^p} dan Lema \ref{lemma:tutx}, diperoleh bahwa untuk setiap $x^*\in Fix(T)$ berlaku
    \begin{align*}
        d(q_n,x^*) &= d\qty((1-c_n)x_n\oplus c_n Tx_n,Tx^*)\\
        &\leq (1-c_n)d(x_n,x^*)+c_n d(Tx_n,x^*)\\
        &\leq (1-c_n)d(x_n,x^*)+\frac{\alpha c_n}{1-\gamma}d(x_n,x^*)\\
        &= \frac{(1-\gamma)(1-c_n)+\alpha c_n}{(1-\gamma)}d(x_n,x^*),\\
        d(y_n,x^*) &= d\qty(T\qty((1-a_n)x_n\oplus a_nq_n),x^*)\\
        &\leq \frac{\alpha}{1-\gamma}d\qty((1-a_n)x_n\oplus a_n q_n,x^*)\\
        &\leq \frac{\alpha}{1-\gamma} \qty[(1-a_n)d(x_n,x^*)+a_n d(q_n,x^*)],\\
        d(x_{n+1},x^*) &= d(Ty_n,Tx^*)\\
        &\leq \frac{\alpha}{(1-\gamma)}d(y_n,x^*)\\
        &\leq \frac{\alpha^2}{(1-\gamma)^2} \qty[(1-a_n)d(x_n,x^*)+a_n d(q_n,x^*)]\\
        &\leq \frac{\alpha^2}{(1-\gamma)^2} \qty[(1-a_n)d(x_n,x^*)+a_n \frac{(1-\gamma)(1-c_n)+\alpha c_n}{(1-\gamma)}d(x_n,x^*)]
    \end{align*}
    Karena $\alpha< 1-\gamma$, didapat 
    \begin{align}
        d(x_{n+1},x^*) &\leq \frac{\alpha^2}{(1-\gamma)^2}d(x_n,x^*),
    \end{align}
    sehingga 
    \begin{align*}
        \lim_{n\to\infty} \dfrac{d(x_{n+1},x^*)}{d(x_n,x^*)} \leq \frac{\alpha^2}{(1-\gamma)^2}<+\infty
    \end{align*}
    Hal ini berarti $\{x_n\}$ memiliki laju konvergensi linear dengan galat asimtotik $\lambda\leq \frac{\alpha^2}{(1-\gamma)^2}$.
\end{bukti}
% Hasil ini secara intuitif dapat dijelaskan dari struktur iterasi Sabri yang melibatkan komposisi pemetaan nonekspansif $T$ secara berlapis pada setiap langkah iterasi, sehingga jarak barisan iterasi terhadap titik tetap berkurang lebih signifikan pada setiap iterasi dibandingkan dengan skema iterasi lainnya.
