\subsection{Masalah Minimalisasi}
Diberikan suatu himpunan tak kosong $X$ dan $f:X\to\mathbb{R}\cup\{\infty\}$ adalah suatu pemetaan. Masalah pencarian titik yang meminimumkan fungsi $f$ dapat diformulasikan sebagai mencari nilai 
\begin{align}\label{eq:minprob}
    x\in X \quad \text{sehingga}\quad f(x)\leq f(y), \quad \text{untuk setiap}\quad y\in X.
\end{align}
Permasalahan ini merupakan permasalahan penting dalam bidang optimasi dan analisis tak linier. Selanjutnya, diperkenalkan operator resolvent dari suatu fungsi.
\begin{defn}
    Diberikan $(X,d,G)$ adalah ruang $CAT _p(0)$ dan $f:X\to \mathbb{R}\cup\{+\infty\}$ adalah suatu fungsi. Untuk $\lambda>0$, operator resolvent $\lambda$ dari $f$ didefinisikan sebagai
    \begin{align}
        J^f_{\lambda}(x) = \text{argmin}_{y\in X} \qty[f(y)+\frac{1}{2\lambda}\qty(d(x,y))^2].
    \end{align}
\end{defn}
\begin{exam}
    Diberikan $(\mathbb{R}^2,\|\cdot\|_2)$ adalah ruang Banach yang juga merupakan $CAT_p(0)$. Misalkan $f:\mathbb{R}^2\to \mathbb{R}\cup\{+\infty\}$ adalah fungsi yang didefinisikan sebagai $f(x_1,x_2) = |x_1| + |x_2|$ untuk setiap $(x_1,x_2)\in \mathbb{R}^2$. Untuk $\lambda>0$ dan $x=(x_1,x_2)\in \mathbb{R}^2$, operator resolvent $J^f_{\lambda}$ dari $f$ adalah sebagai berikut.
    \begin{align*}
        J^f_{\lambda}(x) &= \text{argmin}_{y\in \mathbb{R}^2} \qty[|y_1| + |y_2| + \frac{1}{2\lambda}\qty(\norm{(x_1,y_2)-(y_1,y_2)}_2)^2]\\
        &= \text{argmin}_{y\in \mathbb{R}^2} \qty[|y_1| + |y_2| + \frac{1}{2\lambda}\qty((x_1 - y_1)^2 + (x_2 - y_2)^2)].
    \end{align*}
    % Dengan memisahkan variabel $y_1$ dan $y_2$, diperoleh
    % \begin{align*}  
    %     J^f_{\lambda}(x) &= \left(\text{argmin}_{y_1\in \mathbb{R}} \qty[|y_1| + \frac{1}{2\lambda}(x_1 - y_1)^2], \text{argmin}_{y_2\in \mathbb{R}} \qty[|y_2| + \frac{1}{2\lambda}(x_2 - y_2)^2]\right)\\
    %     &= \left(sgn(x_1)\max\qty(|x_1| - \lambda, 0), sgn(x_2)\max\qty(|x_2| - \lambda, 0)\right).
    % \end{align*}
\end{exam}

Untuk fungsi yang memenuhi kondisi konveks dan \textit{proper lower semi-continuous}, himpunan solusi dari masalah \eqref{eq:minprob} sama dengan himpunan titik tetap dari operator resolvent $J^f_{\lambda}$ (lihat proposisi 6.5 pada \cite{ArizaRuiz2014}). Berikut ini diberikan definisi dari fungsi yang memenuhi kondisi tersebut dengan contoh yang diberikan pada Contoh \ref{con:fkonvk}.

\begin{defn}\cite{Khamsi2017}\label{def:konvgeodesik}
    Diberikan $(X,d,G)$ adalah ruang $CAT _p(0)$. Suatu fungsi $f:X\to \mathbb{R}\cup \{+\infty\}$ disebut konveks secara geodesik jika untuk setiap $t\in(0,1)$ dan $x,y\in X$, berlaku
    \begin{align*}
        f(tx\oplus (1-t)y)\leq tf(x)+(1-t)f(y).
    \end{align*}
\end{defn}
\begin{defn}\cite{Salisu2022}
    Diberikan $(X,d,G)$ adalah ruang $CAT _p(0)$. Suatu fungsi $f:X\to \mathbb{R}\cup \{+\infty\}$ disebut \textit{proper} jika himpunan $D(f):=\{x\in X\mid f(x)<+\infty\}\neq \emptyset$.
\end{defn}
\begin{defn}\cite{Salisu2022}
    Diberikan $(X,d,G)$ adalah ruang $CAT _p(0)$. Suatu fungsi $f:X\to \mathbb{R}\cup \{+\infty\}$ disebut \textit{lower semi-continuous} pada suatu titik $x\in D(f)$ jika $f(x)\leq \liminf_{n\to\infty} f(x_n)$ untuk setiap barisan $\{x_n\}$ yang konvergen di $D(f)$ dengan limit $x\in X$. Jika $f$ \textit{lower semi-continuous} pada setiap titik di $D(f)$, maka $f$ disebut \textit{lower semi-continuous} pada $X$.
\end{defn}

Berdasarkan hal tersebut, didapatkan Teorema berikut. 
\begin{thm}\label{thm:apl1}
    Diberikan $(X,d,G)$ adalah ruang $CAT _p(0)$ dan $f:X\to\mathbb{R}\cup\{+\infty\}$ adalah fungsi yang memenuhi kondisi konveks dan \textit{proper lower semi-continuous}. Untuk $x\in X$, barisan $\{x_n\}$ yang didefinisikan sebagai 
    \begin{align}
        \begin{cases}
            q_n &= J^f_{\lambda}\qty((1-c_n)x_n\oplus c_n J^f_{\lambda} (x_n))\\
            y_n &= J^f_{\lambda} \qty(J^f_{\lambda} (q_n))\\
            x_{n+1} &=J^f_{\lambda} \qty((1-a_n)J^f_{\lambda} (q_n)\oplus a_n J^f_{\lambda} (y_n)),
        \end{cases}
    \end{align}
    dengan $\{a_n\},\{c_n\}\subseteq [a,b]\subset (0,1)$ konvergen-$\Delta$ ke solusi dari permasalahan \eqref{eq:minprob}. Jika $X$ kompak, maka $\{x_n\}$ konvergen kuat. 
\end{thm}
\begin{bukti}
    Dengan menggunakan Lema 4 di \cite{Jost1995}, diketahui bahwa $J^f_{\lambda}$ merupakan pemetaan nonekspansif, sehingga merupakan pemetaan $(\alpha,\beta,\gamma)$-nonekspansif juga. Akibatnya, dengan Teorema \ref{thm:konvD} diperoleh bahwa $\{x_n\}$ konvergen-$\Delta$ ke titik tetap dari $J^f_{\lambda}$. Jika $X$ kompak, Teorema \ref{thm:konvK}, menjamin bahwa konvergensinya kuat. Karena titik tetap dari $J^f_{\lambda}$ sama dengan solusi dari permasalahan \eqref{eq:minprob}, artinya Teorema \ref{thm:apl1} terbukti. 
\end{bukti}

Sebagai gambaran, dilakukan simulasi untuk contoh berikut ini. 
\begin{exam}\label{con:fkonvk}
    Diberikan $(X,d,G)$ adalah ruang $CAT _p(0)$ sebagaimana Contoh \ref{con:Catp} dan $W=\{(x_1,x_2,0,0,\dots)\mid x_1,x_2\in X\}$. Suatu fungsi $f:W\to \mathbb{R}\cup\{+\infty\}$ yang didefinisikan sebagai 
    \begin{align}
        f\qty(x) = 20\qty|x_2-x_1^3|^3+|26-x_1|^3, \quad \text{untuk}\quad x=(x_1,x_2,0,0,\dots)\in W,
    \end{align}
    merupakan fungsi tak konveks pada definisi klasiknya, tetapi konveks secara geodesik dengan geodesik sebagaimana pada Contoh \ref{con:Catp}. Lebih lanjut, $f$ merupakan fungsi yang \textit{proper lower semicontinuous} dan mencapai nilai minimum saat $x_1=26$ dan $x_2=26^3$. 
\end{exam}
Penjelasan dari Contoh \ref{con:fkonvk} diberikan sebagai berikut. 
\begin{enumerate}
    \item Pertama, ditunjukkan bahwa $f$ tak konveks dalam definisi klasiknya, yaitu $f$ memenuhi ketaksamaan
    \begin{align*}
        f\qty((1-t)w + tz) \leq (1-t)f(w) + t f(z),
    \end{align*}
    untuk setiap $t\in(0,1)$ dan $w,z\in W$. Untuk menunjukkan bahwa $f$ tidak konveks dalam definisi klasik, digunakan matriks Hessian, yaitu 
    
    \begin{align*}
        H = \begin{bmatrix}
           \dfrac{\partial^2 f}{\partial x_1^2} & \dfrac{\partial^2 f}{\partial x_1x_2} \vspace*{0.3cm}\\ 
           \dfrac{\partial^2 f}{\partial x_2x_1} & \dfrac{\partial^2 f}{\partial x_2^2}
        \end{bmatrix}.
    \end{align*}
    Ingat kembali bahwa suatu fungsi $f:W\to \mathbb{R}$ adalah konveks jika dan hanya jika matriks Hessian $H$ adalah matriks semidefinit positif untuk setiap $x\in W$ (lihat \cite{Boyd2004}).
    Dapat dihitung bahwa 
    \begin{align*}
        \dfrac{\partial f}{\partial x_1} &= -180(x_2x_1^2-x_1^5)|x_2-x_1^3|-3(26-x_1)|26-x_1|.\\
        \dfrac{\partial^2 f}{\partial x_1^2} &= -180\qty((2x_1x_2-5x_1^4)|x_2-x_1^3|-\frac{3x_1^4(x_2-x_1^3)^2}{|x_2-x_1^3|}) + \frac{6(26-x_1)^2}{|26-x_1|}.\\
        \dfrac{\partial^2 f}{\partial x_1x_2} &= \frac{-360x_1^2(x_2-x_1^3)^2}{|x_2-x_1^3|}.\\
        \dfrac{\partial f}{\partial x_2} &= 60(x_2-x_1^3)|x_2-x_1|^3.\\
        \dfrac{\partial^2 f}{\partial x_2 x_1} &= \frac{-360x_1^2(x_2-x_1^3)^2}{|x_2-x_1^3|}.\\
        \dfrac{\partial^2 f}{\partial x_2^2} &= 60\qty(|x_2-x_1^3|+\frac{(x_2-x_1^3)^2}{|x_2-x_1^3|}).
    \end{align*}
    Jika diambil $x_1=1$ dan $x_2=5$, diperoleh 
    \begin{align*}
        H = \begin{bmatrix}
            -1290 & -1440\\
            -1440 & 480
        \end{bmatrix}.
    \end{align*}
    Diperhatikan bahwa $H$ memiliki persamaan karakteristik $p(\lambda)=\lambda^2+810\lambda-2692800$ dengan akar-akar 
    \begin{align*}
        \lambda_{1,2} = \frac{-810 \pm \sqrt{810^2-4(2692800)}}{2},
    \end{align*}
    yang jelas memiliki akar negatif. Jadi ada $x\in W$ sehingga matriks Hessian $H$ memiliki nilai eigen negatif, yang berarti $H$ bukan matriks semidefinit positif. Akibatnya $f$ bukanlah fungsi konveks dalam definisi klasik. 
    \item Kedua, ditunjukkan bahwa $f$ memenuhi Definisi \ref{def:konvgeodesik}, yaitu fungsi konveks secara geodesik. Sebelum itu, diperhatikan bahwa fungsi $p(x)=|x|^3$ adalah fungsi konveks, karena $p''(x)=\frac{6x^2}{|x|}\geq 0$ untuk setiap $x\in \mathbb{R}\backslash \{0\}$. Dengan fakta tersebut, didapatkan bahwa untuk setiap $t\in(0,1)$ dan $w,z\in X$ berlaku
    \begin{align*}
        f\qty((1-t)w\oplus tz) =~& f\big((1-t)w_1+tz_1, \qty((1-t)w_1+tz_1)^3\\
        &\quad\,\,    -(1-t)(w_1^3-w_2)-t(z_1^3-z_2)\big)\\
        =~& 20\qty|(1-t)(w_2-w_1^3)+t(z_2-z_1)^3|^3\\
        ~&+\qty|26-\qty((1-t)w_1+tz_1)|^3\\
        \leq ~& 20 \qty((1-t)|w_2-w_1^3|^3+t|z_2-z_1^3|^3)\\
        ~&+ \qty |(1-t)(26-w_1)+t(26-z_1)|^3\\
        \leq ~& (1-t) \qty(20|w_2-w_1^3|^3+|26-w_1|^3)\\
        ~&+ t\qty(20|z_2-z_1^3|^3+|26-z_1|^3)\\
        =~& (1-t)f(w)+tf(z).
    \end{align*}
    Jadi $f$ adalah fungsi konveks secara geodesik. 
\end{enumerate}
Selanjutnya, diperhatikan bahwa $(0,0,0,\dots)\in D(f)$ sehingga $D(f)=\{x\in X\mid f(x)<+\infty\}\neq\emptyset$, yang berarti $f$ adalah fungsi \textit{proper}. Diperhatikan pula bahwa $f$ adalah fungsi kontinu, yang berarti juga memenuhi kondisi \textit{lower semicontinuous}. Dalam hal ini kondisi pada Teorema \ref{thm:konvK} terpenuhi. Diperhatikan bahwa 
nilai minimum dari fungsi $f$ pada Contoh \ref{con:fkonvk} adalah $0$, yang dicapai saat $x_2-x_1^3=0$ dan $26-x_1=0$, atau pada titik $x^*=(26, 17576, 0,0,\dots)$. Dengan demikian, berdasarkan Teorema \ref{thm:apl1}, barisan yang dihasilkan dari skema iterasi Sabri dan JK konvergen ke titik $x^*$. 

Untuk ilustrasi, dilakukan simulasi numerik untuk Contoh \ref{con:fkonvk} dengan perangkat lunak Google Colab yang menggunakan bahasa pemrograman python. Kode dari program ini dapat dilihat pada Lampiran A.2. Pada simulasi ini digunakan parameter $\lambda=20$, $a_n=0.34$, dan $c_n=0.93$, serta nilai awal $x_0=(10, 10, 0, 0, \dots)$. Simulasi dilakukan sebanyak 55 iterasi yang hasilnya dapat dilihat pada \ref{tab:simulasimin} dan \ref{tab:errsimulasi}. Galat dari simulasi ini juga digambarkan sebagai grafik dengan koordinat-$y$ berskala log yang disajikan pada \ref{fig:galatsim}.
\begin{longtable}{|c|c|c|}
\caption{\centering Hasil simulasi sebanyak 55 iterasi untuk Contoh \ref{con:fkonvk} dengan skema iterasi Sabri dan JK}\label{tab:simulasimin}\\
\hline
\textbf{n} & \textbf{Sabri} & \textbf{JK} \\
\hline
\endfirsthead 
\multicolumn{3}{c}{ \thetable{} -- Lanjutan dari halaman sebelumnya} \\
\hline
\textbf{n} & \textbf{Sabri} & \textbf{JK} \\
\hline
\endhead 
\hline
% \multicolumn{3}{|c|}{{Lanjut pada halaman berikutnya}} \\
\endfoot 
\hline
\endlastfoot
\hline
0 & (10.00000, 10.00000, 0, 0, \dots) & (10.00000, 10.00000, 0, 0, \dots) \\
1 & (25.04431, 15708.11708, 0, 0, \dots) & (22.68123, 11668.32829, 0, 0, \dots) \\
2 & (25.98980, 17555.31676, 0, 0, \dots) & (25.96654, 17508.22787, 0, 0, \dots) \\
3 & (25.99672, 17569.35288, 0, 0, \dots) & (25.99313, 17562.07883, 0, 0, \dots) \\
4 & (25.99816, 17572.26781, 0, 0, \dots) & (25.99666, 17569.23601, 0, 0, \dots) \\
5 & (25.99874, 17573.44601, 0, 0, \dots) & (25.99787, 17571.67427, 0, 0, \dots) \\
6 & (25.99905, 17574.07075, 0, 0, \dots) & (25.99845, 17572.85677, 0, 0, \dots) \\
7 & (25.99924, 17574.45456, 0, 0, \dots) & (25.99879, 17573.54447, 0, 0, \dots) \\
8 & (25.99937, 17574.71316, 0, 0, \dots) & (25.99901, 17573.99088, 0, 0, \dots) \\
9 & (25.99946, 17574.89874, 0, 0, \dots) & (25.99916, 17574.30275, 0, 0, \dots) \\
10 & (25.99953, 17575.03819, 0, 0, \dots) & (25.99928, 17574.53235, 0, 0, \dots) \\
11 & (25.99958, 17575.14668, 0, 0, \dots) & (25.99936, 17574.70814, 0, 0, \dots) \\
12 & (25.99962, 17575.23343, 0, 0, \dots) & (25.99943, 17574.84690, 0, 0, \dots) \\
13 & (25.99966, 17575.30434, 0, 0, \dots) & (25.99949, 17574.95913, 0, 0, \dots) \\
14 & (25.99969, 17575.36336, 0, 0, \dots) & (25.99953, 17575.05170, 0, 0, \dots) \\
15 & (25.99971, 17575.41324, 0, 0, \dots) & (25.99957, 17575.12934, 0, 0, \dots) \\
16 & (25.99973, 17575.45593, 0, 0, \dots) & (25.99960, 17575.19536, 0, 0, \dots) \\
17 & (25.99975, 17575.49287, 0, 0, \dots) & (25.99963, 17575.25217, 0, 0, \dots) \\
18 & (25.99977, 17575.52516, 0, 0, \dots) & (25.99966, 17575.30157, 0, 0, \dots) \\
19 & (25.99978, 17575.55360, 0, 0, \dots) & (25.99968, 17575.34490, 0, 0, \dots) \\
20 & (25.99979, 17575.57886, 0, 0, \dots) & (25.99970, 17575.38321, 0, 0, \dots) \\
21 & (25.99980, 17575.60142, 0, 0, \dots) & (25.99971, 17575.41733, 0, 0, \dots) \\
22 & (25.99981, 17575.62171, 0, 0, \dots) & (25.99973, 17575.44790, 0, 0, \dots) \\
23 & (25.99982, 17575.64004, 0, 0, \dots) & (25.99974, 17575.47544, 0, 0, \dots) \\
24 & (25.99983, 17575.65669, 0, 0, \dots) & (25.99975, 17575.50039, 0, 0, \dots) \\
25 & (25.99984, 17575.67187, 0, 0, \dots) & (25.99976, 17575.52309, 0, 0, \dots) \\
26 & (25.99985, 17575.68578, 0, 0, \dots) & (25.99978, 17575.54382, 0, 0, \dots) \\
27 & (25.99985, 17575.69856, 0, 0, \dots) & (25.99978, 17575.56285, 0, 0, \dots) \\
28 & (25.99986, 17575.71034, 0, 0, \dots) & (25.99979, 17575.58035, 0, 0, \dots) \\
29 & (25.99986, 17575.72124, 0, 0, \dots) & (25.99980, 17575.59652, 0, 0, \dots) \\
30 & (25.99987, 17575.73136, 0, 0, \dots) & (25.99981, 17575.61150, 0, 0, \dots) \\
31 & (25.99987, 17575.74077, 0, 0, \dots) & (25.99982, 17575.62541, 0, 0, \dots) \\
32 & (25.99988, 17575.74954, 0, 0, \dots) & (25.99982, 17575.63836, 0, 0, \dots) \\
33 & (25.99988, 17575.75775, 0, 0, \dots) & (25.99983, 17575.65045, 0, 0, \dots) \\
34 & (25.99988, 17575.76543, 0, 0, \dots) & (25.99983, 17575.66177, 0, 0, \dots) \\
35 & (25.99989, 17575.77265, 0, 0, \dots) & (25.99984, 17575.67237, 0, 0, \dots) \\
36 & (25.99989, 17575.77943, 0, 0, \dots) & (25.99984, 17575.68234, 0, 0, \dots) \\
37 & (25.99989, 17575.78583, 0, 0, \dots) & (25.99985, 17575.69172, 0, 0, \dots) \\
38 & (25.99990, 17575.79186, 0, 0, \dots) & (25.99985, 17575.70056, 0, 0, \dots) \\
39 & (25.99990, 17575.79757, 0, 0, \dots) & (25.99986, 17575.70892, 0, 0, \dots) \\
40 & (25.99990, 17575.80297, 0, 0, \dots) & (25.99986, 17575.71682, 0, 0, \dots) \\
41 & (25.99991, 17575.80809, 0, 0, \dots) & (25.99986, 17575.72430, 0, 0, \dots) \\
42 & (25.99991, 17575.81295, 0, 0, \dots) & (25.99987, 17575.73141, 0, 0, \dots) \\
43 & (25.99991, 17575.81758, 0, 0, \dots) & (25.99987, 17575.73815, 0, 0, \dots) \\
44 & (25.99991, 17575.82198, 0, 0, \dots) & (25.99987, 17575.74457, 0, 0, \dots) \\
45 & (25.99991, 17575.82617, 0, 0, \dots) & (25.99988, 17575.75068, 0, 0, \dots) \\
46 & (25.99992, 17575.83017, 0, 0, \dots) & (25.99988, 17575.75651, 0, 0, \dots) \\
47 & (25.99992, 17575.83400, 0, 0, \dots) & (25.99988, 17575.76207, 0, 0, \dots) \\
48 & (25.99992, 17575.83765, 0, 0, \dots) & (25.99989, 17575.76739, 0, 0, \dots) \\
49 & (25.99992, 17575.84115, 0, 0, \dots) & (25.99989, 17575.77247, 0, 0, \dots) \\
50 & (25.99992, 17575.84450, 0, 0, \dots) & (25.99989, 17575.77733, 0, 0, \dots) \\
51 & (25.99992, 17575.84771, 0, 0, \dots) & (25.99989, 17575.78200, 0, 0, \dots) \\
52 & (25.99993, 17575.85079, 0, 0, \dots) & (25.99989, 17575.78647, 0, 0, \dots) \\
53 & (25.99993, 17575.85375, 0, 0, \dots) & (25.99990, 17575.79076, 0, 0, \dots) \\
54 & (25.99993, 17575.85660, 0, 0, \dots) & (25.99990, 17575.79489, 0, 0, \dots) \\
55 & (25.99993, 17575.85934, 0, 0, \dots) & (25.99990, 17575.79885, 0, 0, \dots) \\
\hline
\end{longtable}


\begin{longtable}{|c|>{\centering\arraybackslash}p{4cm} | >{\centering\arraybackslash}p{4cm} |}
\caption{\centering Galat simulasi $d(x_n,x^*)$ untuk Contoh \ref{con:fkonvk} dengan skema iterasi Sabri dan JK}\label{tab:errsimulasi}\\
\hline
{\textbf{n}} & {\textbf{Galat Sabri}} & {\textbf{Galat JK}} \\
\hline
\endfirsthead 
\multicolumn{3}{c}{\centering \thetable{} -- Lanjutan dari halaman sebelumnya} \\
\hline
{\textbf{n}} & {\textbf{Sabri}} & {\textbf{JK}} \\
\hline
\endhead 
% \multicolumn{3}{c}{{Lanjut pada halaman berikutnya}} \\
\hline
\endfoot 
\hline
\endlastfoot
\hline
1 & $9.5\times10^{-1}$ & $3.3\times10^{0}$ \\
2 & $1.0\times10^{-2}$ & $3.3\times10^{-2}$ \\
3 & $3.2\times10^{-3}$ & $6.8\times10^{-3}$ \\
4 & $1.8\times10^{-3}$ & $3.3\times10^{-3}$ \\
5 & $1.2\times10^{-3}$ & $2.1\times10^{-3}$ \\
6 & $9.5\times10^{-4}$ & $1.5\times10^{-3}$ \\
7 & $7.6\times10^{-4}$ & $1.2\times10^{-3}$ \\
8 & $6.3\times10^{-4}$ & $9.9\times10^{-4}$ \\
9 & $5.4\times10^{-4}$ & $8.3\times10^{-4}$ \\
10 & $4.7\times10^{-4}$ & $7.2\times10^{-4}$ \\
11 & $4.2\times10^{-4}$ & $6.3\times10^{-4}$ \\
12 & $3.7\times10^{-4}$ & $5.6\times10^{-4}$ \\
13 & $3.4\times10^{-4}$ & $5.1\times10^{-4}$ \\
14 & $3.1\times10^{-4}$ & $4.6\times10^{-4}$ \\
15 & $2.8\times10^{-4}$ & $4.2\times10^{-4}$ \\
16 & $2.6\times10^{-4}$ & $3.9\times10^{-4}$ \\
17 & $2.5\times10^{-4}$ & $3.6\times10^{-4}$ \\
18 & $2.3\times10^{-4}$ & $3.4\times10^{-4}$ \\
19 & $2.2\times10^{-4}$ & $3.2\times10^{-4}$ \\
20 & $2.0\times10^{-4}$ & $3.0\times10^{-4}$ \\
21 & $1.9\times10^{-4}$ & $2.8\times10^{-4}$ \\
22 & $1.8\times10^{-4}$ & $2.7\times10^{-4}$ \\
23 & $1.7\times10^{-4}$ & $2.5\times10^{-4}$ \\
24 & $1.6\times10^{-4}$ & $2.4\times10^{-4}$ \\
25 & $1.6\times10^{-4}$ & $2.3\times10^{-4}$ \\
26 & $1.5\times10^{-4}$ & $2.2\times10^{-4}$ \\
27 & $1.4\times10^{-4}$ & $2.1\times10^{-4}$ \\
28 & $1.4\times10^{-4}$ & $2.0\times10^{-4}$ \\
29 & $1.3\times10^{-4}$ & $1.9\times10^{-4}$ \\
30 & $1.3\times10^{-4}$ & $1.9\times10^{-4}$ \\
31 & $1.2\times10^{-4}$ & $1.8\times10^{-4}$ \\
32 & $1.2\times10^{-4}$ & $1.7\times10^{-4}$ \\
33 & $1.1\times10^{-4}$ & $1.7\times10^{-4}$ \\
34 & $1.1\times10^{-4}$ & $1.6\times10^{-4}$ \\
35 & $1.1\times10^{-4}$ & $1.6\times10^{-4}$ \\
36 & $1.0\times10^{-4}$ & $1.5\times10^{-4}$ \\
37 & $1.0\times10^{-4}$ & $1.5\times10^{-4}$ \\
38 & $1.0\times10^{-4}$ & $1.4\times10^{-4}$ \\
39 & $9.9\times10^{-5}$ & $1.4\times10^{-4}$ \\
40 & $9.7\times10^{-5}$ & $1.3\times10^{-4}$ \\
41 & $9.4\times10^{-5}$ & $1.3\times10^{-4}$ \\
42 & $9.2\times10^{-5}$ & $1.3\times10^{-4}$ \\
43 & $8.9\times10^{-5}$ & $1.2\times10^{-4}$ \\
44 & $8.7\times10^{-5}$ & $1.2\times10^{-4}$ \\
45 & $8.5\times10^{-5}$ & $1.2\times10^{-4}$ \\
46 & $8.3\times10^{-5}$ & $1.2\times10^{-4}$ \\
47 & $8.1\times10^{-5}$ & $1.1\times10^{-4}$ \\
48 & $8.0\times10^{-5}$ & $1.1\times10^{-4}$ \\
49 & $7.8\times10^{-5}$ & $1.1\times10^{-4}$ \\
50 & $7.6\times10^{-5}$ & $1.0\times10^{-4}$ \\
51 & $7.5\times10^{-5}$ & $1.0\times10^{-4}$ \\
52 & $7.3\times10^{-5}$ & $1.0\times10^{-4}$ \\
53 & $7.2\times10^{-5}$ & $1.0\times10^{-4}$ \\
54 & $7.0\times10^{-5}$ & $1.0\times10^{-4}$ \\
55 & $6.9\times10^{-5}$ & $9.9\times10^{-5}$ \\
\hline
\end{longtable}
\begin{figure}[H]
    \centering
    \includegraphics[width=0.9\linewidth]{Bab_4/galatsim.png}
    \caption{Galat $d(x_n,x^*)$ (skala log) vs iterasi}
    \label{fig:galatsim}
\end{figure}
