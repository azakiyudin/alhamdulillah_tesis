\subsection{Masalah Minimalisasi}
Diberikan suatu himpunan tak kosong $X$ dan $f:X\to\mathbb{R}\cup\{\infty\}$ adalah suatu pemetaan. Masalah pencarian titik yang meminimumkan fungsi $f$ dapat diformulasikan sebagai mencari nilai 
\begin{align}\label{eq:minprob}
    x\in X \quad \text{sehingga}\quad f(x)\leq f(y), \quad \text{untuk setiap}\quad y\in X.
\end{align}
Permasalahan ini merupakan permasalahan penting dalam bidang optimasi dan analisis tak linier. Selanjutnya, diperkenalkan operator resolvent dari suatu fungsi.
\begin{defn}
    Diberikan $(X,d,G)$ adalah ruang $CAT _p(0)$ dan $f:X\to \mathbb{R}\cup\{+\infty\}$ adalah suatu fungsi. Untuk $\lambda>0$, operator resolvent $\lambda$ dari $f$ didefinisikan sebagai
    \begin{align}
        J^f_{\lambda}(x) = \text{argmin}_{y\in X} \qty[f(y)+\frac{1}{2\lambda}\qty(d(x,y))^2].
    \end{align}
\end{defn}
\begin{exam}
    Diberikan $(\mathbb{R}^2,\|\cdot\|_2)$ adalah ruang Banach yang juga merupakan $CAT_p(0)$. Misalkan $f:\mathbb{R}^2\to \mathbb{R}\cup\{+\infty\}$ adalah fungsi yang didefinisikan sebagai $f(x_1,x_2) = |x_1| + |x_2|$ untuk setiap $(x_1,x_2)\in \mathbb{R}^2$. Untuk $\lambda>0$ dan $x=(x_1,x_2)\in \mathbb{R}^2$, operator resolvent $J^f_{\lambda}$ dari $f$ adalah sebagai berikut.
    \begin{align*}
        J^f_{\lambda}(x) &= \text{argmin}_{y\in \mathbb{R}^2} \qty[|y_1| + |y_2| + \frac{1}{2\lambda}\qty(\norm{(x_1,y_2)-(y_1,y_2)}_2)^2]\\
        &= \text{argmin}_{y\in \mathbb{R}^2} \qty[|y_1| + |y_2| + \frac{1}{2\lambda}\qty((x_1 - y_1)^2 + (x_2 - y_2)^2)].
    \end{align*}
    % Dengan memisahkan variabel $y_1$ dan $y_2$, diperoleh
    % \begin{align*}  
    %     J^f_{\lambda}(x) &= \left(\text{argmin}_{y_1\in \mathbb{R}} \qty[|y_1| + \frac{1}{2\lambda}(x_1 - y_1)^2], \text{argmin}_{y_2\in \mathbb{R}} \qty[|y_2| + \frac{1}{2\lambda}(x_2 - y_2)^2]\right)\\
    %     &= \left(sgn(x_1)\max\qty(|x_1| - \lambda, 0), sgn(x_2)\max\qty(|x_2| - \lambda, 0)\right).
    % \end{align*}
\end{exam}

Untuk fungsi yang memenuhi kondisi konveks dan \textit{proper lower semi-continuous}, himpunan solusi dari masalah \eqref{eq:minprob} sama dengan himpunan titik tetap dari operator resolvent $J^f_{\lambda}$ (lihat proposisi 6.5 pada \cite{ArizaRuiz2014}). Berikut ini diberikan definisi dari fungsi yang memenuhi kondisi tersebut dengan contoh yang diberikan pada Contoh \ref{con:fkonvk}.

\begin{defn}\cite{Khamsi2017}\label{def:konvgeodesik}
    Diberikan $(X,d,G)$ adalah ruang $CAT _p(0)$. Suatu fungsi $f:X\to \mathbb{R}\cup \{+\infty\}$ disebut konveks secara geodesik jika untuk setiap $t\in(0,1)$ dan $x,y\in X$, berlaku
    \begin{align*}
        f(tx\oplus (1-t)y)\leq tf(x)+(1-t)f(y).
    \end{align*}
\end{defn}
\begin{defn}\cite{Salisu2022}
    Diberikan $(X,d,G)$ adalah ruang $CAT _p(0)$. Suatu fungsi $f:X\to \mathbb{R}\cup \{+\infty\}$ disebut \textit{proper} jika himpunan $D(f):=\{x\in X\mid f(x)<+\infty\}\neq \emptyset$.
\end{defn}
\begin{defn}\cite{Salisu2022}
    Diberikan $(X,d,G)$ adalah ruang $CAT _p(0)$. Suatu fungsi $f:X\to \mathbb{R}\cup \{+\infty\}$ disebut \textit{lower semi-continuous} pada suatu titik $x\in D(f)$ jika $f(x)\leq \liminf_{n\to\infty} f(x_n)$ untuk setiap barisan $\{x_n\}_{n=1}^{\infty}$ yang konvergen di $D(f)$ dengan limit $x\in X$. Jika $f$ \textit{lower semi-continuous} pada setiap titik di $D(f)$, maka $f$ disebut \textit{lower semi-continuous} pada $X$.
\end{defn}

Berdasarkan hal tersebut, didapatkan Teorema berikut. 
\begin{thm}\label{thm:apl1}
    Diberikan $(X,d,G)$ adalah ruang $CAT _p(0)$ dan $f:X\to\mathbb{R}\cup\{+\infty\}$ adalah fungsi yang memenuhi kondisi konveks dan \textit{proper lower semi-continuous}. Untuk $x\in X$, barisan $\{x_n\}_{n=1}^{\infty}$ yang didefinisikan sebagai 
    \begin{align}
        \begin{cases}
            q_n &= J^f_{\lambda}\qty((1-c_n)x_n\oplus c_n J^f_{\lambda} (x_n))\\
            y_n &= J^f_{\lambda} \qty(J^f_{\lambda} (q_n))\\
            x_{n+1} &=J^f_{\lambda} \qty((1-a_n)J^f_{\lambda} (q_n)\oplus a_n J^f_{\lambda} (y_n)),
        \end{cases}
    \end{align}
    dengan $\{a_n\}_{n=1}^{\infty},\{c_n\}_{n=1}^{\infty}\subseteq [a,b]\subset (0,1)$ konvergen-$\Delta$ ke solusi dari permasalahan \eqref{eq:minprob}. Jika $X$ kompak, maka $\{x_n\}_{n=1}^{\infty}$ konvergen kuat. 
\end{thm}
\begin{bukti}
    Dengan menggunakan Lema 4 di \cite{Jost1995}, diketahui bahwa $J^f_{\lambda}$ merupakan pemetaan nonekspansif, sehingga merupakan pemetaan $(\alpha,\beta,\gamma)$-nonekspansif juga. Akibatnya, dengan Teorema \ref{thm:konvD} diperoleh bahwa $\{x_n\}_{n=1}^{\infty}$ konvergen-$\Delta$ ke titik tetap dari $J^f_{\lambda}$. Jika $X$ kompak, Teorema \ref{thm:konvK}, menjamin bahwa konvergensinya kuat. Karena titik tetap dari $J^f_{\lambda}$ sama dengan solusi dari permasalahan \eqref{eq:minprob}, artinya Teorema \ref{thm:apl1} terbukti. 
\end{bukti}

Sebagai gambaran, dilakukan simulasi untuk contoh berikut ini. 
\begin{exam}\label{con:fkonvk}
    Diberikan $(X,d,G)$ adalah ruang $CAT _p(0)$ sebagaimana Contoh \ref{con:Catp} dan $W=\{(x_1,x_2,0,0,\dots)\mid x_1,x_2\in X\}$. Suatu fungsi $f:W\to \mathbb{R}\cup\{+\infty\}$ yang didefinisikan sebagai 
    \begin{align}
        f\qty(x) = 20\qty|x_2-x_1^3|^3+|26-x_1|^3, \quad \text{untuk}\quad x=(x_1,x_2,0,0,\dots)\in W,
    \end{align}
    merupakan fungsi tak konveks pada definisi klasiknya, tetapi konveks secara geodesik dengan geodesik sebagaimana pada Contoh \ref{con:Catp}. Lebih lanjut, $f$ merupakan fungsi yang \textit{proper lower semicontinuous} dan mencapai nilai minimum saat $x_1=26$ dan $x_2=26^3$. 
\end{exam}
Penjelasan dari Contoh \ref{con:fkonvk} diberikan sebagai berikut. 
\begin{enumerate}
    \item Pertama, ditunjukkan bahwa $f$ tak konveks dalam definisi klasiknya, yaitu $f$ memenuhi ketaksamaan
    \begin{align*}
        f\qty((1-t)w + tz) \leq (1-t)f(w) + t f(z),
    \end{align*}
    untuk setiap $t\in(0,1)$ dan $w,z\in W$. Untuk menunjukkan bahwa $f$ tidak konveks dalam definisi klasik, digunakan matriks Hessian, yaitu 
    
    \begin{align*}
        H = \begin{bmatrix}
           \dfrac{\partial^2 f}{\partial x_1^2} & \dfrac{\partial^2 f}{\partial x_1x_2} \vspace*{0.3cm}\\ 
           \dfrac{\partial^2 f}{\partial x_2x_1} & \dfrac{\partial^2 f}{\partial x_2^2}
        \end{bmatrix}.
    \end{align*}
    Ingat kembali bahwa suatu fungsi $f:W\to \mathbb{R}$ adalah konveks jika dan hanya jika matriks Hessian $H$ adalah matriks semidefinit positif untuk setiap $x\in W$ (lihat \cite{Boyd2004}).
    Dapat dihitung bahwa 
    \begin{align*}
        \dfrac{\partial f}{\partial x_1} &= -180(x_2x_1^2-x_1^5)|x_2-x_1^3|-3(26-x_1)|26-x_1|.\\
        \dfrac{\partial^2 f}{\partial x_1^2} &= -180\qty((2x_1x_2-5x_1^4)|x_2-x_1^3|-\frac{3x_1^4(x_2-x_1^3)^2}{|x_2-x_1^3|}) + \frac{6(26-x_1)^2}{|26-x_1|}.\\
        \dfrac{\partial^2 f}{\partial x_1x_2} &= \frac{-360x_1^2(x_2-x_1^3)^2}{|x_2-x_1^3|}.\\
        \dfrac{\partial f}{\partial x_2} &= 60(x_2-x_1^3)|x_2-x_1|^3.\\
        \dfrac{\partial^2 f}{\partial x_2 x_1} &= \frac{-360x_1^2(x_2-x_1^3)^2}{|x_2-x_1^3|}.\\
        \dfrac{\partial^2 f}{\partial x_2^2} &= 60\qty(|x_2-x_1^3|+\frac{(x_2-x_1^3)^2}{|x_2-x_1^3|}).
    \end{align*}
    Jika diambil $x_1=1$ dan $x_2=5$, diperoleh 
    \begin{align*}
        H = \begin{bmatrix}
            -1290 & -1440\\
            -1440 & 480
        \end{bmatrix}.
    \end{align*}
    Diperhatikan bahwa $H$ mempunyai persamaan karakteristik $p(\lambda)=\lambda^2+810\lambda-2692800$ dengan akar-akar 
    \begin{align*}
        \lambda_{1,2} = \frac{-810 \pm \sqrt{810^2-4(2692800)}}{2},
    \end{align*}
    yang jelas mempunyai akar negatif. Jadi ada $x\in W$ sehingga matriks Hessian $H$ mempunyai nilai eigen negatif, yang berarti $H$ bukan matriks semidefinit positif. Akibatnya $f$ bukanlah fungsi konveks dalam definisi klasik. 
    \item Kedua, ditunjukkan bahwa $f$ memenuhi Definisi \ref{def:konvgeodesik}, yaitu fungsi konveks secara geodesik. Sebelum itu, diperhatikan bahwa fungsi $p(x)=|x|^3$ adalah fungsi konveks, karena $p''(x)=\frac{6x^2}{|x|}\geq 0$ untuk setiap $x\in \mathbb{R}\backslash \{0\}$. Dengan fakta tersebut, didapatkan bahwa untuk setiap $t\in(0,1)$ dan $w,z\in X$ berlaku
    \begin{align*}
        f\qty((1-t)w\oplus tz) =~& f\big((1-t)w_1+tz_1, \qty((1-t)w_1+tz_1)^3\\
        &\quad\,\,    -(1-t)(w_1^3-w_2)-t(z_1^3-z_2)\big)\\
        =~& 20\qty|(1-t)(w_2-w_1^3)+t(z_2-z_1)^3|^3\\
        ~&+\qty|26-\qty((1-t)w_1+tz_1)|^3\\
        \leq ~& 20 \qty((1-t)|w_2-w_1^3|^3+t|z_2-z_1^3|^3)\\
        ~&+ \qty |(1-t)(26-w_1)+t(26-z_1)|^3\\
        \leq ~& (1-t) \qty(20|w_2-w_1^3|^3+|26-w_1|^3)\\
        ~&+ t\qty(20|z_2-z_1^3|^3+|26-z_1|^3)\\
        =~& (1-t)f(w)+tf(z).
    \end{align*}
    Jadi $f$ adalah fungsi konveks secara geodesik. 
\end{enumerate}
Selanjutnya, diperhatikan bahwa $(0,0,0,\dots)\in D(f)$ sehingga $D(f)=\{x\in X\mid f(x)<+\infty\}\neq\emptyset$, yang berarti $f$ adalah fungsi \textit{proper}. Diperhatikan pula bahwa $f$ adalah fungsi kontinu, yang berarti juga memenuhi kondisi \textit{lower semicontinuous}. Dalam hal ini kondisi pada Teorema \ref{thm:konvK} terpenuhi. Diperhatikan bahwa 
nilai minimum dari fungsi $f$ pada Contoh \ref{con:fkonvk} adalah $0$, yang dicapai saat $x_2-x_1^3=0$ dan $26-x_1=0$, atau pada titik $x^*=(26, 17576, 0,0,\dots)$. Dengan demikian, berdasarkan Teorema \ref{thm:apl1}, barisan yang dihasilkan dari skema iterasi Sabri dan JK konvergen ke titik $x^*$. 

Untuk ilustrasi, dilakukan simulasi numerik untuk Contoh \ref{con:fkonvk} dengan perangkat lunak Google Colab yang menggunakan bahasa pemrograman python. Kode dari program ini dapat dilihat pada Lampiran A.2. Pada simulasi ini digunakan parameter $\lambda=20$, $a_n=0.74$, dan $c_n=0.93$, serta nilai awal $x_0=(10, 15, 0, 0, \dots)$. Simulasi dilakukan sebanyak 72 iterasi yang hasilnya dapat dilihat pada \ref{tab:simulasimin} dan \ref{tab:errsimulasi}. Galat dari simulasi ini juga digambarkan sebagai grafik dengan koordinat-$y$ berskala log yang disajikan pada \ref{fig:galatsim}.
\begin{longtable}{|c|c|c|}
\caption{\centering Hasil simulasi sebanyak 72 iterasi untuk Contoh \ref{con:fkonvk} dengan skema iterasi Sabri dan JK}\label{tab:simulasimin}\\
\hline
\textbf{n} & \textbf{Sabri} & \textbf{JK} \\
\hline
\endfirsthead 
\multicolumn{3}{c}{ \thetable{} -- Lanjutan dari halaman sebelumnya} \\
\hline
\textbf{n} & \textbf{Sabri} & \textbf{JK} \\
\hline
\endhead 
\hline
% \multicolumn{3}{|c|}{{Lanjut pada halaman berikutnya}} \\
\endfoot 
\hline
\endlastfoot
\hline
0 & (10.00000, 15.00000, 0, 0, \dots) & (10.00000, 15.00000, 0, 0, \dots) \\
1 & (25.77581, 17125.21877, 0, 0, \dots) & (23.07560, 12287.24784, 0, 0, \dots) \\
2 & (25.99383, 17563.49526, 0, 0, \dots) & (25.97620, 17527.77515, 0, 0, \dots) \\
3 & (25.99765, 17571.24240, 0, 0, \dots) & (25.99458, 17565.01472, 0, 0, \dots) \\
4 & (25.99860, 17573.17051, 0, 0, \dots) & (25.99726, 17570.44169, 0, 0, \dots) \\
5 & (25.99902, 17574.00765, 0, 0, \dots) & (25.99821, 17572.37307, 0, 0, \dots) \\
6 & (25.99925, 17574.46912, 0, 0, \dots) & (25.99868, 17573.33297, 0, 0, \dots) \\
7 & (25.99939, 17574.75968, 0, 0, \dots) & (25.99896, 17573.90002, 0, 0, \dots) \\
8 & (25.99949, 17574.95880, 0, 0, \dots) & (25.99915, 17574.27215, 0, 0, \dots) \\
9 & (25.99956, 17575.10350, 0, 0, \dots) & (25.99928, 17574.53424, 0, 0, \dots) \\
10 & (25.99961, 17575.21329, 0, 0, \dots) & (25.99937, 17574.72839, 0, 0, \dots) \\
11 & (25.99965, 17575.29936, 0, 0, \dots) & (25.99945, 17574.87778, 0, 0, \dots) \\
12 & (25.99969, 17575.36861, 0, 0, \dots) & (25.99951, 17574.99618, 0, 0, \dots) \\
13 & (25.99972, 17575.42551, 0, 0, \dots) & (25.99955, 17575.09225, 0, 0, \dots) \\
14 & (25.99974, 17575.47307, 0, 0, \dots) & (25.99959, 17575.17173, 0, 0, \dots) \\
15 & (25.99976, 17575.51342, 0, 0, \dots) & (25.99962, 17575.23855, 0, 0, \dots) \\
16 & (25.99978, 17575.54807, 0, 0, \dots) & (25.99965, 17575.29549, 0, 0, \dots) \\
17 & (25.99979, 17575.57814, 0, 0, \dots) & (25.99968, 17575.34458, 0, 0, \dots) \\
18 & (25.99980, 17575.60448, 0, 0, \dots) & (25.99970, 17575.38733, 0, 0, \dots) \\
19 & (25.99982, 17575.62774, 0, 0, \dots) & (25.99972, 17575.42489, 0, 0, \dots) \\
20 & (25.99983, 17575.64844, 0, 0, \dots) & (25.99973, 17575.45814, 0, 0, \dots) \\
21 & (25.99984, 17575.66696, 0, 0, \dots) & (25.99975, 17575.48779, 0, 0, \dots) \\
22 & (25.99984, 17575.68364, 0, 0, \dots) & (25.99976, 17575.51438, 0, 0, \dots) \\
23 & (25.99985, 17575.69874, 0, 0, \dots) & (25.99977, 17575.53837, 0, 0, \dots) \\
24 & (25.99986, 17575.71246, 0, 0, \dots) & (25.99978, 17575.56011, 0, 0, \dots) \\
25 & (25.99986, 17575.72500, 0, 0, \dots) & (25.99979, 17575.57990, 0, 0, \dots) \\
26 & (25.99987, 17575.73649, 0, 0, \dots) & (25.99980, 17575.59801, 0, 0, \dots) \\
27 & (25.99988, 17575.74707, 0, 0, \dots) & (25.99981, 17575.61462, 0, 0, \dots) \\
28 & (25.99988, 17575.75683, 0, 0, \dots) & (25.99982, 17575.62992, 0, 0, \dots) \\
29 & (25.99988, 17575.76587, 0, 0, \dots) & (25.99982, 17575.64407, 0, 0, \dots) \\
30 & (25.99989, 17575.77426, 0, 0, \dots) & (25.99983, 17575.65717, 0, 0, \dots) \\
31 & (25.99989, 17575.78208, 0, 0, \dots) & (25.99984, 17575.66935, 0, 0, \dots) \\
32 & (25.99990, 17575.78937, 0, 0, \dots) & (25.99984, 17575.68070, 0, 0, \dots) \\
33 & (25.99990, 17575.79620, 0, 0, \dots) & (25.99985, 17575.69130, 0, 0, \dots) \\
34 & (25.99990, 17575.80259, 0, 0, \dots) & (25.99985, 17575.70122, 0, 0, \dots) \\
35 & (25.99991, 17575.80860, 0, 0, \dots) & (25.99986, 17575.71052, 0, 0, \dots) \\
36 & (25.99991, 17575.81425, 0, 0, \dots) & (25.99986, 17575.71927, 0, 0, \dots) \\
37 & (25.99991, 17575.81958, 0, 0, \dots) & (25.99987, 17575.72750, 0, 0, \dots) \\
38 & (25.99991, 17575.82462, 0, 0, \dots) & (25.99987, 17575.73527, 0, 0, \dots) \\
39 & (25.99992, 17575.82938, 0, 0, \dots) & (25.99987, 17575.74261, 0, 0, \dots) \\
40 & (25.99992, 17575.83389, 0, 0, \dots) & (25.99988, 17575.74955, 0, 0, \dots) \\
41 & (25.99992, 17575.83817, 0, 0, \dots) & (25.99988, 17575.75613, 0, 0, \dots) \\
42 & (25.99992, 17575.84223, 0, 0, \dots) & (25.99988, 17575.76237, 0, 0, \dots) \\
43 & (25.99992, 17575.84610, 0, 0, \dots) & (25.99989, 17575.76831, 0, 0, \dots) \\
44 & (25.99993, 17575.84978, 0, 0, \dots) & (25.99989, 17575.77395, 0, 0, \dots) \\
45 & (25.99993, 17575.85329, 0, 0, \dots) & (25.99989, 17575.77933, 0, 0, \dots) \\
46 & (25.99993, 17575.85664, 0, 0, \dots) & (25.99989, 17575.78446, 0, 0, \dots) \\
47 & (25.99993, 17575.85984, 0, 0, \dots) & (25.99990, 17575.78936, 0, 0, \dots) \\
48 & (25.99993, 17575.86290, 0, 0, \dots) & (25.99990, 17575.79404, 0, 0, \dots) \\
49 & (25.99993, 17575.86583, 0, 0, \dots) & (25.99990, 17575.79852, 0, 0, \dots) \\
50 & (25.99994, 17575.86864, 0, 0, \dots) & (25.99990, 17575.80280, 0, 0, \dots) \\
51 & (25.99994, 17575.87133, 0, 0, \dots) & (25.99990, 17575.80691, 0, 0, \dots) \\
52 & (25.99994, 17575.87392, 0, 0, \dots) & (25.99991, 17575.81085, 0, 0, \dots) \\
53 & (25.99994, 17575.87640, 0, 0, \dots) & (25.99991, 17575.81464, 0, 0, \dots) \\
54 & (25.99994, 17575.87879, 0, 0, \dots) & (25.99991, 17575.81827, 0, 0, \dots) \\
55 & (25.99994, 17575.88109, 0, 0, \dots) & (25.99991, 17575.82177, 0, 0, \dots) \\
56 & (25.99994, 17575.88330, 0, 0, \dots) & (25.99991, 17575.82514, 0, 0, \dots) \\
57 & (25.99994, 17575.88543, 0, 0, \dots) & (25.99992, 17575.82838, 0, 0, \dots) \\
58 & (25.99994, 17575.88749, 0, 0, \dots) & (25.99992, 17575.83150, 0, 0, \dots) \\
59 & (25.99995, 17575.88947, 0, 0, \dots) & (25.99992, 17575.83451, 0, 0, \dots) \\
60 & (25.99995, 17575.89138, 0, 0, \dots) & (25.99992, 17575.83742, 0, 0, \dots) \\
61 & (25.99995, 17575.89323, 0, 0, \dots) & (25.99992, 17575.84022, 0, 0, \dots) \\
62 & (25.99995, 17575.89502, 0, 0, \dots) & (25.99992, 17575.84294, 0, 0, \dots) \\
63 & (25.99995, 17575.89675, 0, 0, \dots) & (25.99992, 17575.84556, 0, 0, \dots) \\
64 & (25.99995, 17575.89842, 0, 0, \dots) & (25.99993, 17575.84809, 0, 0, \dots) \\
65 & (25.99995, 17575.90004, 0, 0, \dots) & (25.99993, 17575.85055, 0, 0, \dots) \\
66 & (25.99995, 17575.90161, 0, 0, \dots) & (25.99993, 17575.85292, 0, 0, \dots) \\
67 & (25.99995, 17575.90313, 0, 0, \dots) & (25.99993, 17575.85522, 0, 0, \dots) \\
68 & (25.99995, 17575.90461, 0, 0, \dots) & (25.99993, 17575.85745, 0, 0, \dots) \\
69 & (25.99995, 17575.90604, 0, 0, \dots) & (25.99993, 17575.85962, 0, 0, \dots) \\
70 & (25.99995, 17575.90742, 0, 0, \dots) & (25.99993, 17575.86172, 0, 0, \dots) \\
71 & (25.99996, 17575.90877, 0, 0, \dots) & (25.99993, 17575.86375, 0, 0, \dots) \\
72 & (25.99996, 17575.91008, 0, 0, \dots) & (25.99993, 17575.86573, 0, 0, \dots) \\
\hline
\end{longtable}


\begin{longtable}{|c|>{\centering\arraybackslash}p{4cm} | >{\centering\arraybackslash}p{4cm} |}
\caption{\centering Galat simulasi $d(x_n,x_{n+1})$ untuk Contoh \ref{con:fkonvk} dengan skema iterasi Sabri dan JK}\label{tab:errsimulasi}\\
\hline
{\textbf{n}} & {\textbf{Galat Sabri}} & {\textbf{Galat JK}} \\
\hline
\endfirsthead 
\multicolumn{3}{c}{\centering \thetable{} -- Lanjutan dari halaman sebelumnya} \\
\hline
{\textbf{n}} & {\textbf{Sabri}} & {\textbf{JK}} \\
\hline
\endhead 
% \multicolumn{3}{c}{{Lanjut pada halaman berikutnya}} \\
\hline
\endfoot 
\hline
\endlastfoot
\hline
1 & $9.8\times10^{2}$ & $9.8\times10^{2}$ \\
2 & $2.1\times10^{-1}$ & $2.9\times10^{0}$ \\
3 & $3.8\times10^{-3}$ & $1.8\times10^{-2}$ \\
4 & $9.5\times10^{-4}$ & $2.6\times10^{-3}$ \\
5 & $4.1\times10^{-4}$ & $9.5\times10^{-4}$ \\
6 & $2.2\times10^{-4}$ & $4.7\times10^{-4}$ \\
7 & $1.4\times10^{-4}$ & $2.7\times10^{-4}$ \\
8 & $9.8\times10^{-5}$ & $1.8\times10^{-4}$ \\
9 & $7.1\times10^{-5}$ & $1.2\times10^{-4}$ \\
10 & $5.4\times10^{-5}$ & $9.5\times10^{-5}$ \\
11 & $4.2\times10^{-5}$ & $7.3\times10^{-5}$ \\
12 & $3.4\times10^{-5}$ & $5.8\times10^{-5}$ \\
13 & $2.8\times10^{-5}$ & $4.7\times10^{-5}$ \\
14 & $2.3\times10^{-5}$ & $3.9\times10^{-5}$ \\
15 & $1.9\times10^{-5}$ & $3.2\times10^{-5}$ \\
16 & $1.7\times10^{-5}$ & $2.8\times10^{-5}$ \\
17 & $1.4\times10^{-5}$ & $2.4\times10^{-5}$ \\
18 & $1.2\times10^{-5}$ & $2.1\times10^{-5}$ \\
19 & $1.1\times10^{-5}$ & $1.8\times10^{-5}$ \\
20 & $1.0\times10^{-5}$ & $1.6\times10^{-5}$ \\
21 & $9.1\times10^{-6}$ & $1.4\times10^{-5}$ \\
22 & $8.2\times10^{-6}$ & $1.3\times10^{-5}$ \\
23 & $7.4\times10^{-6}$ & $1.1\times10^{-5}$ \\
24 & $6.7\times10^{-6}$ & $1.0\times10^{-5}$ \\
25 & $6.1\times10^{-6}$ & $9.7\times10^{-6}$ \\
26 & $5.6\times10^{-6}$ & $8.9\times10^{-6}$ \\
27 & $5.2\times10^{-6}$ & $8.1\times10^{-6}$ \\
28 & $4.8\times10^{-6}$ & $7.5\times10^{-6}$ \\
29 & $4.4\times10^{-6}$ & $6.9\times10^{-6}$ \\
30 & $4.1\times10^{-6}$ & $6.4\times10^{-6}$ \\
31 & $3.8\times10^{-6}$ & $6.0\times10^{-6}$ \\
32 & $3.5\times10^{-6}$ & $5.5\times10^{-6}$ \\
33 & $3.3\times10^{-6}$ & $5.2\times10^{-6}$ \\
34 & $3.1\times10^{-6}$ & $4.8\times10^{-6}$ \\
35 & $2.9\times10^{-6}$ & $4.5\times10^{-6}$ \\
36 & $2.7\times10^{-6}$ & $4.3\times10^{-6}$ \\
37 & $2.6\times10^{-6}$ & $4.0\times10^{-6}$ \\
38 & $2.4\times10^{-6}$ & $3.8\times10^{-6}$ \\
39 & $2.3\times10^{-6}$ & $3.6\times10^{-6}$ \\
40 & $2.2\times10^{-6}$ & $3.4\times10^{-6}$ \\
41 & $2.1\times10^{-6}$ & $3.2\times10^{-6}$ \\
42 & $2.0\times10^{-6}$ & $3.0\times10^{-6}$ \\
43 & $1.9\times10^{-6}$ & $2.9\times10^{-6}$ \\
44 & $1.8\times10^{-6}$ & $2.7\times10^{-6}$ \\
45 & $1.7\times10^{-6}$ & $2.6\times10^{-6}$ \\
46 & $1.6\times10^{-6}$ & $2.5\times10^{-6}$ \\
47 & $1.5\times10^{-6}$ & $2.4\times10^{-6}$ \\
48 & $1.5\times10^{-6}$ & $2.3\times10^{-6}$ \\
49 & $1.4\times10^{-6}$ & $2.2\times10^{-6}$ \\
50 & $1.3\times10^{-6}$ & $2.1\times10^{-6}$ \\
51 & $1.3\times10^{-6}$ & $2.0\times10^{-6}$ \\
52 & $1.2\times10^{-6}$ & $1.9\times10^{-6}$ \\
53 & $1.2\times10^{-6}$ & $1.8\times10^{-6}$ \\
54 & $1.1\times10^{-6}$ & $1.7\times10^{-6}$ \\
55 & $1.1\times10^{-6}$ & $1.7\times10^{-6}$ \\
56 & $1.0\times10^{-6}$ & $1.6\times10^{-6}$ \\
57 & $1.0\times10^{-6}$ & $1.5\times10^{-6}$ \\
58 & $1.0\times10^{-6}$ & $1.5\times10^{-6}$ \\
59 & $9.7\times10^{-7}$ & $1.4\times10^{-6}$ \\
60 & $9.4\times10^{-7}$ & $1.4\times10^{-6}$ \\
61 & $9.1\times10^{-7}$ & $1.3\times10^{-6}$ \\
62 & $8.8\times10^{-7}$ & $1.3\times10^{-6}$ \\
63 & $8.5\times10^{-7}$ & $1.2\times10^{-6}$ \\
64 & $8.2\times10^{-7}$ & $1.2\times10^{-6}$ \\
65 & $7.9\times10^{-7}$ & $1.2\times10^{-6}$ \\
66 & $7.7\times10^{-7}$ & $1.1\times10^{-6}$ \\
67 & $7.4\times10^{-7}$ & $1.1\times10^{-6}$ \\
68 & $7.2\times10^{-7}$ & $1.0\times10^{-6}$ \\
69 & $7.0\times10^{-7}$ & $1.0\times10^{-6}$ \\
70 & $6.8\times10^{-7}$ & $1.0\times10^{-6}$ \\
71 & $6.6\times10^{-7}$ & $1.0\times10^{-6}$ \\
72 & $6.4\times10^{-7}$ & $9.7\times10^{-7}$ \\
\hline
\end{longtable}
\begin{figure}[H]
    \centering
    \includegraphics[width=0.9\linewidth]{Bab_4/galatsim.png}
    \caption{Galat $d(x_n,x_{n+1})$ (skala log) vs iterasi}
    \label{fig:galatsim}
\end{figure}
