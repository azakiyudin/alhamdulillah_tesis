\section{Konvergensi Skema Iterasi Sabri untuk Pemetaan $(\alpha,\beta,\gamma)$-nonekspansif}
Pada bagian ini disajikan skema iterasi Sabri pada ruang $CAT_p(0)$ dan hasil terkait konvergensi dari skema iterasi Sabri untuk pemetaan $(\alpha,\beta,\gamma)$-nonekspansif di ruang tersebut. Terdapat dua lema penting yang digunakan untuk membuktikan konvergensi skema tersebut, yaitu pada Lema \ref{Lema:dxnx*} yang menunjukkan bahwa barisan $d(x_{n},x^*)$ adalah barisan turun, dengan $x^*$ adalah titik tetap dari pemetaan $(\alpha,\beta,\gamma)$-nonekspansif, serta pada Lema \ref{Lema:xntxn0} yang menunjukkan bahwa barisan limit dari barisan $d(x_n,Tx_n)$ adalah 0. Kemudian, disajikan syarat eksistensi titik tetap dari pemetaan $(\alpha,\beta,\gamma)$-nonekspansif pada Teorema \ref{thm:fixtnonemp}. Hasil konvergensi dari skema ini diberikan pada Teorema \ref{thm:konvD} yang memberikan syarat cukup untuk konvergensi$-\Delta$, sedangkan syarat cukup untuk konvergensi kuat diberikan oleh Teorema \ref{thm:konvK}. 

Berikut ini diberikan skema iterasi Sabri pada ruang $CAT_p(0)$.
\begin{defn}[\textbf{Skema Iterasi Sabri}]\label{defn:sabri}
Diberikan $(X,d,G)$ adalah ruang $CAT _p(0)$ dan $W$ adalah himpunan bagian tak kosong dari $X$ yang konveks. Untuk suatu pemetaan $T:W\to W$, $x_0\in W$, dan $n\in\mathbb{N}\cup \{0\} $ didefinisikan skema iterasi Sabri pada ruang $CAT_p(0)$ sebagai berikut:
    \begin{align}\label{eq:sabricat}
       \begin{cases}
            q_n&= T\left((1-c_n)x_n \oplus c_nTx_n\right),\\
        y_n&= T(Tq_n),\\
        x_{n+1} &= T\left((1-a_n)Tq_n\oplus a_n Ty_n\right).
       \end{cases}
    \end{align}
    dengan $\{a_n\}\subseteq [0,1]$ dan $\{c_n\}\subseteq [0,1]$.
\end{defn}
Berikut ini adalah diagram alir dari skema iterasi Sabri pada ruang $CAT_p(0)$.
\begin{figure}[H]
    \centering
    \tikzstyle{startstop} = [ellipse, 
minimum width=5.5cm, 
minimum height=0.5cm,
text centered, 
draw=black]

\tikzstyle{io} = [trapezium, 
trapezium stretches=true, % A later addition
trapezium left angle=70, 
trapezium right angle=110, 
minimum width=6cm, 
minimum height=0.5cm, text centered, 
draw=black]

\tikzstyle{process} = [rectangle, 
minimum height=0.5cm, 
text centered, 
text width=8.5cm, 
draw=black]

\tikzstyle{decision} = [diamond, 
minimum width=2cm, 
minimum height=0.5cm, 
text centered, 
aspect=1.8,
inner sep=2pt,
draw=black]
\tikzstyle{arrow} = [thick,->,>=stealth]

\begin{tikzpicture}[node distance=1.4cm]

% ===== Nodes =====
\node (start) [startstop] 
{Mulai};

\node (input) [io, below of=start,align=center,yshift=-0.3cm] 
    {Inisialisasi nilai awal $x_0\in W$, $n=0$, batas galat $\varepsilon>0$, \\
    batas iterasi $N$, dan parameter iterasi $\{a_n\}_{n=1}^{\infty},\{c_n\}_{n=1}^{\infty}\subseteq (0,1)$};

\node (qn) [process, below of=input,yshift=-0.3cm] 
{$q_n = T\big((1-c_n)x_n \oplus c_n T x_n\big)$};

\node (yn) [process, below of=qn] 
{$y_n = T(Tq_n)$};

\node (xn) [process, below of=yn] 
{$x_{n+1} = T\big((1-a_n)Tq_n \oplus a_n Ty_n\big)$};

\node (decision) [decision, below of=xn, yshift=-1.4cm,align=center] 
{Apakah\\$d(x_{n+1},x^*)<\varepsilon$?\\atau $n\geq N$?};

\node (stop) [startstop, below of=decision, yshift=-1.4cm] 
{Selesai};

% ===== Arrows =====
\draw [arrow] (start) -- (input);
\draw [arrow] (input) -- (qn);
\draw [arrow] (qn) -- (yn);
\draw [arrow] (yn) -- (xn);
\draw [arrow] (xn) -- (decision);
\draw [arrow] (decision) -- node[anchor=east]{Ya} (stop);

\draw [arrow] (decision.east) -- ++(3,0) 
node[anchor=west]{Tidak}
|- (qn.east);

\end{tikzpicture}

    \caption{Diagram Alir Skema Iterasi Sabri pada Ruang $CAT_p(0)$}
    \label{fig:skemasabricat0}
\end{figure}

Untuk selanjutnya, dinotasikan $W$ sebagai himpunan bagian tak kosong yang konveks dari ruang $CAT_p(0)$ $(X,d,G)$, serta $Fix(T)$ adalah himpunan semua titik tetap dari pemetaan $T$.
\begin{lemma}\label{Lema:dxnx*}
Diberikan $T:W\to W$ adalah pemetaan $(\alpha,\beta,\gamma)$-nonekspansif dengan $Fix(T)\neq \emptyset$. Jika $\{ x_n\}$ adalah barisan yang dikonstruksi melalui skema iterasi Sabri \eqref{eq:sabricat}, maka $d(x_{n+1},x^*)\leq d(x_n,x^*)$ untuk setiap $x^*\in Fix(T)$.
\end{lemma}
\begin{bukti}
    Diambil sebarang $x^*\in Fix(T)$, berdasarkan Lema \ref{lema:d,d^p} dan \ref{lemma:tutx}, didapatkan bahwa 
    \begin{align*}
        d(q_n,x^*)&=d\left(T\qty((1-c_n)x_n\oplus c_nTx_n), Tx^*\right)\\
        &\leq d\qty((1-c_n)x_n\oplus c_nTx_n,x^*)\\
        &\leq (1-c_n)d(x_n,x^*)+c_nd(Tx_n,x^*)\\
        &\leq (1-c_n)d(x_n,x^*)+c_nd(x_n,x^*)\\
        &= d(x_n,x^*).
    \end{align*}
    Dengan cara yang sama, didapatkan pula
    \begin{align*}
        d(y_n,x^*)&=d\qty(T(Tq_n),x^*)\\
        &\leq d(Tq_n,x^*)\\
        &\leq d(q_n,x^*)\\
        &\leq d(x_n,x^*),
    \end{align*}
    serta 
    \begin{align*}
        d(x_{n+1},x^*)&= d\qty(T\qty((1-a_n)Tq_n\oplus a_n Ty_n), Tx^*)\\
        &\leq d\qty((1-a_n)Tq_n\oplus a_n Ty_n,x^*)\\
        &\leq (1-a_n)d(Tq_n,x^*)+a_n d(Ty_n,x^*)\\
        &\leq (1-a_n)d(q_n,x^*)+a_nd(y_n,x^*)\\
        &\leq (1-a_n)d(x_n,x^*)+a_n d(x_n,x^*)\\
        &= d(x_n,x^*).
    \end{align*}
\end{bukti}
\begin{lemma}\label{Lema:xntxn0}
    Diberikan $T:W\to W$ adalah pemetaan $(\alpha,\beta,\gamma)$-nonekspansif dengan $Fix(T)\neq \emptyset$. Jika $\{x_n\}$ adalah barisan yang dikonstruksi melalui skema iterasi Sabri \eqref{eq:sabricat} dengan $\{a_n\},\{c_n\}\subset (0,1)$, maka $\lim_{n\to\infty} d(x_n,Tx_n)=0$.
\end{lemma}
\begin{bukti}
    Diambil sebarang $x^*\in Fix(T)$, berdasarkan Lema \ref{lema:d,d^p} dan \ref{lemma:tutx}, didapatkan bahwa
    \begin{align*}
        \qty(d(q_n,x^*))^p &= \qty(d\qty(T\qty((1-c_n)x_n\oplus c_n Tx_n),x^*))^p\\
        &\leq \qty(d\qty((1-c_n)x_n\oplus c_nTx_n,x^*))^p\\
        &\leq (1-c_n)\qty(d(x_n,x^*))^p+c_n \qty(d(Tx_n,x^*))^p-\frac{c_n(1-c_n)}{2^{p-1}}\qty(d(x_n,Tx_n))^p.
    \end{align*}
    Karena $a\leq c_k\leq b$, diperoleh $-c_n(1-c_n)\leq a(1-b)$ sehingga 
    \begin{align*}
        \qty(d(q_n,x^*))^p&\leq (1-c_n)\qty(d(x_n,x^*))^p+c_n \qty(d(x_n,x^*))^p-\frac{a(1-b)}{2^{p-1}} \qty(d(x_n,Tx_n))^p\\
        &= \qty(d(x_n,x^*))^p-\frac{a(1-b)}{2^{p-1}}\qty(d(x_n,Tx_n))^p.
    \end{align*}
    Dengan cara yang sama, didapatkan pula
    \begin{align*}
        \qty(d(y_n,x^*))^p&=\qty(d\qty(T(Tq_n),x^*))^p\\
        &\leq \qty(d(q_n,x^*))^p\\
        &\leq \qty(d(x_n,x^*))^p-\frac{a(1-b)}{2^{p-1}}\qty(d(x_n,Tx_n))^p,
    \end{align*}
    serta
    \begin{align*}
        \qty(d(x_{n+1},x^*))^p =&~ \qty(d\qty(T\qty((1-a_n)Tq_n\oplus a_n Ty_n),x^*))^p\\
        \leq &~\qty(d\qty((1-a_n)Tq_n\oplus a_n Ty_n,x^*))^p\\
        \leq&~ (1-a_n)\qty(d(Tq_n,x^*))^p+a_n \qty(d(Ty_n,x^*))^p-\frac{c_n(1-c_n)}{2^{p-1}}\qty(d(Tq_n,Ty_n))^p\\
        \leq&~ (1-a_n)\qty(d(q_n,x^*))^p+a_n \qty(d(y_n,x^*))^p\\
        \leq&~ (1-a_n)\qty[\qty(d(x_n,x^*))^p-\frac{a(1-b)}{2^{p-1}}\qty(d(x_n,Tx_n))^p]\\
        &~+a_n \qty[\qty(d(x_n,x^*))^p-\frac{a(1-b)}{2^{p-1}}\qty(d(x_n,Tx_n))^p]\\
        =&~\qty(d(x_n,x^*))^p-\frac{a(1-b)}{2^{p-1}}\qty(d(x_n,Tx_n))^p.
    \end{align*}
    Akibatnya 
    \begin{align*}
     \qty(d(x_n,Tx_n))^p&\leq \frac{2^{p-1}}{a(1-b)}\qty[\qty(d(x_n,x^*))^p-\qty(d(x_{n+1},x^*))^p].     
    \end{align*}
    Selanjutnya, berdasarkan Lema \ref{Lema:dxnx*}, diperoleh bahwa $\{d(x_n,x^*)\}$ adalah barisan turun dan terbatas di bawah dengan batas bawah 0, sehingga $\{d(x_n,x^*)\}$ konvergen. Hal ini berarti $u_n = \qty(d(x_n,x^*))^p$ juga konvergen dan diperoleh $\lim_{n\to\infty} (u_n-u_{n+1})=0$. Akibatnya, didapatkan 
    \begin{align*}
        0\leq \lim_{n\to \infty} \qty(d(x_n,Tx_n))^p&\leq \lim_{n\to \infty} \frac{2^{p-1}}{a(1-b)}\qty[\qty(d(x_n,x^*))^p-\qty(d(x_{n+1},x^*))^p] =0.
    \end{align*}
    Jadi $\lim_{n\to\infty}d(x_n,Tx_n)=0$.
\end{bukti}


Teorema berikut ini menyajikan syarat eksistensi titik tetap dari pemetaan $(\alpha,\beta,\gamma)$-nonekspansif.
\begin{thm}\label{thm:fixtnonemp}
    Diberikan $T:W\to W$ adalah pemetaan $(\alpha,\beta,\gamma)$-nonekspansif. Jika $\{x_n\}$ adalah barisan yang dikonstruksi melalui skema iterasi Sabri \eqref{eq:sabricat} sehingga $\{x_n\}$ terbatas dan $\lim_{n\to\infty} d(x_n,Tx_n)=0$, maka $Fix(T)\neq\emptyset$.
\end{thm}
\begin{bukti}
    Diambil sebarang $x^*\in A(\{x_n\})$, maka dengan ketaksamaan \ref{eq:ineqabcnoneks}, diperoleh bahwa 
    \begin{align*}
        d(x_n,Tx^*)\leq \frac{\alpha}{1-\gamma}d(x_n,x^*)+\frac{1+\beta}{1-\gamma}d(x_n,Tx_n).
    \end{align*}
    Karena $\lim_{n\to\infty} d(x_n,Tx_n)=0$, diperoleh 
    \begin{align*}
        \limsup_{n\to\infty}d(x_n, Tx^*)=\limsup_{n\to\infty}d(x_n,x^*)\leq \limsup_{n\to\infty} d(x_n,x^*).
    \end{align*}
    Dari sini didapatkan $Tx^*\in A(\{x_n\})$, sehingga berdasarkan Lema \ref{Lema:asimtotik}, diperoleh bahwa $A(\{x_n\})$ tepat memiliki satu elemen, yang berarti $x^*=Tx^*$. Dengan demikian $Fix(T)\neq \emptyset$.
\end{bukti}
Selanjutnya, dua teorema berikut ini menyajikan hasil konvergensi dari skema Sabri, yakni konvergensi-$\Delta$ dan konvergensi kuat, untuk aproksimasi titik tetap dari pemetaan $(\alpha,\beta,\gamma)$-nonekspansif
\begin{thm}\label{thm:konvD}
    Diberikan $(X,d,G)$ adalah ruang $CAT _p(0)$ yang lengkap dan $W$ adalah himpunan bagian tak kosong dari $X$ yang tertutup dan konveks. Jika $T:W\to W$ adalah pemetaan $(\alpha,\beta,\gamma)$-nonekspansif dengan $Fix(T)\neq \emptyset$ dan $\{x_n\}$ adalah barisan yang dikonstruksi melalui skema iterasi Sabri \eqref{eq:sabricat} dengan $\{a_n\},\{c_n\}\subset (0,1)$, maka $\{x_n\}$ konvergen-$\Delta$ ke suatu titik tetap dari $T$.
\end{thm}
\begin{bukti}
    Diamati bahwa $T$ merupakan pemetaan $(\alpha,\beta,\gamma)$-nonekspansif. Dari Lema \ref{Lema:dxnx*}, didapatkan bahwa $\{d(x_n,x^*)\}$ adalah barisan turun dan terbatas di bawah dengan batas bawah 0, sehingga $\{d(x_n,x^*)\}$ konvergen untuk setiap $x^*\in Fix(T)$.
    % \begin{align}
    %     \lim_{n\to\infty} d(x_n,x^*)=0 \quad \quad \text{untuk setiap } x^*\in Fix(T).
    % \end{align}
     Kemudian, karena $Fix(T)\neq \emptyset$, maka berdasarkan Lema \ref{Lema:xntxn0}, didapatkan 
     \begin{align}
         \lim_{n\to\infty} d(x_n,Tx_n)=0.
     \end{align}
     Selanjutnya, dari Lema \ref{Lema:demi}, didapat bahwa $T$ memiliki sifat demiclosedness. Hal ini berarti semua kondisi pada Teorema \ref{thm:kondisikonvD} terpenuhi, sehingga $\{x_n\}$ konvergen$-\Delta$.
\end{bukti}
Untuk mendapatkan hasil konvergensi kuat, diperlukan syarat tambahan, yaitu himpunan $W$ haruslah merupakan himpunan kompak. 
\begin{thm}\label{thm:konvK}
     Diberikan $(X,d,G)$ adalah ruang $CAT _p(0)$ yang lengkap dan $W$ adalah himpunan bagian tak kosong dari $X$ yang tertutup, konveks, dan kompak. Jika $T:W\to W$ adalah pemetaan $(\alpha,\beta,\gamma)$-nonekspansif dengan $Fix(T)\neq \emptyset$ dan $\{x_n\}$ adalah barisan yang dikonstruksi melalui skema iterasi Sabri \eqref{eq:sabricat} dengan $\{a_n\},\{c_n\}\subset (0,1)$, maka $\{x_n\}$ konvergen kuat ke suatu titik tetap dari $T$.
\end{thm}
\begin{bukti}
    Diperhatikan bahwa $W$ adalah himpunan kompak, ini berarti ada subbarisan $\{x_{n_k}\}$ dari $\{x_n\}$ yang konvergen kuat ke $x\in W$. Akibatnya $\{x_{n_k}\}$ juga konvergen$-\Delta$ ke $x\in W$.  Dengan menggunakan fakta bahwa $W$ himpunan kompak dan $Fix(T)\neq\emptyset$, maka berdasarkan Lema \ref{Lema:xntxn0}, diperoleh
    \begin{align*}
        \lim_{k\to\infty} d(x_{n_k},Tx_{n_k})=\lim_{k\to\infty} d(x_n,Tx_n)=0.
    \end{align*}
    Kemudian dengan menggunakan sifat demiclosedness dari $T$, didapat bahwa $x\in Fix(T)$. Selanjutnya dari Lema \ref{Lema:dxnx*} didapatkan
    % \begin{align*}
    %     \lim_{k\to\infty} d(x_{n_k},x)=0.
    % \end{align*}
    % Dengan menggunakan fakta bahwa $W$ himpunan kompak, didapatkan
    \begin{align*}
        \lim_{k\to\infty} d(x_{n},x)=0.
    \end{align*}
    Dengan demikian, barisan $\{x_n\}$ konvergen kuat ke titik tetap dari $T$.
\end{bukti}

\begin{thm}
    Laju
\end{thm}
\begin{bukti}
    Diperhatikan bahwa untuk setiap $x^*\in Fix(T)$ berlaku
    \begin{align*}
        d(q_n,x^*) &= d\qty(T\qty((1-c_n)x_n\oplus c_n Tx_n),Tx^*)\\
        &\leq \frac{\alpha}{1-\gamma}d\qty((1-c_n)x_n\oplus c_n Tx_n, x^* )\\
        &\leq \frac{\alpha}{1-\gamma}\qty[(1-c_n)d(x_n,x^*)+c_n d(Tx_n,x^*)]\\
        &\leq \frac{\alpha}{1-\gamma}\qty[(1-c_n)d(x_n,x^*)+\frac{\alpha c_n}{1-\gamma}d(x_n,x^*)]\\
        &= \frac{\alpha(1-\gamma)(1-c_n)+\alpha^2 c_n}{(1-\gamma)^2}d(x_n,x^*).
    \end{align*}
    \begin{align*}
        d(y_n,x^*) &= d\qty(T(Tq_n),x^*)\\
        &\leq \frac{\alpha}{1-\gamma}d(Tq_n,x^*)\\
        &\leq \qty(\frac{\alpha}{1-\gamma})^2 d(q_n,x^*).
    \end{align*}
    \begin{align*}
        d(x_{n+1},x^*) &= d\qty(T\qty((1-a_n)Tq_n\oplus a_n Ty_n),Tx^*)\\
        &\leq \frac{\alpha}{1-\gamma}d\qty((1-a_n)Tq_n\oplus a_n Ty_n,x^*)\\
        &\leq \frac{\alpha}{1-\gamma}\qty[(1-a_n)d(Tq_n,x^*)+a_n d(Ty_n,x^*)]\\
        &\leq \qty(\frac{\alpha}{1-\gamma})^2\qty[(1-a_n)d(q_n,x^*)+a_n d(y_n,x^*)]\\
        &\leq \qty(\frac{\alpha}{1-\gamma})^2\qty[(1-a_n)d(q_n,x^*)+a_n \qty(\frac{\alpha}{1-\gamma})^2d(q_n,x^*)]\\
        &= \qty(\frac{\alpha}{1-\gamma})^2\qty[\frac{(1-a_n)(1-\gamma)^2+a_n\alpha^2}{(1-\gamma)^2}d(q_n,x^*)]\\
        &\leq \frac{\alpha^2}{(1-\gamma)^4} \qty((1-a_n)(1-\gamma)^2+a_n\alpha^2)\times \frac{\alpha(1-\gamma)(1-c_n)+\alpha^2 c_n}{(1-\gamma)^2}d(x_n,x^*)\\
        &= \frac{\alpha^3}{(1-\gamma)^6}\qty((1-a_n)(1-\gamma)^2+a_n\alpha^2)\qty((1-c_n)(1-\gamma)+c_n \alpha)d(x_n,x^*).
    \end{align*}
    Karena $\alpha\leq 1-\gamma$, didapat 
    \begin{align*}
        d(x_{n+1},x^*) &\leq \frac{\alpha^3}{(1-\gamma)^6}\qty(1-\gamma)^2\qty(1-\gamma)d(x_n,x^*) = \frac{\alpha^3}{(1-\gamma)^3}d(x_n,x^*),
    \end{align*}
    sehingga 
    \begin{align*}
        d(x_{n+1},x^*) &\leq \qty(\frac{\alpha}{1-\gamma})^3 d(x_n,x^*)\\
        &\leq  \qty(\frac{\alpha}{1-\gamma})^6 d(x_{n-1},x^*)\\
        &~~\vdots\\
        &\leq  \qty(\frac{\alpha}{1-\gamma})^{3(n+1)} d(x_0,x^*).
    \end{align*}
    Jika terdapat $\alpha+\gamma\neq 1$, didapatkan $\lim_{n\to\infty} d(x_n,x^*)=0$.
\end{bukti}
\begin{bukti}
    \begin{align*}
        d(q_n,x^*) &=d\qty(T\qty((1-c_n)x_n\oplus c_n Tx_n),Tx^*)\\
        &\leq bd\qty((1-c_n)x_n\oplus c_n Tx_n, x^* )\\
        &\leq bm\qty[(1-c_n)d(x_n,x^*)+c_n d(Tx_n,x^*)]\\
        &\leq bm\qty[(1-c_n)d(x_n,x^*)+bc_nd(x_n,x^*)]\\
        &= bm\qty(1-c_n(1-b))d(x_n,x^*).
    \end{align*}
    \begin{align*}
        d(y_n,x^*) &= d\qty(T(Tq_n),x^*)\\
        &\leq bd(Tq_n,x^*)\\
        &\leq b^2 d(q_n,x^*).
    \end{align*}
    \begin{align*}
        d(x_{n+1},x^*) &= d\qty(T\qty((1-a_n)Tq_n\oplus a_n Ty_n),Tx^*)\\
        &\leq bd\qty((1-a_n)Tq_n\oplus a_n Ty_n,x^*)\\
        &\leq bm\qty[(1-a_n)d(Tq_n,x^*)+a_n d(Ty_n,x^*)]\\
        &\leq b^2m\qty[(1-a_n)d(q_n,x^*)+a_n d(y_n,x^*)]\\
        &\leq b^2m\qty[(1-a_n)d(q_n,x^*)+a_n b^2d(q_n,x^*)]\\
        &= b^2m\qty[(1-a_n(1-b^2))d(q_n,x^*)]\\
        &\leq b^2m (1-a_n(1-b^2))(bm(1-c_n(1-b)))d(x_n,x^*)\\
        &= b^3m^2 (1-a_n(1-b^2))(1-c_n(1-b))d(x_n,x^*).
    \end{align*}
\end{bukti}
