\section{Pemetaan $(\alpha,\beta,\gamma)$-nonekspansif di Ruang $CAT_p(0)$}
Pada bagian ini, disajikan definisi, contoh, dan sifat-sifat dari pemetaan $(\alpha,\beta,\gamma)$-nonekspansif di ruang $CAT_p(0)$. Sifat yang didapatkan dari pemetaan ini disajikan dalam Lema \ref{Lema:dxnx*}, yang menunjukkan bahwa pemetaan ini memenuhi kondisi nonekspansif kuasi. 
Kemudian, pada Lema \ref{Lema:ineqabcnoneks} didapatkan ketaksamaan penting yang melibatkan pemetaan tersebut. Selain itu, didapatkan pula bahwa pemetaan ini memenuhi sifat \textit{demiclosedness} yang ditunjukkan oleh Lema \ref{Lema:demi}. Tiga Lema tersebut berperan penting untuk pembuktian konvergensi skema iterasi Sabri untuk aproksimasi titik tetap dari pemetaan $(\alpha,\beta,\gamma)$-nonekspansif.

Berikut ini disajikan definisi dari pemetaan $(\alpha,\beta,\gamma)$-nonekspansif di ruang $CAT_p(0)$.
\begin{defn}\label{defn:abcnoncat}
    Diberikan $(X,d,G)$ adalah ruang $CAT _p(0)$ dan $W$ adalah himpunan bagian tak kosong dari $X$, pemetaan $f:W\to W$ disebut sebagai pemetaan $(\alpha,\beta,\gamma)$-nonekspansif jika terdapat bilangan real $\alpha,\beta,\gamma\in \mathbb{R}^+\cup\{0\}$ dengan $\alpha+\gamma\leq 1,\gamma\in[0,1)$ sehingga untuk setiap $x,y\in W$ berlaku
    \begin{align}
        d(Tx,Ty)\leq \alpha d(x,y)+\beta d(x,Tx)+\gamma d(x, Ty). \label{eq:abcnoncat}
    \end{align}
\end{defn}
    Selanjutnya, diberikan contoh dari pemetaan $(\alpha,\beta,\gamma)$-nonekspansif di ruang $CAT_p(0)$. Namun, sebelum itu diberikan dulu contoh ruang yang digunakan sebagai berikut.

    Contoh berikut ini merupakan contoh dari pemetaan $(\alpha,\beta,\gamma)$-nonekspansif dengan ruang $CAT_p(0)$ yang diberikan pada Contoh \ref{con:Catp}.

    \begin{exam}\label{con:abcnon}
        Diberikan $(X,d,G)$ adalah ruang $CAT _p(0)$ sebagaimana dalam Contoh \ref{con:Catp}. Diberikan pula $W=\{(w_1,w_2,0,0,\cdots)\mid w_1\in[1,5], w_2\in[1,125]\}\subset X$. dan pemetaan $T:W\to W$ yang didefinisikan sebagai 
        \begin{align*}
            T((w_1,w_2,0,0,\cdots))=\begin{cases}
                \qty(\frac{w_1+3}{4},\frac{(w_1+3)^3}{64},0,0,\cdots), \quad &\text{jika } w_1\in [1,3)\\
                \qty(\frac{w_1+2}{4}, \frac{(w_1+2)^3}{64},0,0,\cdots), \quad &\text{jika } w_1\in [3,5].
            \end{cases}
        \end{align*}
        Pemetaan $T$ adalah pemetaan $(\frac{1}{4},\frac{1}{3},0)$-nonekspansif, tetapi bukan pemetaan nonekspansif. Titik tetap dari $T$ adalah $(1,1,0,0,\cdots)$.
    \end{exam}
        Penjelasan dari Contoh \ref{con:abcnon} diuraikan berikut ini.\\
        Diambil sebarang $u,v\in W$ dengan $u=(u_1,u_2,0,0,\cdots), ~v=(v_1,v_2,0,0,\cdots)$. Dimisalkan $D(u,v)=\alpha d(u,v)+\beta d(u,Tu)+\gamma d(u,Tv)$. 
        \begin{enumerate}[label={\textbf{Kasus \arabic*.}},align=left]
            \item Untuk $u_1,v_1\in [1,3)$, diperoleh 
            \begin{align*}
                D(u,v) =& ~\frac{1}{4}\left(|u_1-v_1|^3+|u_1^3-u_2-v_1^3+v_2|^3\right)^{\frac{1}{3}} \\
                &+ \frac{1}{3} \left(\left|u_1-\frac{u_1+3}{4}\right|^3+\bigg|u_1^3-u_2-\qty(\frac{u_1+3}{4})^3+\frac{(u_1+3)^3}{64}\bigg|^3\right)^{\frac{1}{3}}\\
                &+0\times \left(\left|u_1-\frac{v_1+3}{4}\right|^3+\bigg|u_1^3-u_2 -\qty(\frac{v_1+3}{4})^3+\frac{(v_1+3)^3}{64}\bigg|^3\right)^{\frac{1}{3}}\\
                \geq &~ \frac{1}{4}\left(|u_1-v_1|^3\right)^{\frac{1}{3}}\\
                =&~ \left|\frac{u_1+3}{4}-\frac{v_1+3}{4}\right|\\
                =&~ \Bigg(\left|\frac{u_1+3}{4}-\frac{v_1+3}{4}\right|^3           \\    &+\bigg|\qty(\frac{u_1+3}{4})^3-\frac{(u_1+3)^3}{64}-\qty(\frac{v_1+3}{4})^3+\frac{(v_1+3)^3}{64}\bigg|^3\Bigg)^{\frac{1}{3}}\\
                =& ~ d(Tu,Tv).
            \end{align*}
            \item Untuk $u_1,v_1\in [3,5]$, diperoleh 
            \begin{align*}
                D(u,v) =& ~\frac{1}{4}\left(|u_1-v_1|^3+|u_1^3-u_2-v_1^3+v_2|^3\right)^{\frac{1}{3}} \\
                &+ \frac{1}{3} \Bigg(\left|u_1-\frac{u_1+2}{4}\right|^3+\bigg|u_1^3-u_2-\qty(\frac{u_1+2}{4})^3+\frac{(u_1+2)^3}{64}\bigg|^3\Bigg)^{\frac{1}{3}}\\
                &+0\times \Bigg(\left|u_1-\frac{v_1+2}{4}\right|^3+\bigg|u_1^3-u_2 -\qty(\frac{v_1+2}{4})^3+\qty(\frac{v_1+2}{4})^3\bigg|^3\Bigg)^{\frac{1}{3}}\\
                \geq &~ \frac{1}{4}\left(|u_1-v_1|^3\right)^{\frac{1}{3}}\\
                =&~ \left|\frac{u_1+2}{4}-\frac{v_1+2}{4}\right|\\
                =&~ \Bigg(\left|\frac{u_1+2}{4}-\frac{v_1+2}{4}\right|^3           \\    &+\bigg|\qty(\frac{u_1+2}{4})^3-\frac{(u_1+2)^3}{64}-\qty(\frac{v_1+2}{4})^3+\frac{(v_1+2)^3}{64}\bigg|^3\Bigg)^{\frac{1}{3}}\\
                =& ~ d(Tu,Tv).
            \end{align*}
            \item Untuk $u_1\in [3,5]$ dan $v_1\in[1,3)$, diperoleh 
            \begin{align*}
                D(u,v) =& ~\frac{1}{4}\left(|u_1-v_1|^3+|u_1^3-u_2-v_1^3+v_2|^3\right)^{\frac{1}{3}} \\
                &+ \frac{1}{3} \Bigg(\left|u_1-\frac{u_1+2}{4}\right|^3+\bigg|u_1^3-u_2-\left(\frac{u_1+2}{4}\right)^3+\frac{(u_1+2)^3}{64}\bigg|^3\Bigg)^{\frac{1}{3}}\\
                &+0\times \Bigg(\left|u_1-\frac{v_1+3}{4}\right|^3+\bigg|u_1^3-u_2 -\qty(\frac{v_1+3}{4})^3+\frac{(v_1+3)^3}{64}\bigg|^3\Bigg)^{\frac{1}{3}}\\
                \geq &~ \frac{1}{4}\left(|u_1-v_1|^3\right)^{\frac{1}{3}}+ \frac{1}{3}\left(\qty|u_1-\frac{u_1+2}{4}|^3\right)^{\frac{1}{3}}.
            \end{align*}
            Karena $u_1\in[3,5]$, maka $\frac{1}{12}|3u_1-2|\geq \frac{7}{12}> \frac{1}{4}$, sehingga 
            \begin{align*}
                D(u,v) \geq&~ \frac{1}{4}|u_1-v_1|+\frac{1}{4}\\
                \geq&~ \qty|\frac{u_1}{4}-\frac{v_1}{4}-\frac{1}{4}|\\
                =&~ \Bigg(\qty|\frac{u_1+2}{4}-\frac{v_1+3}{4}|^3\\
                &+\qty|\qty(\frac{u_1+2}{4})^3-\frac{(u_1+2)^3}{64}-\qty(\frac{v_1+2}{4})^3+\frac{(v_1+2)^3}{64}|^3\Bigg)^{\frac{1}{3}}\\
                =&~ d(Tu,Tv).
            \end{align*}
            \item Untuk $u_1\in [1,3)$ dan $v_1\in[3,5]$, diperoleh 
            \begin{align*}
                D(u,v) =& ~\frac{1}{4}\left(|u_1-v_1|^3+|u_1^3-u_2-v_1^3+v_2|^3\right)^{\frac{1}{3}} \\
                &+ \frac{1}{3} \Bigg(\left|u_1-\frac{u_1+3}{4}\right|^3+\bigg|u_1^3-u_2-\left(\frac{u_1+3}{4}\right)^3+\frac{(u_1+3)^3}{64}\bigg|^3\Bigg)^{\frac{1}{3}}\\
                &+0\times \Bigg(\left|u_1-\frac{v_1+2}{4}\right|^3+\bigg|u_1^3-u_2 -\qty(\frac{v_1+2}{4})^3+\frac{(v_1+2)^3}{64}\bigg|^3\Bigg)^{\frac{1}{3}}\\
                \geq &~ \frac{1}{4}\left(|u_1-v_1|^3\right)^{\frac{1}{3}}+ \frac{1}{3}\left(\qty|u_1-\frac{u_1+3}{4}|^3\right)^{\frac{1}{3}} + \frac{3}{4}\left(\qty|u_1-\frac{v_1+2}{4}|^3\right)^{\frac{1}{3}}\\
                =&~ \frac{1}{4}|u_1-v_1|+\frac{1}{4}|u_1-1|.
            \end{align*}
            Diperhatikan bahwa 
            \begin{align*}
                d(Tu,Tv) =&~\Bigg(\qty|\frac{u_1+3}{4}-\frac{v_1+2}{4}|^3\\
                &+ \qty|\qty(\frac{u_1+3}{4})^3-\frac{(u_1+3)^3}{64}-\qty(\frac{v_1+2}{4})^3+\frac{(v_1+3)^3}{64}|^3\Bigg)^{\frac{1}{3}}\\
                =& ~ \frac{1}{4}|u_1-v_1+1|,
            \end{align*}
            sehingga untuk membuktikan bahwa $D(u,v)\geq d(Tu,Tv)$, akan dibuktikan bahwa 
            \begin{align*}
                f(u_1,v_1):=|u_1-v_1|+|u_1-1|-|u_1-v_1+1|\geq 0,
            \end{align*}
            untuk setiap $u_1\in [1,3)$ dan $v_1\in[3,5]$. Diamati bahwa $1\leq u_1<3\leq v_1$, sehingga 
            \begin{align*}
                f(u_1,v_1)&=v_1-u_1+u_1-1-|u_1-v_1+1|\\
                &= v_1-1-|u_1-v_1+1|.
            \end{align*}
            Jika $u_1-v_1+1< 0$, didapatkan 
            \begin{align*}
                f(u_1,v_1) = v_1-1-\qty(-(u_1-v_1+1)) = u_1\geq 1. 
            \end{align*}
            Selanjutnya, jika $u_1-v_1+1\geq 0$, diperoleh $u_1\geq v_1-1\geq 2$
            \begin{align*}
                f(u_1,v_1) = v_1-1-u_1+v_1-1 = 2v_1-u_1-2,
            \end{align*}
            sehingga 
            \begin{align*}
                \dfrac{\partial f}{\partial u_1} &= -1\neq 0 \quad \text{dan} \quad \dfrac{\partial f}{\partial v_1} = 2\neq 0.
            \end{align*}
            Hal ini berarti nilai minimumnya terdapat pada titik-titik batas. Karena $u_1\in[1,3)$ dan $u_1\geq 2$ serta $v_1\in [3,5]$, diperoleh $-3<-u_1\leq-2$ sehingga
            \begin{align*}
                f(u_1,v_1) = 2v_1-u_1-2>2v_1-5\geq 1.
            \end{align*}
                Dari uraian tersebut didapatkan bahwa $f(u_1,v_1)\geq 0$ untuk setiap $u_1\in[1,3)$ dan $v_1\in [3,5]$ sehingga berlaku pula $\alpha d(u,v)+\beta d(u,Tu)+\gamma d(u,Tv)\geq d(Tu,Tv)$.
        \end{enumerate}
        Karena semua tinjauan kasus di atas menghasilkan $\alpha d(u,v)+\beta d(u,Tu)+\gamma d(u,Tv)\geq d(Tu,Tv)$ untuk setiap $u_1,v_1\in[1,5]$, maka $T$ adalah pemetaan $(\alpha,\beta,\gamma)$-nonekspansif dengan $\alpha=\frac{1}{4},~\beta=\frac{1}{3},$ dan $\gamma=0$. Akan tetapi, $T$ bukan pemetaan nonekspansif karena untuk $u=\qty(\frac{29}{10},\frac{29^3}{1000},0,0,\dots)$ dan $v=(3,27,0,0,\dots)$, didapatkan 
            \begin{align*}
            d(Tu,Tv)=&~\Bigg(\mqty|\frac{\frac{29}{10}+3}{4}-\frac{3+2}{4}|^3\\
            &+\bigg|\bigg(\frac{\frac{29}{10}+3}{4}\bigg)^3-\frac{(\frac{29}{10}+3)^3}{64}-\frac{(3+2)^3}{64}+\bigg(\frac{3+2}{4}\bigg)^3\bigg|^3\Bigg)^{\frac{1}{3}}\\
                =&~\frac{9}{40}\\
                >&~ \frac{1}{10}\\                =&~\qty(\left|3-\frac{29}{10}\right|^3+\qty|\qty(\frac{29}{10})^3-\frac{29^3}{1000}-3^3+27|^3)^{\frac{1}{3}} \\
                =&~d(u,v).
            \end{align*}
    Selanjutnya, untuk mendapatkan titik tetap dari $T$, dicari $w=(w_1,w_2,0,0,\cdots)\in W$ sehingga $T(w)=w$. Dari definisi $T$, terdapat dua kemungkinan, yaitu
    \begin{align*}
        &\text{(i) } w_1=\frac{w_1+3}{4}, ~w_2=\frac{(w_1+3)^3}{64}, \quad \text{atau}\\
        &\text{(ii) } w_1=\frac{w_1+2}{4}, ~w_2=\frac{(w_1+2)^3}{64}.
    \end{align*}
    Dari (i) diperoleh $w_1=w_2=1\in [1,3)$, sedangkan dari (ii) diperoleh $w_1=\frac{2}{3}\notin [3,5]$. Dengan demikian, titik tetap dari $T$ adalah $(1,1,0,0,\cdots)$.
    % \begin{exam}\label{con:abcnon}
        Diberikan $(X,d,G)$ adalah ruang $CAT _p(0)$ sebagaimana dalam Contoh \ref{con:Catp}. Diberikan pula $W=\{(w_1,w_2,0,0,\cdots)\mid w_1\in[1,5], w_2\in[1,125]\}\subset X$. dan pemetaan $T:W\to W$ yang didefinisikan sebagai 
        \begin{align*}
            T((w_1,w_2,0,0,\cdots))=\begin{cases}
                \qty(\frac{w_1+3}{4},\frac{(w_1+3)^3}{64},0,0,\cdots), \quad &\text{jika } w_1\in [1,3)\\
                \qty(\frac{w_1+2}{4}, \frac{(w_1+2)^3}{64},0,0,\cdots), \quad &\text{jika } w_1\in [3,5].
            \end{cases}
        \end{align*}
        Pemetaan $T$ adalah pemetaan $(\frac{1}{4},\frac{1}{3},0)$-nonekspansif, tetapi bukan pemetaan nonekspansif. Titik tetap dari $T$ adalah $(1,1,0,0,\cdots)$.
    \end{exam}
        Penjelasan dari Contoh \ref{con:abcnon} diuraikan berikut ini.\\
        Diambil sebarang $u,v\in W$ dengan $u=(u_1,u_2,0,0,\cdots), ~v=(v_1,v_2,0,0,\cdots)$. Dimisalkan $D(u,v)=\alpha d(u,v)+\beta d(u,Tu)+\gamma d(u,Tv)$. 
        \begin{enumerate}[label={\textbf{Kasus \arabic*.}},align=left]
            \item Untuk $u_1,v_1\in [1,3)$, diperoleh 
            \begin{align*}
                D(u,v) =& ~\frac{1}{4}\left(|u_1-v_1|^3+|u_1^3-u_2-v_1^3+v_2|^3\right)^{\frac{1}{3}} \\
                &+ \frac{1}{3} \left(\left|u_1-\frac{u_1+3}{4}\right|^3+\bigg|u_1^3-u_2-\qty(\frac{u_1+3}{4})^3+\frac{(u_1+3)^3}{64}\bigg|^3\right)^{\frac{1}{3}}\\
                &+0\times \left(\left|u_1-\frac{v_1+3}{4}\right|^3+\bigg|u_1^3-u_2 -\qty(\frac{v_1+3}{4})^3+\frac{(v_1+3)^3}{64}\bigg|^3\right)^{\frac{1}{3}}\\
                \geq &~ \frac{1}{4}\left(|u_1-v_1|^3\right)^{\frac{1}{3}}\\
                =&~ \left|\frac{u_1+3}{4}-\frac{v_1+3}{4}\right|\\
                =&~ \Bigg(\left|\frac{u_1+3}{4}-\frac{v_1+3}{4}\right|^3           \\    &+\bigg|\qty(\frac{u_1+3}{4})^3-\frac{(u_1+3)^3}{64}-\qty(\frac{v_1+3}{4})^3+\frac{(v_1+3)^3}{64}\bigg|^3\Bigg)^{\frac{1}{3}}\\
                =& ~ d(Tu,Tv).
            \end{align*}
            \item Untuk $u_1,v_1\in [3,5]$, diperoleh 
            \begin{align*}
                D(u,v) =& ~\frac{1}{4}\left(|u_1-v_1|^3+|u_1^3-u_2-v_1^3+v_2|^3\right)^{\frac{1}{3}} \\
                &+ \frac{1}{3} \Bigg(\left|u_1-\frac{u_1+2}{4}\right|^3+\bigg|u_1^3-u_2-\qty(\frac{u_1+2}{4})^3+\frac{(u_1+2)^3}{64}\bigg|^3\Bigg)^{\frac{1}{3}}\\
                &+\frac{3}{4} \Bigg(\left|u_1-\frac{v_1+2}{4}\right|^3+\bigg|u_1^3-u_2 -\qty(\frac{v_1+2}{4})^3+\qty(\frac{v_1+2}{4})^3\bigg|^3\Bigg)^{\frac{1}{3}}\\
                \geq &~ \frac{1}{4}\left(|u_1-v_1|^3\right)^{\frac{1}{3}}\\
                =&~ \left|\frac{u_1+2}{4}-\frac{v_1+2}{4}\right|\\
                =&~ \Bigg(\left|\frac{u_1+2}{4}-\frac{v_1+2}{4}\right|^3           \\    &+\bigg|\qty(\frac{u_1+2}{4})^3-\frac{(u_1+2)^3}{64}-\qty(\frac{v_1+2}{4})^3+\frac{(v_1+2)^3}{64}\bigg|^3\Bigg)^{\frac{1}{3}}\\
                =& ~ d(Tu,Tv).
            \end{align*}
            \item Untuk $u_1\in [3,5]$ dan $v_1\in[1,3)$, diperoleh 
            \begin{align*}
                D(u,v) =& ~\frac{1}{4}\left(|u_1-v_1|^3+|u_1^3-u_2-v_1^3+v_2|^3\right)^{\frac{1}{3}} \\
                &+ \frac{1}{3} \Bigg(\left|u_1-\frac{u_1+2}{4}\right|^3+\bigg|u_1^3-u_2-\left(\frac{u_1+2}{4}\right)^3+\frac{(u_1+2)^3}{64}\bigg|^3\Bigg)^{\frac{1}{3}}\\
                &+\frac{3}{4} \Bigg(\left|u_1-\frac{v_1+3}{4}\right|^3+\bigg|u_1^3-u_2 -\qty(\frac{v_1+3}{4})^3+\frac{(v_1+3)^3}{64}\bigg|^3\Bigg)^{\frac{1}{3}}\\
                \geq &~ \frac{1}{4}\left(|u_1-v_1|^3\right)^{\frac{1}{3}}+ \frac{1}{3}\left(\qty|u_1-\frac{u_1+2}{4}|^3\right)^{\frac{1}{3}}.
            \end{align*}
            Karena $u_1\in[3,5]$, maka $\frac{1}{12}|3u_1-2|\geq \frac{7}{12}> \frac{1}{4}$, sehingga 
            \begin{align*}
                D(u,v) \geq&~ \frac{1}{4}|u_1-v_1|+\frac{1}{4}\\
                \geq&~ \qty|\frac{u_1}{4}-\frac{v_1}{4}-\frac{1}{4}|\\
                =&~ \Bigg(\qty|\frac{u_1+2}{4}-\frac{v_1+3}{4}|^3\\
                &+\qty|\qty(\frac{u_1+2}{4})^3-\frac{(u_1+2)^3}{64}-\qty(\frac{v_1+2}{4})^3+\frac{(v_1+2)^3}{64}|^3\Bigg)^{\frac{1}{3}}\\
                =&~ d(Tu,Tv).
            \end{align*}
            \item Untuk $u_1\in [1,3)$ dan $v_1\in[3,5]$, diperoleh 
            \begin{align*}
                D(u,v) =& ~\frac{1}{4}\left(|u_1-v_1|^3+|u_1^3-u_2-v_1^3+v_2|^3\right)^{\frac{1}{3}} \\
                &+ \frac{1}{3} \Bigg(\left|u_1-\frac{u_1+3}{4}\right|^3+\bigg|u_1^3-u_2-\left(\frac{u_1+3}{4}\right)^3+\frac{(u_1+3)^3}{64}\bigg|^3\Bigg)^{\frac{1}{3}}\\
                &+\frac{3}{4} \Bigg(\left|u_1-\frac{v_1+2}{4}\right|^3+\bigg|u_1^3-u_2 -\qty(\frac{v_1+2}{4})^3+\frac{(v_1+2)^3}{64}\bigg|^3\Bigg)^{\frac{1}{3}}\\
                \geq &~ \frac{1}{4}\left(|u_1-v_1|^3\right)^{\frac{1}{3}}+ \frac{1}{3}\left(\qty|u_1-\frac{u_1+3}{4}|^3\right)^{\frac{1}{3}} + \frac{3}{4}\left(\qty|u_1-\frac{v_1+2}{4}|^3\right)^{\frac{1}{3}}\\
                =&~ \frac{1}{4}|u_1-v_1|+\frac{1}{4}|u_1-1|+\frac{3}{16}|4u_1-v_1-2|.
            \end{align*}
            Diperhatikan bahwa 
            \begin{align*}
                d(Tu,Tv) =&~\Bigg(\qty|\frac{u_1+3}{4}-\frac{v_1+2}{4}|^3\\
                &+ \qty|\qty(\frac{u_1+3}{4})^3-\frac{(u_1+3)^3}{64}-\qty(\frac{v_1+2}{4})^3+\frac{(v_1+3)^3}{64}|^3\Bigg)^{\frac{1}{3}}\\
                =& ~ \frac{1}{4}|u_1-v_1+1|,
            \end{align*}
            sehingga untuk membuktikan bahwa $D(u,v)\geq d(Tu,Tv)$, akan dibuktikan bahwa 
            \begin{align*}
                f(u_1,v_1):=|u_1-v_1|+|u_1-1|+\frac{3}{4}|4u_1-v_1-2|-|u_1-v_1+1|\geq 0,
            \end{align*}
            untuk setiap $u_1\in [1,3)$ dan $v_1\in[3,5]$. Diamati bahwa $1\leq u_1<3\leq v_1$, sehingga 
            \begin{align*}
                f(u_1,v_1)&=v_1-u_1+u_1-1+\frac{3}{4}|4u_1-v_1-2|-|u_1-v_1+1|\\
                &= v_1-1+\frac{3}{4}|4u_1-v_1-2|-|u_1-v_1+1|.
            \end{align*}
            Dari sini, diperoleh bahwa 
            \begin{align*}
                \dfrac{\partial f}{\partial u_1} &= \frac{3(4u_1-v_1-2)}{4|4u_1-v_1-2|}\times 4-\frac{u_1-v_1+1}{|u_1-v_1+1|}\\
                &= 3\sgn{(4u_1-v_1-2)}-\sgn{(u_1-v_1+1)}\neq 0,
            \intertext{serta}
            \frac{\partial f}{\partial v_1} &= 1+\frac{3(4u_1-v_1-2)}{4|4u_1-v_1-2|}\times (-1) -\frac{u_1-v_1+1}{|u_1-v_1+1|}\times (-1) \\
            &= 1 + \frac{3}{4}\sgn{(4u_1-v_1-2)} - \sgn{(u_1-v_1+1)}\neq 0,
            \end{align*}
            artinya $f(u_1,v_1)$ tidak mempunyai titik pelana. Walaupun begitu, $f(u_1,v_1)$ mempunyai titik kritis, yaitu saat $|4u_1-v_1-2|=0$ atau $|u_1-v_1+1|=0$. Hal ini berarti nilai minimumnya berada pada titik kritis atau pada titik-titik batas domainnya. 
            \begin{enumerate}
                \item[\textbf{(a)}] \textbf{pada titik kritis}\\
                Jika $|4u_1-v_1-2|=0$, didapat $v_1=4u_1-2$ sehingga 
                \begin{align*}
                    f(u_1,v_1) = 4u_1-3-3|u_1-1|.
                \end{align*}
                Karena $u_1\geq 1$ didapat 
                \begin{align*}
                    f(u_1,v_1)=4u_1-3-3u_1+3=u_1\geq 1 >0.
                \end{align*}
                Kemudian, jika $|u_1-v_1+1|=0$, didapat $v_1=u_1+1$ sehingga
                \begin{align*}
                    f(u_1,v_1)=u_1+\frac{9}{4}|u_1-1|\geq u_1\geq 1>0.
                \end{align*}
                \item[\textbf{(b)}] \textbf{pada titik batas domain}\\
                \begin{enumerate}
                    \item Jika $u_1=1$, didapatkan 
                    \begin{align*}
                        f(u_1,v_1)\geq f(1,v_1)=v_1-1+\frac{3}{4}|2-v_1|-|2-v_1|.
                    \end{align*}
                    Karena $v_1\geq 3$, diperoleh 
                    \begin{align*}
                        f(u_1,v_1) &\geq v_1-1-\frac{1}{4}(v_1-2)= \frac{3}{4}v_1-\frac{1}{3}\geq \frac{7}{4}>0.
                    \end{align*}
                    \item Jika $u_1\to 3^{-}$, didapatkan 
                    \begin{align*}
                        f(u_1,v_1)\geq \lim_{u_1\to 3^-} f(u_1,v_1)=v_1-1+\frac{3}{4}|10-v_1|-|4-v_1|.
                    \end{align*}
                    \begin{enumerate}
                        \item Jika $v_1\in[4,5]$, diperoleh 
                        \begin{align*}
                            f(u_1,v_1)&\geq v_1-1+\frac{3}{4}(10-v_1)-(v_1-4)\\
                            &= \frac{21}{2}-\frac{3}{4}v_1 \\
                            &\geq \frac{21}{2}-\frac{15}{4}\\
                            &=\frac{27}{4}\\
                            &>0.
                        \end{align*}
                        \item Jika $v_1\in [3,4]$, didapatkan 
                        \begin{align*}
                            f(u_1,v_1)&\geq v_1-1+\frac{3}{4}(10-v_1)-(4-v_1)=\frac{5}{4}v_1+\frac{5}{2}>0.
                        \end{align*}
                    \end{enumerate}
                    \item Jika $v_1=3$, didapatkan 
                    \begin{align*}
                        f(u_1,v_1)\geq f(u_1,3) = 2+\frac{3}{4}|4u_1-5|-|u_1-2|.
                    \end{align*}
                    Karena $u_1\in [1,3)$, maka nilai maksimum dari $|u_1-2|$ adalah 1 sehingga 
                    \begin{align*}
                        f(u_1,v_1)\geq 2+\frac{3}{4}|4u_1-5|-1\geq 1>0.
                    \end{align*}
                    \item Jika $v_1=5$, didapatkan 
                    \begin{align*}
                        f(u_1,v_1)\geq f(u_1,5)=4+\frac{3}{4}|4u_1-7|-|u_1-4|.
                    \end{align*}
                    Karena $u_1\in[1,3)$, maka 
                    \begin{align*}
                        f(u_1,v_1)&\geq 4+\frac{3}{4}|4u_1-7|-(4-u_1)\\
                        &=\frac{3}{4}|4u_1-7|+u_1\\
                        &\geq u_1\\
                        &>0.
                    \end{align*}
                \end{enumerate}
                Dari uraian tersebut didapatkan bahwa $f(u_1,v_1)\geq 0$ untuk setiap $u_1\in[1,3)$ dan $v_1\in [3,5]$ sehingga berlaku pula $\alpha d(u,v)+\beta d(u,Tu)+\gamma d(u,Tv)\geq d(Tu,Tv)$.
            \end{enumerate}
        \end{enumerate}
        Karena semua tinjauan kasus di atas menghasilkan $\alpha d(u,v)+\beta d(u,Tu)+\gamma d(u,Tv)\geq d(Tu,Tv)$ untuk setiap $u_1,v_1\in[1,5]$, maka $T$ adalah pemetaan $(\alpha,\beta,\gamma)$-nonekspansif dengan $\alpha=\frac{1}{4},~\beta=\frac{1}{3},$ dan $\gamma=\frac{3}{4}$. Akan tetapi, $T$ bukan pemetaan nonekspansif karena untuk $u=\qty(\frac{29}{10},\frac{29^3}{1000},0,0,\dots)$ dan $v=(3,27,0,0,\dots)$, didapatkan 
            \begin{align*}
            d(Tu,Tv)=&~\Bigg(\mqty|\frac{\frac{29}{10}+3}{4}-\frac{3+2}{4}|^3\\
            &+\bigg|\bigg(\frac{\frac{29}{10}+3}{4}\bigg)^3-\frac{(\frac{29}{10}+3)^3}{64}-\frac{(3+2)^3}{64}+\bigg(\frac{3+2}{4}\bigg)^3\bigg|^3\Bigg)^{\frac{1}{3}}\\
                =&~\frac{9}{40}\\
                >&~ \frac{1}{10}\\                =&~\qty(\left|3-\frac{29}{10}\right|^3+\qty|\qty(\frac{29}{10})^3-\frac{29^3}{1000}-3^3+27|^3)^{\frac{1}{3}} \\
                =&~d(u,v).
            \end{align*}
    Selanjutnya, untuk mendapatkan titik tetap dari $T$, dicari $w=(w_1,w_2,0,0,\cdots)\in W$ sehingga $T(w)=w$. Dari definisi $T$, terdapat dua kemungkinan, yaitu
    \begin{align*}
        &\text{(i) } w_1=\frac{w_1+3}{4}, ~w_2=\frac{(w_1+3)^3}{64}, \quad \text{atau}\\
        &\text{(ii) } w_1=\frac{w_1+2}{4}, ~w_2=\frac{(w_1+2)^3}{64}.
    \end{align*}
    Dari (i) diperoleh $w_1=w_2=1\in [1,3)$, sedangkan dari (ii) diperoleh $w_1=\frac{2}{3}\notin [3,5]$. Dengan demikian, titik tetap dari $T$ adalah $(1,1,0,0,\cdots)$.
    
    % \begin{exam}\label{con:abcnoncat}
    %     Diberikan ruang $(\ell_3,\norm{\cdot}_3)$ dengan metrik $d(u,v)=\left(\sum_{i=1}^{+\infty} |u_i-v_i|^3\right)^{\frac{1}{3}}$. Berdasarkan Contoh \ref{con:lpcatp}, ruang tersebut adalah ruang $CAT_p(0)$. Diberikan pula $W=\{(x,y,0,0,\cdots) ~|~ y\in[1,5]\}\subset \ell_3$ dan pemetaan $T:W\to W$ yang didefinisikan sebagai 
    %     \begin{align*}
    %         T\left((x,y,0,0,\cdots)\right) = \begin{cases}
    %             (\frac{1}{3}, \frac{y+3}{4},0,0,\cdots), \qquad &\text{jika } y\in [1,3)\\
    %             (\frac{1}{3}, \frac{y+2}{4},0,0,\cdots),\qquad &\text{jika } y\in [3,5].
    %         \end{cases}
    %     \end{align*}
    %     Pemetaan $T$ adalah pemetaan $(\frac{1}{4},\frac{1}{3},\frac{3}{4})$-nonekspansif tetapi bukan pemetaan nonekspansif.

    %     Penjelasan dari contoh tersebut diberikan di bawah ini.\\
    % % \end{exam}
    % % \begin{bukti} 
    % Diambil sebarang $u,v\in W$ dengan $u=(x_1,y_1,0,0,\cdots), ~v=(x_2,y_2,0,0,\cdots)$.
    %     \begin{enumerate}[label={\textbf{Kasus \arabic*.}},align=left]
    %         \item Untuk $y_1,y_2\in [1,3)$, diperoleh 
    %         \begin{align*}
    %             \alpha d(u,v)+\beta d(u,Tu)+\gamma d(u,Tv) =& ~\frac{1}{4}\left(|x_1-x_2|^3+|y_1-y_2|^3\right)^{\frac{1}{3}} \\
    %             &+ \frac{1}{3} \left(\left|x_1-\frac{1}{3}\right|^3+\left|y_1-\frac{y_1+3}{4}\right|^3\right)^{\frac{1}{3}}\\
    %             &+\frac{3}{4} \left(\left|x_1-\frac{1}{3}\right|^3+\left|y_1-\frac{y_2+3}{4}\right|^3\right)^{\frac{1}{3}}\\
    %             \geq &~ \frac{1}{4}\left(|x_1-x_2|^3+|y_1-y_2|^3\right)^{\frac{1}{3}}\\
    %             \geq &~ \frac{1}{4}\left(|y_1-y_2|^3\right)^{\frac{1}{3}}\\
    %             =&~ \left|\frac{y_1+3}{4}-\frac{y_2+3}{4}\right|\\
    %             =&~ \left(\left|\frac{1}{3}-\frac{1}{3}\right|^3+\left|\frac{y_1+3}{4}-\frac{y_2+3}{4}\right|^3\right)^{\frac{1}{3}}\\
    %             =& ~ d(Tu,Tv).
    %         \end{align*}
    %         \item Untuk $y_1,y_2\in [3,5]$, diperoleh 
    %         \begin{align*}
    %              \alpha d(u,v)+\beta d(u,Tu)+\gamma d(u,Tv) =& ~\frac{1}{4}\left(|x_1-x_2|^3+|y_1-y_2|^3\right)^{\frac{1}{3}} \\
    %             &+ \frac{1}{3} \left(\left|x_1-\frac{1}{3}\right|^3+\left|y_1-\frac{y_1+2}{4}\right|^3\right)^{\frac{1}{3}}\\
    %             &+\frac{3}{4} \left(\left|x_1-\frac{1}{3}\right|^3+\left|y_1-\frac{y_2+2}{4}\right|^3\right)^{\frac{1}{3}}\\
    %             \geq &~ \frac{1}{4}\left(|x_1-x_2|^3+|y_1-y_2|^3\right)^{\frac{1}{3}}\\
    %             \geq &~ \frac{1}{4}\left(|y_1-y_2|^3\right)^{\frac{1}{3}}\\
    %             =&~ \left|\frac{y_1+2}{4}-\frac{y_2+2}{4}\right|\\
    %             =&~ \left(\left|\frac{1}{3}-\frac{1}{3}\right|^3+\left|\frac{y_1+2}{4}-\frac{y_2+2}{4}\right|^3\right)^{\frac{1}{3}}\\
    %             =& ~ d(Tu,Tv).
    %         \end{align*}
    %         \item Untuk $y_1\in[3,5]$ dan $y_2\in [1,3)$, diperoleh 
    %         \begin{align*}
    %             \alpha d(u,v)+\beta d(u,Tu)+\gamma d(u,Tv) =& ~\frac{1}{4}\left(|x_1-x_2|^3+|y_1-y_2|^3\right)^{\frac{1}{3}} \\
    %             &+ \frac{1}{3} \left(\left|x_1-\frac{1}{3}\right|^3+\left|y_1-\frac{y_1+2}{4}\right|^3\right)^{\frac{1}{3}}\\
    %              &+\frac{3}{4} \left(\left|x_1-\frac{1}{3}\right|^3+\left|y_1-\frac{y_2+3}{4}\right|^3\right)^{\frac{1}{3}}\\
    %              \geq &~ ~\frac{1}{4}\left(|x_1-x_2|^3+|y_1-y_2|^3\right)^{\frac{1}{3}} \\
    %              &+ \frac{1}{3} \left(\left|x_1-\frac{1}{3}\right|^3+\left|y_1-\frac{y_1+2}{4}\right|^3\right)^{\frac{1}{3}}\\
    %             \geq & ~\frac{1}{4}|y_1-y_2| + \frac{1}{3} \left|y_1-\frac{y_1+2}{4}\right|\\
    %             =&~ \frac{1}{4}|y_1-y_2| + \frac{1}{12} \left|3y_1-2\right|.
    %         \end{align*}
    %         Karena $y_1\in[3,5]$, maka $\frac{1}{12}|3y_1-2|\geq \frac{7}{12}>\frac{1}{4}$, sehingga
    %         \begin{align*}
    %             \alpha d(u,v)+\beta d(u,Tu)+\gamma d(u,Tv) \geq & ~ \frac{1}{4}|y_1-y_2| +\frac{1}{4}\\
    %             \geq&~ \left|\frac{y_1}{4}-\frac{y_2}{4}-\frac{1}{4}\right|\\
    %             \geq &~ \left(\left|\frac{1}{3}-\frac{1}{3}\right|^3+\left|\frac{y_1+2}{4}-\frac{y_2+3}{4}\right|^3\right)^\frac{1}{3}\\
    %             =&~ d(Tu,Tv).
    %         \end{align*}
    %         \item Untuk $y_1\in[1,3)$ dan $y_2\in[3,5]$, diperoleh 
    %         \begin{align*}
    %             \alpha d(u,v)+\beta d(u,Tu)+\gamma d(u,Tv) =& ~\frac{1}{4}\left(|x_1-x_2|^3+|y_1-y_2|^3\right)^{\frac{1}{3}} \\
    %             &+ \frac{1}{3} \left(\left|x_1-\frac{1}{3}\right|^3+\left|y_1-\frac{y_1+3}{4}\right|^3\right)^{\frac{1}{3}}\\
    %              &+\frac{3}{4} \left(\left|x_1-\frac{1}{3}\right|^3+\left|y_1-\frac{y_2+2}{4}\right|^3\right)^{\frac{1}{3}}\\
    %              \geq &~ \frac{1}{4}|y_1-y_2|+\frac{1}{3}\left|y_1-\frac{y_1+3}{4}\right|\\
    %              &+\frac{3}{4}\left|y_1-\frac{y_2+2}{4}\right|
    %         \end{align*}
    %         Diperhatikan bahwa 
    %         $$d(Tu,Tv)=\left(\left|\frac{1}{3}-\frac{1}{3}\right|^3+\left|\frac{y_1+3}{4}-\frac{y_2+2}{4}\right|^3\right)^\frac{1}{3} = \frac{1}{4}|y_1-y_2+1|,$$ sehingga untuk membuktikan bahwa $ \alpha d(u,v)+\beta d(u,Tu)+\gamma d(u,Tv)\geq d(Tu,Tv)$, akan dibuktikan bahwa
    %         \begin{align*}
    %             f(y_1,y_2) &=|y_1-y_2|+|y_1-1|+\frac{3}{4}\left|4y_1-y_2-2\right| - |y_1-y_2+1| \geq 0,
    %         \end{align*}
    %         untuk setiap $y_1\in[1,3)$ dan $y_2\in [3,5]$. Diamati bahwa $1\leq y_1<3\leq y_2$, sehingga 
    %         \begin{align*}
    %             f(y_1,y_2) &= y_2-y_1+y_1-1 +\frac{3}{4}|4y_1-y_2-2|-|y_1-y_2+1|\\
    %             &= y_2 -1+\frac{3}{4}|4y_1-y_2-2| - |y_1-y_2+1|.
    %         \end{align*}
    %         Dari sini, diperoleh bahwa 
    %         \begin{align*}
    %         \frac{\partial f}{\partial y_1} &= \frac{3(4y_1-y_2-2)}{4|4y_1-y_2-2|}\times 4 -\frac{y_1-y_2+1}{|y_1-y_2+1|}\\
    %         &= 3\sgn{(4y_1-y_2-2)}-\sgn{(y_1-y_2+1)} \neq 0, 
    %         \intertext{serta}
    %         \frac{\partial f}{\partial y_2} &= 1+\frac{3(4y_1-y_2-2)}{4|4y_1-y_2-2|}\times (-1) -\frac{y_1-y_2+1}{|y_1-y_2+1|}\times (-1) \\
    %         &= 1 + \frac{3}{4}\sgn{(4y_1-y_2-2)} - \sgn{(y_1-y_2+1)}\neq 0,
    %         \end{align*}
    %         artinya $f(y_1,y_2)$ tidak memiliki titik pelana. Walaupun begitu, $f(y_1,y_2)$ memiliki titik kritis, yaitu saat $|4y_1-y_2-2|=0$ atau $|y_1-y_2+1|=0$. Hal ini berarti nilai minimumnya berada pada titik kritis atau pada titik-titik batas domainnya. 
    %         \begin{enumerate}
    %             \item[\textbf{(a)}] \textbf{pada titik kritis}\\
    %             Jika $|4y_1-y_2-2|=0$, didapat $y_2=4y_1-2$ sehingga 
    %             \begin{align*}
    %                 f(y_1,y_2) = 4y_1-3-3|y_1-1|.
    %             \end{align*}
    %             Karena $y_1\geq 1$ didapat 
    %             \begin{align*}
    %                 f(y_1,y_2)=4y_1-3-3y_1+3=y_1\geq 1 >0.
    %             \end{align*}
    %             Kemudian, jika $|y_1-y_2+1|=0$, didapat $y_2=y_1+1$ sehingga
    %             \begin{align*}
    %                 f(y_1,y_2)=y_1+\frac{9}{4}|y_1-1|\geq y_1\geq 1>0.
    %             \end{align*}
    %             \item[\textbf{(b)}] \textbf{pada titik batas domain}\\
    %             \begin{enumerate}
    %                 \item Jika $y_1=1$, didapatkan 
    %                 \begin{align*}
    %                     f(y_1,y_2)\geq f(1,y_2)=y_2-1+\frac{3}{4}|2-y_2|-|2-y_2|.
    %                 \end{align*}
    %                 Karena $y_2\geq 3$, diperoleh 
    %                 \begin{align*}
    %                     f(y_1,y_2) &\geq y_2-1-\frac{1}{4}(y_2-2)= \frac{3}{4}y_2-\frac{1}{3}\geq \frac{7}{4}>0.
    %                 \end{align*}
    %                 \item Jika $y_1\to 3^{-}$, didapatkan 
    %                 \begin{align*}
    %                     f(y_1,y_2)\geq \lim_{y_1\to 3^-} f(y_1,y_2)=y_2-1+\frac{3}{4}|10-y_2|-|4-y_2|.
    %                 \end{align*}
    %                 \begin{enumerate}
    %                     \item Jika $y_2\in[4,5]$, diperoleh 
    %                     \begin{align*}
    %                         f(y_1,y_2)&\geq y_2-1+\frac{3}{4}(10-y_2)-(y_2-4)\\
    %                         &= \frac{21}{2}-\frac{3}{4}y_2 \\
    %                         &\geq \frac{21}{2}-\frac{15}{4}\\
    %                         &=\frac{27}{4}\\
    %                         &>0.
    %                     \end{align*}
    %                     \item Jika $y_2\in [3,4]$, didapatkan 
    %                     \begin{align*}
    %                         f(y_1,y_2)&\geq y_2-1+\frac{3}{4}(10-y_2)-(4-y_2)=\frac{5}{4}y_2+\frac{5}{2}>0.
    %                     \end{align*}
    %                 \end{enumerate}
    %                 \item Jika $y_2=3$, didapatkan 
    %                 \begin{align*}
    %                     f(y_1,y_2)\geq f(y_1,3) = 2+\frac{3}{4}|4y_1-5|-|y_1-2|.
    %                 \end{align*}
    %                 Karena $y_1\in [1,3)$, maka nilai maksimum dari $|y_1-2|$ adalah 1 sehingga 
    %                 \begin{align*}
    %                     f(y_1,y_2)\geq 2+\frac{3}{4}|4y_1-5|-1\geq 1>0.
    %                 \end{align*}
    %                 \item Jika $y_2=5$, didapatkan 
    %                 \begin{align*}
    %                     f(y_1,y_2)\geq f(y_1,5)=4+\frac{3}{4}|4y_1-7|-|y_1-4|.
    %                 \end{align*}
    %                 Karena $y_1\in[1,3)$, maka 
    %                 \begin{align*}
    %                     f(y_1,y_2)&\geq 4+\frac{3}{4}|4y_1-7|-(4-y_1)\\
    %                     &=\frac{3}{4}|4y_1-7|+y_1\\
    %                     &\geq y_1\\
    %                     &>0.
    %                 \end{align*}
    %             \end{enumerate}
    %             Dari uraian tersebut didapatkan bahwa $f(y_1,y_2)\geq 0$ untuk setiap $y_1\in[1,3)$ dan $y_2\in [3,5]$ sehingga berlaku pula $\alpha d(u,v)+\beta d(u,Tu)+\gamma d(u,Tv)\geq d(Tu,Tv)$.
    %         \end{enumerate}
    %     \end{enumerate}
    %     Karena semua tinjauan kasus di atas menghasilkan $\alpha d(u,v)+\beta d(u,Tu)+\gamma d(u,Tv)\geq d(Tu,Tv)$ untuk setiap $y_1,y_2\in[1,5]$, maka $T$ adalah pemetaan $(\alpha,\beta,\gamma)$-nonekspansif dengan $\alpha=\frac{1}{4},~\beta=\frac{1}{3},$ dan $\gamma=\frac{3}{4}$. Akan tetapi, $T$ bukan pemetaan nonekspansif karena untuk $u=\qty(\frac{29}{10},\frac{29^3}{1000},0,0,\dots)$ dan $v=(3,27,0,0,\dots)$, didapatkan 
    %         \begin{align*}
    %         d(Tu,Tv)=&~\Bigg(\mqty|\frac{\frac{29}{10}+3}{4}-\frac{3+2}{4}|^3\\
    %         &+\qty|\qty(\frac{\frac{29}{10}+3}{4})^3-\frac{(\frac{29}{10}+3)^3}{64}-\frac{(3+2)^3}{64}+\qty(\frac{3+2}{4})^3|^3\Bigg)^{\frac{1}{3}}\\
    %             =&~\frac{9}{40}\\
    %             >&~ \frac{1}{10}\\                =&~\qty(\left|3-\frac{29}{10}\right|^3+\qty|\qty(\frac{29}{10})^3-\frac{29^3}{1000}-3^3+27|^3)^{\frac{1}{3}} \\
    %             =&~d(u,v).
    %         \end{align*}
    % \end{exam}
    Selanjutnya, berikut ini diberikan tiga lema penting untuk pembuktian konvergensi skema iterasi Sabri. 
    \begin{lemma}\label{lemma:tutx}
    Diberikan $(X,d,G)$ adalah ruang $CAT _p(0)$ dan $W$ adalah himpunan bagian tak kosong dari $X$, serta $T:W\to W$ adalah pemetaan $(\alpha,\beta,\gamma)$-nonekspansif. Jika $u$ adalah titik tetap dari $T$, maka untuk setiap $x\in W$ berlaku $d(Tx,Tu)\leq d(x,u)$. 
    \end{lemma}
    \begin{bukti}
        Diperhatikan bahwa $u$ titik tetap dari $T$ sehingga berlaku $u=Tu$, diperoleh 
        \begin{align*}
            d(u,Tx) = d(Tu,Tx)&\leq \alpha d(u,x)+\beta d(u,Tu)+\gamma d(u,Tx)\\
            &= \alpha d(u,x) +\beta d(u,u)+ \gamma d(u,Tx)\\
            &= \alpha d(x,u) + \gamma d(u,Tx).
        \end{align*}
        Dari sini didapatkan $(1-\gamma) d(u, Tx) \leq \alpha d(x,u)$. Kemudian, karena $\alpha+\gamma\leq 1$, didapat $\alpha \leq 1-\gamma$ sehingga 
        \begin{align*}
            \frac{\alpha}{1-\gamma}\leq 1.
        \end{align*}
        Akibatnya
        \begin{align}
           d(Tx,Tu)= d(Tu,Tx)= d(u, Tx) \leq \frac{\alpha}{1-\gamma} d(x,u) \leq d(x,u).\label{ineq:titap}
        \end{align}
    \end{bukti}
    \begin{thm}
        Diberikan $(X,d,G)$ adalah ruang $CAT_p(0)$ dan $W$ adalah himpunan bagian tak kosong dari $X$, serta $T:W\to W$ adalah pemetaan $(\alpha,\beta,\gamma)$-nonekspansif. Jika $\alpha+\gamma\neq 1$ dan $T$ memiliki titik tetap, maka titik tetap dari $T$ tunggal.
    \end{thm}
    \begin{bukti}
        Misalkan $u$ dan $v$ adalah titik tetap dari $T$, maka berdasarkan Ketaksamaan \eqref{ineq:titap} diperoleh 
        \begin{align*}
            d(u,v)=d(Tu,Tv) \leq \frac{\alpha}{1-\gamma} d(u,v) 
        \end{align*} 
        Karena $\alpha,\gamma\in [0,1]$, $\gamma\neq 1$, dan $\alpha+\gamma\neq 1$, didapatkan bahwa $0\leq \frac{\alpha}{1-\gamma}<1$ sehingga $d(u,v)=0$. Dengan demikian $u=v$, yang berarti bahwa titik tetap dari $T$ tunggal. 
    \end{bukti}
    \begin{lemma}\label{Lema:ineqabcnoneks}
        Diberikan $(X,d,G)$ adalah ruang $CAT _p(0)$ dan $W$ adalah himpunan bagian tak kosong dari $X$, serta $T:W\to W$ adalah pemetaan $(\alpha,\beta,\gamma)$-nonekspansif. Untuk setiap $x,y\in W$, ketaksamaan berikut ini berlaku:
        \begin{equation}\label{eq:ineqabcnoneks}
            d(x,Ty) \leq \frac{\alpha}{1-\gamma}d(x,y)+ \frac{1+\beta}{1-\gamma}d(x,Tx).
        \end{equation}
    \end{lemma}
    \begin{bukti}
        Diamati bahwa untuk setiap $x,y\in W$ berlaku 
        \begin{align*}
            d(x,Ty) &\leq d(x,Tx)+d(Tx,Ty)\\
            &\leq d(x,Tx) + \alpha d(x,y) +\beta d(x,Tx)+\gamma d(x,Ty)\\
            &= (1+\beta)d(x,Tx) + \alpha d(x,y) + \gamma d(x,Ty). 
        \end{align*}
        Akibatnya 
        \begin{align*}
            (1-\gamma)d(x,Ty) &\leq (1+\beta)d(x,Tx) + \alpha d(x,y) \\
            \Longleftrightarrow \qquad \quad d(x,Ty)&\leq \frac{\alpha}{1-\gamma}d(x,y) + \frac{1+\beta}{1-\gamma}d(x,Tx).
        \end{align*}
    \end{bukti}
    \begin{lemma}\label{Lema:demi}
        Diberikan $(X,d,G)$ adalah ruang $CAT _p(0)$ dan $W$ adalah himpunan bagian tak kosong dari $X$. Jika $T:W\to W$ adalah pemetaan $(\alpha,\beta,\gamma)$-nonekspansif, maka $T$ memiliki sifat \textbf{demiclosedness}. 
    \end{lemma}
    \begin{bukti}
        Diambil sebarang barisan $\{x_n\}\subseteq W$ yang terbatas dan konvergen-$\Delta$ ke $u_0\in W$, serta memenuhi $\lim_{n\to\infty} d(x_n,Tx_n)=0$. Berdasarkan definisi \ref{defn:konvD}, diperoleh bahwa $u_0\in A(\{x_n\})$. Kemudian, menggunakan ketaksamaan \eqref{eq:ineqabcnoneks}, didapatkan 
        \begin{align*}
            d(x_n, Tu_0)&\leq \dfrac{\alpha}{1-\gamma}d(x_n, u_0)+\dfrac{1+\beta}{1-\gamma}d(x_n, Tx_n).
        \end{align*}
        Selanjutnya, didapatkan bahwa 
        \begin{align*}
            \limsup_{n\to \infty} d(x_n, Tu_0)\leq \limsup_{n\to \infty} \dfrac{\alpha}{1-\gamma} d(x_n, u_0) \leq \limsup_{n\to \infty} d(x_n, u_0).
        \end{align*}
        Hal ini berarti $Tu_0\in A(\{x_n\})$. Kemudian, berdasarkan Lema \ref{Lema:asimtotik}, dipunyai bahwa $A(\{x_n\})$ tepat memiliki satu elemen, yang berarti $u_0=Tu_0$. Jadi $T$ memiliki sifat \textbf{demiclosedness}. 
    \end{bukti}
