\begin{exam}\label{con:abcnon}
        Diberikan $(X,d,G)$ adalah ruang $CAT _p(0)$ sebagaimana dalam Contoh \ref{con:Catp}. Diberikan pula $W=\{(w_1,w_2,0,0,\cdots)\mid w_1\in[1,5], w_2\in[1,125]\}\subset X$. dan pemetaan $T:W\to W$ yang didefinisikan sebagai 
        \begin{align*}
            T((w_1,w_2,0,0,\cdots))=\begin{cases}
                \qty(\frac{w_1+3}{4},\frac{(w_1+3)^3}{64},0,0,\cdots), \quad &\text{jika } w_1\in [1,3)\\
                \qty(\frac{w_1+2}{4}, \frac{(w_1+2)^3}{64},0,0,\cdots), \quad &\text{jika } w_1\in [3,5].
            \end{cases}
        \end{align*}
        Pemetaan $T$ adalah pemetaan $(\frac{1}{4},\frac{1}{3},0)$-nonekspansif, tetapi bukan pemetaan nonekspansif. Titik tetap dari $T$ adalah $(1,1,0,0,\cdots)$.
    \end{exam}
        Penjelasan dari Contoh \ref{con:abcnon} diuraikan berikut ini.\\
        Diambil sebarang $u,v\in W$ dengan $u=(u_1,u_2,0,0,\cdots), ~v=(v_1,v_2,0,0,\cdots)$. Dimisalkan $D(u,v)=\alpha d(u,v)+\beta d(u,Tu)+\gamma d(u,Tv)$. 
        \begin{enumerate}[label={\textbf{Kasus \arabic*.}},align=left]
            \item Untuk $u_1,v_1\in [1,3)$, diperoleh 
            \begin{align*}
                D(u,v) =& ~\frac{1}{4}\left(|u_1-v_1|^3+|u_1^3-u_2-v_1^3+v_2|^3\right)^{\frac{1}{3}} \\
                &+ \frac{1}{3} \left(\left|u_1-\frac{u_1+3}{4}\right|^3+\bigg|u_1^3-u_2-\qty(\frac{u_1+3}{4})^3+\frac{(u_1+3)^3}{64}\bigg|^3\right)^{\frac{1}{3}}\\
                &+0\times \left(\left|u_1-\frac{v_1+3}{4}\right|^3+\bigg|u_1^3-u_2 -\qty(\frac{v_1+3}{4})^3+\frac{(v_1+3)^3}{64}\bigg|^3\right)^{\frac{1}{3}}\\
                \geq &~ \frac{1}{4}\left(|u_1-v_1|^3\right)^{\frac{1}{3}}\\
                =&~ \left|\frac{u_1+3}{4}-\frac{v_1+3}{4}\right|\\
                =&~ \Bigg(\left|\frac{u_1+3}{4}-\frac{v_1+3}{4}\right|^3           \\    &+\bigg|\qty(\frac{u_1+3}{4})^3-\frac{(u_1+3)^3}{64}-\qty(\frac{v_1+3}{4})^3+\frac{(v_1+3)^3}{64}\bigg|^3\Bigg)^{\frac{1}{3}}\\
                =& ~ d(Tu,Tv).
            \end{align*}
            \item Untuk $u_1,v_1\in [3,5]$, diperoleh 
            \begin{align*}
                D(u,v) =& ~\frac{1}{4}\left(|u_1-v_1|^3+|u_1^3-u_2-v_1^3+v_2|^3\right)^{\frac{1}{3}} \\
                &+ \frac{1}{3} \Bigg(\left|u_1-\frac{u_1+2}{4}\right|^3+\bigg|u_1^3-u_2-\qty(\frac{u_1+2}{4})^3+\frac{(u_1+2)^3}{64}\bigg|^3\Bigg)^{\frac{1}{3}}\\
                &+\frac{3}{4} \Bigg(\left|u_1-\frac{v_1+2}{4}\right|^3+\bigg|u_1^3-u_2 -\qty(\frac{v_1+2}{4})^3+\qty(\frac{v_1+2}{4})^3\bigg|^3\Bigg)^{\frac{1}{3}}\\
                \geq &~ \frac{1}{4}\left(|u_1-v_1|^3\right)^{\frac{1}{3}}\\
                =&~ \left|\frac{u_1+2}{4}-\frac{v_1+2}{4}\right|\\
                =&~ \Bigg(\left|\frac{u_1+2}{4}-\frac{v_1+2}{4}\right|^3           \\    &+\bigg|\qty(\frac{u_1+2}{4})^3-\frac{(u_1+2)^3}{64}-\qty(\frac{v_1+2}{4})^3+\frac{(v_1+2)^3}{64}\bigg|^3\Bigg)^{\frac{1}{3}}\\
                =& ~ d(Tu,Tv).
            \end{align*}
            \item Untuk $u_1\in [3,5]$ dan $v_1\in[1,3)$, diperoleh 
            \begin{align*}
                D(u,v) =& ~\frac{1}{4}\left(|u_1-v_1|^3+|u_1^3-u_2-v_1^3+v_2|^3\right)^{\frac{1}{3}} \\
                &+ \frac{1}{3} \Bigg(\left|u_1-\frac{u_1+2}{4}\right|^3+\bigg|u_1^3-u_2-\left(\frac{u_1+2}{4}\right)^3+\frac{(u_1+2)^3}{64}\bigg|^3\Bigg)^{\frac{1}{3}}\\
                &+\frac{3}{4} \Bigg(\left|u_1-\frac{v_1+3}{4}\right|^3+\bigg|u_1^3-u_2 -\qty(\frac{v_1+3}{4})^3+\frac{(v_1+3)^3}{64}\bigg|^3\Bigg)^{\frac{1}{3}}\\
                \geq &~ \frac{1}{4}\left(|u_1-v_1|^3\right)^{\frac{1}{3}}+ \frac{1}{3}\left(\qty|u_1-\frac{u_1+2}{4}|^3\right)^{\frac{1}{3}}.
            \end{align*}
            Karena $u_1\in[3,5]$, maka $\frac{1}{12}|3u_1-2|\geq \frac{7}{12}> \frac{1}{4}$, sehingga 
            \begin{align*}
                D(u,v) \geq&~ \frac{1}{4}|u_1-v_1|+\frac{1}{4}\\
                \geq&~ \qty|\frac{u_1}{4}-\frac{v_1}{4}-\frac{1}{4}|\\
                =&~ \Bigg(\qty|\frac{u_1+2}{4}-\frac{v_1+3}{4}|^3\\
                &+\qty|\qty(\frac{u_1+2}{4})^3-\frac{(u_1+2)^3}{64}-\qty(\frac{v_1+2}{4})^3+\frac{(v_1+2)^3}{64}|^3\Bigg)^{\frac{1}{3}}\\
                =&~ d(Tu,Tv).
            \end{align*}
            \item Untuk $u_1\in [1,3)$ dan $v_1\in[3,5]$, diperoleh 
            \begin{align*}
                D(u,v) =& ~\frac{1}{4}\left(|u_1-v_1|^3+|u_1^3-u_2-v_1^3+v_2|^3\right)^{\frac{1}{3}} \\
                &+ \frac{1}{3} \Bigg(\left|u_1-\frac{u_1+3}{4}\right|^3+\bigg|u_1^3-u_2-\left(\frac{u_1+3}{4}\right)^3+\frac{(u_1+3)^3}{64}\bigg|^3\Bigg)^{\frac{1}{3}}\\
                &+\frac{3}{4} \Bigg(\left|u_1-\frac{v_1+2}{4}\right|^3+\bigg|u_1^3-u_2 -\qty(\frac{v_1+2}{4})^3+\frac{(v_1+2)^3}{64}\bigg|^3\Bigg)^{\frac{1}{3}}\\
                \geq &~ \frac{1}{4}\left(|u_1-v_1|^3\right)^{\frac{1}{3}}+ \frac{1}{3}\left(\qty|u_1-\frac{u_1+3}{4}|^3\right)^{\frac{1}{3}} + \frac{3}{4}\left(\qty|u_1-\frac{v_1+2}{4}|^3\right)^{\frac{1}{3}}\\
                =&~ \frac{1}{4}|u_1-v_1|+\frac{1}{4}|u_1-1|+\frac{3}{16}|4u_1-v_1-2|.
            \end{align*}
            Diperhatikan bahwa 
            \begin{align*}
                d(Tu,Tv) =&~\Bigg(\qty|\frac{u_1+3}{4}-\frac{v_1+2}{4}|^3\\
                &+ \qty|\qty(\frac{u_1+3}{4})^3-\frac{(u_1+3)^3}{64}-\qty(\frac{v_1+2}{4})^3+\frac{(v_1+3)^3}{64}|^3\Bigg)^{\frac{1}{3}}\\
                =& ~ \frac{1}{4}|u_1-v_1+1|,
            \end{align*}
            sehingga untuk membuktikan bahwa $D(u,v)\geq d(Tu,Tv)$, akan dibuktikan bahwa 
            \begin{align*}
                f(u_1,v_1):=|u_1-v_1|+|u_1-1|+\frac{3}{4}|4u_1-v_1-2|-|u_1-v_1+1|\geq 0,
            \end{align*}
            untuk setiap $u_1\in [1,3)$ dan $v_1\in[3,5]$. Diamati bahwa $1\leq u_1<3\leq v_1$, sehingga 
            \begin{align*}
                f(u_1,v_1)&=v_1-u_1+u_1-1+\frac{3}{4}|4u_1-v_1-2|-|u_1-v_1+1|\\
                &= v_1-1+\frac{3}{4}|4u_1-v_1-2|-|u_1-v_1+1|.
            \end{align*}
            Dari sini, diperoleh bahwa 
            \begin{align*}
                \dfrac{\partial f}{\partial u_1} &= \frac{3(4u_1-v_1-2)}{4|4u_1-v_1-2|}\times 4-\frac{u_1-v_1+1}{|u_1-v_1+1|}\\
                &= 3\sgn{(4u_1-v_1-2)}-\sgn{(u_1-v_1+1)}\neq 0,
            \intertext{serta}
            \frac{\partial f}{\partial v_1} &= 1+\frac{3(4u_1-v_1-2)}{4|4u_1-v_1-2|}\times (-1) -\frac{u_1-v_1+1}{|u_1-v_1+1|}\times (-1) \\
            &= 1 + \frac{3}{4}\sgn{(4u_1-v_1-2)} - \sgn{(u_1-v_1+1)}\neq 0,
            \end{align*}
            artinya $f(u_1,v_1)$ tidak mempunyai titik pelana. Walaupun begitu, $f(u_1,v_1)$ mempunyai titik kritis, yaitu saat $|4u_1-v_1-2|=0$ atau $|u_1-v_1+1|=0$. Hal ini berarti nilai minimumnya berada pada titik kritis atau pada titik-titik batas domainnya. 
            \begin{enumerate}
                \item[\textbf{(a)}] \textbf{pada titik kritis}\\
                Jika $|4u_1-v_1-2|=0$, didapat $v_1=4u_1-2$ sehingga 
                \begin{align*}
                    f(u_1,v_1) = 4u_1-3-3|u_1-1|.
                \end{align*}
                Karena $u_1\geq 1$ didapat 
                \begin{align*}
                    f(u_1,v_1)=4u_1-3-3u_1+3=u_1\geq 1 >0.
                \end{align*}
                Kemudian, jika $|u_1-v_1+1|=0$, didapat $v_1=u_1+1$ sehingga
                \begin{align*}
                    f(u_1,v_1)=u_1+\frac{9}{4}|u_1-1|\geq u_1\geq 1>0.
                \end{align*}
                \item[\textbf{(b)}] \textbf{pada titik batas domain}\\
                \begin{enumerate}
                    \item Jika $u_1=1$, didapatkan 
                    \begin{align*}
                        f(u_1,v_1)\geq f(1,v_1)=v_1-1+\frac{3}{4}|2-v_1|-|2-v_1|.
                    \end{align*}
                    Karena $v_1\geq 3$, diperoleh 
                    \begin{align*}
                        f(u_1,v_1) &\geq v_1-1-\frac{1}{4}(v_1-2)= \frac{3}{4}v_1-\frac{1}{3}\geq \frac{7}{4}>0.
                    \end{align*}
                    \item Jika $u_1\to 3^{-}$, didapatkan 
                    \begin{align*}
                        f(u_1,v_1)\geq \lim_{u_1\to 3^-} f(u_1,v_1)=v_1-1+\frac{3}{4}|10-v_1|-|4-v_1|.
                    \end{align*}
                    \begin{enumerate}
                        \item Jika $v_1\in[4,5]$, diperoleh 
                        \begin{align*}
                            f(u_1,v_1)&\geq v_1-1+\frac{3}{4}(10-v_1)-(v_1-4)\\
                            &= \frac{21}{2}-\frac{3}{4}v_1 \\
                            &\geq \frac{21}{2}-\frac{15}{4}\\
                            &=\frac{27}{4}\\
                            &>0.
                        \end{align*}
                        \item Jika $v_1\in [3,4]$, didapatkan 
                        \begin{align*}
                            f(u_1,v_1)&\geq v_1-1+\frac{3}{4}(10-v_1)-(4-v_1)=\frac{5}{4}v_1+\frac{5}{2}>0.
                        \end{align*}
                    \end{enumerate}
                    \item Jika $v_1=3$, didapatkan 
                    \begin{align*}
                        f(u_1,v_1)\geq f(u_1,3) = 2+\frac{3}{4}|4u_1-5|-|u_1-2|.
                    \end{align*}
                    Karena $u_1\in [1,3)$, maka nilai maksimum dari $|u_1-2|$ adalah 1 sehingga 
                    \begin{align*}
                        f(u_1,v_1)\geq 2+\frac{3}{4}|4u_1-5|-1\geq 1>0.
                    \end{align*}
                    \item Jika $v_1=5$, didapatkan 
                    \begin{align*}
                        f(u_1,v_1)\geq f(u_1,5)=4+\frac{3}{4}|4u_1-7|-|u_1-4|.
                    \end{align*}
                    Karena $u_1\in[1,3)$, maka 
                    \begin{align*}
                        f(u_1,v_1)&\geq 4+\frac{3}{4}|4u_1-7|-(4-u_1)\\
                        &=\frac{3}{4}|4u_1-7|+u_1\\
                        &\geq u_1\\
                        &>0.
                    \end{align*}
                \end{enumerate}
                Dari uraian tersebut didapatkan bahwa $f(u_1,v_1)\geq 0$ untuk setiap $u_1\in[1,3)$ dan $v_1\in [3,5]$ sehingga berlaku pula $\alpha d(u,v)+\beta d(u,Tu)+\gamma d(u,Tv)\geq d(Tu,Tv)$.
            \end{enumerate}
        \end{enumerate}
        Karena semua tinjauan kasus di atas menghasilkan $\alpha d(u,v)+\beta d(u,Tu)+\gamma d(u,Tv)\geq d(Tu,Tv)$ untuk setiap $u_1,v_1\in[1,5]$, maka $T$ adalah pemetaan $(\alpha,\beta,\gamma)$-nonekspansif dengan $\alpha=\frac{1}{4},~\beta=\frac{1}{3},$ dan $\gamma=\frac{3}{4}$. Akan tetapi, $T$ bukan pemetaan nonekspansif karena untuk $u=\qty(\frac{29}{10},\frac{29^3}{1000},0,0,\dots)$ dan $v=(3,27,0,0,\dots)$, didapatkan 
            \begin{align*}
            d(Tu,Tv)=&~\Bigg(\mqty|\frac{\frac{29}{10}+3}{4}-\frac{3+2}{4}|^3\\
            &+\bigg|\bigg(\frac{\frac{29}{10}+3}{4}\bigg)^3-\frac{(\frac{29}{10}+3)^3}{64}-\frac{(3+2)^3}{64}+\bigg(\frac{3+2}{4}\bigg)^3\bigg|^3\Bigg)^{\frac{1}{3}}\\
                =&~\frac{9}{40}\\
                >&~ \frac{1}{10}\\                =&~\qty(\left|3-\frac{29}{10}\right|^3+\qty|\qty(\frac{29}{10})^3-\frac{29^3}{1000}-3^3+27|^3)^{\frac{1}{3}} \\
                =&~d(u,v).
            \end{align*}
    Selanjutnya, untuk mendapatkan titik tetap dari $T$, dicari $w=(w_1,w_2,0,0,\cdots)\in W$ sehingga $T(w)=w$. Dari definisi $T$, terdapat dua kemungkinan, yaitu
    \begin{align*}
        &\text{(i) } w_1=\frac{w_1+3}{4}, ~w_2=\frac{(w_1+3)^3}{64}, \quad \text{atau}\\
        &\text{(ii) } w_1=\frac{w_1+2}{4}, ~w_2=\frac{(w_1+2)^3}{64}.
    \end{align*}
    Dari (i) diperoleh $w_1=w_2=1\in [1,3)$, sedangkan dari (ii) diperoleh $w_1=\frac{2}{3}\notin [3,5]$. Dengan demikian, titik tetap dari $T$ adalah $(1,1,0,0,\cdots)$.
    