\lampiran{Kode Python untuk Simulasi Rekonstruksi Gambar}
\begin{lstlisting}[caption={Kode Rekonstruksi Gambar}]
import numpy as np
import matplotlib.pyplot as plt
import pandas as pd
import os
from google.colab import files 
from skimage import io, color, transform
from skimage.data import shepp_logan_phantom, grass, moon, coins, text, brain
from skimage.transform import radon, iradon
from skimage.metrics import peak_signal_noise_ratio, mean_squared_error

image_size = 512
theta = np.linspace(0., 180., 180, endpoint=False)
n_iterations = 60
alpha_n = 0.7
beta_n = 0.5
gamma_n = 0.4
checkpoints = [5, 10, 30, 60]

def load_and_prep_image(uploaded_filename, target_size=512):
    print(f"-> Memproses gambar: {uploaded_filename}")
    image = io.imread(uploaded_filename)

    # Jika gambar berwarna (RGB/RGBA), ubah ke Grayscale
    if image.ndim == 3:
        if image.shape[2] == 4: # RGBA
            image = color.rgba2rgb(image)
        image = color.rgb2gray(image)

    # Resize ke target_size (128x128)
    image_resized = transform.resize(image, (target_size, target_size), anti_aliasing=True)
    return image_resized

def gambar():
    print("=== UPLOAD GAMBAR ===")
    uploaded = files.upload()

    if not uploaded:
        print("Tidak ada file yang diupload. Menggunakan default Shepp-Logan Phantom.")
        original_image = shepp_logan_phantom()
        original_image = transform.resize(original_image, (image_size, image_size), anti_aliasing=True)
    else:
        filename = next(iter(uploaded))
        original_image = load_and_prep_image(filename, target_size=image_size)
    return original_image

original_image = gambar()

def gammaoptm():
    print("-> Menghitung step size optimal...")
    # Menggunakan random noise untuk estimasi operator
    x_test = np.random.rand(image_size, image_size)
    for _ in range(10):
        Ax = A(x_test)
        AtAx = A_adjoint(Ax)
        norm_val = np.linalg.norm(AtAx)
        x_test = AtAx / norm_val

    Lx = A_adjoint(A(x_test))
    L = np.linalg.norm(Lx) / np.linalg.norm(x_test)
    gamma_step = 1.9 / L
    #gamma_step = 0.001
    print(f"   Gamma Step: {gamma_step:.5f}")
    return gamma_step

def A(x):
        return radon(x, theta=theta, circle=False)

def A_adjoint(sinogram):
        return iradon(sinogram, theta=theta, circle=False, output_size=image_size, filter_name=None)

def P_C(x):
        return np.clip(x, 0, 1)

Q = A(original_image)
gamma_step = gammaoptm()
def T(x):
    Ax = A(x)
    residual = Ax - Q
    gradient = A_adjoint(residual)
    return P_C(x - gamma_step * gradient)

def sabriIter():
    x = np.zeros((image_size, image_size))

    saved_images = {}
    saved_metrics = {}
    data_mse = []
    data_psnr = []

    print(f"-> Memulai Rekonstruksi ({n_iterations} iterasi)...")

    for n in range(n_iterations):
        Tx = T(x)
        z = T((1 - gamma_n) * x + gamma_n * Tx)
        Tz = T(z)
        y = T(Tz)
        Ty = T(y)
        x_next = T((1 - alpha_n) * Tz + alpha_n * Ty)

        x = x_next

        current_psnr = peak_signal_noise_ratio(original_image, x, data_range=1)
        current_mse = mean_squared_error(original_image, x)
        data_mse.append(current_mse)
        data_psnr.append(current_psnr)

        current_iter = n + 1
        if current_iter in checkpoints:
            saved_images[current_iter] = x.copy()
            saved_metrics[current_iter] = (current_psnr, current_mse)
            print(f"   Iterasi {current_iter}:")
            print(f"     PSNR = {current_psnr:.2f} dB")
            print(f"     MSE  = {current_mse:.6f}")
    return saved_images, saved_metrics, data_mse, data_psnr

saved_images, saved_metrics, data_mse, data_psnr = sabriIter()

table = pd.DataFrame({
    'Iterasi n': np.arange(1, len(data_mse) + 1),
    'mse': data_mse,
    'psnr': data_psnr
})
print(table.to_latex(index=False))

def tampilgambar():
    fig, axes = plt.subplots(3, 2, figsize=(10,15))
    ax = axes.ravel()

    # Tampilkan Gambar Asli
    im0 = ax[0].imshow(original_image, cmap='gray', vmin=0, vmax=1)
    ax[0].set_title('Gambar Asli')

    # Tampilkan Sinogram
    ax[1].imshow(Q, cmap='gray', aspect='auto')
    ax[1].set_title('Sinogram')

    # Tampilkan Hasil Iterasi
    for i, iteration in enumerate(checkpoints):
        if iteration in saved_images:
            idx = i + 2
            if idx < 6:
                psnr_val, mse_val = saved_metrics[iteration]
                im = ax[idx].imshow(saved_images[iteration], cmap='gray', vmin=0, vmax=1)
                title_text = (f"Iterasi {iteration}\n"
                              f"PSNR: {psnr_val:.2f} dB\n"
                              f"MSE: {mse_val:.5f}")
                ax[idx].set_title(title_text)
                plt.colorbar(im, ax=ax[idx], fraction=0.046, pad=0.04)

    plt.tight_layout()
    plt.show()

tampilgambar()
\end{lstlisting}