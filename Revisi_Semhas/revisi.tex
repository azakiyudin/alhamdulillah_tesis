\documentclass[10pt,openany,a4paper]{article}
\usepackage{tikz, pgfplots, tkz-euclide,calc}
    \usetikzlibrary{patterns,snakes,shapes.arrows}
\usepackage{fancyhdr}
\usepackage{enumerate,hyperref,enumitem}
\usepackage{cancel}
\usepackage{varwidth}

% TAMBAHKAN PACKAGE SENDIRI KALAU KURANG

\usepackage{geometry}
\geometry{
	total = {210mm, 148.5mm},
	left = 30mm,
	right = 30mm,
	top = 30mm,
	bottom = 30mm,
}

% \fancyhead{}
% \fancyfoot{}
% \fancyhead[r]{}
% \fancyhead[l]{\fbox{\large{\textbf{@ahmadzakiyudin\_}}}}
\renewcommand{\headrulewidth}{0pt}
\renewcommand{\footrulewidth}{0pt}
\usepackage{graphicx}
\usepackage{subcaption}
\usepackage{soul}
\usepackage{hyperref}
\usepackage{float}
\usepackage{setspace}
\usepackage{afterpage}
\usepackage{xcolor}
\usepackage[nosectionbib]{apacite}
\usepackage{lmodern}
\usepackage{mathtools}
\usepackage{upgreek}
\usepackage{siunitx}
\usepackage{physics}
\usepackage{hyperref}
\renewcommand{\baselinestretch}{1.25}
\usepackage{multicol}
\usepackage{ragged2e,scalerel,amsmath,multicol,gensymb,setspace,
fancyhdr,amsfonts,tikz,pgfplots,nccmath,enumerate,verbatim,amsthm,physics}
\usepgfplotslibrary{polar,fillbetween}
\usepgflibrary{shapes.geometric}
\usetikzlibrary{calc,patterns,arrows}
\newcommand\mylog[1]{\mathop{{}^{#1}\mathrm{log}}}
\pgfplotsset{compat=1.15}
\pgfplotsset{my style/.append style={axis x line=middle, axis y line=
middle, xlabel={$x$}, ylabel={$y$}, axis equal }}
\usepackage{etoolbox}
\makeatother
\newtheorem{defn}{Definisi}[section]
\newtheorem{teo}[defn]{Teorema}
\newtheorem{thm}{Teorema}[section]
\newtheorem{lemma}{Lemma}
\newtheorem{lemmas}[defn]{Lemma}

\newtheorem{cor}[defn]{Akibat}
\newtheorem{asumsi}[defn]{Asumsi}
\theoremstyle{definition}
\newtheorem{con}{Contoh}[section]
\theoremstyle{definition}
\theoremstyle{plain}
\newtheorem{prop}{Proposisi}[section]
\newcommand{\longdiv}{\smash{\mkern-0.43mu\vstretch{1.31}{\hstretch{.7}{)}}\mkern-5.2mu\vstretch{1.31}{\hstretch{.7}{)}}}}

% \pagestyle{fancy}
% \fancyhf{}
% \lhead{Halaman \thepage}
% \rhead{Pembahasan Soal ETS 2023/2024 \\ (\href{https://instagram.com/ahmadzakiyudin_/}{@ahmadzakiyudin\_})}
\hypersetup{
    colorlinks=true,
    linkcolor=blue,
    filecolor=blue,      
    urlcolor=blue,
}
\newenvironment{jawab}[2][tes]{
\fbox{\begin{minipage}{14.65cm}
\textbf{Kata kunci materi:} #1.\\
\textbf{Pembahasan:}\\ #2
\end{minipage}}}{}

\newenvironment{jawab2}[2][tes]{
\fbox{\begin{minipage}{14.65cm}
 #2
\end{minipage}}}{}

\newenvironment{jawaban}[2][tes]{
\colorlet{oldcolor}{.}
        \color{red}
        \setlength{\fboxrule}{1pt}\boxed{\color{oldcolor}#2}\color{oldcolor}.
}{}

\newenvironment{jawaban2}[2][tes]{
\colorlet{oldcolor}{.}
        \color{red}
        \setlength{\fboxrule}{1pt}\fbox{\color{oldcolor}#2}\color{oldcolor}.
}{}

\newenvironment{penting}[2][tes]{
\colorlet{oldcolor}{.}
\color{blue}
\fbox{\color{oldcolor}#2}\color{oldcolor}
}{}
\renewcommand{\proofname}{Bukti}
\renewcommand{\thethm}{\arabic{chapter}.\arabic{thm}}

\newcommand{\normlp}[1]{\left\|#1\right\|_{L^p_{\lambda}(0,T;H)}}% Fungsi norm (||x||)
\newcommand{\R}{\mathbb{R}}
\renewcommand{\natural}{\mathbb{N}}
\newcommand{\Xpl}{X^R_{p,\lambda}}
\newcommand{\Xpls}{X^R_{p,\lambda^*}}
\newcommand{\qpl}{q^R_{p,\lambda}}
\newcommand{\qpls}{q^R_{p,\lambda^*}}

\begin{document}
\section*{Revisi Seminar Hasil}
Berikut adalah revisi dari beberapa masukan yang diperoleh pada seminar hasil:
\begin{itemize}
    \item Dr. Mahmud Yunus, S. Si, M. Si:
    \begin{enumerate}
        \item Perbaikan penulisan di seluruh buku tesis.\\
        \textbf{Revisi:} Sudah diperbaiki di seluruh dokumen tesis, utamanya pada definisi ruang metrik geodesik yaitu Definisi 2.2.1.
        \item Perlu dipertimbangkan untuk menyesuaikan judul dan pembahasan.\\
        \textbf{Revisi:} Judul dan pembahasan sudah disesuaikan dengan masukan tersebut.
        \item Skema iterasi ini skemanya seperti apa?\\
        \textbf{Revisi:} Ditambahkan \textit{flowchart} yang menunjukkan skema iterasi Sabri, yaitu pada Gambar 4.1.
        % \item Apa jaminan kode yang dibuat sesuai dengan teori yang ada?\\
        % \textbf{Revisi:} Ditambahkan \textit{flowchart} yang menunjukkan alur kerja kode, yaitu pada Gambar 4.1.
    \end{enumerate}
    \item Dr. Sunarsini, S. Si, M. Si:
    \begin{enumerate}
        \item Jarak spasi pada abstrak bahasa Inggris.\\
        \textbf{Revisi:} Sudah diperbaiki pada halaman v. 
        \begin{figure}[H]
    \centering
    \includegraphics[width=0.85\textwidth]{1_abstrak.jpg}
        \end{figure}
        

        \item Penjelasan simbol metrik pada daftar notasi.\\
        \textbf{Revisi:} Sudah diperbaiki pada halaman xv.
        \begin{figure}[H]
    \centering
    \includegraphics[width=0.85\textwidth]{2_metrik.jpg}
        \end{figure}
        

        \item Konstanta $C$ pada bagian latar belakang.\\
        \textbf{Revisi:} Diganti menjadi $c$ pada pada halaman 1.
        \begin{figure}[H]
    \centering
    \includegraphics[width=0.85\textwidth]{3_konstanc.jpg}
        \end{figure}
        
\newpage
        \item Penamaan subbab 2.2 menjadi Ruang Metrik Geodesik.\\
        \textbf{Revisi:} Diganti menjadi "Ruang Metrik Geodesik" pada halaman 7.
        \begin{figure}[H]
    \centering
    \includegraphics[width=0.85\textwidth]{4_metrik_geodesik.jpg}
        \end{figure}
        

        \item Contoh ruang metrik geodesik dan bukan ruang metrik geodesik.\\
        \textbf{Revisi:} Ditambahkan contoh dan penjelasannya pada halaman 8, yaitu Contoh 2.2.1 dan Contoh 2.2.2.
        \begin{figure}[H]
    \centering
    \includegraphics[width=0.85\textwidth]{5_contoh_geodesik.jpg}
        \end{figure}
        

        \item Contoh segitiga geodesik.\\
        \textbf{Revisi:} Ditambahkan contoh dan penjelasannya pada halaman 9, yaitu Contoh 2.2.3.
        \begin{figure}[H]
    \centering
    \includegraphics[width=0.85\textwidth]{6_contoh_segitiga_geodesik.jpg}
        \end{figure}
        

        \item Contoh titik pada segitiga geodesik.\\
        \textbf{Revisi:} Ditambahkan contoh dan penjelasannya pada halaman 9, yaitu Contoh 2.2.3.
        \begin{figure}[H]
    \centering
    \includegraphics[width=0.85\textwidth]{6.1_contoh_segitiga_geodesik.jpg}
        \end{figure}
        
\newpage
        \item Penulisan deskripsi ruang $CAT_p(0)$.\\
        \textbf{Revisi:} Sudah diperbaiki pada halaman 9.
        \begin{figure}[H]
    \centering
    \includegraphics[width=0.85\textwidth]{7_ruang_catp.jpg}
        \end{figure}
        

        \item Contoh segitiga komparasi.\\
        \textbf{Revisi:} Ditambahkan contoh dan penjelasannya pada halaman 10, yaitu Contoh 2.3.1.
        \begin{figure}[H]
    \centering
    \includegraphics[width=0.85\textwidth]{8_contoh_segitiga_komparasi.jpg}
        \end{figure}
        

        \item Contoh ruang $CAT(0)$.\\
        \textbf{Revisi:} Ditambahkan contoh dan penjelasannya pada halaman 10, yaitu Contoh 2.3.2 dan Contoh 2.3.3.
        \begin{figure}[H]
    \centering
    \includegraphics[width=0.85\textwidth]{9_contoh_ruang_catp.jpg}
        \end{figure}
        

        \item Penjelasan ruang $\ell_p$ adalah ruang geodesik.\\
        \textbf{Revisi:} Ditambahkan penjelasannya pada halaman 11 Ditambahkan pula contoh ruang $CAT_p(0)$ yang bukan ruang Banach dan bukan ruang $CAT(0)$ pada Contoh 2.3.6.
        \begin{figure}[H]
    \centering
    \includegraphics[width=0.85\textwidth]{10_lp_geodesik.jpg}
        \end{figure}
        
\newpage
        \item Contoh barisan konvergen di ruang $CAT_p(0)$.\\
        \textbf{Revisi:} Ditambahkan contoh barisan konvergen-$\Delta$, yaitu Contoh 2.3.10 pada halaman 16, dan tidak konvergen-$\Delta$, yaitu Contoh 2.3.11 pada halaman 17, serta contoh konvergen kuat pada Contoh 2.3.12 halaman 18.
        \begin{figure}[H]
    \centering
    \includegraphics[width=0.85\textwidth]{11_contoh_konvergen.jpg}
        \end{figure}
        

        \item Contoh pemetaan yang memenuhi sifat demiclosedness.\\
        \textbf{Revisi:} Ditambahkan contoh dan penjelasannya pada halaman 18, yaitu Contoh 2.3.13.
        \begin{figure}[H]
    \centering
    \includegraphics[width=0.85\textwidth]{12_contoh_demiclosedness.jpg}
        \end{figure}
        
        \item Penulisan notasi pangkat metrik.\\
        \textbf{Revisi:} Ditambahkan penjelasan pada halaman 19. 
        \begin{figure}[H]
            \centering
            \includegraphics[width=0.85\textwidth]{13_penulisan_metrik.jpg}
        \end{figure}

 \item Definisi pemetaan $\alpha$-nonekspansif.\\
        \textbf{Revisi:} Perbaikan Definisi 2.4.5 pada halaman 23.
        \begin{figure}[H]
    \centering
    \includegraphics[width=0.85\textwidth]{16_definisi_alpha.jpg}
        \end{figure}
\newpage
        \item Contoh fungsi yang memenuhi kondisi konveks secara geodesik serta proper lower semicontinuous.\\
        \textbf{Revisi:} Definisi dipindahkan ke bab 4.4 dan ditambahkan contoh serta penjelasannya pada Contoh 4.4.2 halaman 57.
        \begin{figure}[H]
    \centering
    \includegraphics[width=0.85\textwidth]{14_contoh_fungsi.jpg}
    \end{figure}
        \begin{figure}[H]
    \centering
    \includegraphics[width=0.85\textwidth]{14.1_definisi_fungsi.jpg}
        \end{figure}
        \begin{figure}[H]
    \centering
    \includegraphics[width=0.85\textwidth]{23_definisi_konveks.jpg}
        \end{figure}
        

        \item Contoh pemetaan nonekspansif dan perumumannya.\\
        \textbf{Revisi:} Ditambahkan contoh dan penjelasannya, yaitu Contoh 2.4.1-2.4.7 pada halaman 20-27.
        \begin{figure}[H]
    \centering
    \includegraphics[width=0.85\textwidth]{15_contoh_pemetaan.jpg}
        \end{figure}
        

       
        \newpage

    \item Kalimat pengantar pemetaan $(\alpha,\beta,\gamma)$-nonekspansif.\\
    \textbf{Revisi:} Perbaikan kalimat pengantar pada halaman 26.
        \begin{figure}[H]
    \centering
    \includegraphics[width=0.85\textwidth]{17_bahasa_pemetaan_abc.jpg}
        \end{figure}
        

        \item Penulisan bagian aproksimasi titik tetap pemetaan nonekspansif.\\
        \textbf{Revisi:} Sudah diperbaiki pada halaman 27-29.
        \begin{figure}[H]
    \centering
    \includegraphics[width=0.85\textwidth]{18_titik_picard.jpg}
        \end{figure}
        \begin{figure}[H]
    \centering 
    \includegraphics[width=0.85\textwidth]{19_susunan_paragraf.jpg}
        \end{figure}
        

        \item Penjelasan titik tetap dari pemetaan yang digunakan.\\
        \textbf{Revisi:} Ditambahkan penjelasannya pada Contoh 4.1.1 halaman 36-42.
        \begin{figure}[H]
    \centering
    \includegraphics[width=0.85\textwidth]{20_titap_pemetaan.jpg}
        \end{figure}
        

        \item Contoh operator resolvent.\\
        \textbf{Revisi:} Ditambahkan contoh dan penjelasannya pada halaman 55-56, yaitu Contoh 4.4.1.

        \begin{figure}[H]
    \centering
    \includegraphics[width=0.85\textwidth]{21_contoh_resolvent.jpg}
        \end{figure}

        \item Pemilihan titik pada matriks hessian.\\
        \textbf{Revisi:} Ditambahkan penjelasannya pada halaman 57.
        \begin{figure}[H]
    \centering
    \includegraphics[width=0.85\textwidth]{22_hessian.jpg}
        \end{figure}
        
        \item Penjelasan pemetaan $(\alpha,\beta,\gamma)$-nonekspansif di ruang Hilbert.\\
        \textbf{Revisi:} Ditambahkan penjelasannya dengan diagram venn pada Gambar 4.5 halaman 66.
        \begin{figure}[H]
    \centering
    \includegraphics[width=0.85\textwidth]{24_pemetaan_hilbert.jpg}
        \end{figure}
        

        \item Poin 2 di bagian kesimpulan.\\
        \textbf{Revisi:} Bagian yang ditandai dihapus.
        \begin{figure}[H]
    \centering
    \includegraphics[width=0.85\textwidth]{25_kesimpulan_p2.jpg}
        \end{figure}
        

        \item Poin 3 di bagian kesimpulan.\\
         \textbf{Revisi:} Sudah diperjelas pada halaman 74.
        \begin{figure}[H]
    \centering
    \includegraphics[width=0.85\textwidth]{26_kesimpulan_p3.jpg}
        \end{figure}
       

    \end{enumerate}
    \newpage 
    \item Dr. Imam Mukhlash, S. Si, MT:
    \begin{enumerate}
        \item Mengapa Sabri lebih cepat konvergen, untuk semua kondisi?\\
        \textbf{Revisi:} Ditambahkan penjelasannya pada halaman 55.
        \item Geodesik $\to$ tomografi?\\
        \textbf{Revisi:} Ditambahkan Gambar 4.4 dan 4.5 yang menjelaskan alur geodesik dalam rekonstruksi citra tomografi, yaitu pada halaman 65 dan 66.
        \item Metrik rekonstruksi?\\
        \textbf{Revisi:} Sudah ada pada halaman 69-71.
        \item Flowchart/algoritma Sabri?\\
        \textbf{Revisi:} Ditambahkan flowchart pada halaman 44, yaitu Gambar 4.1.
    \end{enumerate}
\end{itemize}

\end{document}
