\documentclass[a4paper, 12pt]{article}

\usepackage{geometry}
\usepackage{amsfonts}
\usepackage{tabularx}
\usepackage{fancyhdr}
\usepackage{amssymb}
\usepackage{amsmath}
\usepackage{multicol}
\usepackage{graphicx}
\usepackage{array}
\usepackage{lastpage}
\geometry{a4paper, portrait, top=3cm, left=3cm, right=3cm, bottom=3cm}
\setlength{\multicolsep}{5.0pt plus 2.0pt minus 1.5pt}% 50% of original values
\renewcommand{\baselinestretch}{1}
\newcommand{\kali}{\otimes}
\newcommand{\plus}{\oplus}
\newcommand{\homotopic}{\simeq}
\newcommand{\homeq}{\cong}
\newcommand{\iso}{\approx}
\DeclareMathOperator{\ho}{Ho}
\DeclareMathOperator*{\colim}{colim}
\newcommand{\Real}{\mathbb{R}}

\newcommand{\dis}{\displaystyle}
\newcommand*{\defeq}{\stackrel{\text{def}}{=}}
\newcommand{\C}{\mathbb{C}}
\newcommand{\Q}{\mathbb{Q}}
\newcommand{\Z}{\mathbb{Z}}
\newcommand{\N}{\mathbb{N}}
\newcommand{\PP}{\mathrm{P}}
\newcommand{\QQ}{\mathrm{Q}}
\newcommand{\KQ}{\mathrm{H}}
\newcommand{\M}{\mathcal{M}}
\newcommand{\W}{\mathcal{W}}
\newcommand{\Riemann}{\mathcal{R}}
\newcommand{\epsi}{\varepsilon}
\newcommand{\nor}{\parallel}
\newcommand{\deltaepsilon}{\delta_\varepsilon}
\newcommand{\pdot}{\dot{\mathcal{P}}}
\newcommand{\p}{\mathcal{P}}
\newcommand{\itilde}{\tilde{\imath}}
\newcommand{\jtilde}{\tilde{\jmath}}
\newcommand{\ihat}{\hat{\imath}}
\newcommand{\jhat}{\hat{\jmath}}
%

\newcounter{choice}
\renewcommand\thechoice{\alph{choice}}
\newcommand\choicelabel{\thechoice.}

\newenvironment{choices}%
{\list{\choicelabel}%
	{\usecounter{choice}\def\makelabel##1{\hspace{0.3cm}\llap{##1}}%
		\settowidth{\leftmargin}{\hskip\labelsep\hskip 0em}%
		\def\choice{%
			\item
		} % choice
		\labelwidth\leftmargin\advance\labelwidth-\labelsep
		\topsep=0pt
		\partopsep=0pt
	}%
}%
{\endlist}

\newenvironment{oneparchoices}%
{%
	\setcounter{choice}{0}%
	\def\choice{%
		\refstepcounter{choice}%
		\ifnum\value{choice}>1\relax
		\penalty -50\hskip 2cm plus 1em
		\fi
		\choicelabel
		\nobreak\enskip
	}% choice
	% If we're continuing the paragraph containing the question,
	% then leave a bit of space before the first choice:
	\ifvmode\else\enskip\fi
	\ignorespaces
}%
{}

\begin{document}
 \sffamily
\begin{center}\large{
\textbf{Coretan bab 4}}
\end{center}
Matriks operator koin perjalanan kuantum tak beraturan 
$$
\mathrm{H}_n=\left[\begin{array}{ll}
    a_n &b_n  \\
     c_n&d_n 
\end{array}\right]
$$
dimana $a_n,b_n,c_n,d_n\in \Real_{\max} \setminus \{\varepsilon\}$. Didefinisikan matriks $\mathrm{P}_n$ dan  $\mathrm{Q}_n$ sedemikian hingga $\mathrm{H}_n=\mathrm{P}_n\plus \mathrm{Q}_n$.
$$
\mathrm{P}_n=\left[\begin{array}{cc}
    a_n &b_n  \\
     \varepsilon&\varepsilon 
\end{array}\right]~~\text{ dan }~~\mathrm{Q}_n=\left[\begin{array}{cc}
    \varepsilon &\varepsilon  \\
     c_n&d_n 
\end{array}\right]
$$
Model evolusi waktu total perjalanan kuantum tak beratruan $$
\varphi(n,k)=\left(P_n\kali \varphi(n-1,k+1)\right)\plus\left(Q_n\kali \varphi(n-1,k-1)\right)
$$
dimana inisial state
$$
\varepsilon(0,k)=\begin{cases}
\left[\begin{array}{l}
     \alpha\\\beta
\end{array}\right],~~\text{ untuk  }k=0\\\\\left[\begin{array}{l}
     \varepsilon\\ \varepsilon
\end{array}\right],~~\text{ untuk  }k\neq0
\end{cases}
$$
dengan $\alpha,\beta$ tidak semuanya $\varepsilon$.\newline
Didefinisikan $A(n,k)$ disebut dengan matriks \textit{decision} yang memenuhi 
$$
\varphi(n,k)=\left(P_n\kali \varphi(n-1,k+1)\right)\plus\left(Q_n\kali \varphi(n-1,k-1)\right)=A(n,k)\varphi(0,0)
$$
Sebagai contoh untuk $n=4,k=2$\
$$
\begin{array}{lll}
\lefteqn{\varphi(4,2)}\\&=&\left(P_4\kali \varphi(3,3)\right)\plus\left(Q_4\kali \varphi(3,1)\right)\\
&=&
\left(P_4\kali[\left(P_3\kali \varphi(2,4)\right)\plus\left(Q_3\kali \varphi(2,2)\right)]\right)\plus\left(Q_4\kali [\left(P_3\kali \varphi(2,2)\right)\plus\left(Q_3\kali \varphi(2,0)\right)]\right)\\
&=&\left(P_4\kali[\left(Q_3\kali \varphi(2,2)\right)]\right)\plus\left(Q_4\kali [\left(P_3\kali \varphi(2,2)\right)\plus\left(Q_3\kali \varphi(2,0)\right)]\right)\\
&=&\left(P_4\kali[\left(Q_3\kali [\left(P_2\kali \varphi(1,3)\right)\plus\left(Q_2\kali \varphi(1,1)\right)]\right)]\right)\\
&&\plus\left(Q_4\kali [\left(P_3\kali [\left(P_2\kali \varphi(1,3)\right)\plus\left(Q_2\kali \varphi(1,1)\right)]\right)\right.\\
&&\plus\left.\left(Q_3\kali [\left(P_2\kali \varphi(1,1)\right)\plus\left(Q_2\kali \varphi(1,-1)\right)]\right)]\right)\\
&=&\left(P_4\kali[\left(Q_3\kali \varphi(2,2)\right)]\right)\plus\left(Q_4\kali [\left(P_3\kali \varphi(2,2)\right)\plus\left(Q_3\kali \varphi(2,0)\right)]\right)\\
&=&\left(P_4\kali[\left(Q_3\kali [\left(Q_2\kali [\left(P_1\kali \varphi(0,2)\right)\plus\left(Q_1\kali \varphi(0,0)\right)]\right)]\right)]\right)\\
&&\plus\left(Q_4\kali [\left(P_3\kali [\left(Q_2\kali [\left(P_1\kali \varphi(0,2)\right)\plus\left(Q_1\kali \varphi(0,0)\right)]\right)]\right)\right.\\
&&\plus\left.\left(Q_3\kali [\left(P_2\kali [\left(P_1\kali \varphi(0,2)\right)\plus\left(Q_1\kali \varphi(0,0)\right)]\right)\right.\right.\\&&\left.\left.\plus\left(Q_2\kali  [\left(P_1\kali \varphi(0,0)\right)\plus\left(Q_1\kali \varphi(0,-2)\right)]\right)]\right)]\right)\\
&=&\left(P_4\kali[\left(Q_3\kali [\left(Q_2\kali [\left(Q_1\kali \varphi(0,0)\right)]\right)]\right)]\right)\plus\left(Q_4\kali [\left(P_3\kali [\left(Q_2\kali [\left(Q_1\kali \varphi(0,0)\right)]\right)]\right)\right.\\
&&\plus\left.\left(Q_3\kali [\left(P_2\kali [\left(Q_1\kali \varphi(0,0)\right)]\right)\plus\left(Q_2\kali  [\left(P_1\kali \varphi(0,0)\right)]\right)]\right)]\right)\\
&=&\left[(P_4\kali Q_3\kali Q_2\kali Q_1)\plus(Q_4\kali P_3\kali Q_2\kali Q_1)\right.\\&&\left.\plus(Q_4\kali Q_3\kali P_2\kali Q_1)\plus(Q_4\kali Q_3\kali Q_2\kali P_1)\right]\varphi(0,0)
\end{array}
$$
Sehingga diperoleh 
$$
A(4,2)=(P_4\kali Q_3\kali Q_2\kali Q_1)\plus(Q_4\kali P_3\kali Q_2\kali Q_1)\plus(Q_4\kali Q_3\kali P_2\kali Q_1)\plus(Q_4\kali Q_3\kali Q_2\kali P_1)
$$
\newpage
Diberikan gambar evolusi waktu total perjalanan kuantum aljabar max-plus tak beraturan
\begin{figure}[h] 
	\begin{center}
	\includegraphics[scale=0.45]{evolusi.png}
		\caption{Evolusi Waktu total perjalanan kuantum waktu diskrit tak beraturan}
		\vspace{-20pt}
			\label{g2}
	\end{center}
\end{figure}
\end{document}
