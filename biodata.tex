\Biodata{foto}
Lahir di Madiun, 15 Maret 2003, penulis memiliki nama lengkap Ahmad Hisbu Zakiyudin. Sebagian memanggil dengan Zaki, sebagian lagi dengan Hisbu, Bubu, Udin, atau Ahmad. Enam tahun penulis menempuh pendidikan dasar di MI Nurul Huda Sawahan, dua tahun menempuh pendidikan menengah pertama di MTsN Kota Madiun, dan tiga tahun menempuh pendidikan menengah atas di MAN 2 Kota Madiun. Kemudian pada saat pandemi mulai Covid-19 mulai menyebar, yaitu pada tahun 2020, penulis melanjutkan pendidikan tinggi di Departemen Matematika, Fakultas Sains dan Analitika Data (FSAD), Institut Teknologi Sepuluh Nopember (ITS) Surabaya. Penulis menyelesaikan program Sarjana pada Departemen Matematika ITS pada tahun 2020, dengan judul tugas akhir "Aplikasi Teorema Titik Tetap pada Ruang Metrik Kuasi untuk Masalah Eksistensi dan Ketunggalan Solusi dari Persamaan Cauchy Tidak Homogen". Hasil dari tugas akhir tersebut juga dipublikasikan dalam Jurnal Ilmiah terindeks Scopus dengan peringkat Q3, yaitu pada Jurnal Nonlinear Dynamics \& Systems Theory yang terbit pada Edisi ke-5 Tahun 2025. Penulis kemudian melanjutkan pendidikan pascasarjana pada program Magister by Riset di Departemen Matematika ITS. Selain itu, penulis juga sempat mengikuti dua konferensi ilmiah internasional sebagai \textit{presenter}, yaitu pada The International Conference in Mathematical Analysis and Applications (ICONMAA) di tahun 2024 dan International Conference on Mathematics: Pure, Applied, and Computation (ICoMPAC) di tahun 2025. 

Adapun kritik dan saran atau informasi lebih lanjut mengenai Tesis ini dapat ditujukan ke penulis melalui email penulis, yakni \href{mailto:azakiyudin1790@gmail.com}{azakiyudin1790@gmail.com} atau LinkedIn penulis, yaitu Ahmad Hisbu Zakiyudin atau bisa diakses melalui \url{www.linkedin.com/in/ahmadhisbuzakiyudin}.