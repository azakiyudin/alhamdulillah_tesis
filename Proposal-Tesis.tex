\documentclass{ProposalTesis-ITSpdf}
\usepackage{enumerate}
\usepackage[table]{xcolor}
\usepackage{ragged2e}
\usepackage{array}
% \newcolumntype{L}[1]{>{\raggedright\let\newline\\\arraybackslash\hspace{0pt}}m{#1}}
% \newcolumntype{C}[1]{>{\centering\let\newline\\\arraybackslash\hspace{0pt}}m{#1}}
% \newcolumntype{R}[1]{>{\raggedleft\let\newline\\\arraybackslash\hspace{0pt}}m{#1}}
\DeclareMathOperator{\sgn}{sgn}
\newcommand{\R}{\mathbb{R}}
\usepackage{setspace}
\usepackage{pdfpages}
\usepackage{physics}
\usepackage{float}
\usepackage{tikz}
\usetikzlibrary{shapes.geometric,arrows}
\usepackage{todonotes}
\usepackage{bm}
\usepackage{siunitx}

\RequirePackage{mathptmx,amsthm,enumitem}
\usepackage{palatino,url}
\usepackage{longtable}
\RequirePackage{euscript} %
\onehalfspacing
\definecolor{mygreen}{RGB}{0,154,78}
\definecolor{blue}{RGB}{0,103,172}
\definecolor{green}{RGB}{0,154,78}
\newtheorem{defn}{Definisi}[section]
\tolerance=1
\emergencystretch=\maxdimen
\hyphenpenalty=10000
\hbadness=10000
\usepackage{cleveref}
\usetikzlibrary{calc}
\usepackage{listings}
\usepackage{xcolor}

\lstset{
  language=Python,
  basicstyle=\ttfamily\small,
  numbers=left,
  numberstyle=\tiny,
  stepnumber=1,
  numbersep=5pt,
  frame=single,
  breaklines=true,
  captionpos=b,
  showstringspaces=false,
  keywordstyle=\color{blue},
  commentstyle=\color{gray},
  stringstyle=\color{red},
}
\renewcommand{\lstlistingname}{Lampiran}
\begin{document}


%\Judul{Aproksimasi Iterasi Sabri}

\JudulEng{Approximation}

\Nama{Joko Cahyono}

\NRP{1214 201 007}

\Jurusan{Matematika}

\Department{Mathematics}

\BidangStudi{Pemodelan Matematika dan Komputasi}

\Tahun{2015}

\Fakultas{Matematika dan Ilmu Pengetahuan Alam}{MIPA}

\Faculty{Mathematics and Natural Sciences}

\Pembimbing{Dr. Subiono, MS}
           {}

\NIPPembimbing{19570411 198403 1 001}
              {}

\Penguji{Dr. Dieky Adzkiya, S.Si, M.Si}
        {Dr. Imam Mukhlash, S.Si, MT}
        {Dr. Mahmud Yunus, M.Si}

\NIPPenguji{19830517 200812 1 003}
           {19700831 199403 1 003}
           {19620407 198703 1 005}
          
\TanggalDisetujui{28 Oktober 2015}


\Judul{Konvergensi Skema Iterasi Sabri untuk Kelas Pemetaan Nonekspansif Diperumum di Ruang Geodesik beserta Aplikasinya} %%Tuliskan Judul Yang sesuai (bhs.Indonesia)

\JudulEng{Convergence of Sabri's Iterative Scheme for The Generalized Class of Nonexpansive Mappings in Geodesic Spaces and Its Applications} % Tuiskan Judul Yang seuai (bhs.Inggris)

\Nama{Ahmad Hisbu Zakiyudin} % Tuliskan Nama penulis proposal tesis

\NRP{6002241016} % Tuliskan Nrp penulis proposal tesis

\Jurusan{Matematika} % Tuliskan Departemen penulis proposal Tesis

\Department{Mathematics} % Tuliskan Departement Penulis proposal tesis


\Tahun{2026} % Tuliskan tahun pembuatan proposal tesis

\Fakultas{Sains dan Analitika Data}{FSAD} % Tuliskan Fakultas pembuatan tesis

\Faculty{Science and Data analytics} % Tuliskan Fakultas pembuatan tesis (bhs. Inggris)

\Pembimbing{Dr.mont. Kistosil Fahim, S. Si., M. Si.} % Tuliskan nama pembimbing 1           % Tuliskan nama pembimbing2, bila hanya satu jangan diisi
{}             % Tuliskan nama pembimbing3, bila hanya dua jangan diisi

\NIPPembimbing{19910522 201504 1 001}    % Isi dengan NIP pembimbing
{}    % Isi dengan NIP pembimbing kedua bila ada
{}    % Isi dengan NIP pembimbing ketiga bila ada

\Penguji{Dr. Mahmud Yunus, M.Si}   % Isi nama penguji 1 yang sesuai
{Dr. Sunarsini, S.Si, M.Si}     % Isi nama penguji 2 yang sesuai
{Dr. Imam Mukhlash, S.Si, MT}         % Isi nama penguji 3 yang bila ada bila tidak kosongkan

\NIPPenguji{19620407 198703 1 005}  % Isi Nip penguji 1 yang sesuai
{19691004 199402 2 001}             % Isi Nip penguji 2 yang sesuai
{19700831 199403 1 003}             % Isi Nip penguji 3 yang sesuai bila ada bila tidak kosongkan
\Kadep{Dr. Didik Khusnul Arif, S.Si, M.Si}

\NIPKadep{19730930 199702 1 001}
\TanggalDisetujui{2 September 2025} %  Isi dengan tanggal seminar
\allowdisplaybreaks

\BagianAwal
\Cover
\LembarJudul
\TitlePage
\LembarPengesahan


\begin{Abstrak}
Teori titik tetap, yang diawali oleh Teorema Titik Tetap Banach untuk pemetaan kontraktif, telah berkembang secara signifikan untuk mencakup kelas pemetaan yang lebih luas seperti pemetaan nonekspansif dan generalisasinya. Berbagai skema iterasi, seperti Mann, Ishikawa, dan Agarwal, Thakur, JK dan yang lainnya, telah dikembangkan untuk aproksimasi titik tetap dari kelas-kelas pemetaan tersebut di berbagai ruang, baik itu di ruang linear seperti ruang Banach maupun di ruang tak linear seperti ruang geodesik. Baru-baru ini, Sabri dkk. mengusulkan skema iterasi baru dengan tingkat konvergensi yang lebih cepat dibanding skema iterasi lainnya pada ruang Banach konveks seragam. Pada penelitian ini, didapatkan syarat cukup untuk konvergensi-$\Delta$ dan kuat dari skema iterasi Sabri untuk aproksimasi titik tetap dari pemetaan $(\alpha,\beta,\gamma)$-nonekspansif di ruang metrik geodesik $CAT_p(0)$. Berdasarkan percobaan numerik, skema iterasi ini memiliki laju konvergensi yang lebih cepat dibandingkan skema iterasi JK, Thakur, Agarwal, dan Abbas. Selain itu, didapatkan pula aplikasi dari skema ini untuk permasalahan optimasi, tepatnya untuk minimalisasi fungsi dan rekonstruksi citra tomografi. 
\katakunci{aproksimasi titik tetap, pemetaan nonekspansif, ruang geodesik, skema iterasi, masalah optimasi}

\end{Abstrak}
 %%  menampilkan Abstrak Bhs.Indonesia

\begin{Abstract}
The Fixed Point theory, originating from Banach's Fixed Point Theorem for contractive mappings, has developed significantly to encompass a wider classes of mappings, such as nonexpansive mappings and their generalizations. Various iterative schemes, such as Mann, Ishikawa, and Agarwal, Thakur, JK, and others, have been developed for fixed point approximations of these classes of mappings in various spaces, both in linear spaces such as Banach spaces and in nonlinear spaces such as geodesic spaces. Recently, Sabri et al. proposed a new iterative scheme with faster convergence rates than previous iterative schemes in uniformly convex Banach spaces. In this study, we will analyze the convergence of Sabri's iterative scheme in geodesic spaces for approximating fixed points of $(\alpha,\beta,\gamma)$-nonexpansive mappings. In this study, we obtain the $\Delta$ and strong convergence results of Sabri's iteration scheme for the fixed point approximation of the $(\alpha,\beta,\gamma)$-non-expansive mapping in the geodesic metric space $CAT_p(0)$. Based on numerical experiments, this iteration scheme has a faster convergence rate than the JK, Thakur, Agarwal, and Abbas iteration schemes. In addition, we obtain applications of this scheme for optimization problems, specifically for function minimization and tomography image reconstruction.
	\keywords{fixed point approximation, nonexpansive mapping, geodesic space, iterative scheme, optimization problems}
\end{Abstract}
 %%  menampilkan Abstrak Bhs.Inggris
\KataPengantar
\setstretch{1.25}
 Puji syukur kehadirat Allah SWT karena atas berkah, rahmat, dan Ridho-Nya sehingga penulis dapat menyelesaikan tesis dengan judul:
\begin{center}
\textbf{"KONVERGENSI SKEMA ITERASI SABRI UNTUK KELAS\\
PEMETAAN NONEKSPANSIF DIPERUMUM DI RUANG GEODESIK\\
BESERTA APLIKASINYA"}
\end{center}
sebagai salah satu persyaratan akademis dalam menyelesaikan
Program Magister Departemen Matematika, Fakultas Sains dan Analitika Data, Institut Teknologi Sepuluh Nopember Surabaya. Tesis ini dapat diselesaikan dengan baik berkat kerja sama, bantuan, dan dukungan dari banyak pihak. Sehubungan dengan hal itu, penulis ingin mengucapkan terima kasih dan penghargaan kepada:
\begin{enumerate}[nosep]
	\item Institut Teknologi Sepuluh Nopember yang telah memberikan beasiswa Fresh Graduate pada penulis.
\item Bapak Nursalim, Ibu Kusnul, dan adik Nadhira, serta keluarga besar Imam Zuhdi yang telah memberikan dukungan, nasehat, dan
motivasi sehingga penulis dapat menyelesaikan tesis ini.

\item Bapak Dr.mont. Kistosil Fahim, S. Si., M. Si. selaku dosen pembimbing sekaligus dosen wali yang telah memberikan arahan dan bimbingan dengan penuh kesabaran kepada penulis.

\item Bapak Dr. Mahmud Yunus, M.Si, Ibu Dr. Sunarsini, S.Si, M.Si, dan Bapak Imam Mukhlash S.Si, M.T selaku dosen penguji yang telah memberikan arahan dalam tesis ini.


\item Seluruh keluarga besar Departemen Matematika Institut Teknologi Sepuluh Nopember atas dukungan dan bantuannya.


\end{enumerate}
% Penulis menyadari bahwa tugas akhir ini masih jauh dari
% kesempurnaan. Oleh karena itu, 
% Penulis mengharapkan saran dan kritik dari pembaca. 
Akhir kata, semoga Tesis ini bermanfaat bagi semua pihak yang berkepentingan.

\begin{flushright}
\begin{tabular}{r}
	 Surabaya, 27 Januari 2026 \vspace{1cm}\\\\
	Ahmad Hisbu Zakiyudin
\end{tabular}
\end{flushright}


\DaftarIsi %% menampilkan daftar isi

\DaftarGambar

\DaftarTabel %% menampilkan tabel

\DaftarNotasi
\begin{flushleft}
\begin{longtable}{lll}
\endfirsthead 
\endhead 
\endfoot 
\endlastfoot
$\R$ &:& Himpunan semua bilangan real\\
$\R^+$ &:& Himpunan semua bilangan real positif\\
$\N$ &:& Himpunan semua bilangan asli\\
$\R^n$ &:& Himpunan semua vektor kolom dengan $n$ komponen real\\
$\mathbf{0}$ &:& Vektor yang semua anggotanya bernilai 0\\
$d(x,y)$ &:& Metrik antara titik $x$ dan $y$\\
$d^p(x,y)$ &:& Perkalian dari $d(x,y)$ dengan $d(x,y)$ sebanyak $p$ kali\\
$:=$ &:& Didefinisikan\\
$\neq$ &:& Tidak sama dengan\\
$\geq$ &:& Lebih besar atau sama dengan\\
$\leq$ &:& Lebih kecil atau sama dengan\\
$>$ &:& Lebih besar\\
$<$ &:& Lebih kecil\\
$\nless$ &:& Tidak kurang dari\\
$\subset$ &:& Himpunan bagian sejati\\
$\subseteq$ &:& Himpunan bagian tak sejati\\
$\cup$ &:& Operasi gabungan pada himpunan\\
$\cap$ &:& Operasi irisan pada himpunan\\
    $\emptyset$ &:& Himpunan kosong\\
    $\infty$ &:& Tak hingga\\
    $\|x\|$ &:& Norm dari $x$\\
    $\|x\|_p$ &:& Norm $p$ dari $x$\\
    $\%$ &:& Persentase\\
$|x|$ &:& Harga mutlak dari $x$\\
$\partial$ &:& Turunan parsial\\
$\displaystyle \lim_{n\rightarrow \infty} x_n$ &:& Limit barisan $(x_n)$\\
$\sup$ &:& Supremum (batas atas terkecil)\\
$\limsup$ &:& Limit superior\\
$Tx$ &:& Hasil pemetaan dari $x$ oleh $T$\\
$Fix(T)$ &:& Titik tetap dari pemetaan $T$\\
$\triangle$ &:& Segitiga \\
$\overline{x}$ &:& Titik komparasi\\
$\overline{\triangle}$ &:& Segitiga komparasi\\
$sgn(x)$ &:& Fungsi tanda dari $x$\\
$\oplus$ &:& Operasi konveks antara dua titik pada ruang geodesik (Oplus)\\
$J^f_{\lambda} (x)$ &:& Operator resolvent $\lambda$ dari $f$\\
$P_C$ &:& Proyeksi titik terdekat pada himpunan $C$\\
$A^*$ &:& Transformasi adjoin dari transformasi $A$\\
$\displaystyle argmin_{y\in X} f(y)$ &:& Nilai $y\in X$ sehingga $f(y)$ minimum\\ 
    
\end{longtable}
\end{flushleft}
\cleardoublepage



\BagianInti %%% menampilkan bagian inti proposal tesis yang berisi:

\chapter{PENDAHULUAN}
 Pada bab ini dijelaskan tentang latar belakang, rumusan masalah, batasan masalah, tujuan penelitian, dan manfaat penulisan Tesis. 
\section{Latar Belakang}
Teorema Titik Tetap Banach merupakan temuan awal yang mengilhami berkembangnya teori mengenai titik tetap dan berbagai macam aplikasinya di bidang matematika maupun sains terapan. Teorema ini menyatakan bahwa jika $T:W\to W $ adalah suatu pemetaan kontraktif pada suatu himpunan bagian tertutup $W$ dari ruang Banach $B$, yaitu terdapat konstanta $c\in [0,1)$ sehingga
\begin{equation}
    \|Tx-Ty\|\leq c\|x-y\|
\end{equation}
untuk setiap $x,y\in W$, maka pemetaan $T$ memiliki tepat satu titik tetap \cite{Banach1922}. Jika konstanta $c$ boleh bernilai 1, pemetaan tersebut dinamakan pemetaan nonekspansif. Pemetaan ini merupakan perluasan dari pemetaan kontraktif yang memainkan peranan penting dalam teori titik tetap, terutama dalam memastikan eksistensi serta konvergensi solusi dari berbagai permasalahan. Namun, berbeda dengan pemetaan kontraktif, pemetaan nonekspansif tidak selalu memiliki titik tetap yang tunggal. Selain itu, terdapat syarat cukup untuk keberadaan titik tetap pada kelas pemetaan ini, yaitu jika $W$ merupakan himpunan bagian tertutup dan terbatas dari ruang Banach konveks seragam \cite{Browder1965}. 

Dengan menggunakan teori titik tetap, eksistensi solusi dari suatu model matematika dapat dijamin dengan mendefinisikan pemetaan yang bersesuaian. Walaupun demikian, solusi tersebut tidak selalu dalam bentuk eksak, atau mungkin sulit untuk diperoleh secara analitik, sehingga diperlukan solusi numerik. Untuk mendapatkan solusi numeriknya, diperlukan suatu algoritma iterasi yang konvergen ke solusi dari model tersebut. 

Pada pemetaan kontraktif Banach, skema iterasi Picard merupakan skema iterasi paling sederhana yang dapat digunakan untuk mendapatkan solusi numerik dari suatu model matematika. Namun, skema ini tidak selalu konvergen untuk pemetaan nonekspansif \cite{Krasnoselski}. Pada tahun 1953, Mann memperkenalkan skema iterasinya yang konvergen untuk pemetaan nonekspansif, tetapi skemanya juga tidak selalu konvergen pada pemetaan pseudo-kontraktif \cite{mann1953}. Mengatasi hal tersebut, Ishikawa mengembangkan skema iterasi dua tahap untuk mendapatkan nilai aproksimasi titik tetap dari pemetaan pseudo-kontraktif \cite{Ishikawa1974}. Berbagai variasi dan pengembangan skema iterasi untuk pemetaan nonekspansif ini juga terus bermunculan, sebagaimana oleh \cite{Noor2000,agarwal,abbas,Thakur2016,Ahmad2021}. Tidak hanya itu, berbagai jenis pemetaan nonekspansif juga dikembangkan oleh berbagai peneliti, antara lain: pemetaan Suzuki nonekspansif \cite{Suzuki2008}, pemetaan $\alpha$-nonekspansif \cite{Aoyama2011}, pemetaan yang memenuhi kondisi $(C_\mu)$ \cite{GARCIAFALSET2011185}, pemetaan tipe Reich-Suzuki nonekspansif \cite{Pant2019}, serta pemetaan $(\alpha,\beta,\gamma)$-nonekspansif \cite{Ullah2023}. 



Konvergensi dari skema iterasi terhadap titik tetap dari berbagai macam jenis pemetaan juga diteliti di berbagai macam jenis ruang, baik itu ruang linear seperti ruang Hilbert dan ruang Banach, maupun di ruang geodesik. (lihat \cite{ArizaRuiz2014,Bridson1999,dehghan,Kirk2014,kumam}, dan referensi di dalamnya). Pada dasarnya, ruang geodesik adalah ruang yang memiliki suatu geodesik, atau lintasan terpendek di antara dua titik. Ruang ini memungkinkan struktur nonlinier sehingga model matematika nonlinier dapat dimodelkan dengan lebih akurat. Salah satu contoh dari ruang ini adalah ruang $CAT(0)$ yang lengkap atau dapat disebut juga ruang Hadamard. Pada tahun 2017, Khamsi dkk. mengembangkan ruang yang lebih umum dari $CAT(0)$ yaitu ruang $CAT_p(0)$ dan membuktikan bahwa pemetaan nonekspansif pada himpunan bagian tak kosong yang tertutup, terbatas, dan konveks, dari ruang $CAT_p(0)$ selalu memiliki titik tetap \cite{Khamsi2017}. Kemudian Salisu dkk. mendapatkan konvergensi dari skema iterasi JK untuk pemetaan nonekspansif yang memenuhi kondisi $(C_\mu)$ di ruang $CAT_p(0)$. Mereka juga mendapatkan aplikasinya untuk masalah optimasi. Kemudian mereka memberikan hasil eksperimen numeriknya \cite{Salisu2022}.

Baru-baru ini, Sabri dkk. memperkenalkan skema iterasi baru untuk aproksimasi titik tetap pada pemetaan tipe Reich-Suzuki nonekspansif di ruang Banach konveks seragam. Melalui eksperimen numerik, mereka mendapatkan tingkat konvergensi yang lebih cepat dibanding beberapa skema iterasi lainnya \cite{sabri2025}.
Pada penelitian ini, akan diselidiki syarat-syarat konvergensi dari skema iterasi Sabri di ruang $CAT_p(0)$, khususnya untuk pemetaan nonekspansif tipe $(\alpha,\beta,\gamma)$ yang diberikan oleh Ullah dkk. Kemudian akan dilakukan eksperimen numerik dengan skema iterasi tersebut serta didapatkan aplikasinya pada masalah optimasi. 



\section{Rumusan Masalah}
Dari latar belakang yang telah dipaparkan sebelumnya, maka konsep baru yang diangkat dalam perumusan masalah ini adalah sebagai berikut:
\begin{enumerate}
\item Bagaimana syarat perlu untuk konvergensi dari skema iterasi Sabri di ruang $CAT_p(0)$ dalam aproksimasi titik tetap dari pemetaan $(\alpha,\beta,\gamma)$-nonekspansif?
\item Bagaimana syarat cukup untuk konvergensi dari skema iterasi Sabri di ruang $CAT_p(0)$ dalam aproksimasi titik tetap dari pemetaan $(\alpha,\beta,\gamma)$-nonekspansif?
\item Bagaimana laju konvergensi dari skema iterasi Sabri untuk mendapatkan aproksimasi titik tetap dari pemetaan $(\alpha,\beta,\gamma)$-nonekspansif dibanding dengan skema iterasi lainnya secara numerik?
\item Bagaimana aplikasi dari skema iterasi Sabri untuk pemetaan $(\alpha,\beta,\gamma)$-nonekspansif di ruang $CAT_p(0)$ pada masalah optimasi?
\end{enumerate}

\section{Batasan Masalah}
Batasan masalah dalam penelitian ini adalah :
\begin{enumerate}
    \item Laju konvergensi yang dibandingkan adalah laju dari skema iterasi Sabri, JK, Thakur, dan Agarwal berdasarkan banyaknya iterasi yang diperlukan. 
    %\item Masalah optimasi yang dibahas adalah masalah kelayakan terpisah atau lebih umumnya dikenal sebagai \textit{split feasibility problem}.
     \item Terdapat dua masalah optimasi yang dibahas, yaitu
\begin{enumerate}
\item permasalahan minimalisasi fungsi, lebih tepatnya mencari titik $x\in X$ sehingga $f(x)\leq f(y)$ untuk setiap $y\in X$, dengan $X$ adalah suatu himpunan tak kosong dan $f:X\to \mathbb{R}$ adalah suatu fungsi yang memenuhi kondisi konveks \textit{proper lower semicontinuous}.
\item permasalahan \textit{split feasibility problem} yang kemudian diterapkan dalam rekonstruksi citra tomografi.
\end{enumerate}
\end{enumerate}
\section{Tujuan Penelitian}
Tujuan dari penelitian ini adalah :
\begin{enumerate}
   \item Mendapatkan syarat perlu untuk konvergensi dari skema iterasi Sabri di ruang $CAT_p(0)$ dalam aproksimasi titik tetap dari pemetaan $(\alpha,\beta,\gamma)$-nonekspansif.
   \item Mendapatkan syarat cukup untuk konvergensi dari skema iterasi Sabri di ruang $CAT_p(0)$ dalam aproksimasi titik tetap dari pemetaan $(\alpha,\beta,\gamma)$-nonekspansif.
\item Mengetahui laju konvergensi dari skema iterasi Sabri untuk mendapatkan aproksimasi titik tetap dari pemetaan $(\alpha, \beta,\gamma)$-nonekspansif dibanding dengan skema iterasi lainnya secara numerik.
\item Mendapatkan aplikasi dari skema iterasi Sabri untuk pemetaan $(\alpha,\beta,\gamma)$-nonekspansif di ruang $CAT_p(0)$ pada masalah optimasi.

\end{enumerate}
\section{Manfaat Penelitian}
Manfaat dari penelitian ini adalah :
\begin{enumerate}
    \item Sebagai kontribusi pada pengembangan teori titik tetap, khususnya pada analisis konvergensi skema iterasi untuk kelas pemetaan yang lebih umum dan ruang yang lebih umum, yaitu pemetaan $(\alpha,\beta,\gamma)$-nonekspansif di ruang $CAT_p(0)$. 
    \item Sebagai bahan kajian dan perbandingan mengenai laju konvergensi antara skema iterasi Sabri dengan skema iterasi lainnya.
    \item Sebagai alternatif penyelesaian masalah optimasi, khususnya untuk minimalisasi fungsi dan rekonstruksi citra. 
\end{enumerate}

 %%  menampilkan Bab I

\chapter{KAJIAN PUSTAKA DAN DASAR TEORI}
Pada bagian ini dipaparkan penelitian terdahulu yang berkaitan dengan penelitian ini. Demikian juga diberikan beberapa definisi, lemma, teorema
serta beberapa contoh terkait sebagai dasar teori yang menjadi referensi dalam penelitian.

\section{Penelitian Terdahulu}
Bermula dari teorema titik tetap Banach dengan skema iterasi Picard, penelitian mengenai titik tetap dan aproksimasinya terus berkembang hingga saat ini. Dari yang awalnya pemetaan kontraksi, kemudian dikembangkan menjadi pemetaan nonekspansif dan berbagai perumumannya. Tidak seperti pemetaan kontraktif yang titik tetapnya terjamin ada dan tunggal jika di ruang Banach lengkap, pemetaan nonekspansif mensyaratkan beberapa kondisi lain, yaitu titik tetapnya ada jika pemetaannya adalah pemetaan diri sendiri dari himpunan bagian tertutup dan terbatas dari ruang Banach konveks seragam. Walaupun sudah dijamin ada, titik tetapnya juga tidak selalu tunggal \cite{Browder1965}. 


Untuk mendapatkan titik tetap dari pemetaan nonekspansif yang sulit diperoleh secara langsung, diperlukan berbagai macam skema iterasi yang konvergen ke titik tetapnya dengan mendapatkan nilai aproksimasinya. Skema iterasi pertama yang konvergen untuk pemetaan nonekspansif diperoleh oleh Mann pada tahun 1953 \cite{mann1953}. Kemudian Ishikawa mengembangkannya menjadi skema iterasi dua tahap karena skema iterasi Mann gagal konvergen pada pemetaan pseudo-kontraktif \cite{Ishikawa1974}. Dari hasil yang diperoleh Mann dan Ishikawa, Noor kemudian mengembangkan skema iterasi untuk aproksimasi titik tetap menjadi skema iterasi tiga tahap \cite{Noor2000}. Selanjutnya, pada tahun 2007 Agarwal mengenalkan skema iterasinya dengan memodifikasi skema iterasi yang dikenalkan oleh Mann \cite{agarwal}. Pada tahun berikutnya, Suzuki memperkenalkan pemetaan nonekspansif yang baru dengan melemahkan kondisi pemetaannya. Pemetaan baru tersebut ia namakan sebagai pemetaan yang memenuhi kondisi $(C)$ \cite{Suzuki2008}. García-Falset dkk. kemudian juga mengembangkan pemetaan tersebut dengan memperumum kondisinya \cite{GARCIAFALSET2011185}. 

Penelitian lain terkait skema iterasi untuk pemetaan nonekspansif diperoleh Abbas dan Nazir. Mereka mengenalkan skema iterasi baru yang konvergen untuk pemetaan nonekspansif biasa dan mendapatkan aplikasinya untuk masalah minimalisasi dengan konstrain dan masalah \textit{feasibility} \cite{abbas}. Berikutnya, Thakur juga mengenalkan skema iterasi baru yang konvergen lebih cepat daripada skema iterasi sebelum-sebelumnya untuk pemetaan nonekspansif Suzuki di ruang Banach \cite{Thakur2016}.  

Pemetaan nonekspansif beserta aproksimasi titik tetapnya tidak hanya dipelajari dalam ruang Banach, tetapi juga dalam ruang metrik lain, khususnya ruang metrik geodesik. Ruang ini merupakan ruang yang memiliki struktur geodesik, yaitu lintasan terpendek di antara dua titik. Ruang ini memiliki keunggulan dalam hal struktur nonlinier, sehingga dapat memodelkan suatu permasalahan nonlinier secara lebih akurat. Salah satu contoh dari ruang ini adalah ruang $CAT(0)$. Kemudian ruang tersebut diperumum oleh Khamsi dkk. menjadi ruang $CAT_p(0)$ \cite{Khamsi2017}. Salah satu hasil aproksimasi titik tetap di ruang tersebut didapatkan oleh Calder dkk. dengan mengenalkan skema iterasi Agarwal pertubasi di ruang $CAT_p(0)$. Mereka menggunakan tiga pemetaan nonekspansif biasa dan mendapatkan hasil konvergensi-$\Delta$ dengan skema iterasi tersebut \cite{CALDERN2021}. 

Dalam ruang Banach, Pant dan Pandey mengusulkan perumuman baru dari pemetaan nonekspansif yang dikenal sebagai pemetaan nonekspansif tipe Reich–Suzuki. Skema iterasi yang mereka gunakan adalah skema iterasi Thakur dan ruang yang digunakan adalah ruang hiperbolik \cite{Pant2019}. Kemudian, Ahmad dkk. mengenalkan skema iterasi baru yang dinamakan skema iterasi JK yang konvergen untuk pemetaan nonekspansif yang memenuhi kondisi $(C)$ \cite{Ahmad2021}. Hasil menarik juga diperoleh Ullah dkk. dengan mengenalkan pemetaan $(\alpha,\beta,\gamma)$-nonekspansif yang juga perumuman dari pemetaan nonekspansif \cite{Ullah2023}. Arif dkk. mendapatkan hasil konvergensi dari skema iterasi JK untuk pemetaan jenis tersebut \cite{Arif2025}. 

Selain mengalami berbagai pengembangan dalam bentuk perumuman kelas pemetaan, pemetaan nonekspansif beserta skema iterasi yang menyertainya juga memiliki beragam aplikasi di berbagai bidang. Perumuman tersebut mencakup perluasan dari pemetaan nonekspansif klasik ke berbagai kelas pemetaan yang lebih umum dengan kondisi kontraktivitas yang lebih lemah. Beberapa aplikasi yang diperoleh antara lain pada persamaan diferensial, persamaan integral, masalah optimasi, rekonstruksi citra, dan bidang terkait lainnya \cite{Byrne2003,Ramage2023,Salisu2022}.
\todo{sitasi} 

Hasil terbaru di ruang Banach diperoleh Sabri dkk. dengan mengenalkan skema iterasinya yang konvergen untuk pemetaan nonekspansif tipe Reich-Suzuki. Skema iterasi yang mereka peroleh secara numerik juga lebih cepat konvergen dibanding skema iterasi sebelumnya \cite{sabri2025}. Hasil terbaru juga, Salisu dkk. meneliti terkait pemetaan nonekspansif yang memenuhi kondisi $(C_\lambda)$ di ruang $CAT_p(0)$. Mereka mendapatkan bahwa pemetaan tersebut memiliki sifat \textit{demiclosedness}. Kemudian, skema iterasi yang mereka gunakan adalah skema iterasi JK. Dengan skema iterasi tersebut, mereka mendapatkan beberapa teorema terkait dengan konvergen-$\Delta$ maupun konvergen kuat untuk pemetaan nonekspansif yang memenuhi kondisi $(C_\lambda)$. Lebih lanjut lagi, mereka juga mendapatkan beberapa aplikasi dari hasil yang mereka peroleh sekaligus memberikan hasil numeriknya \cite{Salisu2022}. 


\section{Ruang Metrik Geodesik}
% \subsection{Definisi dan Notasi Ruang $CAT_p(0)$}
Pada bagian ini dikenalkan konsep terkait ruang metrik geodesik. Secara intuisi, ruang metrik geodesik adalah ruang metrik dengan sifat terdapat jalur terpendek di antara dua titik dari ruang metrik tersebut. Jalur terpendek di sini tidak selalu garis lurus seperti yang ada di $\mathbb{R}^2$, tetapi jalurnya bisa melengkung atau berkelok asalkan lintasannya mempunyai panjang minimum terhadap metriknya. Berikut diberikan definisi ruang metrik geodesik secara formal. 
\begin{defn}\cite{Bridson1999}
    Diberikan $(X,d)$ adalah ruang metrik dan $x,y\in X$. Pemetaan kontinu $G: \qty[0, 1] \to X$ disebut geodesik yang menghubungkan $x$ dan $y$ jika memenuhi 
    \begin{enumerate}
        \item[(a)] $G(0)=x$,
        \item[(b)] $G(1)=y$
        \item[(c)] $d\qty(G(t),G(s))=|t-s|d(x,y)$ untuk setiap $t,s\in [0,1]$.
    \end{enumerate}
 Ruang metrik $(X,d)$ disebut sebagai ruang metrik geodesik jika setiap elemen $x,y\in X$ dihubungkan oleh suatu geodesik.
\end{defn}
Untuk selanjutnya, ruang metrik geodesik dinotasikan sebagai $(X,d,G)$. Pada ruang metrik geodesik, himpunan titik-titik pada geodesik yang menghubungkan $x$ dan $y$ dituliskan sebagai $[x\sim  y]$. Kemudian, berikut diberikan contoh ruang metrik geodesik dan bukan ruang metrik geodesik.
\begin{exam}\cite{Bridson1999}\label{con:geodesik}
    Setiap ruang bernorma $(V,\norm{\cdot})$ dengan metrik $d(v,w)=\|v-w\|$ untuk setiap $v,w\in V$ adalah ruang metrik geodesik dengan geodesik yang menghubungkan $v$ dan $w$ didefinisikan sebagai
    \begin{align*}
        G(t) = (1-t)v + tw, \quad \text{untuk setiap } t\in [0,1].
    \end{align*}
    
\end{exam}
Penjelasan dari Contoh \ref{con:geodesik} adalah sebagai berikut.

    Diperhatikan bahwa $G(0)=v$, $G(1)=w$, dan
    \begin{align*}
        d(G(t),G(s)) &= \norm{(1-t)v + tw - (1-s)v - sw} \\
        &= \norm{(t-s)(w-v)} \\
        &= |t-s|\norm{w-v} \\
        &= |t-s|d(v,w),
    \end{align*}
    untuk setiap $t,s\in [0,1]$. Dengan demikian, $G$ adalah geodesik yang menghubungkan $v$ dan $w$, sehingga setiap ruang bernorma adalah ruang metrik geodesik.
\begin{exam}
    Diberikan $X=[0,1]\subset \mathbb{R}$ dengan metrik 
    \begin{align*}
        d(x,y)=\begin{cases}
            1 & \text{jika } x\neq y\\
            0 & \text{jika } x=y.
        \end{cases}
    \end{align*}
    Ruang metrik $(X,d)$ bukan ruang metrik geodesik karena untuk $x=0$ dan $y=1$, serta $t=\frac{1}{2}$, tidak ada $G(t)\in X$ sehingga $$d\qty(0,G(t))=d\qty(G(t),1)=\frac{1}{2}d(0,1)=\frac{1}{2}.$$
\end{exam}
Konsep segitiga geodesik merupakan konsep dasar yang digunakan dalam mendefinisikan ruang $CAT(0)$ dan ruang $CAT_p(0)$. Berikut ini diberikan definisi segitiga geodesik.
\begin{defn}\cite{Bridson1999}
    Diberikan $(X,d,G)$ ruang metrik geodesik. Suatu segitiga geodesik $\triangle(p, q,r)$ di ruang metrik geodesik berisi tiga titik $p, q,r\in X$ sebagai titik sudut dan geodesik di antara dua titik tersebut, yaitu $[p\sim  q], [p\sim r], [q\sim  r]$, disebut sebagai sisi dari segitiganya. 
\end{defn}
Untuk selanjutnya, titik yang terletak pada suatu segitiga geodesik $\triangle(x_1, x_2,x_3)$ dituliskan sebagai $z\in \triangle(x_1,x_2,x_3)$ jika $z\in [x_i\sim x_j]$ untuk suatu $i,j\in \{1,2,3\}$ dengan $i\neq j$. Titik $z\in[x_i\sim x_j]$ direpresentasikan oleh $z=(1-t)x_i\oplus tx_j$ dengan $t=\frac{d(x_i,z)}{d(x_i,x_j)}$ dan $\oplus$ adalah operasi geodesik.
% \begin{defn}\cite{Bridson1999}
%     Diberikan ruang metrik geodesik $(X,d)$ dengan segitiga geodesik $\triangle(x_1,x_2,x_3)$. Titik yang terletak pada suatu sisi segitiga geodesik tersebut disebut sebagai titik pada segitiga geodesik, yang dituliskan sebagai $z\in \triangle(x_1,x_2,x_3)$ jika $z\in [x_i,x_j]$ untuk suatu $i,j\in \{1,2,3\}$ dengan $i\neq j$. Titik tersebut dapat dituliskan sebagai $z=(1-t)x_i\oplus tx_j$ dengan $t=\frac{d(x_i,z)}{d(x_i,x_j)}$ dan $\oplus$ adalah operasi geodesik.
%     \end{defn}
Contoh dari segitiga geodesik di ruang metrik geodesik diberikan sebagai berikut.
\begin{exam}
    Sebagaimana pada Contoh \ref{con:geodesik}, setiap ruang bernorma $(V,\norm{\cdot})$ adalah ruang metrik geodesik. Misalkan dipilih $V=\mathbb{R}^2$ dan $p=(0,0), q=(1,0), r=(0,1)\in \mathbb{R}^2$, maka segitiga geodesik $\triangle(p,q,r)$ di ruang tersebut mempunyai titik sudut $p,q,r$ dan sisi-sisi segitiga geodesik yang didefinisikan sebagai
    \begin{align*}
        [p\sim  q] &= \{(1-t)p + tq \mid t\in [0,1]\} = \{(t,0) \mid t\in [0,1]\},\\
        [p\sim r] &= \{(1-t)p + tr \mid t\in [0,1]\} = \{(0,t) \mid t\in [0,1]\},\\
        [q\sim  r] &= \{(1-t)q + tr \mid t\in [0,1]\} = \{(1-t,t) \mid t\in [0,1]\}.
    \end{align*}
    Titik $z=(0.5,0)$ terletak pada sisi $[p\sim  q]$ dari segitiga geodesik tersebut dan dapat dituliskan sebagai $z=(1-t)p \oplus tq$ dengan $t=0.5$ dan operasi geodesik $\oplus$ sama dengan operasi penjumlahan vektor.
\end{exam}
Untuk selanjutnya, pada suatu segitiga geodesik $\triangle(x_1,x_2,x_3)$, setiap titik $z\in [x_i\sim  x_j]$ dengan $i,j\in \{1,2,3\}$ dan $i\neq j$, dinotasikan sebagai $z=(1-t)x_i\oplus tx_j$ dengan $t=\frac{d(x_i,z)}{d(x_i,x_j)}$. 

\section{Ruang $CAT_p(0)$}
Sebelum membahas mengenai ruang $CAT_p(0)$, terlebih dahulu dikenalkan ruang $CAT(0)$, sebab ruang $CAT_p(0)$ adalah perumuman dari ruang $CAT(0)$. Ruang $CAT(0)$ sejatinya adalah ruang geodesik, tetapi dengan syarat tambahan bahwa setiap segitiga geodesik di ruang tersebut memenuhi ketaksamaan $CAT(0)$. Untuk itu, berikut diberikan konsep dari segitiga komparasi. 

% segitiga geodesik lebih ramping daripada segitiga pembanding di $\mathbb{R}^2$. Untuk itu berikut diberikan konsep dari segitiga komparasi. 
\begin{defn}\cite{Bridson1999}
    Diberikan segitiga geodesik $\triangle (p,q,r)$ pada ruang metrik geodesik $(X,d)$. Suatu segitiga komparasi $\overline{\triangle}(\overline{p},\overline{q},\overline{r})$ adalah suatu segitiga di ruang bernorma $(\mathbb{E},\|\cdot\|)$ yang memenuhi $d(p,q)=\|\overline{p}-\overline{q}\|, d(p,r)=\|\overline{p}-\overline{r}\|$, dan $d(q,r)=\|\overline{q}-\overline{r}\|$.
\end{defn}
\begin{exam}\label{con:segkom}
    Diberikan ruang metrik $(\mathbb{R}^2,\tilde{d})$ dengan metrik $\tilde{d}$ yang didefinisikan untuk setiap $x=(x_1,x_2), y=(y_1,y_2)\in \mathbb{R}^2$ sebagai 
    \begin{align*}
        \tilde{d}(x,y) = \sqrt{(x_1 - y_1)^2 + \qty(x_1^2-x_2 + y_2-y_1^2)^2}.
    \end{align*}
    Ruang metrik tersebut adalah ruang metrik geodesik dengan geodesik yang menghubungkan $x$ dan $y$ didefinisikan sebagai
    \begin{align*}
        G(t) = \left((1-t)x_1 + ty_1, \qty((1-t)x_1 + ty_1)^2-(1-t)(x_1^2-x_2)-t(y_1^2-y_2)\right),
    \end{align*}
    untuk setiap $t\in [0,1]$ \cite{Eskandani2018}.
    Untuk setiap segitiga geodesik $\triangle(p,q,r)$ dengan $p=(p_1,p_2)$, $q=(q_1,q_2)$, dan $r=(r_1,r_2)\in(\mathbb{R}^2,\tilde{d})$, segitiga komparasi $\overline{\triangle}(\overline{p},\overline{q},\overline{r})$ di ruang Euclid $(\mathbb{R}^2,\|\cdot\|_2)$ dapat dibentuk dengan memilih titik $\overline{p}=(p_1,p_1^2-p_2)$, $\overline{q}=(q_1,q_1^2-q_2)$, $\overline{r}=(r_1,r_1^2-r_2)\in \mathbb{R}^2$ sehingga berlaku  
    \begin{align*}
        \tilde{d}(p,q) &= \sqrt{(p_1 - q_1)^2 + \qty(p_1^2-p_2 + q_2-q_1^2)^2} = \|\overline{p}-\overline{q}\|_2,\\
        \tilde{d}(p,r) &= \sqrt{(p_1 - r_1)^2 + \qty(p_1^2-p_2 + r_2-r_1^2)^2} = \|\overline{p}-\overline{r}\|_2,\\
        \tilde{d}(q,r) &= \sqrt{(q_1 - r_1)^2 + \qty(q_1^2-q_2 + r_2-r_1^2)^2} = \|\overline{q}-\overline{r}\|_2.
    \end{align*}
\end{exam}
Selanjutnya, berikut ini diberikan definisi ruang $CAT(0)$.
\begin{defn}\cite{Bridson1999}
    Diberikan ruang metrik geodesik $(X,d)$ dan ruang Euclid $(\mathbb{R}^2,\|\cdot\|_2)$. Ruang $(X,d)$ disebut sebagai ruang $CAT(0)$ jika untuk setiap segitiga geodesik $\triangle(p,q,r)\in X$ dan $x,y\in \triangle(p,q,r)$, terdapat segitiga komparasi $\overline{\triangle}(\overline{p},\overline{q},\overline{r})\in \mathbb{R}^2$ sehingga untuk setiap $\overline{x},\overline{y}\in\overline{\triangle}$ berlaku $d(x,y)\leq \|\overline{x}-\overline{y}\|_2$.
\end{defn}
\begin{exam}
    Sebagaimana Contoh \ref{con:segkom}, ruang metrik geodesik $(\mathbb{R}^2,\tilde{d})$ adalah ruang $CAT(0)$ karena untuk setiap segitiga geodesik $\triangle(p,q,r)$ di $(\mathbb{R}^2,\tilde{d})$ dan $x,y\in \triangle(p,q,r)$, terdapat segitiga komparasi $\overline{\triangle}(\overline{p},\overline{q},\overline{r})$ di ruang Euclid $(\mathbb{R}^2,\|\cdot\|_2)$ sehingga untuk setiap $\overline{x},\overline{y}\in \overline{\triangle}(\overline{p},\overline{q},\overline{r})$ berlaku $d(x,y)\leq \|\overline{x}-\overline{y}\|_2$.
\end{exam}
% \begin{figure}
%     \centering
%     \begin{tikzpicture}[scale=1.1, line cap=round, line join=round]

% =====================
% Gambar kiri
% =====================
\begin{scope}
% Titik-titik
\path (0,0) coordinate (p) (2.4,2.2) coordinate (q) (4.2,0.3) coordinate (r);
\path (1.0964,0.726) coordinate (A) (2.075,0.427) coordinate (B);

% Segitiga utama
\draw (p) arc(-90-15+42.51:-90+15+42.51:{3.256/(2*sin(15))}) -- (q);
\draw (r) arc(90-15+4.0856:90+15+4.0856:{4.21/(2*sin(15))}) -- (p);
\draw (q) arc(-90-15-46.5481577:-90+15-46.5481577:{2.617/(2*sin(15))}) -- (r);


% Garis dalam
% \draw (A) arc(-90-15-16.99:-90+15-16.99:{1.02325/(2*sin(15))});
\draw (A) -- (B);

% Titik
\fill (p) circle (1.5pt) node[below left] {$p$};
\fill (q) circle (1.5pt) node[above] {$q$};
\fill (r) circle (1.5pt) node[below right] {$r$};

\fill (A) circle (1.5pt) node[above left] {$x$};
\fill (B) circle (1.5pt) node[below right] {$y$};

% Tanda ruas sama panjang
\end{scope}

% =====================
% Gambar kanan
% =====================
\begin{scope}[xshift=6cm]
% Titik-titik
\coordinate (pb) at (0,0);
\coordinate (qb) at (2.4,2.2);
\coordinate (rb) at (4.2,0.3);

\coordinate (xb) at (0.4*2.4,0.4*2.2);
\coordinate (yb) at (0.5*4.2,0.5*0.3);

% Segitiga utama
\draw (pb)  --(qb) --(rb)--cycle;

% Garis dalam
\draw (xb)--(yb);

% Titik
\fill (pb) circle (1.5pt) node[below left] {$\overline{p}$};
\fill (qb) circle (1.5pt) node[above] {$\overline{q}$};
\fill (rb) circle (1.5pt) node[below right] {$\overline{r}$};

\fill (xb) circle (1.5pt) node[above left] {$\overline{x}$};
\fill (yb) circle (1.5pt) node[below right] {$\overline{y}$};

% Tanda ruas sama panjang
\end{scope}

\end{tikzpicture}

%     \caption{Caption}
%     \label{fig:placeholder}
% \end{figure}
% \begin{tikzpicture}[
%     year/.style={
%         rectangle,
%         rounded corners,
%         draw=black,
%         fill=#1,
%         minimum width=2.8cm,
%         minimum height=0.9cm,
%         font=\bfseries,
%         align=center
%     },
%     desc/.style={
%         align=left,
%         font=\small,
%         text width=10cm
%     },
%     arrow/.style={
%         thick, ->, >=stealth
%     }
% ]

% % Nodes Tahun
% \node[year=cyan!40] (y2025) at (0,0) {$\dots$ - 2025};
% \node[year=red] (mulai) at (0, -2) {mulai};
% \node[year=blue!55] (y2026) at (0,-3.8) {2026};
% \node[year=red] (selesai) at (0,-5.6) {selesai};
% \node[year=cyan!40] (y2027) at (0,-7.6) {2027};
% \node[year=cyan!40] (y2028) at (0,-9.6) {2028};
% \node[year=cyan!40] (y2029) at (0,-11.6) {2029};
% \node[year=cyan!40] (y2030) at (0,-13.6) {2030};

% % Deskripsi
% \node[desc] at (7,0) {
% Aproksimasi titik tetap dari pemetaan nonekspansif di ruang Banach.
% };

% \node[desc] at (7,-2) {
% Studi pendahuluan (proposal)
% };
% \node[desc] at (7,-3.8) {
% - Konvergensi skema iterasi Sabri untuk pemetaan $(\alpha,\beta,\gamma)$-nonekspansif di ruang CAT$_p(0)$.\\
% - Aplikasi pada rekonstruksi gambar.

% };
% \node[desc] at (7,-5.6) {
% Submit hasil penelitian ke jurnal Scopus peringkat Q2, naskah tesis, dan penyusunan laporan akhir. 
% };

% \node[desc] at (7,-7.6) {
% Pengembangan skema iterasi Sabri dan perbandingan laju konvergensi secara analitik.
% };

% \node[desc] at (7,-9.6) {
% Aplikasi skema iterasi Sabri pada pelacakan kendali gerak robot berlengan ganda.
% };

% \node[desc] at (7,-11.6) {
% Aplikasi skema iterasi Sabri pada permasalahan pemulihan sinyal.
% };

% \node[desc] at (7,-13.6) {
% Generalisasi aproksimasi titik tetap pada ruang geodesik nonlinier lainnya.
% };

% % Panah
% \draw[arrow] (y2025) -- (mulai);
% \draw[arrow] (mulai) -- (y2026);
% \draw[arrow] (y2026) -- (selesai);
% \draw[arrow] (selesai) -- (y2027);
% \draw[arrow] (y2027) -- (y2028);
% \draw[arrow] (y2028) -- (y2029);
% \draw[arrow] (y2029) -- (y2030);

% \end{tikzpicture}
% \begin{tikzpicture}[
%     >=stealth,
%     main/.style={draw, fill=blue!15, rounded corners, align=center, font=\small},
%     box/.style={draw, fill=red!10, rounded corners, align=center, font=\small},
%     line/.style={thick}
% ]

% % Spine
% \draw[line] (0,0) -- (12,0);

% % Head
% \node[main, minimum width=2cm, minimum height=1cm] 
%     at (12,0) {Aproksimasi titik tetap\\ dari pemetaan \\$(\alpha,\beta,\gamma)$-nonekspansif\\ di ruang $CAT_p(0)$ \\beserta aplikasinya};

% % === Upper bones ===
% % Pemetaan nonekspansif
% \draw[line] (3,0) -- (2,4);
% \node[box, minimum height=1cm, minimum width=2.5cm] at (1.3,4.5)
% {Pemetaan nonekspansif\\ diperumum
% % mencakup lebih banyak pemetaan 
% % aplikasi lebih luas
% % hasil terbaru abcnonekspansif
% % Suzuki\\
% % Reich--Suzuki\\
% % $(\alpha,\beta,\gamma)$-nonekspansif\\
% % Skema iterasi Sabri
% };
% \node at (0,1) {Suzuki};

% % Ruang geodesik
% \draw[line] (8,0) -- (6.5,2);
% \node[box, minimum width=4cm] at (6,2.4)
% {Ruang Geodesik
% % \footnotesize
% % struktur nonlinear 
% % hasil terbaru ruang cat_p
% % Ruang CAT(0)\\
% % Ruang CAT$_p$(0)\\
% % Struktur nonlinier\\
% % Konveksitas geodesik
% };

% % === Lower bones ===
% % Urgensitas
% \draw[line] (4,0) -- (1.3,-4.3);
% \node[box, minimum width=4.5cm] at (6,-4.6)
% {Urgensitas Penelitian
% % \footnotesize
% % Keterbatasan hasil di ruang Banach\\
% % Minimnya studi di CAT$_p$(0)\\
% % Kebutuhan skema iterasi yang efisien
% };

% % Aplikasi
% \draw[line] (8,0) -- (6,-4.2);
% \node[box, minimum width=4cm] at (1,-4.6)
% {Aplikasi Optimasi
% % \footnotesize
% % Rekonstruksi gambar\\
% % Masalah minimisasi konveks\\
% % Formulasi sebagai titik tetap\\
% };

% \end{tikzpicture}


Khamsi mengembangkan ruang yang lebih umum, yaitu ruang $CAT_p(0)$, dengan $p\geq 2$, dengan mengganti segitiga komparasi pada ruang $CAT(0)$ dari yang awalnya segitiga di $\mathbb{R}^2$ menjadi segitiga di $\ell_p$. Berikut ini definisi dari ruang $CAT_p(0)$. 
\begin{defn}\cite{Khamsi2017}\label{defn:catp0}
    Diberikan ruang metrik geodesik $(X,d)$ dan ruang $(\ell_p,\|\cdot\|_p)$ untuk $p\geq 2$. Ruang $(X,d)$ disebut sebagai ruang $CAT_p(0)$, untuk $p\geq 2$, jika untuk setiap segitiga geodesik $\triangle(p,q,r) \in X$ dan $x,y\in\triangle(p,q,r)$, terdapat segitiga komparasi $\overline{\triangle}(\overline{p},\overline{q},\overline{r})\in \ell_p$ sehingga untuk setiap $\overline{x},\overline{y}\in \overline{\triangle}$ berlaku $d(x,y)\leq \|\overline{x}-\overline{y}\|_p$.
\end{defn}
\begin{exam}\cite{Khamsi2017}\label{con:lpcatp}
Setiap ruang $(\ell_p, \|\cdot\|_p)$ dengan $p\geq 2$ adalah ruang $CAT_p(0)$. 
\end{exam}
Penjelasan dari Contoh \ref{con:lpcatp} adalah sebagai berikut.

Diketahui bahwa setiap ruang $(\ell_p, \|\cdot\|_p)$ dengan $p\geq 2$ adalah ruang metrik geodesik. Misalkan $\triangle(p,q,r)$ adalah segitiga geodesik di $(\ell_p, \|\cdot\|_p)$ dengan $p,q,r\in \ell_p$ dan $x,y\in \triangle(p,q,r)$. Dapat dipilih segitiga komparasi $\overline{\triangle}(\overline{p},\overline{q},\overline{r})\in (\ell_p, \|\cdot\|_p)$ dengan $\overline{p}=p$, $\overline{q}=q$, dan $\overline{r}=r$ sehingga untuk setiap $\overline{x},\overline{y}\in \overline{\triangle}$ berlaku $\overline{x}=x$ dan $\overline{y}=y$. Dengan demikian, diperoleh
\begin{align*}  
    d(x,y) &= \|x-y\|_p = \|\overline{x}-\overline{y}\|_p,
\end{align*}
yang menunjukkan bahwa ruang $(\ell_p, \|\cdot\|_p)$ dengan $p\geq 2$ adalah ruang $CAT_p(0)$.

Selanjutnya, berikut ini diberikan sebuah lema yang menyatakan bahwa ruang $CAT_p(0)$ dengan $p>2$ bukan merupakan ruang $CAT(0)$.
\begin{lemma}\cite{Khamsi2017}\label{lema:p=2}
Untuk $p>2$ ruang $CAT_p(0)$ bukan merupakan ruang $CAT(0)$.
\end{lemma}

 
Berikut ini diberikan beberapa konsep yang berlaku pada ruang $CAT_p(0)$ dengan $p\geq 2$.
\begin{defn}\cite{Khamsi2017} \label{defn:konveks}
    Diberikan $(X,d)$ adalah ruang $CAT_p(0)$. Suatu himpunan bagian $\mathcal{M}\subseteq X$ disebut konveks jika $[x\sim y]\subseteq \mathcal{M}$ untuk setiap $x,y\in \mathcal{M}$.
\end{defn}
\begin{exam}
    Dipilih ruang $CAT_p(0)$, yaitu ruang $CAT(0)$ $(\mathbb{R}^2,\tilde{d})$ pada Contoh \ref{con:segkom} dan himpunan bagian $\mathcal{M}=\{(0,b)\in \mathbb{R}^2\mid b\in \mathbb{R}\}\subset \mathbb{R}^2$. Diperhatikan bahwa untuk setiap $x=(0,x_2), y=(0,y_2)\in \mathcal{M}$, geodesik yang menghubungkan $x$ dan $y$ adalah $[x\sim y] = \{G(t)\mid t\in[0,1], G(0)=x, G(1)=y\}$ dengan
    \begin{align*}
         G(t) &= \left((1-t)0 + t0, \qty((1-t)0 + t0)^2-(1-t)(0^2 - x_2)-t(0^2 - y_2)\right)\\
         &=\left(0, (1-t)x_2 + ty_2\right).
    \end{align*}
    Karena $x_2,y_2\in \mathbb{R}$, maka $(1-t)x_2 + ty_2\in \mathbb{R}$, sehingga $[x\sim y]=\{(0, (1-t)x_2 + ty_2)\mid t\in[0,1]\}\subseteq \mathcal{M}$. Dengan demikian, himpunan bagian $\mathcal{M}$ adalah himpunan konveks di ruang $(\mathbb{R}^2,\tilde{d})$.
\end{exam}
\begin{exam}
    Dipilih ruang $CAT_p(0)$, yaitu ruang $CAT(0)$ $(\mathbb{R}^2,\tilde{d})$ pada Contoh \ref{con:segkom} dan himpunan bagian $\mathcal{M}=\{(b,0)\in \mathbb{R}^2\mid b\in \mathbb{R}\}\subset \mathbb{R}^2$. Himpunan bagian tersebut tidak konveks karena terdapat $x=(1,0), y=(-1,0)\in \mathcal{M}$ sehingga 
    \begin{align*}
        G(t) &= \left((1-t)1 + t(-1), \qty((1-t)1 + t(-1))^2-(1-t)(1^2 - 0)-t((-1)^2 - 0)\right)\\
        &=\left(1-2t, (1-2t)^2 - 1\right).
    \end{align*}
    Untuk $t=\dfrac{1}{2}$, diperoleh $G\left(\dfrac{1}{2}\right) = \left(0, -\dfrac{1}{2}\right)\notin \mathcal{M}$, sehingga himpunan bagian $\mathcal{M}$ tidak konveks di ruang $(\mathbb{R}^2,\tilde{d})$.
\end{exam}
% \todo{himpunan kompak}
% \begin{defn}

% \end{defn}
\begin{defn}\cite{Salisu2022}\label{defn:asimtotik}
    Diberikan $(X,d)$ adalah ruang $CAT_p(0)$ dan $\{x_n\}$ adalah barisan terbatas di $X$. Pusat asimtotik dari barisan $\{x_n\}$ di suatu $CAT_p(0)$ didefinisikan sebagai 
    \begin{equation}
        A(\{x_n\}) := \{x\in X:\limsup_{n\to\infty} d(x,x_n) = \inf_{y\in X} \limsup_{n\to \infty} d(y,x_n)\}
    \end{equation}
\end{defn}
\todo{cek}
\begin{exam}
    Diberikan ruang $CAT_p(0)$, yaitu ruang $CAT(0)$ $(\mathbb{R}^2,\tilde{d})$ pada Contoh \ref{con:segkom} dan barisan $\{x_n\}\subseteq (\mathbb{R}^2,\tilde{d})$ dengan $x_n=(0,\dfrac{1}{n})$ untuk setiap $n\in \mathbb{N}$. Barisan tersebut adalah barisan terbatas karena untuk setiap $n,m\in \mathbb{N}$, diperoleh
    \begin{align*}
        \tilde{d}(x_n,x_m) &= \sqrt{(0-0)^2 + \qty(0^2 - \dfrac{1}{n} + \dfrac{1}{m} - 0^2)^2} = \left|\dfrac{1}{n} - \dfrac{1}{m}\right| \leq 1.
    \end{align*}
    Selanjutnya, ditentukan pusat asimtotik dari barisan $\{x_n\}$. Diketahui bahwa untuk setiap $x=(x_1,x_2)\in (\mathbb{R}^2,\tilde{d})$, berlaku
    \begin{align*}
        \limsup_{n\to\infty} \tilde{d}(x,x_n) &= \limsup_{n\to\infty} \sqrt{(x_1 - 0)^2 + \qty(x_1^2 - x_2 + \dfrac{1}{n} - 0^2)^2}\\
        &= \sqrt{x_1^2 + (x_1^2 - x_2)^2}.
    \end{align*}
    Sehingga diperoleh
    \begin{align*}
        \inf_{y\in X} \limsup_{n\to\infty} \tilde{d}(y,x_n) &= \inf_{(y_1,y_2)\in \mathbb{R}^2} \sqrt{y_1^2 + (y_1^2 - y_2)^2}.
    \end{align*}
    Nilai minimum dari fungsi $f(y_1,y_2)=\sqrt{y_1^2 + (y_1^2 - y_2)^2}$ adalah $0$ yang terjadi pada titik $(0,0)$. Dengan demikian, diperoleh
    \begin{align*}
        A(\{x_n\}) &= \{(x_1,x_2)\in \mathbb{R}^2 : \limsup_{n\to\infty} \tilde{d}((x_1,x_2),x_n) = 0\}\\
        &= \{(0,0)\}.
    \end{align*}
\end{exam}
\begin{defn}\cite{Salisu2022}\label{defn:konvD}
    Diberikan $(X,d)$ adalah ruang $CAT_p(0)$ dan $\{x_n\}$ adalah barisan terbatas di $X$. Barisan $\{x_n\}$ disebut sebagai konvergen-$\Delta$ ke suatu titik $x$ di $X$ (dinotasikan $\Delta-\lim_{n\to\infty} x_n=x$) jika $\{x\}$ adalah pusat asimtotik dari setiap subbarisan $\{x_{n_k}\}$ dari $\{x_n\}$.
\end{defn}
\todo{cek}
\begin{exam}
    Diberikan ruang $CAT_p(0)$, yaitu ruang $CAT(0)$ $(\mathbb{R}^2,\tilde{d})$ pada Contoh \ref{con:segkom} dan barisan $\{x_n\}\subseteq (\mathbb{R}^2,\tilde{d})$ dengan $x_n=(0,\dfrac{1}{n})$ untuk setiap $n\in \mathbb{N}$. Barisan tersebut adalah barisan terbatas karena untuk setiap $n,m\in \mathbb{N}$, diperoleh
    \begin{align*}
        \tilde{d}(x_n,x_m) &= \sqrt{(0-0)^2 + \qty(0^2 - \dfrac{1}{n} + \dfrac{1}{m} - 0^2)^2} = \left|\dfrac{1}{n} - \dfrac{1}{m}\right| \leq 1.
    \end{align*}
    Selanjutnya, ditentukan pusat asimtotik dari setiap subbarisan $\{x_{n_k}\}$ dari $\{x_n\}$. Diketahui bahwa untuk setiap $x=(x_1,x_2)\in (\mathbb{R}^2,\tilde{d})$, berlaku
    \begin{align*}
        \limsup_{k\to\infty} \tilde{d}(x,x_{n_k}) &= \limsup_{k\to\infty} \sqrt{(x_1 - 0)^2 + \qty(x_1^2 - x_2 + \dfrac{1}{n_k} - 0^2)^2}\\
        &= \sqrt{x_1^2 + (x_1^2 - x_2)^2}.
    \end{align*}
    Sehingga diperoleh
    \begin{align*}
        \inf_{y\in X} \limsup_{k\to\infty} \tilde{d}(y,x_{n_k}) &= \inf_{(y_1,y_2)\in \mathbb{R}^2} \sqrt{y_1^2 + (y_1^2 - y_2)^2}.
    \end{align*}
    Nilai minimum dari fungsi $f(y_1,y_2)=\sqrt{y_1^2 + (y_1^2 - y_2)^2}$ adalah $0$ yang terjadi pada titik $(0,0)$. Dengan demikian, diperoleh
    \begin{align*}
        A(\{x_{n_k}\}) &= \{(x_1,x_2)\in \mathbb{R}^2 : \limsup_{k\to\infty} \tilde{d}((x_1,x_2),x_{n_k}) = 0\}\\
        &= \{(0,0)\}.
    \end{align*}
    Karena pusat asimtotik dari setiap subbarisan $\{x_{n_k}\}$ adalah $\{(0,0)\}$, maka barisan $\{x_n\}$ konvergen-$\Delta$ ke titik $(0,0)$, yaitu $\Delta-\lim_{n\to\infty} x_n = (0,0)$.
\end{exam}
\begin{defn}\cite{Salisu2022}\label{defn:konvK}
    Diberikan $(X,d)$ adalah ruang $CAT_p(0)$ dan $\{x_n\}$ adalah barisan di $X$. Barisan $\{x_n\}$ disebut konvergen kuat ke suatu titik $x$ di $X$ (dinotasikan $\lim_{n\to\infty} d(x_n,x)=0$) jika untuk setiap $\varepsilon>0$, terdapat $n_0\in \mathbb{N}$ sehingga untuk setiap $n\geq n_0$ berlaku $d(x_n,x)<\varepsilon$.
\end{defn}
\todo{cek}
\begin{exam}
    Diberikan ruang $CAT_p(0)$, yaitu ruang $CAT(0)$ $(\mathbb{R}^2,\tilde{d})$ pada Contoh \ref{con:segkom} dan barisan $\{x_n\}\subseteq (\mathbb{R}^2,\tilde{d})$ dengan $x_n=(0,\dfrac{1}{n})$ untuk setiap $n\in \mathbb{N}$. Barisan tersebut konvergen kuat ke titik $x=(0,0)$ karena untuk setiap $\varepsilon>0$, dapat dipilih $n_0\in \mathbb{N}$ sehingga untuk setiap $n\geq n_0$ berlaku
    \begin{align*}
        \tilde{d}(x_n,x) &= \sqrt{(0-0)^2 + \qty(0^2 - \dfrac{1}{n} - 0^2)^2} = \dfrac{1}{n} < \varepsilon.
    \end{align*}
\end{exam}
\begin{defn}\cite{Salisu2022}\label{defn:demi}
    Diberikan $(X,d)$ adalah suatu ruang $CAT_p(0)$. Pemetaan $T:X\to X$ disebut memiliki sifat \textbf{demiclosedness} jika untuk setiap barisan $\{x_n\}\subseteq X$ yang konvergen-$\Delta$ dan memenuhi $\lim_{n\to \infty} d(x_n,Tx_n)=0$, maka berlaku $x=Tx$. 
\end{defn}
\todo{cek}
\begin{exam}
    Diberikan ruang $CAT_p(0)$, yaitu ruang $CAT(0)$ $(\mathbb{R}^2,\tilde{d})$ pada Contoh \ref{con:segkom} dan pemetaan $T:(\mathbb{R}^2,\tilde{d})\to (\mathbb{R}^2,\tilde{d})$ yang didefinisikan sebagai $T(x_1,x_2) = \left(\dfrac{x_1}{2}, \dfrac{x_2}{2}\right)$ untuk setiap $(x_1,x_2)\in \mathbb{R}^2$. Diketahui bahwa titik tetap dari $T$ adalah $(0,0)$. Misalkan $\{x_n\}\subseteq (\mathbb{R}^2,\tilde{d})$ adalah barisan dengan $x_n=\left(\dfrac{1}{n}, \dfrac{1}{n}\right)$ untuk setiap $n\in \mathbb{N}$. Barisan tersebut konvergen-$\Delta$ ke titik $(0,0)$ karena pusat asimtotik dari setiap subbarisan $\{x_{n_k}\}$ dari $\{x_n\}$ adalah $\{(0,0)\}$. Selain itu, diperoleh
    \begin{align*}
        \lim_{n\to\infty} \tilde{d}(x_n,Tx_n) &= \lim_{n\to\infty} \sqrt{\left(\dfrac{1}{n} - \dfrac{1}{2n}\right)^2 + \qty(\dfrac{1}{n^2} - \dfrac{1}{n} + \dfrac{1}{2n} - \dfrac{1}{4n})^2}\\
        &= \lim_{n\to\infty} \sqrt{\left(\dfrac{1}{2n}\right)^2 + \left(-\dfrac{3}{4n} + \dfrac{1}{n^2}\right)^2} = 0.
    \end{align*}
    Dengan demikian, karena $\Delta-\lim_{n\to\infty} x_n = (0,0)$ dan $\lim_{n\to\infty} \tilde{d}(x_n,Tx_n)=0$, maka diperoleh $x=Tx$, yaitu $(0,0)=T(0,0)$. Oleh karena itu, pemetaan $T$ memiliki sifat demiclosedness.
\end{exam}

Selanjutnya, berikut ini diberikan ketaksamaan penting yang berlaku pada ruang $CAT_p(0)$. Dalam tesis ini dinotasikan $d^p(x,y)=[d(x,y)]^p$.
\begin{lemma}\cite{CALDERN2021}\label{lema:d,d^p}
    Diberikan $(X,d)$ adalah suatu ruang $CAT_p(0)$. Jika $x,y,z\in X$ dan $t\in[0,1]$, maka 
    \begin{enumerate}
        \item $d((1-t)x\oplus tz, y)\leq (1-t)d(x,y)+td(z,y)$. 
        \item $d^p((1-t)x\oplus tz, y)\leq (1-t)d^p(x,y)+td^p(z,y)-\dfrac{t(1-t)}{2^{p-1}}d^p(x,z)$.
    \end{enumerate}
\end{lemma}
% \begin{remark}\cite{CALDERN2021}
%     Suatu ruang metrik geodesik merupakan ruang $CAT_p(0)$ jika dan hanya jika memenuhi 
% \end{remark}
\begin{lemma}\cite{Salisu2022}\label{lema:asimtotik}
    Pusat asimtotik dari suatu barisan terbatas di ruang $CAT_p(0)$ memiliki tepat satu elemen. 
\end{lemma}
\begin{thm}\cite{Salisu2022}\label{thm:kondisikonvD}
    Diberikan $(X,d)$ adalah ruang $CAT_p(0)$ yang lengkap dan $W$ adalah himpunan bagian tak kosong dari $X$ yang tertutup dan konveks. Diberikan pula $T:W\to W$ adalah pemetaan yang memiliki titik tetap dan memenuhi sifat demiclosedness \ref{defn:demi}. Jika $\{x_n\}$ adalah barisan di $W$ yang memenuhi $\lim_{n\to\infty}d(x_n,Tx_n)=0$ dan barisan $\{d(x_n,x^*)\}$ konvergen di $\mathbb{R}$ untuk setiap $x^*$ titik tetap dari $T$, maka barisan $\{x_n\}$ konvergen-$\Delta$ ke titik tetap dari $T$.
\end{thm}
\begin{defn}
    Diberikan $(X,d)$ adalah ruang $CAT_p(0)$. Suatu fungsi $f:X\to \mathbb{R}\cup \{+\infty\}$ disebut konveks secara geodesik jika untuk setiap $t\in(0,1)$ dan $x,y\in X$, berlaku
    \begin{align*}
        f(tx\oplus (1-t)y)\leq tf(x)+(1-t)f(y).
    \end{align*}
\end{defn}
\begin{defn}
    Diberikan $(X,d)$ adalah ruang $CAT_p(0)$. Suatu fungsi $f:X\to \mathbb{R}\cup \{+\infty\}$ disebut \textit{proper} jika himpunan $D(f):=\{x\in X\mid f(x)<+\infty\}\neq \emptyset$.
\end{defn}
\begin{defn}
    Diberikan $(X,d)$ adalah ruang $CAT_p(0)$. Suatu fungsi $f:X\to \mathbb{R}\cup \{+\infty\}$ disebut \textit{lower semi-continuous} pada suatu titik $x\in D(f)$ jika $f(x)\leq \liminf_{n\to\infty} f(x_n)$ untuk setiap barisan $\{x_n\}$ yang konvergen di $D(f)$ dengan limit $x\in X$. Jika $f$ \textit{lower semi-continuous} pada setiap titik di $D(f)$, maka $f$ disebut \textit{lower semi-continuous} pada $X$.
\end{defn}
\todo{cek}
\begin{exam}
    Diberikan ruang $CAT_p(0)$, yaitu ruang $CAT(0)$ $(\mathbb{R}^2,\tilde{d})$ pada Contoh \ref{con:segkom} dan fungsi $f:(\mathbb{R}^2,\tilde{d})\to \mathbb{R}$ yang didefinisikan sebagai $f(x_1,x_2) = x_1^2 + x_2^2$ untuk setiap $(x_1,x_2)\in \mathbb{R}^2$. Fungsi tersebut adalah konveks secara geodesik karena untuk setiap $t\in(0,1)$ dan $(x_1,x_2),(y_1,y_2)\in \mathbb{R}^2$, diperoleh
    \begin{align*}
        f\left(t(x_1,x_2)\oplus (1-t)(y_1,y_2)\right) &= f\left(tx_1 + (1-t)y_1, (tx_1 + (1-t)y_1)^2 - t(x_1^2 - x_2) - (1-t)(y_1^2 - y_2)\right)\\
        &= \left(tx_1 + (1-t)y_1\right)^2 + \left((tx_1 + (1-t)y_1)^2 - t(x_1^2 - x_2) - (1-t)(y_1^2 - y_2)\right)^2\\
        &\leq t(x_1^2 + x_2^2) + (1-t)(y_1^2 + y_2^2)\\
        &= tf(x_1,x_2) + (1-t)f(y_1,y_2).
    \end{align*}
    Selain itu, fungsi tersebut adalah \textit{proper} karena himpunan $D(f)=\{(x_1,x_2)\in \mathbb{R}^2\mid f(x_1,x_2)<+\infty\}=\mathbb{R}^2\neq \emptyset$. Fungsi tersebut juga adalah \textit{lower semi-continuous} pada setiap titik di $\mathbb{R}^2$ karena untuk setiap barisan $\{(x_{1n},x_{2n})\}\subseteq \mathbb{R}^2$ yang konvergen ke suatu titik $(x_1,x_2)\in \mathbb{R}^2$, diperoleh
    \begin{align*}
        f(x_1,x_2) &= x_1^2 + x_2^2 = \lim_{n\to\infty} (x_{1n}^2 + x_{2n}^2) = \lim_{n\to\infty} f(x_{1n},x_{2n}) \geq \liminf_{n\to\infty} f(x_{1n},x_{2n}).
    \end{align*}
\end{exam}

\section{Pemetaan Nonekspansif}
Pemetaan nonekspansif merupakan salah satu jenis pemetaan yang memiliki banyak aplikasi di berbagai bidang. Pemetaan ini merupakan perluasan dari kontraktif yang dikenalkan oleh Banach. 
Berikut ini diberikan definisi dari pemetaan nonekspansif. 
\begin{defn}\cite{Browder1965}
    Diberikan $(B,\|\cdot\|)$ adalah ruang Banach dan $W$ adalah himpunan bagian tak kosong dari $B$. Pemetaan $T:W\to W$ disebut pemetaan nonekspansif jika untuk setiap $x,y\in W$ berlaku 
    \begin{equation}
        \|Tx-Ty\|\leq \|x-y\|.
    \end{equation}
\end{defn}
Pemetaan nonekspansif ini juga mengalami perluasan, salah satu perluasannya adalah pemetaan nonekspansif kuasi. Berikut adalah definisi dari pemetaan nonekspansif kuasi.
\begin{defn}
    Diberikan $(B,\|\cdot\|)$ adalah ruang Banach dan $W$ adalah himpunan bagian tak kosong dari $B$. Pemetaan $T:W\to W$ disebut pemetaan nonekspansif kuasi jika untuk setiap titik tetap dari $T$ yaitu $y$ dan $x\in W$ berlaku 
    \begin{equation}
        \|Tx-y\|\leq \|x-y\|.
    \end{equation}
\end{defn}

Pada tahun 2008, Suzuki juga mengenalkan bentuk lain pemetaan nonekspansif, yaitu pemetaan Suzuki nonekspansif atau pemetaan yang memenuhi kondisi $(C)$ yang didefinisikan sebagai berikut
\begin{defn}\cite{Suzuki2008}
    Diberikan $(B,\|\cdot\|)$ adalah ruang Banach dan $W$ adalah himpunan bagian tak kosong dari $B$. Pemetaan $T:W\to W$ disebut pemetaan yang memenuhi kondisi $(C)$ jika untuk setiap $x,y\in W$ berlaku 
    \begin{equation}
        \dfrac{1}{2}\|x-Tx\|\leq \|x-y\| \qquad \Longrightarrow \qquad \|Tx-Ty\|\leq \|x-y\|.
    \end{equation}
\end{defn}

Pemetaan ini merupakan kondisi lemah dari pemetaan nonekspansif karena ketaksamaan nonekspansif hanya berlaku untuk dua titik yang memenuhi kondisi $\frac{1}{2}\|x-Tx\|\leq \|x-y\|$.

García-Falset dkk. memperumum pemetaan Suzuki nonekspansif dengan mengganti konstanta $\frac{1}{2}$ menjadi konstanta $\mu\in (0,1)$. Pemetaan tersebut didefinisikan sebagai berikut. 
\begin{defn}\cite{GARCIAFALSET2011185}
    Diberikan $(B,\|\cdot\|)$ adalah ruang Banach dan $W$ adalah himpunan bagian tak kosong dari $B$. Pemetaan $T:W\to W$ disebut pemetaan yang memenuhi kondisi $(C_{\mu})$ jika untuk setiap $x,y\in W$ berlaku 
    \begin{equation}
        \mu \|x-Tx\|\leq \|x-y\| \qquad \Longrightarrow \qquad \|Tx-Ty\|\leq \|x-y\|
    \end{equation}
\end{defn}

Di sisi lain, Aoyama dan Kohsaka juga mengembangkan pemetaan nonekspansif menjadi pemetaan $\alpha$-nonekspansif. Berikut adalah definisi dari pemetaan $\alpha$-nonekspansif.

\begin{defn}\cite{Aoyama2011}
    Diberikan $(B,\|\cdot\|)$ adalah ruang Banach dan $W$ adalah himpunan bagian tak kosong dari $B$. Pemetaan $T:W\to W$ disebut pemetaan $\alpha$-nonekspansif jika untuk setiap $x,y\in W$ terdapat bilangan real $\alpha<1$ sedemikian hingga
    \begin{equation}
        \|Tx-Ty\|^2\leq \alpha \|x-Ty\|^2+\alpha \|y-Tx\|^2 +(1-2\alpha) \|x-y\|^2.
    \end{equation}
\end{defn}

Pant dan Pandey di tahun 2019 mengenalkan pemetaan pemetaan tipe Reich-Suzuki nonekspansif yang didefinisikan sebagai berikut. 

\begin{defn}\cite{Pant2019}
    Diberikan $(B,\|\cdot\|)$ adalah ruang Banach dan $W$ adalah himpunan bagian tak kosong dari $B$. Pemetaan $T:W\to W$ disebut pemetaan tipe Reich-Suzuki nonekspansif jika untuk setiap $x,y\in W$ terdapat bilangan real $\alpha\in[0,1)$ sedemikian hingga
    \begin{equation}
        \|Tx-Ty\|\leq \alpha \|x-Tx\|+\alpha \|y-Ty\| +(1-2\alpha) \|x-y\|.
    \end{equation}
\end{defn}

Yang terbaru, pada tahun 2023 Ullah dkk. mengenalkan pemetaan $(\alpha,\beta,\gamma)$-nonekspansif. 

\begin{defn}\cite{Ullah2023}
    Diberikan $(B,\|\cdot\|)$ adalah ruang Banach dan $W$ adalah himpunan bagian tak kosong dari $B$. Pemetaan $T:W\to W$ disebut pemetaan $(\alpha,\beta,\gamma)$-nonekspansif jika untuk setiap $x,y\in W$ terdapat bilangan real $\alpha,\beta,\gamma\in \mathbb{R}^+$ dengan $\gamma\in[0,1)$ dan $\alpha+\gamma\leq 1$ sedemikian hingga
    \begin{equation}
        \|Tx-Ty\|\leq \alpha \|x-y\|+\beta \|x-Tx\| +\gamma \|y-Tx\|.
    \end{equation}
\end{defn}
% Berikut ini diberikan contoh dari pemetaan tersebut. 
% \begin{exam}
    
% \end{exam}

\input{Bab_2/2_5_Titik_Tetap_Pemetaan_Nonekspansif.tex}
\section{Aproksimasi Titik Tetap dari Pemetaan Nonekspansif}
Salah satu cara untuk mendapatkan titik tetap dari suatu pemetaan $f$ adalah dengan mendapatkan nilai $x_0$ yang memenuhi persamaan titik tetap yaitu, $x_0=f(x_0)$. Akan tetapi, menyelesaikan persamaan tersebut tidak selalu mudah, khususnya untuk persamaan tak linier. Oleh karena itu, nilai pendekatan atau aproksimasi dari titik tetap diperlukan. 

Untuk melakukan aproksimasi, tentunya diperlukan algoritma atau dalam hal ini adalah skema iterasi yang konvergen ke titik tetap dari pemetaan tersebut. Untuk pemetaan yang bersifat kontraktif, skema iterasi Picard adalah skema iterasi paling sederhana yang dapat digunakan. Berikut adalah skema iterasi Picard
Diberikan $X$ adalah himpunan tak kosong dan $T:X\to X$ adalah suatu pemetaan, serta $x_1\in X$, maka $\{x_n\}$ yang didefinisikan sebagai $x_{n+1}=T(x_n)$ untuk setiap $n\in\mathbb{N}$ adalah barisan yang dihasilkan oleh skema iterasi Picard. 
Namun, skema ini tidak selalu konvergen untuk pemetaan yang bersifat nonekspansif. Sebagai ilustrasi, diberikan contoh berikut 
\begin{exam}
    Diberikan $f:\mathbb{R}\to\mathbb{R}$ dengan $f(x)=1-x$. Pemetaan ini adalah pemetaan nonekspansif dengan titik tetap adalah $x=\frac{1}{2}$, tetapi untuk setiap $x\in \mathbb{R}\backslash \{\frac{1}{2}\}$ iterasi Picard menghasilkan barisan 
    $\{x,1-x,x,1-x,x,\dots\}$ yang divergen.
\end{exam}
Misalkan $W$ adalah himpunan yang tertutup dan konveks dari suatu ruang Banach, $x_0\in W$ dan $\{a_k\},\{b_k\},\{c_k\}$ adalah barisan di $(0,1]$, serta $T:W\to W$ adalah suatu pemetaan nonekspansif. 
Pada tahun 1953, Mann memperkenalkan skema iterasinya yang konvergen untuk pemetaan nonekspansif yang diberikan sebagai berikut \cite{mann1953}
\begin{align}
        x_{k+1} &= (1-a_k)x_k +a_k Tx_k.
\end{align}
Berbagai perkembangan skema iterasi juga terus bermunculan hingga saat ini dengan salah satu tujuannya adalah mencari skema iterasi yang tercepat untuk aproksimasi titik tetap dari suatu pemetaan. Untuk selanjutnya, berikut ini diberikan skema iterasi yang konvergen ke titik tetap dari pemetaan nonekspansif diperumum oleh Agarwal, Abbas dan Nazir, Thakur, Ahmad dkk. serta Sabri dkk. secara berurutan \cite{agarwal,abbas,Thakur2016,Ahmad2021,sabri2025}. 
\begin{align}\label{eq:itagarwal}
    \begin{cases}
        y_k &= (1-b_k)x_k+b_kTx_k,\\
        x_{k+1} &= (1-a_k)Tx_k +a_k Ty_k.
    \end{cases}
\end{align}
\begin{align}\label{eq:itabbas}
    \begin{cases}
        q_k &= (1-c_k)x_k+c_kTx_k,\\
        y_k &= (1-b_k)Tx_k +b_kTq_k,\\
        x_{k+1} &= (1-a_k)Ty_k+a_kTq_k.
    \end{cases}
\end{align}
\begin{align}\label{eq:itthakur}
    \begin{cases}
        q_k &= (1-c_k)x_k+c_kTx_k,\\
        y_k &= T((1-b_k)x_k +b_k q_k),\\
        x_{k+1} &= Ty_k.
    \end{cases}
\end{align}
\begin{align}\label{eq:itjk}
    \begin{cases}
        q_k &= (1-c_k)x_k+c_kTx_k,\\
        y_k &= Tq_k,\\
        x_{k+1} &= T((1-a_k)Tq_k+a_kTy_k).
    \end{cases}
\end{align}
\begin{align}\label{eq:itSabri}
    \begin{cases}
        q_k &= T\qty(\qty(1-c_k)x_k+c_kTx_k),\\
        y_k &= T\qty(Tq_k),\\
        x_{k+1} &= T\qty((1-a_k)Tq_k+a_kTy_k).
    \end{cases}
\end{align}


 %% menampilkan Bab II

\chapter{METODE PENELITIAN}

\indent Pada bab ini diuraikan beberapa tahapan penelitian yang akan dikerjakan untuk mencapai tujuan penelitian.
\section{Tahapan Penelitian}
\begin{itemize}
    \item[(a)] Studi Literatur\\
    Pada tahap ini dilakukan tinjauan pustaka dengan fokus utama meliputi konsep mengenai titik tetap, ruang $CAT_p(0)$, pemetaan nonekspansif dan perumumannya, serta terkait skema iterasi. Akan dilakukan kajian terkait syarat-syarat untuk konvergensi dari suatu skema iterasi untuk mendapatkan nilai aproksimasi titik tetap dari pemetaan nonekspansif dan perumumannya. Kemudian dikaji pula konsep-konsep yang ada pada ruang $CAT_p(0)$, seperti dari definisi, sifat-sifat, dan hasil penting lainnya yang berkaitan dengan aproksimasi titik tetap dari pemetaan nonekspansif di ruang tersebut.
    \item[(b)] Mengkaji Pemetaan $(\alpha,\beta,\gamma)$-nonekspansif di ruang $CAT_p(0)$.\\
    Pada tahap ini pemetaan $(\alpha,\beta,\gamma)$-nonekspansif yang awalnya didefinisikan untuk ruang Banach akan dikaji di ruang $CAT_p(0)$. Beberapa lema dan teorema yang berlaku untuk pemetaan tersebut akan diperluas dalam konteks ruang $CAT_p(0)$. 
    \item[(c)] Mendapatkan Konvergensi dari Skema Iterasi Sabri di Ruang $CAT_p(0)$ untuk Aproksimasi Titik Tetap dari Pemetaan $(\alpha,\beta,\gamma)$-nonekspansif.\\
    Tahapan ini memiliki tujuan untuk membuktikan bahwa barisan yang diperoleh dari skema iterasi Sabri konvergen titik tetap dari pemetaan $(\alpha,\beta,\gamma)$-nonekspansif di ruang $CAT_p(0)$. Beberapa syarat untuk barisan tersebut konvergen, baik itu konvergen-$\Delta$ maupun konvergen kuat akan diselidiki pada tahap ini. 
    \item [(d)] Melakukan Percobaan Numerik.\\
    Untuk memvalidasi hasil teoritis pada tahap sebelumnya, dilakukan eksperimen numerik. Dalam tahap ini, dilakukan percobaan dengan mendefinisikan suatu pemetaan $(\alpha,\beta,\gamma)$-nonekspansif di ruang $CAT_p(0)$, kemudian titik tetap dari pemetaan tersebut akan didekati dengan menggunakan skema iterasi Sabri. Dalam percobaan ini, juga akan digunakan beberapa variasi parameter. Selanjutnya, jumlah iterasi dari skema iterasi Sabri akan dibandingkan dengan beberapa skema iterasi lainnya yang telah ada sebelumnya. Hasil yang diperoleh akan disajikan dalam bentuk tabel. 
    \item [(e)] Mendapatkan Aplikasi dari Skema Iterasi Sabri untuk Masalah Optimasi.\\
    Banyak masalah dalam bidang optimisasi, seperti \textit{convex feasibility problem} atau masalah minimalisasi, dapat diformulasikan ulang sebagai masalah pencarian titik tetap. Dalam tahap ini akan diidentifikasi masalah yang relevan dan menyelesaikannya dengan skema iterasi Sabri.
    \item [(f)] Diseminasi.\\
    Pada tahap diseminasi, dilakukan penulisan artikel ilmiah dan akan dipublikasikan pada seminar internasional atau jurnal internasional bereputasi. 
    
    %ini tulisan
    \item [(g)] Penyusunan Laporan Tesis.\\
   Pada tahap ini, dilakukan penulisan laporan tesis yang meliputi seluruh hasil penelitian, baik itu hasil teoritis maupun hasil numerik. Penjelasan dari hasil tersebut ditulis secara rinci, terstruktur, dan lengkap. Selain itu, kesimpulan dari penelitian dan saran untuk penelitian selanjutnya akan dituliskan dalam laporan tesis.
    
\end{itemize}
\newpage
\section{Diagram Alir Penelitian} 

Diagram alir untuk penelitian ini disajikan dalam  \ref{3_gambar_LangkahUmum} sebagai berikut:
\begin{figure}[H]
        \centering
	\tikzstyle{startstop} = [ellipse, 
minimum width=5.5cm, 
minimum height=0.5cm,
text centered, 
draw=black]

\tikzstyle{io} = [trapezium, 
trapezium stretches=true, % A later addition
trapezium left angle=70, 
trapezium right angle=110, 
minimum width=6cm, 
minimum height=0.5cm, text centered, 
draw=black]

\tikzstyle{process} = [rectangle, 
minimum height=0.5cm, 
text centered, 
text width=8.5cm, 
draw=black]

\tikzstyle{decision} = [diamond, 
minimum width=5cm, 
minimum height=1cm, 
text centered, 
draw=black]
\tikzstyle{arrow} = [thick,->,>=stealth]
\begin{tikzpicture}[node distance=1cm]

\node (start) [startstop] {Mulai};
\node (pro1) [process, below of=start,yshift=-0.3cm] {Studi Literatur};
\node (pro2) [process, below of=pro1, yshift=-0.6cm] {Mengkaji Pemetaan \\$(\alpha,\beta,\gamma)$-nonekspansif di ruang $CAT_p(0)$ };
\node (pro3) [process, below of=pro2, yshift=-1.6cm] {Mendapatkan Konvergensi dari Skema Iterasi Sabri di Ruang $CAT_p(0)$ untuk Aproksimasi Titik Tetap dari Pemetaan $(\alpha,\beta,\gamma)$-nonekspansif };
\node (pro4) [process, below of=pro3, yshift=-1.2cm] {Melakukan Percobaan Numerik };
\node (pro5) [process, below of=pro4,yshift=-0.6cm] {Mendapatkan Aplikasi dari Skema Iterasi Sabri untuk Masalah Optimasi};
\node (pro6) [process, below of=pro5,yshift=-0.6cm] {Diseminasi};
\node (pro7) [process, below of=pro6, yshift=-0.2cm] {Penyusunan Laporan Tesis};

\node (stop) [startstop, below of=pro7, yshift=-0.3cm] {Selesai};

\draw [arrow] (start) -- (pro1);
\draw [arrow] (pro1) -- (pro2);
\draw [arrow] (pro2) -- (pro3);
\draw [arrow] (pro3) -- (pro4);
\draw [arrow] (pro4) -- (pro5);
\draw [arrow] (pro5) -- (pro6);
\draw [arrow] (pro6) -- (pro7);
\draw [arrow] (pro7) -- (stop);

\end{tikzpicture}
	\caption{Blok Diagram Penelitian.}
        \label{3_gambar_LangkahUmum}
\end{figure}
% \newpage
% \section{Jadwal Pelaksanaan}
% Rencana dan jadwal kerja penelitian, serta penyusunan tesis disajikan dalam~\ref{Jadwal} sebagai berikut:

% \begin{table}[th!]
% \caption{Rencana Pelaksanaan Penelitian}
% \centering
% \begin{tabular}{|c|L{3cm}|c|c|c|c|c|c|c|c|c|c|c|c|c|c|c|c|}	
% \hline
% &&\multicolumn{16}{c|}{Bulan ke-}\\
% \cline{3-18}
% \multicolumn{1}{|c|}{No.}&\multicolumn{1}{c|}{Jenis Kegiatan}&\multicolumn{4}{c|}{1}&\multicolumn{4}{c|}{2}&\multicolumn{4}{c|}{3}&\multicolumn{4}{c|}{4}\\\cline{3-18}
% &&1&2&3&4&1&2&3&4&1&2&3&4&1&2&3&4\\\cline{1-18}
% 1&Studi literatur&\cellcolor{gray}&\cellcolor{gray}&\cellcolor{gray}&\cellcolor{gray}&&&&&&&&&&&&\\\hline
% 2& Mengkaji Pemetaan $(\alpha,\beta,\gamma)$-nonekspansif di ruang $CAT_p(0)$ &&&\cellcolor{gray}&\cellcolor{gray}&\cellcolor{gray}&&&&&&&&&&&\\\hline
% 3&  Mendapatkan Konvergensi dari Skema Iterasi Sabri di Ruang $CAT_p(0)$ untuk Aproksimasi Titik Tetap dari Pemetaan $(\alpha,\beta,\gamma)$-nonekspansif &&&&\cellcolor{gray}&\cellcolor{gray}&\cellcolor{gray}&\cellcolor{gray}&\cellcolor{gray}&&&&&&&&\\\hline
% 4&Melalukan Percobaan Numerik &&&&&&&\cellcolor{gray}&\cellcolor{gray}&\cellcolor{gray}&\cellcolor{gray}&\cellcolor{gray}&\cellcolor{gray}&&&&\\\hline
% 5& Mendapatkan Aplikasi dari Skema Iterasi Sabri untuk Masalah Optimasi &&&&&&&&&&\cellcolor{gray}&\cellcolor{gray}&\cellcolor{gray}&\cellcolor{gray}&&&\\\hline
% 6& Diseminasi.&&&&&&&&\cellcolor{gray}&\cellcolor{gray}&\cellcolor{gray}&\cellcolor{gray}&\cellcolor{gray}&\cellcolor{gray}&\cellcolor{gray}&&\\\hline
% 7&Penyusunan Laporan Tesis.&&&&&&&&\cellcolor{gray}&\cellcolor{gray}&\cellcolor{gray}&\cellcolor{gray}&\cellcolor{gray}&\cellcolor{gray}&\cellcolor{gray}&\cellcolor{gray}&\cellcolor{gray}\\\hline

% \end{tabular}

% \label{Jadwal}
% \end{table} 

  %%% menampilkan Bab III
\chapter{HASIL DAN PEMBAHASAN}\label{chap:bab4}
Dalam bab ini disajikan perluasan dari pemetaan $(\alpha,\beta,\gamma)$-nonekspansif pada ruang $CAT_p(0)$ yang meliputi definisi dan sifat-sifat pemetaannya. Kemudian, disajikan pula konvergensi dari skema iterasi Sabri untuk pemetaan tersebut. Selanjutnya, dilakukan simulasi numerik terkait konvergensi dari skema iterasi Sabri untuk aproksimasi titik tetap dari pemetaan $(\alpha,\beta,\gamma)$-nonekspansif. Selain itu, didapatkan pula aplikasi dari skema iterasi Sabri untuk masalah optimasi, khususnya untuk masalah minimalisasi fungsi dan rekonstruksi citra. 
\section{Pemetaan $(\alpha,\beta,\gamma)$-nonekspansif di Ruang $CAT_p(0)$}
Pada bagian ini, disajikan definisi, contoh, dan sifat-sifat dari pemetaan $(\alpha,\beta,\gamma)$-nonekspansif di ruang $CAT_p(0)$. Sifat yang didapatkan dari pemetaan ini disajikan dalam Lema \ref{Lema:dxnx*}, yang menunjukkan bahwa pemetaan ini memenuhi kondisi nonekspansif kuasi. 
Kemudian, pada Lema \ref{Lema:ineqabcnoneks} didapatkan ketaksamaan penting yang melibatkan pemetaan tersebut. Selain itu, didapatkan pula bahwa pemetaan ini memenuhi sifat \textit{demiclosedness} yang ditunjukkan oleh Lema \ref{Lema:demi}. Tiga Lema tersebut berperan penting untuk pembuktian konvergensi skema iterasi Sabri untuk aproksimasi titik tetap dari pemetaan $(\alpha,\beta,\gamma)$-nonekspansif.

Berikut ini disajikan definisi dari pemetaan $(\alpha,\beta,\gamma)$-nonekspansif di ruang $CAT_p(0)$.
\begin{defn}\label{defn:abcnoncat}
    Diberikan $(X,d,G)$ adalah ruang $CAT _p(0)$ dan $W$ adalah himpunan bagian tak kosong dari $X$, pemetaan $f:W\to W$ disebut sebagai pemetaan $(\alpha,\beta,\gamma)$-nonekspansif jika terdapat bilangan real $\alpha,\beta,\gamma\in \mathbb{R}^+\cup\{0\}$ dengan $\alpha+\gamma\leq 1,\gamma\in[0,1)$ sehingga untuk setiap $x,y\in W$ berlaku
    \begin{align}
        d(Tx,Ty)\leq \alpha d(x,y)+\beta d(x,Tx)+\gamma d(x, Ty). \label{eq:abcnoncat}
    \end{align}
\end{defn}
    Selanjutnya, diberikan contoh dari pemetaan $(\alpha,\beta,\gamma)$-nonekspansif di ruang $CAT_p(0)$. Namun, sebelum itu diberikan dulu contoh ruang yang digunakan sebagai berikut.

    Contoh berikut ini merupakan contoh dari pemetaan $(\alpha,\beta,\gamma)$-nonekspansif dengan ruang $CAT_p(0)$ yang diberikan pada Contoh \ref{con:Catp}.

    \begin{exam}\label{con:abcnon}
        Diberikan $(X,d,G)$ adalah ruang $CAT _p(0)$ sebagaimana dalam Contoh \ref{con:Catp}. Diberikan pula $W=\{(w_1,w_2,0,0,\cdots)\mid w_1\in[1,5], w_2\in[1,125]\}\subset X$. dan pemetaan $T:W\to W$ yang didefinisikan sebagai 
        \begin{align*}
            T((w_1,w_2,0,0,\cdots))=\begin{cases}
                \qty(\frac{w_1+3}{4},\frac{(w_1+3)^3}{64},0,0,\cdots), \quad &\text{jika } w_1\in [1,3)\\
                \qty(\frac{w_1+2}{4}, \frac{(w_1+2)^3}{64},0,0,\cdots), \quad &\text{jika } w_1\in [3,5].
            \end{cases}
        \end{align*}
        Pemetaan $T$ adalah pemetaan $(\frac{1}{4},\frac{1}{3},0)$-nonekspansif, tetapi bukan pemetaan nonekspansif. Titik tetap dari $T$ adalah $(1,1,0,0,\cdots)$.
    \end{exam}
        Penjelasan dari Contoh \ref{con:abcnon} diuraikan berikut ini.\\
        Diambil sebarang $u,v\in W$ dengan $u=(u_1,u_2,0,0,\cdots), ~v=(v_1,v_2,0,0,\cdots)$. Dimisalkan $D(u,v)=\alpha d(u,v)+\beta d(u,Tu)+\gamma d(u,Tv)$. 
        \begin{enumerate}[label={\textbf{Kasus \arabic*.}},align=left]
            \item Untuk $u_1,v_1\in [1,3)$, diperoleh 
            \begin{align*}
                D(u,v) =& ~\frac{1}{4}\left(|u_1-v_1|^3+|u_1^3-u_2-v_1^3+v_2|^3\right)^{\frac{1}{3}} \\
                &+ \frac{1}{3} \left(\left|u_1-\frac{u_1+3}{4}\right|^3+\bigg|u_1^3-u_2-\qty(\frac{u_1+3}{4})^3+\frac{(u_1+3)^3}{64}\bigg|^3\right)^{\frac{1}{3}}\\
                &+0\times \left(\left|u_1-\frac{v_1+3}{4}\right|^3+\bigg|u_1^3-u_2 -\qty(\frac{v_1+3}{4})^3+\frac{(v_1+3)^3}{64}\bigg|^3\right)^{\frac{1}{3}}\\
                \geq &~ \frac{1}{4}\left(|u_1-v_1|^3\right)^{\frac{1}{3}}\\
                =&~ \left|\frac{u_1+3}{4}-\frac{v_1+3}{4}\right|\\
                =&~ \Bigg(\left|\frac{u_1+3}{4}-\frac{v_1+3}{4}\right|^3           \\    &+\bigg|\qty(\frac{u_1+3}{4})^3-\frac{(u_1+3)^3}{64}-\qty(\frac{v_1+3}{4})^3+\frac{(v_1+3)^3}{64}\bigg|^3\Bigg)^{\frac{1}{3}}\\
                =& ~ d(Tu,Tv).
            \end{align*}
            \item Untuk $u_1,v_1\in [3,5]$, diperoleh 
            \begin{align*}
                D(u,v) =& ~\frac{1}{4}\left(|u_1-v_1|^3+|u_1^3-u_2-v_1^3+v_2|^3\right)^{\frac{1}{3}} \\
                &+ \frac{1}{3} \Bigg(\left|u_1-\frac{u_1+2}{4}\right|^3+\bigg|u_1^3-u_2-\qty(\frac{u_1+2}{4})^3+\frac{(u_1+2)^3}{64}\bigg|^3\Bigg)^{\frac{1}{3}}\\
                &+0\times \Bigg(\left|u_1-\frac{v_1+2}{4}\right|^3+\bigg|u_1^3-u_2 -\qty(\frac{v_1+2}{4})^3+\qty(\frac{v_1+2}{4})^3\bigg|^3\Bigg)^{\frac{1}{3}}\\
                \geq &~ \frac{1}{4}\left(|u_1-v_1|^3\right)^{\frac{1}{3}}\\
                =&~ \left|\frac{u_1+2}{4}-\frac{v_1+2}{4}\right|\\
                =&~ \Bigg(\left|\frac{u_1+2}{4}-\frac{v_1+2}{4}\right|^3           \\    &+\bigg|\qty(\frac{u_1+2}{4})^3-\frac{(u_1+2)^3}{64}-\qty(\frac{v_1+2}{4})^3+\frac{(v_1+2)^3}{64}\bigg|^3\Bigg)^{\frac{1}{3}}\\
                =& ~ d(Tu,Tv).
            \end{align*}
            \item Untuk $u_1\in [3,5]$ dan $v_1\in[1,3)$, diperoleh 
            \begin{align*}
                D(u,v) =& ~\frac{1}{4}\left(|u_1-v_1|^3+|u_1^3-u_2-v_1^3+v_2|^3\right)^{\frac{1}{3}} \\
                &+ \frac{1}{3} \Bigg(\left|u_1-\frac{u_1+2}{4}\right|^3+\bigg|u_1^3-u_2-\left(\frac{u_1+2}{4}\right)^3+\frac{(u_1+2)^3}{64}\bigg|^3\Bigg)^{\frac{1}{3}}\\
                &+0\times \Bigg(\left|u_1-\frac{v_1+3}{4}\right|^3+\bigg|u_1^3-u_2 -\qty(\frac{v_1+3}{4})^3+\frac{(v_1+3)^3}{64}\bigg|^3\Bigg)^{\frac{1}{3}}\\
                \geq &~ \frac{1}{4}\left(|u_1-v_1|^3\right)^{\frac{1}{3}}+ \frac{1}{3}\left(\qty|u_1-\frac{u_1+2}{4}|^3\right)^{\frac{1}{3}}.
            \end{align*}
            Karena $u_1\in[3,5]$, maka $\frac{1}{12}|3u_1-2|\geq \frac{7}{12}> \frac{1}{4}$, sehingga 
            \begin{align*}
                D(u,v) \geq&~ \frac{1}{4}|u_1-v_1|+\frac{1}{4}\\
                \geq&~ \qty|\frac{u_1}{4}-\frac{v_1}{4}-\frac{1}{4}|\\
                =&~ \Bigg(\qty|\frac{u_1+2}{4}-\frac{v_1+3}{4}|^3\\
                &+\qty|\qty(\frac{u_1+2}{4})^3-\frac{(u_1+2)^3}{64}-\qty(\frac{v_1+2}{4})^3+\frac{(v_1+2)^3}{64}|^3\Bigg)^{\frac{1}{3}}\\
                =&~ d(Tu,Tv).
            \end{align*}
            \item Untuk $u_1\in [1,3)$ dan $v_1\in[3,5]$, diperoleh 
            \begin{align*}
                D(u,v) =& ~\frac{1}{4}\left(|u_1-v_1|^3+|u_1^3-u_2-v_1^3+v_2|^3\right)^{\frac{1}{3}} \\
                &+ \frac{1}{3} \Bigg(\left|u_1-\frac{u_1+3}{4}\right|^3+\bigg|u_1^3-u_2-\left(\frac{u_1+3}{4}\right)^3+\frac{(u_1+3)^3}{64}\bigg|^3\Bigg)^{\frac{1}{3}}\\
                &+0\times \Bigg(\left|u_1-\frac{v_1+2}{4}\right|^3+\bigg|u_1^3-u_2 -\qty(\frac{v_1+2}{4})^3+\frac{(v_1+2)^3}{64}\bigg|^3\Bigg)^{\frac{1}{3}}\\
                \geq &~ \frac{1}{4}\left(|u_1-v_1|^3\right)^{\frac{1}{3}}+ \frac{1}{3}\left(\qty|u_1-\frac{u_1+3}{4}|^3\right)^{\frac{1}{3}} + \frac{3}{4}\left(\qty|u_1-\frac{v_1+2}{4}|^3\right)^{\frac{1}{3}}\\
                =&~ \frac{1}{4}|u_1-v_1|+\frac{1}{4}|u_1-1|.
            \end{align*}
            Diperhatikan bahwa 
            \begin{align*}
                d(Tu,Tv) =&~\Bigg(\qty|\frac{u_1+3}{4}-\frac{v_1+2}{4}|^3\\
                &+ \qty|\qty(\frac{u_1+3}{4})^3-\frac{(u_1+3)^3}{64}-\qty(\frac{v_1+2}{4})^3+\frac{(v_1+3)^3}{64}|^3\Bigg)^{\frac{1}{3}}\\
                =& ~ \frac{1}{4}|u_1-v_1+1|,
            \end{align*}
            sehingga untuk membuktikan bahwa $D(u,v)\geq d(Tu,Tv)$, akan dibuktikan bahwa 
            \begin{align*}
                f(u_1,v_1):=|u_1-v_1|+|u_1-1|-|u_1-v_1+1|\geq 0,
            \end{align*}
            untuk setiap $u_1\in [1,3)$ dan $v_1\in[3,5]$. Diamati bahwa $1\leq u_1<3\leq v_1$, sehingga 
            \begin{align*}
                f(u_1,v_1)&=v_1-u_1+u_1-1-|u_1-v_1+1|\\
                &= v_1-1-|u_1-v_1+1|.
            \end{align*}
            Jika $u_1-v_1+1< 0$, didapatkan 
            \begin{align*}
                f(u_1,v_1) = v_1-1-\qty(-(u_1-v_1+1)) = u_1\geq 1. 
            \end{align*}
            Selanjutnya, jika $u_1-v_1+1\geq 0$, diperoleh $u_1\geq v_1-1\geq 2$
            \begin{align*}
                f(u_1,v_1) = v_1-1-u_1+v_1-1 = 2v_1-u_1-2,
            \end{align*}
            sehingga 
            \begin{align*}
                \dfrac{\partial f}{\partial u_1} &= -1\neq 0 \quad \text{dan} \quad \dfrac{\partial f}{\partial v_1} = 2\neq 0.
            \end{align*}
            Hal ini berarti nilai minimumnya terdapat pada titik-titik batas. Karena $u_1\in[1,3)$ dan $u_1\geq 2$ serta $v_1\in [3,5]$, diperoleh $-3<-u_1\leq-2$ sehingga
            \begin{align*}
                f(u_1,v_1) = 2v_1-u_1-2>2v_1-5\geq 1.
            \end{align*}
                Dari uraian tersebut didapatkan bahwa $f(u_1,v_1)\geq 0$ untuk setiap $u_1\in[1,3)$ dan $v_1\in [3,5]$ sehingga berlaku pula $\alpha d(u,v)+\beta d(u,Tu)+\gamma d(u,Tv)\geq d(Tu,Tv)$.
        \end{enumerate}
        Karena semua tinjauan kasus di atas menghasilkan $\alpha d(u,v)+\beta d(u,Tu)+\gamma d(u,Tv)\geq d(Tu,Tv)$ untuk setiap $u_1,v_1\in[1,5]$, maka $T$ adalah pemetaan $(\alpha,\beta,\gamma)$-nonekspansif dengan $\alpha=\frac{1}{4},~\beta=\frac{1}{3},$ dan $\gamma=0$. Akan tetapi, $T$ bukan pemetaan nonekspansif karena untuk $u=\qty(\frac{29}{10},\frac{29^3}{1000},0,0,\dots)$ dan $v=(3,27,0,0,\dots)$, didapatkan 
            \begin{align*}
            d(Tu,Tv)=&~\Bigg(\mqty|\frac{\frac{29}{10}+3}{4}-\frac{3+2}{4}|^3\\
            &+\bigg|\bigg(\frac{\frac{29}{10}+3}{4}\bigg)^3-\frac{(\frac{29}{10}+3)^3}{64}-\frac{(3+2)^3}{64}+\bigg(\frac{3+2}{4}\bigg)^3\bigg|^3\Bigg)^{\frac{1}{3}}\\
                =&~\frac{9}{40}\\
                >&~ \frac{1}{10}\\                =&~\qty(\left|3-\frac{29}{10}\right|^3+\qty|\qty(\frac{29}{10})^3-\frac{29^3}{1000}-3^3+27|^3)^{\frac{1}{3}} \\
                =&~d(u,v).
            \end{align*}
    Selanjutnya, untuk mendapatkan titik tetap dari $T$, dicari $w=(w_1,w_2,0,0,\cdots)\in W$ sehingga $T(w)=w$. Dari definisi $T$, terdapat dua kemungkinan, yaitu
    \begin{align*}
        &\text{(i) } w_1=\frac{w_1+3}{4}, ~w_2=\frac{(w_1+3)^3}{64}, \quad \text{atau}\\
        &\text{(ii) } w_1=\frac{w_1+2}{4}, ~w_2=\frac{(w_1+2)^3}{64}.
    \end{align*}
    Dari (i) diperoleh $w_1=w_2=1\in [1,3)$, sedangkan dari (ii) diperoleh $w_1=\frac{2}{3}\notin [3,5]$. Dengan demikian, titik tetap dari $T$ adalah $(1,1,0,0,\cdots)$.
    % \begin{exam}\label{con:abcnon}
        Diberikan $(X,d,G)$ adalah ruang $CAT _p(0)$ sebagaimana dalam Contoh \ref{con:Catp}. Diberikan pula $W=\{(w_1,w_2,0,0,\cdots)\mid w_1\in[1,5], w_2\in[1,125]\}\subset X$. dan pemetaan $T:W\to W$ yang didefinisikan sebagai 
        \begin{align*}
            T((w_1,w_2,0,0,\cdots))=\begin{cases}
                \qty(\frac{w_1+3}{4},\frac{(w_1+3)^3}{64},0,0,\cdots), \quad &\text{jika } w_1\in [1,3)\\
                \qty(\frac{w_1+2}{4}, \frac{(w_1+2)^3}{64},0,0,\cdots), \quad &\text{jika } w_1\in [3,5].
            \end{cases}
        \end{align*}
        Pemetaan $T$ adalah pemetaan $(\frac{1}{4},\frac{1}{3},0)$-nonekspansif, tetapi bukan pemetaan nonekspansif. Titik tetap dari $T$ adalah $(1,1,0,0,\cdots)$.
    \end{exam}
        Penjelasan dari Contoh \ref{con:abcnon} diuraikan berikut ini.\\
        Diambil sebarang $u,v\in W$ dengan $u=(u_1,u_2,0,0,\cdots), ~v=(v_1,v_2,0,0,\cdots)$. Dimisalkan $D(u,v)=\alpha d(u,v)+\beta d(u,Tu)+\gamma d(u,Tv)$. 
        \begin{enumerate}[label={\textbf{Kasus \arabic*.}},align=left]
            \item Untuk $u_1,v_1\in [1,3)$, diperoleh 
            \begin{align*}
                D(u,v) =& ~\frac{1}{4}\left(|u_1-v_1|^3+|u_1^3-u_2-v_1^3+v_2|^3\right)^{\frac{1}{3}} \\
                &+ \frac{1}{3} \left(\left|u_1-\frac{u_1+3}{4}\right|^3+\bigg|u_1^3-u_2-\qty(\frac{u_1+3}{4})^3+\frac{(u_1+3)^3}{64}\bigg|^3\right)^{\frac{1}{3}}\\
                &+0\times \left(\left|u_1-\frac{v_1+3}{4}\right|^3+\bigg|u_1^3-u_2 -\qty(\frac{v_1+3}{4})^3+\frac{(v_1+3)^3}{64}\bigg|^3\right)^{\frac{1}{3}}\\
                \geq &~ \frac{1}{4}\left(|u_1-v_1|^3\right)^{\frac{1}{3}}\\
                =&~ \left|\frac{u_1+3}{4}-\frac{v_1+3}{4}\right|\\
                =&~ \Bigg(\left|\frac{u_1+3}{4}-\frac{v_1+3}{4}\right|^3           \\    &+\bigg|\qty(\frac{u_1+3}{4})^3-\frac{(u_1+3)^3}{64}-\qty(\frac{v_1+3}{4})^3+\frac{(v_1+3)^3}{64}\bigg|^3\Bigg)^{\frac{1}{3}}\\
                =& ~ d(Tu,Tv).
            \end{align*}
            \item Untuk $u_1,v_1\in [3,5]$, diperoleh 
            \begin{align*}
                D(u,v) =& ~\frac{1}{4}\left(|u_1-v_1|^3+|u_1^3-u_2-v_1^3+v_2|^3\right)^{\frac{1}{3}} \\
                &+ \frac{1}{3} \Bigg(\left|u_1-\frac{u_1+2}{4}\right|^3+\bigg|u_1^3-u_2-\qty(\frac{u_1+2}{4})^3+\frac{(u_1+2)^3}{64}\bigg|^3\Bigg)^{\frac{1}{3}}\\
                &+\frac{3}{4} \Bigg(\left|u_1-\frac{v_1+2}{4}\right|^3+\bigg|u_1^3-u_2 -\qty(\frac{v_1+2}{4})^3+\qty(\frac{v_1+2}{4})^3\bigg|^3\Bigg)^{\frac{1}{3}}\\
                \geq &~ \frac{1}{4}\left(|u_1-v_1|^3\right)^{\frac{1}{3}}\\
                =&~ \left|\frac{u_1+2}{4}-\frac{v_1+2}{4}\right|\\
                =&~ \Bigg(\left|\frac{u_1+2}{4}-\frac{v_1+2}{4}\right|^3           \\    &+\bigg|\qty(\frac{u_1+2}{4})^3-\frac{(u_1+2)^3}{64}-\qty(\frac{v_1+2}{4})^3+\frac{(v_1+2)^3}{64}\bigg|^3\Bigg)^{\frac{1}{3}}\\
                =& ~ d(Tu,Tv).
            \end{align*}
            \item Untuk $u_1\in [3,5]$ dan $v_1\in[1,3)$, diperoleh 
            \begin{align*}
                D(u,v) =& ~\frac{1}{4}\left(|u_1-v_1|^3+|u_1^3-u_2-v_1^3+v_2|^3\right)^{\frac{1}{3}} \\
                &+ \frac{1}{3} \Bigg(\left|u_1-\frac{u_1+2}{4}\right|^3+\bigg|u_1^3-u_2-\left(\frac{u_1+2}{4}\right)^3+\frac{(u_1+2)^3}{64}\bigg|^3\Bigg)^{\frac{1}{3}}\\
                &+\frac{3}{4} \Bigg(\left|u_1-\frac{v_1+3}{4}\right|^3+\bigg|u_1^3-u_2 -\qty(\frac{v_1+3}{4})^3+\frac{(v_1+3)^3}{64}\bigg|^3\Bigg)^{\frac{1}{3}}\\
                \geq &~ \frac{1}{4}\left(|u_1-v_1|^3\right)^{\frac{1}{3}}+ \frac{1}{3}\left(\qty|u_1-\frac{u_1+2}{4}|^3\right)^{\frac{1}{3}}.
            \end{align*}
            Karena $u_1\in[3,5]$, maka $\frac{1}{12}|3u_1-2|\geq \frac{7}{12}> \frac{1}{4}$, sehingga 
            \begin{align*}
                D(u,v) \geq&~ \frac{1}{4}|u_1-v_1|+\frac{1}{4}\\
                \geq&~ \qty|\frac{u_1}{4}-\frac{v_1}{4}-\frac{1}{4}|\\
                =&~ \Bigg(\qty|\frac{u_1+2}{4}-\frac{v_1+3}{4}|^3\\
                &+\qty|\qty(\frac{u_1+2}{4})^3-\frac{(u_1+2)^3}{64}-\qty(\frac{v_1+2}{4})^3+\frac{(v_1+2)^3}{64}|^3\Bigg)^{\frac{1}{3}}\\
                =&~ d(Tu,Tv).
            \end{align*}
            \item Untuk $u_1\in [1,3)$ dan $v_1\in[3,5]$, diperoleh 
            \begin{align*}
                D(u,v) =& ~\frac{1}{4}\left(|u_1-v_1|^3+|u_1^3-u_2-v_1^3+v_2|^3\right)^{\frac{1}{3}} \\
                &+ \frac{1}{3} \Bigg(\left|u_1-\frac{u_1+3}{4}\right|^3+\bigg|u_1^3-u_2-\left(\frac{u_1+3}{4}\right)^3+\frac{(u_1+3)^3}{64}\bigg|^3\Bigg)^{\frac{1}{3}}\\
                &+\frac{3}{4} \Bigg(\left|u_1-\frac{v_1+2}{4}\right|^3+\bigg|u_1^3-u_2 -\qty(\frac{v_1+2}{4})^3+\frac{(v_1+2)^3}{64}\bigg|^3\Bigg)^{\frac{1}{3}}\\
                \geq &~ \frac{1}{4}\left(|u_1-v_1|^3\right)^{\frac{1}{3}}+ \frac{1}{3}\left(\qty|u_1-\frac{u_1+3}{4}|^3\right)^{\frac{1}{3}} + \frac{3}{4}\left(\qty|u_1-\frac{v_1+2}{4}|^3\right)^{\frac{1}{3}}\\
                =&~ \frac{1}{4}|u_1-v_1|+\frac{1}{4}|u_1-1|+\frac{3}{16}|4u_1-v_1-2|.
            \end{align*}
            Diperhatikan bahwa 
            \begin{align*}
                d(Tu,Tv) =&~\Bigg(\qty|\frac{u_1+3}{4}-\frac{v_1+2}{4}|^3\\
                &+ \qty|\qty(\frac{u_1+3}{4})^3-\frac{(u_1+3)^3}{64}-\qty(\frac{v_1+2}{4})^3+\frac{(v_1+3)^3}{64}|^3\Bigg)^{\frac{1}{3}}\\
                =& ~ \frac{1}{4}|u_1-v_1+1|,
            \end{align*}
            sehingga untuk membuktikan bahwa $D(u,v)\geq d(Tu,Tv)$, akan dibuktikan bahwa 
            \begin{align*}
                f(u_1,v_1):=|u_1-v_1|+|u_1-1|+\frac{3}{4}|4u_1-v_1-2|-|u_1-v_1+1|\geq 0,
            \end{align*}
            untuk setiap $u_1\in [1,3)$ dan $v_1\in[3,5]$. Diamati bahwa $1\leq u_1<3\leq v_1$, sehingga 
            \begin{align*}
                f(u_1,v_1)&=v_1-u_1+u_1-1+\frac{3}{4}|4u_1-v_1-2|-|u_1-v_1+1|\\
                &= v_1-1+\frac{3}{4}|4u_1-v_1-2|-|u_1-v_1+1|.
            \end{align*}
            Dari sini, diperoleh bahwa 
            \begin{align*}
                \dfrac{\partial f}{\partial u_1} &= \frac{3(4u_1-v_1-2)}{4|4u_1-v_1-2|}\times 4-\frac{u_1-v_1+1}{|u_1-v_1+1|}\\
                &= 3\sgn{(4u_1-v_1-2)}-\sgn{(u_1-v_1+1)}\neq 0,
            \intertext{serta}
            \frac{\partial f}{\partial v_1} &= 1+\frac{3(4u_1-v_1-2)}{4|4u_1-v_1-2|}\times (-1) -\frac{u_1-v_1+1}{|u_1-v_1+1|}\times (-1) \\
            &= 1 + \frac{3}{4}\sgn{(4u_1-v_1-2)} - \sgn{(u_1-v_1+1)}\neq 0,
            \end{align*}
            artinya $f(u_1,v_1)$ tidak mempunyai titik pelana. Walaupun begitu, $f(u_1,v_1)$ mempunyai titik kritis, yaitu saat $|4u_1-v_1-2|=0$ atau $|u_1-v_1+1|=0$. Hal ini berarti nilai minimumnya berada pada titik kritis atau pada titik-titik batas domainnya. 
            \begin{enumerate}
                \item[\textbf{(a)}] \textbf{pada titik kritis}\\
                Jika $|4u_1-v_1-2|=0$, didapat $v_1=4u_1-2$ sehingga 
                \begin{align*}
                    f(u_1,v_1) = 4u_1-3-3|u_1-1|.
                \end{align*}
                Karena $u_1\geq 1$ didapat 
                \begin{align*}
                    f(u_1,v_1)=4u_1-3-3u_1+3=u_1\geq 1 >0.
                \end{align*}
                Kemudian, jika $|u_1-v_1+1|=0$, didapat $v_1=u_1+1$ sehingga
                \begin{align*}
                    f(u_1,v_1)=u_1+\frac{9}{4}|u_1-1|\geq u_1\geq 1>0.
                \end{align*}
                \item[\textbf{(b)}] \textbf{pada titik batas domain}\\
                \begin{enumerate}
                    \item Jika $u_1=1$, didapatkan 
                    \begin{align*}
                        f(u_1,v_1)\geq f(1,v_1)=v_1-1+\frac{3}{4}|2-v_1|-|2-v_1|.
                    \end{align*}
                    Karena $v_1\geq 3$, diperoleh 
                    \begin{align*}
                        f(u_1,v_1) &\geq v_1-1-\frac{1}{4}(v_1-2)= \frac{3}{4}v_1-\frac{1}{3}\geq \frac{7}{4}>0.
                    \end{align*}
                    \item Jika $u_1\to 3^{-}$, didapatkan 
                    \begin{align*}
                        f(u_1,v_1)\geq \lim_{u_1\to 3^-} f(u_1,v_1)=v_1-1+\frac{3}{4}|10-v_1|-|4-v_1|.
                    \end{align*}
                    \begin{enumerate}
                        \item Jika $v_1\in[4,5]$, diperoleh 
                        \begin{align*}
                            f(u_1,v_1)&\geq v_1-1+\frac{3}{4}(10-v_1)-(v_1-4)\\
                            &= \frac{21}{2}-\frac{3}{4}v_1 \\
                            &\geq \frac{21}{2}-\frac{15}{4}\\
                            &=\frac{27}{4}\\
                            &>0.
                        \end{align*}
                        \item Jika $v_1\in [3,4]$, didapatkan 
                        \begin{align*}
                            f(u_1,v_1)&\geq v_1-1+\frac{3}{4}(10-v_1)-(4-v_1)=\frac{5}{4}v_1+\frac{5}{2}>0.
                        \end{align*}
                    \end{enumerate}
                    \item Jika $v_1=3$, didapatkan 
                    \begin{align*}
                        f(u_1,v_1)\geq f(u_1,3) = 2+\frac{3}{4}|4u_1-5|-|u_1-2|.
                    \end{align*}
                    Karena $u_1\in [1,3)$, maka nilai maksimum dari $|u_1-2|$ adalah 1 sehingga 
                    \begin{align*}
                        f(u_1,v_1)\geq 2+\frac{3}{4}|4u_1-5|-1\geq 1>0.
                    \end{align*}
                    \item Jika $v_1=5$, didapatkan 
                    \begin{align*}
                        f(u_1,v_1)\geq f(u_1,5)=4+\frac{3}{4}|4u_1-7|-|u_1-4|.
                    \end{align*}
                    Karena $u_1\in[1,3)$, maka 
                    \begin{align*}
                        f(u_1,v_1)&\geq 4+\frac{3}{4}|4u_1-7|-(4-u_1)\\
                        &=\frac{3}{4}|4u_1-7|+u_1\\
                        &\geq u_1\\
                        &>0.
                    \end{align*}
                \end{enumerate}
                Dari uraian tersebut didapatkan bahwa $f(u_1,v_1)\geq 0$ untuk setiap $u_1\in[1,3)$ dan $v_1\in [3,5]$ sehingga berlaku pula $\alpha d(u,v)+\beta d(u,Tu)+\gamma d(u,Tv)\geq d(Tu,Tv)$.
            \end{enumerate}
        \end{enumerate}
        Karena semua tinjauan kasus di atas menghasilkan $\alpha d(u,v)+\beta d(u,Tu)+\gamma d(u,Tv)\geq d(Tu,Tv)$ untuk setiap $u_1,v_1\in[1,5]$, maka $T$ adalah pemetaan $(\alpha,\beta,\gamma)$-nonekspansif dengan $\alpha=\frac{1}{4},~\beta=\frac{1}{3},$ dan $\gamma=\frac{3}{4}$. Akan tetapi, $T$ bukan pemetaan nonekspansif karena untuk $u=\qty(\frac{29}{10},\frac{29^3}{1000},0,0,\dots)$ dan $v=(3,27,0,0,\dots)$, didapatkan 
            \begin{align*}
            d(Tu,Tv)=&~\Bigg(\mqty|\frac{\frac{29}{10}+3}{4}-\frac{3+2}{4}|^3\\
            &+\bigg|\bigg(\frac{\frac{29}{10}+3}{4}\bigg)^3-\frac{(\frac{29}{10}+3)^3}{64}-\frac{(3+2)^3}{64}+\bigg(\frac{3+2}{4}\bigg)^3\bigg|^3\Bigg)^{\frac{1}{3}}\\
                =&~\frac{9}{40}\\
                >&~ \frac{1}{10}\\                =&~\qty(\left|3-\frac{29}{10}\right|^3+\qty|\qty(\frac{29}{10})^3-\frac{29^3}{1000}-3^3+27|^3)^{\frac{1}{3}} \\
                =&~d(u,v).
            \end{align*}
    Selanjutnya, untuk mendapatkan titik tetap dari $T$, dicari $w=(w_1,w_2,0,0,\cdots)\in W$ sehingga $T(w)=w$. Dari definisi $T$, terdapat dua kemungkinan, yaitu
    \begin{align*}
        &\text{(i) } w_1=\frac{w_1+3}{4}, ~w_2=\frac{(w_1+3)^3}{64}, \quad \text{atau}\\
        &\text{(ii) } w_1=\frac{w_1+2}{4}, ~w_2=\frac{(w_1+2)^3}{64}.
    \end{align*}
    Dari (i) diperoleh $w_1=w_2=1\in [1,3)$, sedangkan dari (ii) diperoleh $w_1=\frac{2}{3}\notin [3,5]$. Dengan demikian, titik tetap dari $T$ adalah $(1,1,0,0,\cdots)$.
    
    % \begin{exam}\label{con:abcnoncat}
    %     Diberikan ruang $(\ell_3,\norm{\cdot}_3)$ dengan metrik $d(u,v)=\left(\sum_{i=1}^{+\infty} |u_i-v_i|^3\right)^{\frac{1}{3}}$. Berdasarkan Contoh \ref{con:lpcatp}, ruang tersebut adalah ruang $CAT_p(0)$. Diberikan pula $W=\{(x,y,0,0,\cdots) ~|~ y\in[1,5]\}\subset \ell_3$ dan pemetaan $T:W\to W$ yang didefinisikan sebagai 
    %     \begin{align*}
    %         T\left((x,y,0,0,\cdots)\right) = \begin{cases}
    %             (\frac{1}{3}, \frac{y+3}{4},0,0,\cdots), \qquad &\text{jika } y\in [1,3)\\
    %             (\frac{1}{3}, \frac{y+2}{4},0,0,\cdots),\qquad &\text{jika } y\in [3,5].
    %         \end{cases}
    %     \end{align*}
    %     Pemetaan $T$ adalah pemetaan $(\frac{1}{4},\frac{1}{3},\frac{3}{4})$-nonekspansif tetapi bukan pemetaan nonekspansif.

    %     Penjelasan dari contoh tersebut diberikan di bawah ini.\\
    % % \end{exam}
    % % \begin{bukti} 
    % Diambil sebarang $u,v\in W$ dengan $u=(x_1,y_1,0,0,\cdots), ~v=(x_2,y_2,0,0,\cdots)$.
    %     \begin{enumerate}[label={\textbf{Kasus \arabic*.}},align=left]
    %         \item Untuk $y_1,y_2\in [1,3)$, diperoleh 
    %         \begin{align*}
    %             \alpha d(u,v)+\beta d(u,Tu)+\gamma d(u,Tv) =& ~\frac{1}{4}\left(|x_1-x_2|^3+|y_1-y_2|^3\right)^{\frac{1}{3}} \\
    %             &+ \frac{1}{3} \left(\left|x_1-\frac{1}{3}\right|^3+\left|y_1-\frac{y_1+3}{4}\right|^3\right)^{\frac{1}{3}}\\
    %             &+\frac{3}{4} \left(\left|x_1-\frac{1}{3}\right|^3+\left|y_1-\frac{y_2+3}{4}\right|^3\right)^{\frac{1}{3}}\\
    %             \geq &~ \frac{1}{4}\left(|x_1-x_2|^3+|y_1-y_2|^3\right)^{\frac{1}{3}}\\
    %             \geq &~ \frac{1}{4}\left(|y_1-y_2|^3\right)^{\frac{1}{3}}\\
    %             =&~ \left|\frac{y_1+3}{4}-\frac{y_2+3}{4}\right|\\
    %             =&~ \left(\left|\frac{1}{3}-\frac{1}{3}\right|^3+\left|\frac{y_1+3}{4}-\frac{y_2+3}{4}\right|^3\right)^{\frac{1}{3}}\\
    %             =& ~ d(Tu,Tv).
    %         \end{align*}
    %         \item Untuk $y_1,y_2\in [3,5]$, diperoleh 
    %         \begin{align*}
    %              \alpha d(u,v)+\beta d(u,Tu)+\gamma d(u,Tv) =& ~\frac{1}{4}\left(|x_1-x_2|^3+|y_1-y_2|^3\right)^{\frac{1}{3}} \\
    %             &+ \frac{1}{3} \left(\left|x_1-\frac{1}{3}\right|^3+\left|y_1-\frac{y_1+2}{4}\right|^3\right)^{\frac{1}{3}}\\
    %             &+\frac{3}{4} \left(\left|x_1-\frac{1}{3}\right|^3+\left|y_1-\frac{y_2+2}{4}\right|^3\right)^{\frac{1}{3}}\\
    %             \geq &~ \frac{1}{4}\left(|x_1-x_2|^3+|y_1-y_2|^3\right)^{\frac{1}{3}}\\
    %             \geq &~ \frac{1}{4}\left(|y_1-y_2|^3\right)^{\frac{1}{3}}\\
    %             =&~ \left|\frac{y_1+2}{4}-\frac{y_2+2}{4}\right|\\
    %             =&~ \left(\left|\frac{1}{3}-\frac{1}{3}\right|^3+\left|\frac{y_1+2}{4}-\frac{y_2+2}{4}\right|^3\right)^{\frac{1}{3}}\\
    %             =& ~ d(Tu,Tv).
    %         \end{align*}
    %         \item Untuk $y_1\in[3,5]$ dan $y_2\in [1,3)$, diperoleh 
    %         \begin{align*}
    %             \alpha d(u,v)+\beta d(u,Tu)+\gamma d(u,Tv) =& ~\frac{1}{4}\left(|x_1-x_2|^3+|y_1-y_2|^3\right)^{\frac{1}{3}} \\
    %             &+ \frac{1}{3} \left(\left|x_1-\frac{1}{3}\right|^3+\left|y_1-\frac{y_1+2}{4}\right|^3\right)^{\frac{1}{3}}\\
    %              &+\frac{3}{4} \left(\left|x_1-\frac{1}{3}\right|^3+\left|y_1-\frac{y_2+3}{4}\right|^3\right)^{\frac{1}{3}}\\
    %              \geq &~ ~\frac{1}{4}\left(|x_1-x_2|^3+|y_1-y_2|^3\right)^{\frac{1}{3}} \\
    %              &+ \frac{1}{3} \left(\left|x_1-\frac{1}{3}\right|^3+\left|y_1-\frac{y_1+2}{4}\right|^3\right)^{\frac{1}{3}}\\
    %             \geq & ~\frac{1}{4}|y_1-y_2| + \frac{1}{3} \left|y_1-\frac{y_1+2}{4}\right|\\
    %             =&~ \frac{1}{4}|y_1-y_2| + \frac{1}{12} \left|3y_1-2\right|.
    %         \end{align*}
    %         Karena $y_1\in[3,5]$, maka $\frac{1}{12}|3y_1-2|\geq \frac{7}{12}>\frac{1}{4}$, sehingga
    %         \begin{align*}
    %             \alpha d(u,v)+\beta d(u,Tu)+\gamma d(u,Tv) \geq & ~ \frac{1}{4}|y_1-y_2| +\frac{1}{4}\\
    %             \geq&~ \left|\frac{y_1}{4}-\frac{y_2}{4}-\frac{1}{4}\right|\\
    %             \geq &~ \left(\left|\frac{1}{3}-\frac{1}{3}\right|^3+\left|\frac{y_1+2}{4}-\frac{y_2+3}{4}\right|^3\right)^\frac{1}{3}\\
    %             =&~ d(Tu,Tv).
    %         \end{align*}
    %         \item Untuk $y_1\in[1,3)$ dan $y_2\in[3,5]$, diperoleh 
    %         \begin{align*}
    %             \alpha d(u,v)+\beta d(u,Tu)+\gamma d(u,Tv) =& ~\frac{1}{4}\left(|x_1-x_2|^3+|y_1-y_2|^3\right)^{\frac{1}{3}} \\
    %             &+ \frac{1}{3} \left(\left|x_1-\frac{1}{3}\right|^3+\left|y_1-\frac{y_1+3}{4}\right|^3\right)^{\frac{1}{3}}\\
    %              &+\frac{3}{4} \left(\left|x_1-\frac{1}{3}\right|^3+\left|y_1-\frac{y_2+2}{4}\right|^3\right)^{\frac{1}{3}}\\
    %              \geq &~ \frac{1}{4}|y_1-y_2|+\frac{1}{3}\left|y_1-\frac{y_1+3}{4}\right|\\
    %              &+\frac{3}{4}\left|y_1-\frac{y_2+2}{4}\right|
    %         \end{align*}
    %         Diperhatikan bahwa 
    %         $$d(Tu,Tv)=\left(\left|\frac{1}{3}-\frac{1}{3}\right|^3+\left|\frac{y_1+3}{4}-\frac{y_2+2}{4}\right|^3\right)^\frac{1}{3} = \frac{1}{4}|y_1-y_2+1|,$$ sehingga untuk membuktikan bahwa $ \alpha d(u,v)+\beta d(u,Tu)+\gamma d(u,Tv)\geq d(Tu,Tv)$, akan dibuktikan bahwa
    %         \begin{align*}
    %             f(y_1,y_2) &=|y_1-y_2|+|y_1-1|+\frac{3}{4}\left|4y_1-y_2-2\right| - |y_1-y_2+1| \geq 0,
    %         \end{align*}
    %         untuk setiap $y_1\in[1,3)$ dan $y_2\in [3,5]$. Diamati bahwa $1\leq y_1<3\leq y_2$, sehingga 
    %         \begin{align*}
    %             f(y_1,y_2) &= y_2-y_1+y_1-1 +\frac{3}{4}|4y_1-y_2-2|-|y_1-y_2+1|\\
    %             &= y_2 -1+\frac{3}{4}|4y_1-y_2-2| - |y_1-y_2+1|.
    %         \end{align*}
    %         Dari sini, diperoleh bahwa 
    %         \begin{align*}
    %         \frac{\partial f}{\partial y_1} &= \frac{3(4y_1-y_2-2)}{4|4y_1-y_2-2|}\times 4 -\frac{y_1-y_2+1}{|y_1-y_2+1|}\\
    %         &= 3\sgn{(4y_1-y_2-2)}-\sgn{(y_1-y_2+1)} \neq 0, 
    %         \intertext{serta}
    %         \frac{\partial f}{\partial y_2} &= 1+\frac{3(4y_1-y_2-2)}{4|4y_1-y_2-2|}\times (-1) -\frac{y_1-y_2+1}{|y_1-y_2+1|}\times (-1) \\
    %         &= 1 + \frac{3}{4}\sgn{(4y_1-y_2-2)} - \sgn{(y_1-y_2+1)}\neq 0,
    %         \end{align*}
    %         artinya $f(y_1,y_2)$ tidak memiliki titik pelana. Walaupun begitu, $f(y_1,y_2)$ memiliki titik kritis, yaitu saat $|4y_1-y_2-2|=0$ atau $|y_1-y_2+1|=0$. Hal ini berarti nilai minimumnya berada pada titik kritis atau pada titik-titik batas domainnya. 
    %         \begin{enumerate}
    %             \item[\textbf{(a)}] \textbf{pada titik kritis}\\
    %             Jika $|4y_1-y_2-2|=0$, didapat $y_2=4y_1-2$ sehingga 
    %             \begin{align*}
    %                 f(y_1,y_2) = 4y_1-3-3|y_1-1|.
    %             \end{align*}
    %             Karena $y_1\geq 1$ didapat 
    %             \begin{align*}
    %                 f(y_1,y_2)=4y_1-3-3y_1+3=y_1\geq 1 >0.
    %             \end{align*}
    %             Kemudian, jika $|y_1-y_2+1|=0$, didapat $y_2=y_1+1$ sehingga
    %             \begin{align*}
    %                 f(y_1,y_2)=y_1+\frac{9}{4}|y_1-1|\geq y_1\geq 1>0.
    %             \end{align*}
    %             \item[\textbf{(b)}] \textbf{pada titik batas domain}\\
    %             \begin{enumerate}
    %                 \item Jika $y_1=1$, didapatkan 
    %                 \begin{align*}
    %                     f(y_1,y_2)\geq f(1,y_2)=y_2-1+\frac{3}{4}|2-y_2|-|2-y_2|.
    %                 \end{align*}
    %                 Karena $y_2\geq 3$, diperoleh 
    %                 \begin{align*}
    %                     f(y_1,y_2) &\geq y_2-1-\frac{1}{4}(y_2-2)= \frac{3}{4}y_2-\frac{1}{3}\geq \frac{7}{4}>0.
    %                 \end{align*}
    %                 \item Jika $y_1\to 3^{-}$, didapatkan 
    %                 \begin{align*}
    %                     f(y_1,y_2)\geq \lim_{y_1\to 3^-} f(y_1,y_2)=y_2-1+\frac{3}{4}|10-y_2|-|4-y_2|.
    %                 \end{align*}
    %                 \begin{enumerate}
    %                     \item Jika $y_2\in[4,5]$, diperoleh 
    %                     \begin{align*}
    %                         f(y_1,y_2)&\geq y_2-1+\frac{3}{4}(10-y_2)-(y_2-4)\\
    %                         &= \frac{21}{2}-\frac{3}{4}y_2 \\
    %                         &\geq \frac{21}{2}-\frac{15}{4}\\
    %                         &=\frac{27}{4}\\
    %                         &>0.
    %                     \end{align*}
    %                     \item Jika $y_2\in [3,4]$, didapatkan 
    %                     \begin{align*}
    %                         f(y_1,y_2)&\geq y_2-1+\frac{3}{4}(10-y_2)-(4-y_2)=\frac{5}{4}y_2+\frac{5}{2}>0.
    %                     \end{align*}
    %                 \end{enumerate}
    %                 \item Jika $y_2=3$, didapatkan 
    %                 \begin{align*}
    %                     f(y_1,y_2)\geq f(y_1,3) = 2+\frac{3}{4}|4y_1-5|-|y_1-2|.
    %                 \end{align*}
    %                 Karena $y_1\in [1,3)$, maka nilai maksimum dari $|y_1-2|$ adalah 1 sehingga 
    %                 \begin{align*}
    %                     f(y_1,y_2)\geq 2+\frac{3}{4}|4y_1-5|-1\geq 1>0.
    %                 \end{align*}
    %                 \item Jika $y_2=5$, didapatkan 
    %                 \begin{align*}
    %                     f(y_1,y_2)\geq f(y_1,5)=4+\frac{3}{4}|4y_1-7|-|y_1-4|.
    %                 \end{align*}
    %                 Karena $y_1\in[1,3)$, maka 
    %                 \begin{align*}
    %                     f(y_1,y_2)&\geq 4+\frac{3}{4}|4y_1-7|-(4-y_1)\\
    %                     &=\frac{3}{4}|4y_1-7|+y_1\\
    %                     &\geq y_1\\
    %                     &>0.
    %                 \end{align*}
    %             \end{enumerate}
    %             Dari uraian tersebut didapatkan bahwa $f(y_1,y_2)\geq 0$ untuk setiap $y_1\in[1,3)$ dan $y_2\in [3,5]$ sehingga berlaku pula $\alpha d(u,v)+\beta d(u,Tu)+\gamma d(u,Tv)\geq d(Tu,Tv)$.
    %         \end{enumerate}
    %     \end{enumerate}
    %     Karena semua tinjauan kasus di atas menghasilkan $\alpha d(u,v)+\beta d(u,Tu)+\gamma d(u,Tv)\geq d(Tu,Tv)$ untuk setiap $y_1,y_2\in[1,5]$, maka $T$ adalah pemetaan $(\alpha,\beta,\gamma)$-nonekspansif dengan $\alpha=\frac{1}{4},~\beta=\frac{1}{3},$ dan $\gamma=\frac{3}{4}$. Akan tetapi, $T$ bukan pemetaan nonekspansif karena untuk $u=\qty(\frac{29}{10},\frac{29^3}{1000},0,0,\dots)$ dan $v=(3,27,0,0,\dots)$, didapatkan 
    %         \begin{align*}
    %         d(Tu,Tv)=&~\Bigg(\mqty|\frac{\frac{29}{10}+3}{4}-\frac{3+2}{4}|^3\\
    %         &+\qty|\qty(\frac{\frac{29}{10}+3}{4})^3-\frac{(\frac{29}{10}+3)^3}{64}-\frac{(3+2)^3}{64}+\qty(\frac{3+2}{4})^3|^3\Bigg)^{\frac{1}{3}}\\
    %             =&~\frac{9}{40}\\
    %             >&~ \frac{1}{10}\\                =&~\qty(\left|3-\frac{29}{10}\right|^3+\qty|\qty(\frac{29}{10})^3-\frac{29^3}{1000}-3^3+27|^3)^{\frac{1}{3}} \\
    %             =&~d(u,v).
    %         \end{align*}
    % \end{exam}
    Selanjutnya, berikut ini diberikan tiga lema penting untuk pembuktian konvergensi skema iterasi Sabri. 
    \begin{lemma}\label{lemma:tutx}
    Diberikan $(X,d,G)$ adalah ruang $CAT _p(0)$ dan $W$ adalah himpunan bagian tak kosong dari $X$, serta $T:W\to W$ adalah pemetaan $(\alpha,\beta,\gamma)$-nonekspansif. Jika $u$ adalah titik tetap dari $T$, maka untuk setiap $x\in W$ berlaku $d(Tx,Tu)\leq d(x,u)$. 
    \end{lemma}
    \begin{bukti}
        Diperhatikan bahwa $u$ titik tetap dari $T$ sehingga berlaku $u=Tu$, diperoleh 
        \begin{align*}
            d(u,Tx) = d(Tu,Tx)&\leq \alpha d(u,x)+\beta d(u,Tu)+\gamma d(u,Tx)\\
            &= \alpha d(u,x) +\beta d(u,u)+ \gamma d(u,Tx)\\
            &= \alpha d(x,u) + \gamma d(u,Tx).
        \end{align*}
        Dari sini didapatkan $(1-\gamma) d(u, Tx) \leq \alpha d(x,u)$. Kemudian, karena $\alpha+\gamma\leq 1$, didapat $\alpha \leq 1-\gamma$ sehingga 
        \begin{align*}
            \frac{\alpha}{1-\gamma}\leq 1.
        \end{align*}
        Akibatnya
        \begin{align}
           d(Tx,Tu)= d(Tu,Tx)= d(u, Tx) \leq \frac{\alpha}{1-\gamma} d(x,u) \leq d(x,u).\label{ineq:titap}
        \end{align}
    \end{bukti}
    \begin{thm}
        Diberikan $(X,d,G)$ adalah ruang $CAT_p(0)$ dan $W$ adalah himpunan bagian tak kosong dari $X$, serta $T:W\to W$ adalah pemetaan $(\alpha,\beta,\gamma)$-nonekspansif. Jika $\alpha+\gamma\neq 1$ dan $T$ memiliki titik tetap, maka titik tetap dari $T$ tunggal.
    \end{thm}
    \begin{bukti}
        Misalkan $u$ dan $v$ adalah titik tetap dari $T$, maka berdasarkan Ketaksamaan \eqref{ineq:titap} diperoleh 
        \begin{align*}
            d(u,v)=d(Tu,Tv) \leq \frac{\alpha}{1-\gamma} d(u,v) 
        \end{align*} 
        Karena $\alpha,\gamma\in [0,1]$, $\gamma\neq 1$, dan $\alpha+\gamma\neq 1$, didapatkan bahwa $0\leq \frac{\alpha}{1-\gamma}<1$ sehingga $d(u,v)=0$. Dengan demikian $u=v$, yang berarti bahwa titik tetap dari $T$ tunggal. 
    \end{bukti}
    \begin{lemma}\label{Lema:ineqabcnoneks}
        Diberikan $(X,d,G)$ adalah ruang $CAT _p(0)$ dan $W$ adalah himpunan bagian tak kosong dari $X$, serta $T:W\to W$ adalah pemetaan $(\alpha,\beta,\gamma)$-nonekspansif. Untuk setiap $x,y\in W$, ketaksamaan berikut ini berlaku:
        \begin{equation}\label{eq:ineqabcnoneks}
            d(x,Ty) \leq \frac{\alpha}{1-\gamma}d(x,y)+ \frac{1+\beta}{1-\gamma}d(x,Tx).
        \end{equation}
    \end{lemma}
    \begin{bukti}
        Diamati bahwa untuk setiap $x,y\in W$ berlaku 
        \begin{align*}
            d(x,Ty) &\leq d(x,Tx)+d(Tx,Ty)\\
            &\leq d(x,Tx) + \alpha d(x,y) +\beta d(x,Tx)+\gamma d(x,Ty)\\
            &= (1+\beta)d(x,Tx) + \alpha d(x,y) + \gamma d(x,Ty). 
        \end{align*}
        Akibatnya 
        \begin{align*}
            (1-\gamma)d(x,Ty) &\leq (1+\beta)d(x,Tx) + \alpha d(x,y) \\
            \Longleftrightarrow \qquad \quad d(x,Ty)&\leq \frac{\alpha}{1-\gamma}d(x,y) + \frac{1+\beta}{1-\gamma}d(x,Tx).
        \end{align*}
    \end{bukti}
    \begin{lemma}\label{Lema:demi}
        Diberikan $(X,d,G)$ adalah ruang $CAT _p(0)$ dan $W$ adalah himpunan bagian tak kosong dari $X$. Jika $T:W\to W$ adalah pemetaan $(\alpha,\beta,\gamma)$-nonekspansif, maka $T$ memiliki sifat \textbf{demiclosedness}. 
    \end{lemma}
    \begin{bukti}
        Diambil sebarang barisan $\{x_n\}\subseteq W$ yang terbatas dan konvergen-$\Delta$ ke $u_0\in W$, serta memenuhi $\lim_{n\to\infty} d(x_n,Tx_n)=0$. Berdasarkan definisi \ref{defn:konvD}, diperoleh bahwa $u_0\in A(\{x_n\})$. Kemudian, menggunakan ketaksamaan \eqref{eq:ineqabcnoneks}, didapatkan 
        \begin{align*}
            d(x_n, Tu_0)&\leq \dfrac{\alpha}{1-\gamma}d(x_n, u_0)+\dfrac{1+\beta}{1-\gamma}d(x_n, Tx_n).
        \end{align*}
        Selanjutnya, didapatkan bahwa 
        \begin{align*}
            \limsup_{n\to \infty} d(x_n, Tu_0)\leq \limsup_{n\to \infty} \dfrac{\alpha}{1-\gamma} d(x_n, u_0) \leq \limsup_{n\to \infty} d(x_n, u_0).
        \end{align*}
        Hal ini berarti $Tu_0\in A(\{x_n\})$. Kemudian, berdasarkan Lema \ref{Lema:asimtotik}, dipunyai bahwa $A(\{x_n\})$ tepat memiliki satu elemen, yang berarti $u_0=Tu_0$. Jadi $T$ memiliki sifat \textbf{demiclosedness}. 
    \end{bukti}

\section{Konvergensi Skema Iterasi Sabri untuk Pemetaan $(\alpha,\beta,\gamma)$-nonekspansif}
Pada bagian ini disajikan skema iterasi Sabri pada ruang $CAT_p(0)$ dan hasil terkait konvergensi dari skema iterasi Sabri untuk pemetaan $(\alpha,\beta,\gamma)$-nonekspansif di ruang tersebut. Terdapat dua lema penting yang digunakan untuk membuktikan konvergensi skema tersebut, yaitu pada Lema \ref{Lema:dxnx*} yang menunjukkan bahwa barisan $d(x_{n},x^*)$ adalah barisan turun, dengan $x^*$ adalah titik tetap dari pemetaan $(\alpha,\beta,\gamma)$-nonekspansif, serta pada Lema \ref{Lema:xntxn0} yang menunjukkan bahwa barisan limit dari barisan $d(x_n,Tx_n)$ adalah 0. Kemudian, disajikan syarat eksistensi titik tetap dari pemetaan $(\alpha,\beta,\gamma)$-nonekspansif pada Teorema \ref{thm:fixtnonemp}. Hasil konvergensi dari skema ini diberikan pada Teorema \ref{thm:konvD} yang memberikan syarat cukup untuk konvergensi$-\Delta$, sedangkan syarat cukup untuk konvergensi kuat diberikan oleh Teorema \ref{thm:konvK}. 

Berikut ini diberikan skema iterasi Sabri pada ruang $CAT_p(0)$.
\begin{defn}[\textbf{Skema Iterasi Sabri}]\label{defn:sabri}
Diberikan $(X,d,G)$ adalah ruang $CAT _p(0)$ dan $W$ adalah himpunan bagian tak kosong dari $X$ yang konveks. Untuk suatu pemetaan $T:W\to W$, $x_0\in W$, dan $n\in\mathbb{N}\cup \{0\} $ didefinisikan skema iterasi Sabri pada ruang $CAT_p(0)$ sebagai berikut:
    \begin{align}\label{eq:sabricat}
       \begin{cases}
            q_n&= T\left((1-c_n)x_n \oplus c_nTx_n\right),\\
        y_n&= T(Tq_n),\\
        x_{n+1} &= T\left((1-a_n)Tq_n\oplus a_n Ty_n\right).
       \end{cases}
    \end{align}
    dengan $\{a_n\}\subseteq [0,1]$ dan $\{c_n\}\subseteq [0,1]$.
\end{defn}
Berikut ini adalah diagram alir dari skema iterasi Sabri pada ruang $CAT_p(0)$.
\begin{figure}[H]
    \centering
    \tikzstyle{startstop} = [ellipse, 
minimum width=5.5cm, 
minimum height=0.5cm,
text centered, 
draw=black]

\tikzstyle{io} = [trapezium, 
trapezium stretches=true, % A later addition
trapezium left angle=70, 
trapezium right angle=110, 
minimum width=6cm, 
minimum height=0.5cm, text centered, 
draw=black]

\tikzstyle{process} = [rectangle, 
minimum height=0.5cm, 
text centered, 
text width=8.5cm, 
draw=black]

\tikzstyle{decision} = [diamond, 
minimum width=2cm, 
minimum height=0.5cm, 
text centered, 
aspect=1.8,
inner sep=2pt,
draw=black]
\tikzstyle{arrow} = [thick,->,>=stealth]

\begin{tikzpicture}[node distance=1.4cm]

% ===== Nodes =====
\node (start) [startstop] 
{Mulai};

\node (input) [io, below of=start,align=center,yshift=-0.3cm] 
    {Inisialisasi nilai awal $x_0\in W$, $n=0$, batas galat $\varepsilon>0$, \\
    batas iterasi $N$, dan parameter iterasi $\{a_n\}_{n=1}^{\infty},\{c_n\}_{n=1}^{\infty}\subseteq (0,1)$};

\node (qn) [process, below of=input,yshift=-0.3cm] 
{$q_n = T\big((1-c_n)x_n \oplus c_n T x_n\big)$};

\node (yn) [process, below of=qn] 
{$y_n = T(Tq_n)$};

\node (xn) [process, below of=yn] 
{$x_{n+1} = T\big((1-a_n)Tq_n \oplus a_n Ty_n\big)$};

\node (decision) [decision, below of=xn, yshift=-1.4cm,align=center] 
{Apakah\\$d(x_{n+1},x^*)<\varepsilon$?\\atau $n\geq N$?};

\node (stop) [startstop, below of=decision, yshift=-1.4cm] 
{Selesai};

% ===== Arrows =====
\draw [arrow] (start) -- (input);
\draw [arrow] (input) -- (qn);
\draw [arrow] (qn) -- (yn);
\draw [arrow] (yn) -- (xn);
\draw [arrow] (xn) -- (decision);
\draw [arrow] (decision) -- node[anchor=east]{Ya} (stop);

\draw [arrow] (decision.east) -- ++(3,0) 
node[anchor=west]{Tidak}
|- (qn.east);

\end{tikzpicture}

    \caption{Diagram Alir Skema Iterasi Sabri pada Ruang $CAT_p(0)$}
    \label{fig:skemasabricat0}
\end{figure}

Untuk selanjutnya, dinotasikan $W$ sebagai himpunan bagian tak kosong yang konveks dari ruang $CAT_p(0)$ $(X,d,G)$, serta $Fix(T)$ adalah himpunan semua titik tetap dari pemetaan $T$.
\begin{lemma}\label{Lema:dxnx*}
Diberikan $T:W\to W$ adalah pemetaan $(\alpha,\beta,\gamma)$-nonekspansif dengan $Fix(T)\neq \emptyset$. Jika $\{ x_n\}$ adalah barisan yang dikonstruksi melalui skema iterasi Sabri \eqref{eq:sabricat}, maka $d(x_{n+1},x^*)\leq d(x_n,x^*)$ untuk setiap $x^*\in Fix(T)$.
\end{lemma}
\begin{bukti}
    Diambil sebarang $x^*\in Fix(T)$, berdasarkan Lema \ref{lema:d,d^p} dan \ref{lemma:tutx}, didapatkan bahwa 
    \begin{align*}
        d(q_n,x^*)&=d\left(T\qty((1-c_n)x_n\oplus c_nTx_n), Tx^*\right)\\
        &\leq d\qty((1-c_n)x_n\oplus c_nTx_n,x^*)\\
        &\leq (1-c_n)d(x_n,x^*)+c_nd(Tx_n,x^*)\\
        &\leq (1-c_n)d(x_n,x^*)+c_nd(x_n,x^*)\\
        &= d(x_n,x^*).
    \end{align*}
    Dengan cara yang sama, didapatkan pula
    \begin{align*}
        d(y_n,x^*)&=d\qty(T(Tq_n),x^*)\\
        &\leq d(Tq_n,x^*)\\
        &\leq d(q_n,x^*)\\
        &\leq d(x_n,x^*),
    \end{align*}
    serta 
    \begin{align*}
        d(x_{n+1},x^*)&= d\qty(T\qty((1-a_n)Tq_n\oplus a_n Ty_n), Tx^*)\\
        &\leq d\qty((1-a_n)Tq_n\oplus a_n Ty_n,x^*)\\
        &\leq (1-a_n)d(Tq_n,x^*)+a_n d(Ty_n,x^*)\\
        &\leq (1-a_n)d(q_n,x^*)+a_nd(y_n,x^*)\\
        &\leq (1-a_n)d(x_n,x^*)+a_n d(x_n,x^*)\\
        &= d(x_n,x^*).
    \end{align*}
\end{bukti}
\begin{lemma}\label{Lema:xntxn0}
    Diberikan $T:W\to W$ adalah pemetaan $(\alpha,\beta,\gamma)$-nonekspansif dengan $Fix(T)\neq \emptyset$. Jika $\{x_n\}$ adalah barisan yang dikonstruksi melalui skema iterasi Sabri \eqref{eq:sabricat} dengan $\{a_n\},\{c_n\}\subset (0,1)$, maka $\lim_{n\to\infty} d(x_n,Tx_n)=0$.
\end{lemma}
\begin{bukti}
    Diambil sebarang $x^*\in Fix(T)$, berdasarkan Lema \ref{lema:d,d^p} dan \ref{lemma:tutx}, didapatkan bahwa
    \begin{align*}
        \qty(d(q_n,x^*))^p &= \qty(d\qty(T\qty((1-c_n)x_n\oplus c_n Tx_n),x^*))^p\\
        &\leq \qty(d\qty((1-c_n)x_n\oplus c_nTx_n,x^*))^p\\
        &\leq (1-c_n)\qty(d(x_n,x^*))^p+c_n \qty(d(Tx_n,x^*))^p-\frac{c_n(1-c_n)}{2^{p-1}}\qty(d(x_n,Tx_n))^p.
    \end{align*}
    Karena $a\leq c_k\leq b$, diperoleh $-c_n(1-c_n)\leq a(1-b)$ sehingga 
    \begin{align*}
        \qty(d(q_n,x^*))^p&\leq (1-c_n)\qty(d(x_n,x^*))^p+c_n \qty(d(x_n,x^*))^p-\frac{a(1-b)}{2^{p-1}} \qty(d(x_n,Tx_n))^p\\
        &= \qty(d(x_n,x^*))^p-\frac{a(1-b)}{2^{p-1}}\qty(d(x_n,Tx_n))^p.
    \end{align*}
    Dengan cara yang sama, didapatkan pula
    \begin{align*}
        \qty(d(y_n,x^*))^p&=\qty(d\qty(T(Tq_n),x^*))^p\\
        &\leq \qty(d(q_n,x^*))^p\\
        &\leq \qty(d(x_n,x^*))^p-\frac{a(1-b)}{2^{p-1}}\qty(d(x_n,Tx_n))^p,
    \end{align*}
    serta
    \begin{align*}
        \qty(d(x_{n+1},x^*))^p =&~ \qty(d\qty(T\qty((1-a_n)Tq_n\oplus a_n Ty_n),x^*))^p\\
        \leq &~\qty(d\qty((1-a_n)Tq_n\oplus a_n Ty_n,x^*))^p\\
        \leq&~ (1-a_n)\qty(d(Tq_n,x^*))^p+a_n \qty(d(Ty_n,x^*))^p-\frac{c_n(1-c_n)}{2^{p-1}}\qty(d(Tq_n,Ty_n))^p\\
        \leq&~ (1-a_n)\qty(d(q_n,x^*))^p+a_n \qty(d(y_n,x^*))^p\\
        \leq&~ (1-a_n)\qty[\qty(d(x_n,x^*))^p-\frac{a(1-b)}{2^{p-1}}\qty(d(x_n,Tx_n))^p]\\
        &~+a_n \qty[\qty(d(x_n,x^*))^p-\frac{a(1-b)}{2^{p-1}}\qty(d(x_n,Tx_n))^p]\\
        =&~\qty(d(x_n,x^*))^p-\frac{a(1-b)}{2^{p-1}}\qty(d(x_n,Tx_n))^p.
    \end{align*}
    Akibatnya 
    \begin{align*}
     \qty(d(x_n,Tx_n))^p&\leq \frac{2^{p-1}}{a(1-b)}\qty[\qty(d(x_n,x^*))^p-\qty(d(x_{n+1},x^*))^p].     
    \end{align*}
    Selanjutnya, berdasarkan Lema \ref{Lema:dxnx*}, diperoleh bahwa $\{d(x_n,x^*)\}$ adalah barisan turun dan terbatas di bawah dengan batas bawah 0, sehingga $\{d(x_n,x^*)\}$ konvergen. Hal ini berarti $u_n = \qty(d(x_n,x^*))^p$ juga konvergen dan diperoleh $\lim_{n\to\infty} (u_n-u_{n+1})=0$. Akibatnya, didapatkan 
    \begin{align*}
        0\leq \lim_{n\to \infty} \qty(d(x_n,Tx_n))^p&\leq \lim_{n\to \infty} \frac{2^{p-1}}{a(1-b)}\qty[\qty(d(x_n,x^*))^p-\qty(d(x_{n+1},x^*))^p] =0.
    \end{align*}
    Jadi $\lim_{n\to\infty}d(x_n,Tx_n)=0$.
\end{bukti}


Teorema berikut ini menyajikan syarat eksistensi titik tetap dari pemetaan $(\alpha,\beta,\gamma)$-nonekspansif.
\begin{thm}\label{thm:fixtnonemp}
    Diberikan $T:W\to W$ adalah pemetaan $(\alpha,\beta,\gamma)$-nonekspansif. Jika $\{x_n\}$ adalah barisan yang dikonstruksi melalui skema iterasi Sabri \eqref{eq:sabricat} sehingga $\{x_n\}$ terbatas dan $\lim_{n\to\infty} d(x_n,Tx_n)=0$, maka $Fix(T)\neq\emptyset$.
\end{thm}
\begin{bukti}
    Diambil sebarang $x^*\in A(\{x_n\})$, maka dengan ketaksamaan \ref{eq:ineqabcnoneks}, diperoleh bahwa 
    \begin{align*}
        d(x_n,Tx^*)\leq \frac{\alpha}{1-\gamma}d(x_n,x^*)+\frac{1+\beta}{1-\gamma}d(x_n,Tx_n).
    \end{align*}
    Karena $\lim_{n\to\infty} d(x_n,Tx_n)=0$, diperoleh 
    \begin{align*}
        \limsup_{n\to\infty}d(x_n, Tx^*)=\limsup_{n\to\infty}d(x_n,x^*)\leq \limsup_{n\to\infty} d(x_n,x^*).
    \end{align*}
    Dari sini didapatkan $Tx^*\in A(\{x_n\})$, sehingga berdasarkan Lema \ref{Lema:asimtotik}, diperoleh bahwa $A(\{x_n\})$ tepat memiliki satu elemen, yang berarti $x^*=Tx^*$. Dengan demikian $Fix(T)\neq \emptyset$.
\end{bukti}
Selanjutnya, dua teorema berikut ini menyajikan hasil konvergensi dari skema Sabri, yakni konvergensi-$\Delta$ dan konvergensi kuat, untuk aproksimasi titik tetap dari pemetaan $(\alpha,\beta,\gamma)$-nonekspansif
\begin{thm}\label{thm:konvD}
    Diberikan $(X,d,G)$ adalah ruang $CAT _p(0)$ yang lengkap dan $W$ adalah himpunan bagian tak kosong dari $X$ yang tertutup dan konveks. Jika $T:W\to W$ adalah pemetaan $(\alpha,\beta,\gamma)$-nonekspansif dengan $Fix(T)\neq \emptyset$ dan $\{x_n\}$ adalah barisan yang dikonstruksi melalui skema iterasi Sabri \eqref{eq:sabricat} dengan $\{a_n\},\{c_n\}\subset (0,1)$, maka $\{x_n\}$ konvergen-$\Delta$ ke suatu titik tetap dari $T$.
\end{thm}
\begin{bukti}
    Diamati bahwa $T$ merupakan pemetaan $(\alpha,\beta,\gamma)$-nonekspansif. Dari Lema \ref{Lema:dxnx*}, didapatkan bahwa $\{d(x_n,x^*)\}$ adalah barisan turun dan terbatas di bawah dengan batas bawah 0, sehingga $\{d(x_n,x^*)\}$ konvergen untuk setiap $x^*\in Fix(T)$.
    % \begin{align}
    %     \lim_{n\to\infty} d(x_n,x^*)=0 \quad \quad \text{untuk setiap } x^*\in Fix(T).
    % \end{align}
     Kemudian, karena $Fix(T)\neq \emptyset$, maka berdasarkan Lema \ref{Lema:xntxn0}, didapatkan 
     \begin{align}
         \lim_{n\to\infty} d(x_n,Tx_n)=0.
     \end{align}
     Selanjutnya, dari Lema \ref{Lema:demi}, didapat bahwa $T$ memiliki sifat demiclosedness. Hal ini berarti semua kondisi pada Teorema \ref{thm:kondisikonvD} terpenuhi, sehingga $\{x_n\}$ konvergen$-\Delta$.
\end{bukti}
Untuk mendapatkan hasil konvergensi kuat, diperlukan syarat tambahan, yaitu himpunan $W$ haruslah merupakan himpunan kompak. 
\begin{thm}\label{thm:konvK}
     Diberikan $(X,d,G)$ adalah ruang $CAT _p(0)$ yang lengkap dan $W$ adalah himpunan bagian tak kosong dari $X$ yang tertutup, konveks, dan kompak. Jika $T:W\to W$ adalah pemetaan $(\alpha,\beta,\gamma)$-nonekspansif dengan $Fix(T)\neq \emptyset$ dan $\{x_n\}$ adalah barisan yang dikonstruksi melalui skema iterasi Sabri \eqref{eq:sabricat} dengan $\{a_n\},\{c_n\}\subset (0,1)$, maka $\{x_n\}$ konvergen kuat ke suatu titik tetap dari $T$.
\end{thm}
\begin{bukti}
    Diperhatikan bahwa $W$ adalah himpunan kompak, ini berarti ada subbarisan $\{x_{n_k}\}$ dari $\{x_n\}$ yang konvergen kuat ke $x\in W$. Akibatnya $\{x_{n_k}\}$ juga konvergen$-\Delta$ ke $x\in W$.  Dengan menggunakan fakta bahwa $W$ himpunan kompak dan $Fix(T)\neq\emptyset$, maka berdasarkan Lema \ref{Lema:xntxn0}, diperoleh
    \begin{align*}
        \lim_{k\to\infty} d(x_{n_k},Tx_{n_k})=\lim_{k\to\infty} d(x_n,Tx_n)=0.
    \end{align*}
    Kemudian dengan menggunakan sifat demiclosedness dari $T$, didapat bahwa $x\in Fix(T)$. Selanjutnya dari Lema \ref{Lema:dxnx*} didapatkan
    % \begin{align*}
    %     \lim_{k\to\infty} d(x_{n_k},x)=0.
    % \end{align*}
    % Dengan menggunakan fakta bahwa $W$ himpunan kompak, didapatkan
    \begin{align*}
        \lim_{k\to\infty} d(x_{n},x)=0.
    \end{align*}
    Dengan demikian, barisan $\{x_n\}$ konvergen kuat ke titik tetap dari $T$.
\end{bukti}

\begin{thm}
    Laju
\end{thm}
\begin{bukti}
    Diperhatikan bahwa untuk setiap $x^*\in Fix(T)$ berlaku
    \begin{align*}
        d(q_n,x^*) &= d\qty(T\qty((1-c_n)x_n\oplus c_n Tx_n),Tx^*)\\
        &\leq \frac{\alpha}{1-\gamma}d\qty((1-c_n)x_n\oplus c_n Tx_n, x^* )\\
        &\leq \frac{\alpha}{1-\gamma}\qty[(1-c_n)d(x_n,x^*)+c_n d(Tx_n,x^*)]\\
        &\leq \frac{\alpha}{1-\gamma}\qty[(1-c_n)d(x_n,x^*)+\frac{\alpha c_n}{1-\gamma}d(x_n,x^*)]\\
        &= \frac{\alpha(1-\gamma)(1-c_n)+\alpha^2 c_n}{(1-\gamma)^2}d(x_n,x^*).
    \end{align*}
    \begin{align*}
        d(y_n,x^*) &= d\qty(T(Tq_n),x^*)\\
        &\leq \frac{\alpha}{1-\gamma}d(Tq_n,x^*)\\
        &\leq \qty(\frac{\alpha}{1-\gamma})^2 d(q_n,x^*).
    \end{align*}
    \begin{align*}
        d(x_{n+1},x^*) &= d\qty(T\qty((1-a_n)Tq_n\oplus a_n Ty_n),Tx^*)\\
        &\leq \frac{\alpha}{1-\gamma}d\qty((1-a_n)Tq_n\oplus a_n Ty_n,x^*)\\
        &\leq \frac{\alpha}{1-\gamma}\qty[(1-a_n)d(Tq_n,x^*)+a_n d(Ty_n,x^*)]\\
        &\leq \qty(\frac{\alpha}{1-\gamma})^2\qty[(1-a_n)d(q_n,x^*)+a_n d(y_n,x^*)]\\
        &\leq \qty(\frac{\alpha}{1-\gamma})^2\qty[(1-a_n)d(q_n,x^*)+a_n \qty(\frac{\alpha}{1-\gamma})^2d(q_n,x^*)]\\
        &= \qty(\frac{\alpha}{1-\gamma})^2\qty[\frac{(1-a_n)(1-\gamma)^2+a_n\alpha^2}{(1-\gamma)^2}d(q_n,x^*)]\\
        &\leq \frac{\alpha^2}{(1-\gamma)^4} \qty((1-a_n)(1-\gamma)^2+a_n\alpha^2)\times \frac{\alpha(1-\gamma)(1-c_n)+\alpha^2 c_n}{(1-\gamma)^2}d(x_n,x^*)\\
        &= \frac{\alpha^3}{(1-\gamma)^6}\qty((1-a_n)(1-\gamma)^2+a_n\alpha^2)\qty((1-c_n)(1-\gamma)+c_n \alpha)d(x_n,x^*).
    \end{align*}
    Karena $\alpha\leq 1-\gamma$, didapat 
    \begin{align*}
        d(x_{n+1},x^*) &\leq \frac{\alpha^3}{(1-\gamma)^6}\qty(1-\gamma)^2\qty(1-\gamma)d(x_n,x^*) = \frac{\alpha^3}{(1-\gamma)^3}d(x_n,x^*),
    \end{align*}
    sehingga 
    \begin{align*}
        d(x_{n+1},x^*) &\leq \qty(\frac{\alpha}{1-\gamma})^3 d(x_n,x^*)\\
        &\leq  \qty(\frac{\alpha}{1-\gamma})^6 d(x_{n-1},x^*)\\
        &~~\vdots\\
        &\leq  \qty(\frac{\alpha}{1-\gamma})^{3(n+1)} d(x_0,x^*).
    \end{align*}
    Jika terdapat $\alpha+\gamma\neq 1$, didapatkan $\lim_{n\to\infty} d(x_n,x^*)=0$.
\end{bukti}
\begin{bukti}
    \begin{align*}
        d(q_n,x^*) &=d\qty(T\qty((1-c_n)x_n\oplus c_n Tx_n),Tx^*)\\
        &\leq bd\qty((1-c_n)x_n\oplus c_n Tx_n, x^* )\\
        &\leq bm\qty[(1-c_n)d(x_n,x^*)+c_n d(Tx_n,x^*)]\\
        &\leq bm\qty[(1-c_n)d(x_n,x^*)+bc_nd(x_n,x^*)]\\
        &= bm\qty(1-c_n(1-b))d(x_n,x^*).
    \end{align*}
    \begin{align*}
        d(y_n,x^*) &= d\qty(T(Tq_n),x^*)\\
        &\leq bd(Tq_n,x^*)\\
        &\leq b^2 d(q_n,x^*).
    \end{align*}
    \begin{align*}
        d(x_{n+1},x^*) &= d\qty(T\qty((1-a_n)Tq_n\oplus a_n Ty_n),Tx^*)\\
        &\leq bd\qty((1-a_n)Tq_n\oplus a_n Ty_n,x^*)\\
        &\leq bm\qty[(1-a_n)d(Tq_n,x^*)+a_n d(Ty_n,x^*)]\\
        &\leq b^2m\qty[(1-a_n)d(q_n,x^*)+a_n d(y_n,x^*)]\\
        &\leq b^2m\qty[(1-a_n)d(q_n,x^*)+a_n b^2d(q_n,x^*)]\\
        &= b^2m\qty[(1-a_n(1-b^2))d(q_n,x^*)]\\
        &\leq b^2m (1-a_n(1-b^2))(bm(1-c_n(1-b)))d(x_n,x^*)\\
        &= b^3m^2 (1-a_n(1-b^2))(1-c_n(1-b))d(x_n,x^*).
    \end{align*}
\end{bukti}

\section{Percobaan Numerik}
Pada bagian ini, dilakukan percobaan numerik untuk menguji laju konvergensi dari skema iterasi Sabri dibanding dengan skema iterasi lainnya. Untuk percobaan ini, terlebih dahulu dikenalkan skema iterasi JK yang dikembangkan pada ruang $CAT_p(0)$ oleh Salisu sebagai berikut. 
\begin{defn}[\textbf{Skema Iterasi JK}]\cite{Salisu2022}\label{defn:jk}
% Diberikan $(X,d,G)$ adalah ruang $CAT _p(0)$ dan $W$ adalah himpunan bagian tak kosong dari $X$ yang konveks. 
Untuk suatu pemetaan $T:W\to W$, $x_0\in W$, dan $n\in\mathbb{N}\cup \{0\} $ didefinisikan skema iterasi JK pada ruang $CAT_p(0)$ sebagai berikut:
    \begin{align}\label{eq:jkcat}
       \begin{cases}
            q_n &= (1-c_n)x_n \oplus c_nTx_n,\\
        y_n&= Tq_n,\\
        x_{n+1} &= T\left((1-a_n)Tq_n\oplus a_n Ty_n\right),
       \end{cases}
    \end{align}
    dengan $\{a_n\}\subseteq [0,1]$ dan $\{c_n\}\subseteq [0,1]$.
\end{defn}
Selain itu, berikut ini dikenalkan pula skema iterasi Thakur, Abbas, dan Agarwal pada ruang $CAT_p(0)$. 
\begin{defn}[\textbf{Skema Iterasi Thakur}]\label{defn:thakur}
% Diberikan $(X,d,G)$ adalah ruang $CAT _p(0)$ dan $W$ adalah himpunan bagian tak kosong dari $X$ yang konveks. 
Untuk suatu pemetaan $T:W\to W$, $x_0\in W$, dan $n\in\mathbb{N}\cup \{0\} $ didefinisikan skema iterasi Thakur pada ruang $CAT_p(0)$ sebagai berikut:
    \begin{align}\label{eq:thakurcat}
       \begin{cases}
            q_n&= (1-c_n)x_n \oplus c_nTx_n,\\
        y_n&= T((1-a_n)x_n\oplus a_nq_n),\\
        x_{n+1} &= Ty_n,
       \end{cases}
    \end{align}
    dengan $\{a_n\}\subseteq [0,1]$ dan $\{c_n\}\subseteq [0,1]$.
\end{defn}
\begin{defn}[\textbf{Skema Iterasi Abbas}]\label{defn:abbas}
% Diberikan $(X,d,G)$ adalah ruang $CAT _p(0)$ dan $W$ adalah himpunan bagian tak kosong dari $X$ yang konveks. 
Untuk suatu pemetaan $T:W\to W$, $x_0\in W$, dan $n\in\mathbb{N}\cup \{0\} $ didefinisikan skema iterasi Abbas pada ruang $CAT_p(0)$ sebagai berikut:
    \begin{align}\label{eq:abbascat}
       \begin{cases}
            q_n&= (1-c_n)x_n \oplus c_nTx_n,\\
        y_n&= (1-b_n)Tx_n\oplus b_nTq_n,\\
        x_{n+1} &= (1-a_n)Ty_n\oplus a_n Tq_n,
       \end{cases}
    \end{align}
    dengan $\{a_n\}\subseteq [0,1]$, $\{b_n\}\subseteq [0,1]$, dan $\{c_n\}\subseteq [0,1]$.
\end{defn}
\begin{defn}[\textbf{Skema Iterasi Agarwal}]\label{defn:agarwal}
% Diberikan $(X,d,G)$ adalah ruang $CAT _p(0)$ dan $W$ adalah himpunan bagian tak kosong dari $X$ yang konveks. 
Untuk suatu pemetaan $T:W\to W$, $x_0\in W$, dan $n\in\mathbb{N}\cup \{0\} $ didefinisikan skema iterasi Agarwal pada ruang $CAT_p(0)$ sebagai berikut:
    \begin{align}\label{eq:agarwalcat}
       \begin{cases}
        y_n&= (1-c_n)x_n\oplus a_n Tx_n,\\
        x_{n+1} &= (1-a_n)Tx_n\oplus a_n Ty_n,
       \end{cases}
    \end{align}
    dengan $\{a_n\}\subseteq [0,1]$ dan $\{c_n\}\subseteq [0,1]$.
\end{defn}

Percobaan numerik dilakukan dengan perangkat lunak Google Colab yang menggunakan bahasa pemrograman Python. Digunakan beberapa variasi parameter, baik dari nilai awal $x_0$ maupun parameter di skema iterasinya, yaitu barisan $\{a_n\},\{b_n\},\{c_n\}$.
Pemetaan dan ruang yang digunakan dalam percobaan ini diambil dari Contoh \ref{con:abcnon} yang memiliki titik tetap $x^*=(1,1,0,0,\dots)$. 

Untuk hasil numerik tiap iterasi, digunakan batas galat $d(x_{n+1},x_n)<10^{-6}$, sedangkan untuk jumlah iterasi yang diperlukan untuk konvergen dari masing-masing skema iterasi, digunakan batas galat $d(x_{n+1},x_n)<10^{-16}$. Dalam percobaan ini, laju konvergensi skema iterasi Sabri \eqref{eq:sabricat} dibandingkan dengan Skema iterasi JK \eqref{eq:jkcat}, Thakur \eqref{eq:thakurcat}, Abbas \ref{eq:abbascat}, dan Agarwal \eqref{eq:agarwalcat}. Kode dari percobaan ini dapat dilihat pada Lampiran A.1.

Pada \ref{tab:sabjk}, \ref{tab:sabthkur}, \ref{tab:sababb}, dan \ref{tab:sabag} disajikan hasil numerik dari tiap iterasi dengan nilai awal $(2,3,0,0,\dots)$ dan parameter $a_n=0.92, b_n=0.83, c_n=0.81$ untuk setiap $n\in\mathbb{N}$ hingga batas galat $d(x_{n+1},x_n)<10^{-6}$.

\begin{table}[H]
    \centering
    \caption{Hasil numerik skema iterasi Sabri dan JK}
    \begin{tabular}{ccc}
\hline
\textbf{Iterasi} & \multicolumn{2}{c}{\textbf{Nilai $x_n$ hingga $d(x_{n+1},x_n)<10^{-6}$}} \\
\hline
\textbf{n} & \textbf{Sabri} & \textbf{JK} \\
\hline
0 & (2.000000, 3.000000, 0, 0, \dots) & (2.000000, 3.000000, 0, 0, \dots) \\
1 & (1.000843, 1.002532, 0, 0, \dots) & (1.007605, 1.022988, 0, 0, \dots) \\
2 & (1.000001, 1.000002, 0, 0, \dots) & (1.000058, 1.000174, 0, 0, \dots) \\
3 & (1.000000, 1.000000, 0, 0, \dots) & (1.000000, 1.000001, 0, 0, \dots) \\
4 & (1, 1, 0, 0, \dots) & (1.000000, 1.000000, 0, 0, \dots) \\
\hline
\end{tabular}
    \label{tab:sabjk}
\end{table}

\begin{table}[H]
    \centering
        \caption{Hasil numerik skema iterasi Sabri dan Thakur}
    \begin{tabular}{ccc}
\hline
\textbf{Iterasi} & \multicolumn{2}{c}{\textbf{Nilai $x_n$ hingga $d(x_{n+1},x_n)<10^{-6}$}} \\
\hline
\textbf{n} & \textbf{Sabri} & \textbf{Thakur} \\
\hline
0 & (2.000000, 3.000000, 0, 0, \dots) & (2.000000, 3.000000, 0, 0, \dots) \\
1 & (1.000843, 1.002532, 0, 0, \dots) & (1.027569, 1.085007, 0, 0, \dots) \\
2 & (1.000001, 1.000002, 0, 0, \dots) & (1.000760, 1.002282, 0, 0, \dots) \\
3 & (1.000000, 1.000000, 0, 0, \dots) & (1.000021, 1.000063, 0, 0, \dots) \\
4 & (1, 1, 0, 0, \dots) & (1.000001, 1.000002, 0, 0, \dots) \\
5 & (1, 1, 0, 0, \dots) & (1.000000, 1.000000, 0, 0, \dots) \\
\hline
\end{tabular}
    \label{tab:sabthkur}
\end{table}

\begin{table}[H]
    \centering
        \caption{Hasil numerik skema iterasi Sabri dan Abbas}
    \begin{tabular}{ccc}
\hline
\textbf{Iterasi} & \multicolumn{2}{c}{\textbf{Nilai $x_n$ hingga $d(x_{n+1},x_n)<10^{-6}$}} \\
\hline
\textbf{n} & \textbf{Sabri} & \textbf{Abbas} \\
\hline
0 & (2.000000, 3.000000, 0, 0, \dots) & (2.000000, 3.000000, 0, 0, \dots) \\
1 & (1.000843, 1.002532, 0, 0, \dots) & (1.092754, 1.304869, 0, 0, \dots) \\
2 & (1.000001, 1.000002, 0, 0, \dots) & (1.008603, 1.026033, 0, 0, \dots) \\
3 & (1.000000, 1.000000, 0, 0, \dots) & (1.000798, 1.002396, 0, 0, \dots) \\
4 & (1, 1, 0, 0, \dots) & (1.000074, 1.000222, 0, 0, \dots) \\
5 & (1, 1, 0, 0, \dots) & (1.000007, 1.000021, 0, 0, \dots) \\
6 & (1, 1, 0, 0, \dots) & (1.000001, 1.000002, 0, 0, \dots) \\
7 & (1, 1, 0, 0, \dots) & (1.000000, 1.000000, 0, 0, \dots) \\
\hline
\end{tabular}
    \label{tab:sababb}
\end{table}

% \begin{longtable}{ccc}
% \caption{\centering Hasil numerik skema iterasi Sabri dan Abbas}\\
% \hline
% \textbf{Iterasi} & \multicolumn{2}{c}{\textbf{Nilai $x_n$ hingga $d(x_{n+1},x_n)<10^{-6}$}} \\
% \hline
% \textbf{n} & \textbf{Sabri} & \textbf{Abbas} \\
% \hline
% \endfirsthead 
% \multicolumn{3}{c}{ \thetable{} -- Lanjutan dari halaman sebelumnya} \\
% \hline
% \textbf{n} & \textbf{Sabri} & \textbf{Abbas} \\
% \hline
% \endhead 
% \hline
% 0 & (2.000000, 3.000000, 0, 0, \dots) & (2.000000, 3.000000, 0, 0, \dots) \\
% 1 & (1.007270, 1.021970, 0, 0, \dots) & (1.109842, 1.367048, 0, 0, \dots) \\
% 2 & (1.000053, 1.000159, 0, 0, \dots) & (1.012065, 1.036634, 0, 0, \dots) \\
% 3 & (1.000000, 1.000001, 0, 0, \dots) & (1.001325, 1.003981, 0, 0, \dots) \\
% 4 & (1, 1, 0, 0, \dots) & (1.000146, 1.000437, 0, 0, \dots) \\
% 5 & (1, 1, 0, 0, \dots) & (1.000016, 1.000048, 0, 0, \dots) \\
% 6 & (1, 1, 0, 0, \dots) & (1.000002, 1.000005, 0, 0, \dots) \\
% 7 & (1, 1, 0, 0, \dots) & (1.000000, 1.000001, 0, 0, \dots) \\
% \hline
% \label{tab:sababb}
% \end{longtable}


\begin{table}[H]
    \centering
        \caption{Hasil numerik skema iterasi Sabri dan Agarwal}
    \begin{tabular}{ccc}
\hline
\textbf{Iterasi} & \multicolumn{2}{c}{\textbf{Nilai $x_n$ hingga $d(x_{n+1},x_n)<10^{-6}$}} \\
\hline
\textbf{n} & \textbf{Sabri} & \textbf{Agarwal} \\
\hline
0 & (2.000000, 3.000000, 0, 0, \dots) & (2.000000, 3.000000, 0, 0, \dots) \\
1 & (1.000843, 1.002532, 0, 0, \dots) & (1.110275, 1.368648, 0, 0, \dots) \\
2 & (1.000001, 1.000002, 0, 0, \dots) & (1.012161, 1.036927, 0, 0, \dots) \\
3 & (1.000000, 1.000000, 0, 0, \dots) & (1.001341, 1.004028, 0, 0, \dots) \\
4 & (1, 1, 0, 0, \dots) & (1.000148, 1.000444, 0, 0, \dots) \\
5 & (1, 1, 0, 0, \dots) & (1.000016, 1.000049, 0, 0, \dots) \\
6 & (1, 1, 0, 0, \dots) & (1.000002, 1.000005, 0, 0, \dots) \\
7 & (1, 1, 0, 0, \dots) & (1.000000, 1.000001, 0, 0, \dots) \\
8 & (1, 1, 0, 0, \dots) & (1.000000, 1.000000, 0, 0, \dots) \\
\hline
\end{tabular}
    \label{tab:sabag}
\end{table}

\begin{longtable}{cccccc}
\caption{\centering Galat $d(x_n, x^*)$}\label{tab:galat}\\
\hline
% \textbf{Iterasi} & \multicolumn{5}{c}{\textbf{Nilai Galat $d(x_n, x^*)$}} \\
% \hline
$\mathbf{n}$ & \textbf{Sabri} & \textbf{JK} & \textbf{Thakur} & \textbf{Abbas} & \textbf{Agarwal} \\
\hline
\endfirsthead 
\multicolumn{6}{c}{ \thetable{} -- Lanjutan dari halaman sebelumnya} \\
\hline
$\mathbf{n}$ & \textbf{Sabri} & \textbf{JK} & \textbf{Thakur} & \textbf{Abbas} & \textbf{Agarwal} \\
\hline
\endhead 
\hline
% \multicolumn{3}{|c|}{{Lanjut pada halaman berikutnya}} \\
\endfoot 
\hline
\endlastfoot
\hline
1 & $5.0\times10^{0}$ & $5.0\times10^{0}$ & $5.0\times10^{0}$ & $5.0\times10^{0}$ & $5.0\times10^{0}$ \\
2 & $8.4\times10^{-4}$ & $7.5\times10^{-3}$ & $2.6\times10^{-2}$ & $8.4\times10^{-2}$ & $9.8\times10^{-2}$ \\
3 & $7.1\times10^{-7}$ & $5.7\times10^{-5}$ & $7.3\times10^{-4}$ & $7.8\times10^{-3}$ & $1.0\times10^{-2}$ \\
4 & $5.9\times10^{-10}$ & $4.3\times10^{-7}$ & $2.0\times10^{-5}$ & $7.2\times10^{-4}$ & $1.1\times10^{-3}$ \\
5 & $5.0\times10^{-13}$ & $3.3\times10^{-9}$ & $5.6\times10^{-7}$ & $6.7\times10^{-5}$ & $1.3\times10^{-4}$ \\
6 & $4.4\times10^{-16}$ & $2.5\times10^{-11}$ & $1.5\times10^{-8}$ & $6.2\times10^{-6}$ & $1.4\times10^{-5}$ \\
7 & 0 & $1.9\times10^{-13}$ & $4.2\times10^{-10}$ & $5.7\times10^{-7}$ & $1.5\times10^{-6}$ \\
8 & 0 & $1.5\times10^{-15}$ & $1.1\times10^{-11}$ & $5.3\times10^{-8}$ & $1.7\times10^{-7}$ \\
9 & 0 & 0 & $3.2\times10^{-13}$ & $4.9\times10^{-9}$ & $1.9\times10^{-8}$ \\
10 & 0 & 0 & $9.1\times10^{-15}$ & $4.6\times10^{-10}$ & $2.1\times10^{-9}$ \\
11 & 0 & 0 & $2.2\times10^{-16}$ & $4.2\times10^{-11}$ & $2.3\times10^{-10}$ \\
12 & 0 & 0 & 0 & $3.9\times10^{-12}$ & $2.6\times10^{-11}$ \\
13 & 0 & 0 & 0 & $3.6\times10^{-13}$ & $2.8\times10^{-12}$ \\
14 & 0 & 0 & 0 & $3.4\times10^{-14}$ & $3.1\times10^{-13}$ \\
15 & 0 & 0 & 0 & $2.8\times10^{-15}$ & $3.5\times10^{-14}$ \\
16 & 0 & 0 & 0 & $4.4\times10^{-16}$ & $3.7\times10^{-15}$ \\
17 & 0 & 0 & 0 & 0 & $4.4\times10^{-16}$ \\
\hline
\end{longtable}
% \begin{table}[H]
%     \centering
%         \caption{Galat $d(x_n, x^*)$}
%     \label{tab:galat}
%     \begin{tabular}{cccccc}
%     \hline
%     $\mathbf{n}$ & \textbf{Sabri} & \textbf{JK} & \textbf{Thakur} & \textbf{Abbas} & \textbf{Agarwal} \\
%     \hline
%     1 & 0.007270 & 0.032859 & 0.056397 & 0.109842 & 0.225588 \\
%     2& 0.000053 & 0.001080 & 0.003181 & 0.012065 & 0.050890 \\
% 3 & 0.000000 & 0.000035 & 0.000179 & 0.001325 & 0.011480 \\
% 4 & NaN & 0.000001 & 0.000010 & 0.000146 & 0.002590 \\
% 5 & NaN & 0.000000 & 0.000001 & 0.000016 & 0.000584 \\
% 6 & NaN & NaN & NaN & 0.000002 & 0.000132 \\
% 7 & NaN & NaN & NaN & 0.000000 & 0.000030 \\
% 8 & NaN & NaN & NaN & NaN & 0.000007 \\
% 9 & NaN & NaN & NaN & NaN & 0.000002 \\
% 10 & NaN & NaN & NaN & NaN & 0.000000 \\
% \hline
%     \end{tabular}
% \end{table}
\begin{figure}[H]
    \centering
    \includegraphics[width=\linewidth]{Bab_4/galatpercobaan.png}
    \caption{Galat $d(x_{n+1},x_n)$ (dalam log) vs iterasi}
    \label{fig:galatperc}
\end{figure}
Dengan galat kurang dari $10^{-6}$, skema iterasi Sabri hanya membutuhkan 3 iterasi dibanding dengan JK (4 iterasi), Thakur (5 iterasi), Abbas (7 iterasi), dan Agarwal (8 iterasi). Pada \ref{tab:galat}, diberikan pula nilai galat $d(x_n,x^*)$ dari tiap iterasi. \ref{fig:galatperc} juga memberikan gambaran mengenai penurunan galatnya. 


Selanjutnya, pada \ref{tab:paramkonstan}, \ref{tab:paramturun}, \ref{tab:paramnaik}, \ref{tab:paramnaikturun}, dan \ref{tab:paramturunnaik}, disajikan jumlah iterasi yang diperlukan sampai batas galat $d(x_{n+1},x^*)<10^{-16}$ dari masing-masing skema iterasi dengan nilai awal dan parameter yang berbeda. Nilai awal dipilih secara acak dengan batasan nilainya berada pada domain pemetaan yang digunakan. Pada tabel-tabel tersebut parameter barisan $\{a_n\}$ dan $\{c_n\}$ berturut-turut dipilih dengan kondisi konstan-konstan, turun-turun, naik-naik, naik-turun, dan turun-naik. 
% \begin{longtable}{cccccc}
% \caption{\centering Iterasi dengan parameter $a_n = 0.42, b_n=0.83, c_n=0.31$}\\
% \hline
% $\mathbf{n}$ & \textbf{Sabri} & \textbf{JK} & \textbf{Thakur} & \textbf{Abbas} & \textbf{Agarwal} \\
% \hline
% \endfirsthead 
% \multicolumn{6}{c}{ \thetable{} -- Lanjutan dari halaman sebelumnya} \\
% \hline
% $\mathbf{n}$ & \textbf{Sabri} & \textbf{JK} & \textbf{Thakur} & \textbf{Abbas} & \textbf{Agarwal} \\
% \hline
% \endhead 
% \hline
% (2.0, 3.0, 0, 0, \dots) & 9 & 12 & 14 & 18 & 26 \\
% (3.0, 4.0, 0, 0, \dots) & 9 & 12 & 14 & 19 & 26 \\
% (4.0, 1.0, 0, 0, \dots) & 9 & 12 & 14 & 19 & 26 \\
% (1.5, 4.0, 0, 0, \dots) & 9 & 12 & 14 & 18 & 26 \\
% (2.7, 4.2, 0, 0, \dots) & 9 & 12 & 14 & 19 & 26 \\
% (5.0, 1.8, 0, 0, \dots) & 9 & 12 & 15 & 19 & 27 \\
% (3.1, 3.5, 0, 0, \dots) & 9 & 12 & 14 & 19 & 26 \\
% \hline
% \label{tab:paramkonstan}
% \end{longtable}
\begin{table}[H]
\caption{Iterasi dengan parameter $a_n = 0.92, b_n=0.83, c_n=0.81$}
    \centering
    \begin{tabular}{cccccc}
\hline
\textbf{Nilai Awal} & \multicolumn{5}{c}{\textbf{Jumlah Iterasi}} \\
\hline
$x_0$ & Sabri & JK & Thakur & Abbas & Agarwal \\
\hline
(2.0, 3.0, 0, 0, \dots) & 8 & 10 & 13 & 18 & 19 \\
(3.0, 4.0, 0, 0, \dots) & 8 & 10 & 13 & 18 & 19 \\
(4.0, 1.0, 0, 0, \dots) & 8 & 10 & 13 & 18 & 19 \\
(1.5, 4.0, 0, 0, \dots) & 8 & 10 & 12 & 18 & 19 \\
(2.7, 4.2, 0, 0, \dots) & 8 & 10 & 13 & 18 & 19 \\
(5.0, 1.8, 0, 0, \dots) & 8 & 10 & 13 & 19 & 20 \\
(3.1, 3.5, 0, 0, \dots) & 8 & 10 & 13 & 18 & 19 \\
\hline
\end{tabular}
    \label{tab:paramkonstan}
\end{table}
\begin{table}[H]
\caption{Iterasi dengan parameter $a_n =\frac{n^2}{n^3+1}, b_n=\frac{2}{n+3},c_n=\frac{4n+2}{{7n+4}}.$}
    \centering
    \begin{tabular}{cccccc}
\hline
\textbf{Nilai Awal} & \multicolumn{5}{c}{\textbf{Jumlah Iterasi}} \\
\hline
$x_0$ & Sabri & JK & Thakur & Abbas & Agarwal \\
\hline
(2.0, 3.0, 0, 0, \dots) & 10 & 13 & 15 & 16 & 28 \\
(3.0, 4.0, 0, 0, \dots) & 10 & 13 & 15 & 16 & 28 \\
(4.0, 1.0, 0, 0, \dots) & 10 & 13 & 16 & 17 & 28 \\
(1.5, 4.0, 0, 0, \dots) & 10 & 13 & 15 & 16 & 27 \\
(2.7, 4.2, 0, 0, \dots) & 10 & 13 & 15 & 17 & 28 \\
(5.0, 1.8, 0, 0, \dots) & 10 & 13 & 16 & 17 & 29 \\
(3.1, 3.5, 0, 0, \dots) & 10 & 13 & 15 & 16 & 28 \\
\hline
\end{tabular}
    \label{tab:paramturun}
\end{table}
\begin{table}[H]
\caption{Iterasi dengan parameter $a_n =1- \frac{\sqrt{4n+9}}{2n+13}, b_n=0.8,c_n=1-\frac{n^2}{\sqrt{n^7+3}}.$}
    \centering
    \begin{tabular}{cccccc}
\hline
\textbf{Nilai Awal} & \multicolumn{5}{c}{\textbf{Jumlah Iterasi}} \\
\hline
$x_0$ & Sabri & JK & Thakur & Abbas & Agarwal \\
\hline
(2.0, 3.0, 0, 0, \dots) & 8 & 10 & 13 & 16 & 19 \\
(3.0, 4.0, 0, 0, \dots) & 8 & 10 & 13 & 16 & 19 \\
(4.0, 1.0, 0, 0, \dots) & 8 & 11 & 13 & 16 & 19 \\
(1.5, 4.0, 0, 0, \dots) & 8 & 10 & 13 & 16 & 19 \\
(2.7, 4.2, 0, 0, \dots) & 8 & 11 & 13 & 16 & 19 \\
(5.0, 1.8, 0, 0, \dots) & 8 & 11 & 13 & 17 & 20 \\
(3.1, 3.5, 0, 0, \dots) & 8 & 10 & 13 & 16 & 19 \\
\hline
\end{tabular}
    \label{tab:paramnaik}
\end{table}
\begin{table}[H]
\caption{Iterasi dengan parameter $a_n = 1-\frac{n^3}{n^5+1}, b_n=\frac{1}{n+1},c_n=\frac{3n+4}{5n^2+4}.$}
    \centering
    \begin{tabular}{cccccc}
\hline
\textbf{Nilai Awal} & \multicolumn{5}{c}{\textbf{Jumlah Iterasi}} \\
\hline
$x_0$ & Sabri & JK & Thakur & Abbas & Agarwal \\
\hline
(2.0, 3.0, 0, 0, \dots) & 8 & 11 & 15 & 25 & 27 \\
(3.0, 4.0, 0, 0, \dots) & 8 & 11 & 15 & 25 & 27 \\
(4.0, 1.0, 0, 0, \dots) & 8 & 11 & 15 & 26 & 27 \\
(1.5, 4.0, 0, 0, \dots) & 8 & 11 & 14 & 25 & 26 \\
(2.7, 4.2, 0, 0, \dots) & 8 & 11 & 15 & 26 & 27 \\
(5.0, 1.8, 0, 0, \dots) & 8 & 11 & 15 & 26 & 27 \\
(3.1, 3.5, 0, 0, \dots) & 8 & 11 & 15 & 26 & 27 \\
\hline
\end{tabular}
    \label{tab:paramnaikturun}
\end{table}
\begin{table}[H]
\caption{Iterasi dengan parameter $a_n = \frac{n^5+1}{5n^7+9}, b_n=1-\frac{3}{n+5},c_n=1-\frac{\sqrt{n^3+8}}{5n^5+9}.$}
    \centering
    \begin{tabular}{cccccc}
\hline
\textbf{Nilai Awal} & \multicolumn{5}{c}{\textbf{Jumlah Iterasi}} \\
\hline
$x_0$ & Sabri & JK & Thakur & Abbas & Agarwal \\
\hline
(2.0, 3.0, 0, 0, \dots) & 9 & 11 & 16 & 13 & 29 \\
(3.0, 4.0, 0, 0, \dots) & 9 & 12 & 16 & 13 & 29 \\
(4.0, 1.0, 0, 0, \dots) & 9 & 12 & 16 & 14 & 29 \\
(1.5, 4.0, 0, 0, \dots) & 9 & 11 & 15 & 13 & 28 \\
(2.7, 4.2, 0, 0, \dots) & 9 & 12 & 16 & 13 & 29 \\
(5.0, 1.8, 0, 0, \dots) & 9 & 12 & 16 & 14 & 29 \\
(3.1, 3.5, 0, 0, \dots) & 9 & 12 & 16 & 13 & 29 \\
\hline
\end{tabular}
    \label{tab:paramturunnaik}
\end{table}
Hasil tersebut menunjukkan bahwa skema iterasi Sabri memiliki laju konvergensi yang lebih baik daripada yang lainnya secara numerik. Dengan parameter tertentu, skema tersebut hanya membutuhkan 8 iterasi untuk konvergen ke titik tetapnya dengan galat kurang dari $10^{-16}$. Secara analitik, skema iterasi Sabri lebih cepat karena nilai galat asimtotiknya, yaitu $\lambda\leq \frac{\alpha^3}{(1-\gamma)^3}$, sedangkan untuk JK dan Thakur adalah $\lambda\leq \frac{\alpha^2}{(1-\gamma)^2}$ yang ditunjukkan dalam teorema berikut ini. 
\begin{thm}
Diberikan $(X,d,G)$ adalah ruang $CAT _p(0)$ yang lengkap dan $W$ adalah himpunan bagian tak kosong dari $X$ yang tertutup, konveks, dan kompak. Jika $T:W\to W$ adalah pemetaan $(\alpha,\beta,\gamma)$-nonekspansif dengan $\alpha+\gamma<1$, $Fix(T)\neq \emptyset$, serta $\{x_n\}$ adalah barisan yang dikonstruksi melalui skema iterasi JK \eqref{eq:jkcat} dengan $\{a_n\},\{c_n\}\subset (0,1)$, maka $\{x_n\}$ memiliki laju konvergensi linear dengan galat asimtotik $\lambda\leq \frac{\alpha^2}{(1-\gamma)^2}$.
\end{thm}
\begin{bukti}
    Berdasarkan Lema \ref{lema:d,d^p} dan Lema \ref{lemma:tutx}, diperoleh bahwa untuk setiap $x^*\in Fix(T)$ berlaku
    \begin{align*}
        d(q_n,x^*) &= d\qty((1-c_n)x_n\oplus c_n Tx_n,Tx^*)\\
        &\leq (1-c_n)d(x_n,x^*)+c_n d(Tx_n,x^*)\\
        &\leq (1-c_n)d(x_n,x^*)+\frac{\alpha c_n}{1-\gamma}d(x_n,x^*)\\
        &= \frac{(1-\gamma)(1-c_n)+\alpha c_n}{(1-\gamma)}d(x_n,x^*),\\
        d(y_n,x^*) &= d\qty(Tq_n,x^*)\\
        &\leq \frac{\alpha}{1-\gamma}d(q_n,x^*),\\
        d(x_{n+1},x^*) &= d\qty(T\qty((1-a_n)Tq_n\oplus a_n Ty_n),Tx^*)\\
        &\leq \frac{\alpha}{1-\gamma}d\qty((1-a_n)Tq_n\oplus a_n Ty_n,x^*)\\
        &\leq \frac{\alpha}{1-\gamma}\qty[(1-a_n)d(Tq_n,x^*)+a_n d(Ty_n,x^*)]\\
        &\leq \qty(\frac{\alpha}{1-\gamma})^2\qty[(1-a_n)d(q_n,x^*)+a_n d(y_n,x^*)]\\
        &\leq \qty(\frac{\alpha}{1-\gamma})^2\qty[(1-a_n)d(q_n,x^*)+a_n \frac{\alpha}{1-\gamma}d(q_n,x^*)]\\
        &= \qty(\frac{\alpha}{1-\gamma})^2\qty[\frac{(1-a_n)(1-\gamma)+a_n\alpha}{(1-\gamma)}d(q_n,x^*)]\\
        &\leq \frac{\alpha^2}{(1-\gamma)^3} \qty((1-a_n)(1-\gamma)+a_n\alpha)\times \frac{(1-\gamma)(1-c_n)+\alpha c_n}{(1-\gamma)}d(x_n,x^*)\\
        &= \frac{\alpha^2}{(1-\gamma)^4}\qty((1-a_n)(1-\gamma)+a_n\alpha)\qty((1-c_n)(1-\gamma)+c_n \alpha)d(x_n,x^*).
    \end{align*}
    Karena $\alpha< 1-\gamma$, didapat 
    \begin{align}
        d(x_{n+1},x^*) &\leq \frac{\alpha^2}{(1-\gamma)^4}\qty(1-\gamma)\qty(1-\gamma)d(x_n,x^*) = \frac{\alpha^2}{(1-\gamma)^2}d(x_n,x^*),
    \end{align}
    sehingga 
    \begin{align*}
        \lim_{n\to\infty} \dfrac{d(x_{n+1},x^*)}{d(x_n,x^*)} \leq \frac{\alpha^2}{(1-\gamma)^2}<+\infty
    \end{align*}
    Hal ini berarti $\{x_n\}$ memiliki laju konvergensi linear dengan galat asimtotik $\lambda\leq \frac{\alpha^2}{(1-\gamma)^2}$.
\end{bukti}
\begin{thm}
Diberikan $(X,d,G)$ adalah ruang $CAT _p(0)$ yang lengkap dan $W$ adalah himpunan bagian tak kosong dari $X$ yang tertutup, konveks, dan kompak. Jika $T:W\to W$ adalah pemetaan $(\alpha,\beta,\gamma)$-nonekspansif dengan $\alpha+\gamma<1$, $Fix(T)\neq \emptyset$, serta $\{x_n\}$ adalah barisan yang dikonstruksi melalui skema iterasi Thakur \eqref{eq:thakurcat} dengan $\{a_n\},\{c_n\}\subset (0,1)$, maka $\{x_n\}$ memiliki laju konvergensi linear dengan galat asimtotik $\lambda\leq \frac{\alpha^2}{(1-\gamma)^2}$.
\end{thm}
\begin{bukti}
    Berdasarkan Lema \ref{lema:d,d^p} dan Lema \ref{lemma:tutx}, diperoleh bahwa untuk setiap $x^*\in Fix(T)$ berlaku
    \begin{align*}
        d(q_n,x^*) &= d\qty((1-c_n)x_n\oplus c_n Tx_n,Tx^*)\\
        &\leq (1-c_n)d(x_n,x^*)+c_n d(Tx_n,x^*)\\
        &\leq (1-c_n)d(x_n,x^*)+\frac{\alpha c_n}{1-\gamma}d(x_n,x^*)\\
        &= \frac{(1-\gamma)(1-c_n)+\alpha c_n}{(1-\gamma)}d(x_n,x^*),\\
        d(y_n,x^*) &= d\qty(T\qty((1-a_n)x_n\oplus a_nq_n),x^*)\\
        &\leq \frac{\alpha}{1-\gamma}d\qty((1-a_n)x_n\oplus a_n q_n,x^*)\\
        &\leq \frac{\alpha}{1-\gamma} \qty[(1-a_n)d(x_n,x^*)+a_n d(q_n,x^*)],\\
        d(x_{n+1},x^*) &= d(Ty_n,Tx^*)\\
        &\leq \frac{\alpha}{(1-\gamma)}d(y_n,x^*)\\
        &\leq \frac{\alpha^2}{(1-\gamma)^2} \qty[(1-a_n)d(x_n,x^*)+a_n d(q_n,x^*)]\\
        &\leq \frac{\alpha^2}{(1-\gamma)^2} \qty[(1-a_n)d(x_n,x^*)+a_n \frac{(1-\gamma)(1-c_n)+\alpha c_n}{(1-\gamma)}d(x_n,x^*)]
    \end{align*}
    Karena $\alpha< 1-\gamma$, didapat 
    \begin{align}
        d(x_{n+1},x^*) &\leq \frac{\alpha^2}{(1-\gamma)^2}d(x_n,x^*),
    \end{align}
    sehingga 
    \begin{align*}
        \lim_{n\to\infty} \dfrac{d(x_{n+1},x^*)}{d(x_n,x^*)} \leq \frac{\alpha^2}{(1-\gamma)^2}<+\infty
    \end{align*}
    Hal ini berarti $\{x_n\}$ memiliki laju konvergensi linear dengan galat asimtotik $\lambda\leq \frac{\alpha^2}{(1-\gamma)^2}$.
\end{bukti}
% Hasil ini secara intuitif dapat dijelaskan dari struktur iterasi Sabri yang melibatkan komposisi pemetaan nonekspansif $T$ secara berlapis pada setiap langkah iterasi, sehingga jarak barisan iterasi terhadap titik tetap berkurang lebih signifikan pada setiap iterasi dibandingkan dengan skema iterasi lainnya.

\section{Aplikasi pada Masalah Optimasi}
Dalam bagian ini disajikan bahwa skema iterasi Sabri dapat digunakan pada masalah optimasi, khususnya untuk masalah minimalisasi dan rekonstruksi citra. 
\subsection{Masalah Minimalisasi}
Diberikan suatu himpunan tak kosong $X$ dan $f:X\to\mathbb{R}\cup\{\infty\}$ adalah suatu pemetaan. Masalah pencarian titik yang meminimumkan fungsi $f$ dapat diformulasikan sebagai mencari nilai 
\begin{align}\label{eq:minprob}
    x\in X \quad \text{sehingga}\quad f(x)\leq f(y), \quad \text{untuk setiap}\quad y\in X.
\end{align}
Permasalahan ini merupakan permasalahan penting dalam bidang optimasi dan analisis tak linier. Selanjutnya, diperkenalkan operator resolvent dari suatu fungsi.
\begin{defn}
    Diberikan $(X,d,G)$ adalah ruang $CAT _p(0)$ dan $f:X\to \mathbb{R}\cup\{+\infty\}$ adalah suatu fungsi. Untuk $\lambda>0$, operator resolvent $\lambda$ dari $f$ didefinisikan sebagai
    \begin{align}
        J^f_{\lambda}(x) = \text{argmin}_{y\in X} \qty[f(y)+\frac{1}{2\lambda}\qty(d(x,y))^2].
    \end{align}
\end{defn}
\begin{exam}
    Diberikan $(\mathbb{R}^2,\|\cdot\|_2)$ adalah ruang Banach yang juga merupakan $CAT_p(0)$. Misalkan $f:\mathbb{R}^2\to \mathbb{R}\cup\{+\infty\}$ adalah fungsi yang didefinisikan sebagai $f(x_1,x_2) = |x_1| + |x_2|$ untuk setiap $(x_1,x_2)\in \mathbb{R}^2$. Untuk $\lambda>0$ dan $x=(x_1,x_2)\in \mathbb{R}^2$, operator resolvent $J^f_{\lambda}$ dari $f$ adalah sebagai berikut.
    \begin{align*}
        J^f_{\lambda}(x) &= \text{argmin}_{y\in \mathbb{R}^2} \qty[|y_1| + |y_2| + \frac{1}{2\lambda}\qty(\norm{(x_1,y_2)-(y_1,y_2)}_2)^2]\\
        &= \text{argmin}_{y\in \mathbb{R}^2} \qty[|y_1| + |y_2| + \frac{1}{2\lambda}\qty((x_1 - y_1)^2 + (x_2 - y_2)^2)].
    \end{align*}
    % Dengan memisahkan variabel $y_1$ dan $y_2$, diperoleh
    % \begin{align*}  
    %     J^f_{\lambda}(x) &= \left(\text{argmin}_{y_1\in \mathbb{R}} \qty[|y_1| + \frac{1}{2\lambda}(x_1 - y_1)^2], \text{argmin}_{y_2\in \mathbb{R}} \qty[|y_2| + \frac{1}{2\lambda}(x_2 - y_2)^2]\right)\\
    %     &= \left(sgn(x_1)\max\qty(|x_1| - \lambda, 0), sgn(x_2)\max\qty(|x_2| - \lambda, 0)\right).
    % \end{align*}
\end{exam}

Untuk fungsi yang memenuhi kondisi konveks dan \textit{proper lower semi-continuous}, himpunan solusi dari masalah \eqref{eq:minprob} sama dengan himpunan titik tetap dari operator resolvent $J^f_{\lambda}$ (lihat proposisi 6.5 pada \cite{ArizaRuiz2014}). Berikut ini diberikan definisi dari fungsi yang memenuhi kondisi tersebut dengan contoh yang diberikan pada Contoh \ref{con:fkonvk}.

\begin{defn}\cite{Khamsi2017}\label{def:konvgeodesik}
    Diberikan $(X,d,G)$ adalah ruang $CAT _p(0)$. Suatu fungsi $f:X\to \mathbb{R}\cup \{+\infty\}$ disebut konveks secara geodesik jika untuk setiap $t\in(0,1)$ dan $x,y\in X$, berlaku
    \begin{align*}
        f(tx\oplus (1-t)y)\leq tf(x)+(1-t)f(y).
    \end{align*}
\end{defn}
\begin{defn}\cite{Salisu2022}
    Diberikan $(X,d,G)$ adalah ruang $CAT _p(0)$. Suatu fungsi $f:X\to \mathbb{R}\cup \{+\infty\}$ disebut \textit{proper} jika himpunan $D(f):=\{x\in X\mid f(x)<+\infty\}\neq \emptyset$.
\end{defn}
\begin{defn}\cite{Salisu2022}
    Diberikan $(X,d,G)$ adalah ruang $CAT _p(0)$. Suatu fungsi $f:X\to \mathbb{R}\cup \{+\infty\}$ disebut \textit{lower semi-continuous} pada suatu titik $x\in D(f)$ jika $f(x)\leq \liminf_{n\to\infty} f(x_n)$ untuk setiap barisan $\{x_n\}_{n=1}^{\infty}$ yang konvergen di $D(f)$ dengan limit $x\in X$. Jika $f$ \textit{lower semi-continuous} pada setiap titik di $D(f)$, maka $f$ disebut \textit{lower semi-continuous} pada $X$.
\end{defn}

Berdasarkan hal tersebut, didapatkan Teorema berikut. 
\begin{thm}\label{thm:apl1}
    Diberikan $(X,d,G)$ adalah ruang $CAT _p(0)$ dan $f:X\to\mathbb{R}\cup\{+\infty\}$ adalah fungsi yang memenuhi kondisi konveks dan \textit{proper lower semi-continuous}. Untuk $x\in X$, barisan $\{x_n\}_{n=1}^{\infty}$ yang didefinisikan sebagai 
    \begin{align}
        \begin{cases}
            q_n &= J^f_{\lambda}\qty((1-c_n)x_n\oplus c_n J^f_{\lambda} (x_n))\\
            y_n &= J^f_{\lambda} \qty(J^f_{\lambda} (q_n))\\
            x_{n+1} &=J^f_{\lambda} \qty((1-a_n)J^f_{\lambda} (q_n)\oplus a_n J^f_{\lambda} (y_n)),
        \end{cases}
    \end{align}
    dengan $\{a_n\}_{n=1}^{\infty},\{c_n\}_{n=1}^{\infty}\subseteq [a,b]\subset (0,1)$ konvergen-$\Delta$ ke solusi dari permasalahan \eqref{eq:minprob}. Jika $X$ kompak, maka $\{x_n\}_{n=1}^{\infty}$ konvergen kuat. 
\end{thm}
\begin{bukti}
    Dengan menggunakan Lema 4 di \cite{Jost1995}, diketahui bahwa $J^f_{\lambda}$ merupakan pemetaan nonekspansif, sehingga merupakan pemetaan $(\alpha,\beta,\gamma)$-nonekspansif juga. Akibatnya, dengan Teorema \ref{thm:konvD} diperoleh bahwa $\{x_n\}_{n=1}^{\infty}$ konvergen-$\Delta$ ke titik tetap dari $J^f_{\lambda}$. Jika $X$ kompak, Teorema \ref{thm:konvK}, menjamin bahwa konvergensinya kuat. Karena titik tetap dari $J^f_{\lambda}$ sama dengan solusi dari permasalahan \eqref{eq:minprob}, artinya Teorema \ref{thm:apl1} terbukti. 
\end{bukti}

Sebagai gambaran, dilakukan simulasi untuk contoh berikut ini. 
\begin{exam}\label{con:fkonvk}
    Diberikan $(X,d,G)$ adalah ruang $CAT _p(0)$ sebagaimana Contoh \ref{con:Catp} dan $W=\{(x_1,x_2,0,0,\dots)\mid x_1,x_2\in X\}$. Suatu fungsi $f:W\to \mathbb{R}\cup\{+\infty\}$ yang didefinisikan sebagai 
    \begin{align}
        f\qty(x) = 20\qty|x_2-x_1^3|^3+|26-x_1|^3, \quad \text{untuk}\quad x=(x_1,x_2,0,0,\dots)\in W,
    \end{align}
    merupakan fungsi tak konveks pada definisi klasiknya, tetapi konveks secara geodesik dengan geodesik sebagaimana pada Contoh \ref{con:Catp}. Lebih lanjut, $f$ merupakan fungsi yang \textit{proper lower semicontinuous} dan mencapai nilai minimum saat $x_1=26$ dan $x_2=26^3$. 
\end{exam}
Penjelasan dari Contoh \ref{con:fkonvk} diberikan sebagai berikut. 
\begin{enumerate}
    \item Pertama, ditunjukkan bahwa $f$ tak konveks dalam definisi klasiknya, yaitu $f$ memenuhi ketaksamaan
    \begin{align*}
        f\qty((1-t)w + tz) \leq (1-t)f(w) + t f(z),
    \end{align*}
    untuk setiap $t\in(0,1)$ dan $w,z\in W$. Untuk menunjukkan bahwa $f$ tidak konveks dalam definisi klasik, digunakan matriks Hessian, yaitu 
    
    \begin{align*}
        H = \begin{bmatrix}
           \dfrac{\partial^2 f}{\partial x_1^2} & \dfrac{\partial^2 f}{\partial x_1x_2} \vspace*{0.3cm}\\ 
           \dfrac{\partial^2 f}{\partial x_2x_1} & \dfrac{\partial^2 f}{\partial x_2^2}
        \end{bmatrix}.
    \end{align*}
    Ingat kembali bahwa suatu fungsi $f:W\to \mathbb{R}$ adalah konveks jika dan hanya jika matriks Hessian $H$ adalah matriks semidefinit positif untuk setiap $x\in W$ (lihat \cite{Boyd2004}).
    Dapat dihitung bahwa 
    \begin{align*}
        \dfrac{\partial f}{\partial x_1} &= -180(x_2x_1^2-x_1^5)|x_2-x_1^3|-3(26-x_1)|26-x_1|.\\
        \dfrac{\partial^2 f}{\partial x_1^2} &= -180\qty((2x_1x_2-5x_1^4)|x_2-x_1^3|-\frac{3x_1^4(x_2-x_1^3)^2}{|x_2-x_1^3|}) + \frac{6(26-x_1)^2}{|26-x_1|}.\\
        \dfrac{\partial^2 f}{\partial x_1x_2} &= \frac{-360x_1^2(x_2-x_1^3)^2}{|x_2-x_1^3|}.\\
        \dfrac{\partial f}{\partial x_2} &= 60(x_2-x_1^3)|x_2-x_1|^3.\\
        \dfrac{\partial^2 f}{\partial x_2 x_1} &= \frac{-360x_1^2(x_2-x_1^3)^2}{|x_2-x_1^3|}.\\
        \dfrac{\partial^2 f}{\partial x_2^2} &= 60\qty(|x_2-x_1^3|+\frac{(x_2-x_1^3)^2}{|x_2-x_1^3|}).
    \end{align*}
    Jika diambil $x_1=1$ dan $x_2=5$, diperoleh 
    \begin{align*}
        H = \begin{bmatrix}
            -1290 & -1440\\
            -1440 & 480
        \end{bmatrix}.
    \end{align*}
    Diperhatikan bahwa $H$ mempunyai persamaan karakteristik $p(\lambda)=\lambda^2+810\lambda-2692800$ dengan akar-akar 
    \begin{align*}
        \lambda_{1,2} = \frac{-810 \pm \sqrt{810^2-4(2692800)}}{2},
    \end{align*}
    yang jelas mempunyai akar negatif. Jadi ada $x\in W$ sehingga matriks Hessian $H$ mempunyai nilai eigen negatif, yang berarti $H$ bukan matriks semidefinit positif. Akibatnya $f$ bukanlah fungsi konveks dalam definisi klasik. 
    \item Kedua, ditunjukkan bahwa $f$ memenuhi Definisi \ref{def:konvgeodesik}, yaitu fungsi konveks secara geodesik. Sebelum itu, diperhatikan bahwa fungsi $p(x)=|x|^3$ adalah fungsi konveks, karena $p''(x)=\frac{6x^2}{|x|}\geq 0$ untuk setiap $x\in \mathbb{R}\backslash \{0\}$. Dengan fakta tersebut, didapatkan bahwa untuk setiap $t\in(0,1)$ dan $w,z\in X$ berlaku
    \begin{align*}
        f\qty((1-t)w\oplus tz) =~& f\big((1-t)w_1+tz_1, \qty((1-t)w_1+tz_1)^3\\
        &\quad\,\,    -(1-t)(w_1^3-w_2)-t(z_1^3-z_2)\big)\\
        =~& 20\qty|(1-t)(w_2-w_1^3)+t(z_2-z_1)^3|^3\\
        ~&+\qty|26-\qty((1-t)w_1+tz_1)|^3\\
        \leq ~& 20 \qty((1-t)|w_2-w_1^3|^3+t|z_2-z_1^3|^3)\\
        ~&+ \qty |(1-t)(26-w_1)+t(26-z_1)|^3\\
        \leq ~& (1-t) \qty(20|w_2-w_1^3|^3+|26-w_1|^3)\\
        ~&+ t\qty(20|z_2-z_1^3|^3+|26-z_1|^3)\\
        =~& (1-t)f(w)+tf(z).
    \end{align*}
    Jadi $f$ adalah fungsi konveks secara geodesik. 
\end{enumerate}
Selanjutnya, diperhatikan bahwa $(0,0,0,\dots)\in D(f)$ sehingga $D(f)=\{x\in X\mid f(x)<+\infty\}\neq\emptyset$, yang berarti $f$ adalah fungsi \textit{proper}. Diperhatikan pula bahwa $f$ adalah fungsi kontinu, yang berarti juga memenuhi kondisi \textit{lower semicontinuous}. Dalam hal ini kondisi pada Teorema \ref{thm:konvK} terpenuhi. Diperhatikan bahwa 
nilai minimum dari fungsi $f$ pada Contoh \ref{con:fkonvk} adalah $0$, yang dicapai saat $x_2-x_1^3=0$ dan $26-x_1=0$, atau pada titik $x^*=(26, 17576, 0,0,\dots)$. Dengan demikian, berdasarkan Teorema \ref{thm:apl1}, barisan yang dihasilkan dari skema iterasi Sabri dan JK konvergen ke titik $x^*$. 

Untuk ilustrasi, dilakukan simulasi numerik untuk Contoh \ref{con:fkonvk} dengan perangkat lunak Google Colab yang menggunakan bahasa pemrograman python. Kode dari program ini dapat dilihat pada Lampiran A.2. Pada simulasi ini digunakan parameter $\lambda=20$, $a_n=0.74$, dan $c_n=0.93$, serta nilai awal $x_0=(10, 15, 0, 0, \dots)$. Simulasi dilakukan sebanyak 72 iterasi yang hasilnya dapat dilihat pada \ref{tab:simulasimin} dan \ref{tab:errsimulasi}. Galat dari simulasi ini juga digambarkan sebagai grafik dengan koordinat-$y$ berskala log yang disajikan pada \ref{fig:galatsim}.
\begin{longtable}{|c|c|c|}
\caption{\centering Hasil simulasi sebanyak 72 iterasi untuk Contoh \ref{con:fkonvk} dengan skema iterasi Sabri dan JK}\label{tab:simulasimin}\\
\hline
\textbf{n} & \textbf{Sabri} & \textbf{JK} \\
\hline
\endfirsthead 
\multicolumn{3}{c}{ \thetable{} -- Lanjutan dari halaman sebelumnya} \\
\hline
\textbf{n} & \textbf{Sabri} & \textbf{JK} \\
\hline
\endhead 
\hline
% \multicolumn{3}{|c|}{{Lanjut pada halaman berikutnya}} \\
\endfoot 
\hline
\endlastfoot
\hline
0 & (10.00000, 15.00000, 0, 0, \dots) & (10.00000, 15.00000, 0, 0, \dots) \\
1 & (25.77581, 17125.21877, 0, 0, \dots) & (23.07560, 12287.24784, 0, 0, \dots) \\
2 & (25.99383, 17563.49526, 0, 0, \dots) & (25.97620, 17527.77515, 0, 0, \dots) \\
3 & (25.99765, 17571.24240, 0, 0, \dots) & (25.99458, 17565.01472, 0, 0, \dots) \\
4 & (25.99860, 17573.17051, 0, 0, \dots) & (25.99726, 17570.44169, 0, 0, \dots) \\
5 & (25.99902, 17574.00765, 0, 0, \dots) & (25.99821, 17572.37307, 0, 0, \dots) \\
6 & (25.99925, 17574.46912, 0, 0, \dots) & (25.99868, 17573.33297, 0, 0, \dots) \\
7 & (25.99939, 17574.75968, 0, 0, \dots) & (25.99896, 17573.90002, 0, 0, \dots) \\
8 & (25.99949, 17574.95880, 0, 0, \dots) & (25.99915, 17574.27215, 0, 0, \dots) \\
9 & (25.99956, 17575.10350, 0, 0, \dots) & (25.99928, 17574.53424, 0, 0, \dots) \\
10 & (25.99961, 17575.21329, 0, 0, \dots) & (25.99937, 17574.72839, 0, 0, \dots) \\
11 & (25.99965, 17575.29936, 0, 0, \dots) & (25.99945, 17574.87778, 0, 0, \dots) \\
12 & (25.99969, 17575.36861, 0, 0, \dots) & (25.99951, 17574.99618, 0, 0, \dots) \\
13 & (25.99972, 17575.42551, 0, 0, \dots) & (25.99955, 17575.09225, 0, 0, \dots) \\
14 & (25.99974, 17575.47307, 0, 0, \dots) & (25.99959, 17575.17173, 0, 0, \dots) \\
15 & (25.99976, 17575.51342, 0, 0, \dots) & (25.99962, 17575.23855, 0, 0, \dots) \\
16 & (25.99978, 17575.54807, 0, 0, \dots) & (25.99965, 17575.29549, 0, 0, \dots) \\
17 & (25.99979, 17575.57814, 0, 0, \dots) & (25.99968, 17575.34458, 0, 0, \dots) \\
18 & (25.99980, 17575.60448, 0, 0, \dots) & (25.99970, 17575.38733, 0, 0, \dots) \\
19 & (25.99982, 17575.62774, 0, 0, \dots) & (25.99972, 17575.42489, 0, 0, \dots) \\
20 & (25.99983, 17575.64844, 0, 0, \dots) & (25.99973, 17575.45814, 0, 0, \dots) \\
21 & (25.99984, 17575.66696, 0, 0, \dots) & (25.99975, 17575.48779, 0, 0, \dots) \\
22 & (25.99984, 17575.68364, 0, 0, \dots) & (25.99976, 17575.51438, 0, 0, \dots) \\
23 & (25.99985, 17575.69874, 0, 0, \dots) & (25.99977, 17575.53837, 0, 0, \dots) \\
24 & (25.99986, 17575.71246, 0, 0, \dots) & (25.99978, 17575.56011, 0, 0, \dots) \\
25 & (25.99986, 17575.72500, 0, 0, \dots) & (25.99979, 17575.57990, 0, 0, \dots) \\
26 & (25.99987, 17575.73649, 0, 0, \dots) & (25.99980, 17575.59801, 0, 0, \dots) \\
27 & (25.99988, 17575.74707, 0, 0, \dots) & (25.99981, 17575.61462, 0, 0, \dots) \\
28 & (25.99988, 17575.75683, 0, 0, \dots) & (25.99982, 17575.62992, 0, 0, \dots) \\
29 & (25.99988, 17575.76587, 0, 0, \dots) & (25.99982, 17575.64407, 0, 0, \dots) \\
30 & (25.99989, 17575.77426, 0, 0, \dots) & (25.99983, 17575.65717, 0, 0, \dots) \\
31 & (25.99989, 17575.78208, 0, 0, \dots) & (25.99984, 17575.66935, 0, 0, \dots) \\
32 & (25.99990, 17575.78937, 0, 0, \dots) & (25.99984, 17575.68070, 0, 0, \dots) \\
33 & (25.99990, 17575.79620, 0, 0, \dots) & (25.99985, 17575.69130, 0, 0, \dots) \\
34 & (25.99990, 17575.80259, 0, 0, \dots) & (25.99985, 17575.70122, 0, 0, \dots) \\
35 & (25.99991, 17575.80860, 0, 0, \dots) & (25.99986, 17575.71052, 0, 0, \dots) \\
36 & (25.99991, 17575.81425, 0, 0, \dots) & (25.99986, 17575.71927, 0, 0, \dots) \\
37 & (25.99991, 17575.81958, 0, 0, \dots) & (25.99987, 17575.72750, 0, 0, \dots) \\
38 & (25.99991, 17575.82462, 0, 0, \dots) & (25.99987, 17575.73527, 0, 0, \dots) \\
39 & (25.99992, 17575.82938, 0, 0, \dots) & (25.99987, 17575.74261, 0, 0, \dots) \\
40 & (25.99992, 17575.83389, 0, 0, \dots) & (25.99988, 17575.74955, 0, 0, \dots) \\
41 & (25.99992, 17575.83817, 0, 0, \dots) & (25.99988, 17575.75613, 0, 0, \dots) \\
42 & (25.99992, 17575.84223, 0, 0, \dots) & (25.99988, 17575.76237, 0, 0, \dots) \\
43 & (25.99992, 17575.84610, 0, 0, \dots) & (25.99989, 17575.76831, 0, 0, \dots) \\
44 & (25.99993, 17575.84978, 0, 0, \dots) & (25.99989, 17575.77395, 0, 0, \dots) \\
45 & (25.99993, 17575.85329, 0, 0, \dots) & (25.99989, 17575.77933, 0, 0, \dots) \\
46 & (25.99993, 17575.85664, 0, 0, \dots) & (25.99989, 17575.78446, 0, 0, \dots) \\
47 & (25.99993, 17575.85984, 0, 0, \dots) & (25.99990, 17575.78936, 0, 0, \dots) \\
48 & (25.99993, 17575.86290, 0, 0, \dots) & (25.99990, 17575.79404, 0, 0, \dots) \\
49 & (25.99993, 17575.86583, 0, 0, \dots) & (25.99990, 17575.79852, 0, 0, \dots) \\
50 & (25.99994, 17575.86864, 0, 0, \dots) & (25.99990, 17575.80280, 0, 0, \dots) \\
51 & (25.99994, 17575.87133, 0, 0, \dots) & (25.99990, 17575.80691, 0, 0, \dots) \\
52 & (25.99994, 17575.87392, 0, 0, \dots) & (25.99991, 17575.81085, 0, 0, \dots) \\
53 & (25.99994, 17575.87640, 0, 0, \dots) & (25.99991, 17575.81464, 0, 0, \dots) \\
54 & (25.99994, 17575.87879, 0, 0, \dots) & (25.99991, 17575.81827, 0, 0, \dots) \\
55 & (25.99994, 17575.88109, 0, 0, \dots) & (25.99991, 17575.82177, 0, 0, \dots) \\
56 & (25.99994, 17575.88330, 0, 0, \dots) & (25.99991, 17575.82514, 0, 0, \dots) \\
57 & (25.99994, 17575.88543, 0, 0, \dots) & (25.99992, 17575.82838, 0, 0, \dots) \\
58 & (25.99994, 17575.88749, 0, 0, \dots) & (25.99992, 17575.83150, 0, 0, \dots) \\
59 & (25.99995, 17575.88947, 0, 0, \dots) & (25.99992, 17575.83451, 0, 0, \dots) \\
60 & (25.99995, 17575.89138, 0, 0, \dots) & (25.99992, 17575.83742, 0, 0, \dots) \\
61 & (25.99995, 17575.89323, 0, 0, \dots) & (25.99992, 17575.84022, 0, 0, \dots) \\
62 & (25.99995, 17575.89502, 0, 0, \dots) & (25.99992, 17575.84294, 0, 0, \dots) \\
63 & (25.99995, 17575.89675, 0, 0, \dots) & (25.99992, 17575.84556, 0, 0, \dots) \\
64 & (25.99995, 17575.89842, 0, 0, \dots) & (25.99993, 17575.84809, 0, 0, \dots) \\
65 & (25.99995, 17575.90004, 0, 0, \dots) & (25.99993, 17575.85055, 0, 0, \dots) \\
66 & (25.99995, 17575.90161, 0, 0, \dots) & (25.99993, 17575.85292, 0, 0, \dots) \\
67 & (25.99995, 17575.90313, 0, 0, \dots) & (25.99993, 17575.85522, 0, 0, \dots) \\
68 & (25.99995, 17575.90461, 0, 0, \dots) & (25.99993, 17575.85745, 0, 0, \dots) \\
69 & (25.99995, 17575.90604, 0, 0, \dots) & (25.99993, 17575.85962, 0, 0, \dots) \\
70 & (25.99995, 17575.90742, 0, 0, \dots) & (25.99993, 17575.86172, 0, 0, \dots) \\
71 & (25.99996, 17575.90877, 0, 0, \dots) & (25.99993, 17575.86375, 0, 0, \dots) \\
72 & (25.99996, 17575.91008, 0, 0, \dots) & (25.99993, 17575.86573, 0, 0, \dots) \\
\hline
\end{longtable}


\begin{longtable}{|c|>{\centering\arraybackslash}p{4cm} | >{\centering\arraybackslash}p{4cm} |}
\caption{\centering Galat simulasi $d(x_n,x_{n+1})$ untuk Contoh \ref{con:fkonvk} dengan skema iterasi Sabri dan JK}\label{tab:errsimulasi}\\
\hline
{\textbf{n}} & {\textbf{Galat Sabri}} & {\textbf{Galat JK}} \\
\hline
\endfirsthead 
\multicolumn{3}{c}{\centering \thetable{} -- Lanjutan dari halaman sebelumnya} \\
\hline
{\textbf{n}} & {\textbf{Sabri}} & {\textbf{JK}} \\
\hline
\endhead 
% \multicolumn{3}{c}{{Lanjut pada halaman berikutnya}} \\
\hline
\endfoot 
\hline
\endlastfoot
\hline
1 & $9.8\times10^{2}$ & $9.8\times10^{2}$ \\
2 & $2.1\times10^{-1}$ & $2.9\times10^{0}$ \\
3 & $3.8\times10^{-3}$ & $1.8\times10^{-2}$ \\
4 & $9.5\times10^{-4}$ & $2.6\times10^{-3}$ \\
5 & $4.1\times10^{-4}$ & $9.5\times10^{-4}$ \\
6 & $2.2\times10^{-4}$ & $4.7\times10^{-4}$ \\
7 & $1.4\times10^{-4}$ & $2.7\times10^{-4}$ \\
8 & $9.8\times10^{-5}$ & $1.8\times10^{-4}$ \\
9 & $7.1\times10^{-5}$ & $1.2\times10^{-4}$ \\
10 & $5.4\times10^{-5}$ & $9.5\times10^{-5}$ \\
11 & $4.2\times10^{-5}$ & $7.3\times10^{-5}$ \\
12 & $3.4\times10^{-5}$ & $5.8\times10^{-5}$ \\
13 & $2.8\times10^{-5}$ & $4.7\times10^{-5}$ \\
14 & $2.3\times10^{-5}$ & $3.9\times10^{-5}$ \\
15 & $1.9\times10^{-5}$ & $3.2\times10^{-5}$ \\
16 & $1.7\times10^{-5}$ & $2.8\times10^{-5}$ \\
17 & $1.4\times10^{-5}$ & $2.4\times10^{-5}$ \\
18 & $1.2\times10^{-5}$ & $2.1\times10^{-5}$ \\
19 & $1.1\times10^{-5}$ & $1.8\times10^{-5}$ \\
20 & $1.0\times10^{-5}$ & $1.6\times10^{-5}$ \\
21 & $9.1\times10^{-6}$ & $1.4\times10^{-5}$ \\
22 & $8.2\times10^{-6}$ & $1.3\times10^{-5}$ \\
23 & $7.4\times10^{-6}$ & $1.1\times10^{-5}$ \\
24 & $6.7\times10^{-6}$ & $1.0\times10^{-5}$ \\
25 & $6.1\times10^{-6}$ & $9.7\times10^{-6}$ \\
26 & $5.6\times10^{-6}$ & $8.9\times10^{-6}$ \\
27 & $5.2\times10^{-6}$ & $8.1\times10^{-6}$ \\
28 & $4.8\times10^{-6}$ & $7.5\times10^{-6}$ \\
29 & $4.4\times10^{-6}$ & $6.9\times10^{-6}$ \\
30 & $4.1\times10^{-6}$ & $6.4\times10^{-6}$ \\
31 & $3.8\times10^{-6}$ & $6.0\times10^{-6}$ \\
32 & $3.5\times10^{-6}$ & $5.5\times10^{-6}$ \\
33 & $3.3\times10^{-6}$ & $5.2\times10^{-6}$ \\
34 & $3.1\times10^{-6}$ & $4.8\times10^{-6}$ \\
35 & $2.9\times10^{-6}$ & $4.5\times10^{-6}$ \\
36 & $2.7\times10^{-6}$ & $4.3\times10^{-6}$ \\
37 & $2.6\times10^{-6}$ & $4.0\times10^{-6}$ \\
38 & $2.4\times10^{-6}$ & $3.8\times10^{-6}$ \\
39 & $2.3\times10^{-6}$ & $3.6\times10^{-6}$ \\
40 & $2.2\times10^{-6}$ & $3.4\times10^{-6}$ \\
41 & $2.1\times10^{-6}$ & $3.2\times10^{-6}$ \\
42 & $2.0\times10^{-6}$ & $3.0\times10^{-6}$ \\
43 & $1.9\times10^{-6}$ & $2.9\times10^{-6}$ \\
44 & $1.8\times10^{-6}$ & $2.7\times10^{-6}$ \\
45 & $1.7\times10^{-6}$ & $2.6\times10^{-6}$ \\
46 & $1.6\times10^{-6}$ & $2.5\times10^{-6}$ \\
47 & $1.5\times10^{-6}$ & $2.4\times10^{-6}$ \\
48 & $1.5\times10^{-6}$ & $2.3\times10^{-6}$ \\
49 & $1.4\times10^{-6}$ & $2.2\times10^{-6}$ \\
50 & $1.3\times10^{-6}$ & $2.1\times10^{-6}$ \\
51 & $1.3\times10^{-6}$ & $2.0\times10^{-6}$ \\
52 & $1.2\times10^{-6}$ & $1.9\times10^{-6}$ \\
53 & $1.2\times10^{-6}$ & $1.8\times10^{-6}$ \\
54 & $1.1\times10^{-6}$ & $1.7\times10^{-6}$ \\
55 & $1.1\times10^{-6}$ & $1.7\times10^{-6}$ \\
56 & $1.0\times10^{-6}$ & $1.6\times10^{-6}$ \\
57 & $1.0\times10^{-6}$ & $1.5\times10^{-6}$ \\
58 & $1.0\times10^{-6}$ & $1.5\times10^{-6}$ \\
59 & $9.7\times10^{-7}$ & $1.4\times10^{-6}$ \\
60 & $9.4\times10^{-7}$ & $1.4\times10^{-6}$ \\
61 & $9.1\times10^{-7}$ & $1.3\times10^{-6}$ \\
62 & $8.8\times10^{-7}$ & $1.3\times10^{-6}$ \\
63 & $8.5\times10^{-7}$ & $1.2\times10^{-6}$ \\
64 & $8.2\times10^{-7}$ & $1.2\times10^{-6}$ \\
65 & $7.9\times10^{-7}$ & $1.2\times10^{-6}$ \\
66 & $7.7\times10^{-7}$ & $1.1\times10^{-6}$ \\
67 & $7.4\times10^{-7}$ & $1.1\times10^{-6}$ \\
68 & $7.2\times10^{-7}$ & $1.0\times10^{-6}$ \\
69 & $7.0\times10^{-7}$ & $1.0\times10^{-6}$ \\
70 & $6.8\times10^{-7}$ & $1.0\times10^{-6}$ \\
71 & $6.6\times10^{-7}$ & $1.0\times10^{-6}$ \\
72 & $6.4\times10^{-7}$ & $9.7\times10^{-7}$ \\
\hline
\end{longtable}
\begin{figure}[H]
    \centering
    \includegraphics[width=0.9\linewidth]{Bab_4/galatsim.png}
    \caption{Galat $d(x_n,x_{n+1})$ (skala log) vs iterasi}
    \label{fig:galatsim}
\end{figure}

\subsection{Rekonstruksi Citra}
Selain dapat digunakan dalam minimalisasi fungsi, skema iterasi Sabri juga dapat digunakan dalam permasalahan \textit{split feasibility}, yang memiliki aplikasi nyata dalam rekonstruksi citra. Permasalahan \textit{split feasibility} ini dikenalkan oleh Censor dan Elfving pada tahun 1994, yang diformulasikan sebagai 
\begin{align}\label{eq:sfp}
    \text{Cari titik } x^*\in C \quad \text{yang memenuhi}\quad Ax^*\in Q,
\end{align}
dengan $C$ dan $Q$ berturut-turut adalah himpunan bagian tertutup dan konveks dari ruang Hilbert $H_1$ dan $H_2$, serta $A:H_1\to H_2$ adalah transformasi linier terbatas \cite{Censor1994}. Permasalahan \eqref{eq:sfp} disebut konsisten jika setidaknya memiliki satu solusi. Suatu titik $x^*\in C$ merupakan solusi dari permasalahan tersebut jika dan hanya jika $x^*$ memenuhi persamaan titik tetap 
\begin{align}
    x=P_C(I-\gamma A^*(I-P_Q)A)x,
\end{align}
dengan $P_C$ dan $P_Q$ adalah proyeksi titik terdekat pada $C$ dan $Q$, secara berturut-turut, $\gamma>0$, dan $A^*$ adalah transformasi adjoin dari $A$ \cite{feng2019}. Kemudian, berdasarkan Contoh \ref{con:hilbert}, diketahui bahwa ruang Hilbert merupakan ruang $CAT(0)$ yang juga merupakan ruang $CAT_p(0)$ dengan $p=2$. Selanjutnya, berdasarkan hasil dari Byrne \cite{Byrne2003}, pemetaan $T:C\to C$ yang didefinisikan sebagai $T(x)=P_C(I-\gamma A^*(I-P_Q)A)x$ adalah pemetaan nonekspansif. Oleh karena itu, pemetaan ini juga merupakan pemetaan $(\alpha,\beta,\gamma)$-nonekspansif.

Dari permasalahan tersebut, skema iterasi Sabri dapat digunakan sebagaimana Teorema berikut.
\begin{thm}
    Diberikan ruang Hilbert $H$ yang kompak dan $C$ adalah himpunan bagian yang tertutup dan konveks dari $H$.
    Jika permasalahan \eqref{eq:sfp} konsisten dengan solusi yang tunggal dan $\gamma\in (0,\frac{2}{k})$ dengan $k$ adalah radius spektral dari $A^*A$, maka pemetaan $T:C\to C$ yang didefinisikan sebagai $$T(x)=P_C(I-\gamma A^*(I-P_Q)A)x$$ 
    merupakan pemetaan $(\alpha,\beta,\gamma)$-nonekspansif, serta barisan $\{x_n\}$ yang didefinisikan sebagai
    \begin{flalign}
        \begin{cases}
            q_n &= T\qty((1-c_n)x_n\oplus c_n T(x_n)) \\
            y_n &= T(T(q_n)) \\
            x_{n+1} &= T\qty((1-a_n)T(q_n)\oplus a_n T(y_n)) 
        \end{cases}, 
    \end{flalign}
    dengan $\{a_n\},\{c_n\}\subseteq [a,b]\subset (0,1)$ konvergen kuat ke solusi tunggal $x^*\in C$ dari permasalahan \eqref{eq:sfp} untuk setiap nilai awal $x_0\in C$.
    \begin{bukti}
        Berdasarkan hasil dari Byrne (lihat \cite{Byrne2003}), diperoleh bahwa pemetaan $T$ yang memenuhi kondisi tersebut adalah pemetaan nonekspansif sehingga juga merupakan $(\alpha,\beta,\gamma)$-nonekpansif. Kemudian, berdasarkan Contoh \ref{con:lpcatp}, didapatkan bahwa $H$ adalah ruang $CAT(0)$ yang juga merupakan ruang $CAT_p(0)$ dengan $p=2$. Akibatnya, dengan Teorema \ref{thm:konvK}, diperoleh bahwa $\{x_n\}$ konvergen kuat ke titik tetap dari $T$, yang berarti $\{x_n\}$ konvergen kuat ke solusi tunggal dari $x^*\in C$.
    \end{bukti}
\end{thm}

Hasil ini dapat diterapkan pada permasalahan rekonstruksi citra tomografi. Permasalahan tersebut memiliki peranan penting dalam dunia medis, khususnya dalam rekonstruksi citra dari \textit{CT-Scan}, yang datanya berupa proyeksi yang dikenal sebagai sinogram. Pada umumnya, rekonstruksi citra organ tubuh dari data sinogram dilakukan menggunakan proyeksi mundur berfilter (\textit{filtered back projection}). Dari segi waktu komputasi, metode ini relatif cepat dalam menghasilkan citra. Namun, metode tersebut memiliki keterbatasan dalam kualitas citra yang dihasilkan, akibat tingginya tingkat derau. Untuk mereduksi derau pada citra hasil rekonstruksi, metode ini memerlukan dosis radiasi yang lebih tinggi pada pasien, yang berpotensi menimbulkan dampak negatif pada kesehatan pasien. Sebagai alternatif, rekonstruksi citra dapat dilakukan menggunakan proyeksi mundur tanpa filter yang dikombinasikan dengan algoritma skema iterasi. Pendekatan ini terbukti mampu mereduksi derau secara signifikan, yaitu sekitar 40\% sampai 70\% dibandingkan dengan metode proyeksi mundur berfilter \cite{Ramage2023}. 

Berikut ini diberikan gambaran bagaimana skema iterasi Sabri dapat diterapkan dalam rekonstruksi citra tomografi di ruang $CAT_p(0)$.
\begin{figure}[H]
    \centering
    \tikzstyle{startstop} = [ellipse, 
minimum width=3cm, 
minimum height=0.5cm,
text centered, 
draw=black]

\tikzstyle{io} = [trapezium, 
trapezium stretches=true, % A later addition
trapezium left angle=70, 
trapezium right angle=110, 
minimum width=6cm, 
minimum height=0.5cm, text centered, 
draw=black]

\tikzstyle{process} = [rectangle, 
minimum height=0.5cm, 
text centered, 
text width=4cm, 
draw=black]

\tikzstyle{decision} = [diamond, 
minimum width=2cm, 
minimum height=0.5cm, 
text centered, 
aspect=1.8,
inner sep=2pt,
draw=black]
\tikzstyle{arrow} = [thick,->,>=stealth]
\begin{tikzpicture}[scale=1.2,node distance=1.8cm]

\node (sino) [startstop]
{Data Sinogram};

\node (data2d) [process, below left of=sino, xshift=-2cm,yshift=-0.7cm]
{Data 2D};

\node (data3d) [process, below right of=sino, xshift=2cm]
{Data 3D};

% \node (vec2d) [process, below of=data2d]
% {Vektorisasi\\
% data 2D};

\node (slice) [process, below of=data3d]
{\textit{Slice} menjadi\\
potongan 2D};

\node (rn) [startstop, below of=sino, yshift=-3.5cm, align=center]
{Vektor data $\mathbb{R}^n$\\
(Ruang Euclid)};

\draw [arrow] (sino) -- (data2d);
\draw [arrow] (sino) -- (data3d);

% \draw [arrow] (data2d) -- (vec2d);
\draw [arrow] (data3d) -- (slice);

\draw [arrow] (data2d.south) |- (rn.west);
\draw [arrow] (slice.south) |- (rn.east);

\end{tikzpicture}
    \caption{\centering Diagram Data Sinogram ke Ruang Euclid dalam Rekonstruksi Citra}
\end{figure}
\begin{figure}[H]
\centering
\begin{tikzpicture}[scale=1.1, every node/.style={transform shape}]


\draw[thick,fill=yellow!10] (-6,-3.7) rectangle (6,3.7);
\node at (0,-3.4) {Ruang Metrik Geodesik};

\draw[thick,fill=green!10] (-5,-2.9) rectangle (5,2.9);
\node at (0,-2.55) {Ruang $CAT_p(0)$};

\draw[thick, fill=gray!10] (-4,-2.1) rectangle (4,2.1);
\node at (0,-1.7) {Ruang $CAT(0)$};

\draw[thick, fill=blue!20] (-3,-1.3) rectangle (3,1.3);
\node at (0,-0.9) {Ruang Hilbert};

\draw[thick,fill=red!20] (-2,-0.5) rectangle (2,0.5);
\node at (0,0) {Ruang Euclid $\mathbb{R}^n$};


\end{tikzpicture}
\caption{\centering Hubungan ruang Euclid, Hilbert, $CAT(0)$, $CAT_p(0)$, dan ruang metrik geodesik}
\end{figure}


Pada bagian ini, dilakukan simulasi rekonstruksi citra tomografi berbasis skema iterasi Sabri. Transformasi maju $A$ yang digunakan adalah transformasi Radon, sedangkan himpunan batas $C$ dan $Q$ adalah himpunan bagian yang konveks dari suatu ruang Euclid, yang masing-masing merepresentasikan solusi \textit{feasible} untuk citra dan domain proyeksi. Pada simulasi ini, digunakan bahasa pemrograman Python dengan bantuan pustaka Scikit-image. Data citra yang digunakan adalah Shepp-Logan Phantom $x^*\in \mathbb{R}^n$ dengan resolusi $512\times 512$. Shepp-Logan Phantom dipilih karena merupakan model sintetik yang dirancang untuk menyerupai potongan gambar bagian dalam kepala manusia, sehingga sering digunakan dalam penelitian tomografi untuk menguji kemampuan algoritma rekonstruksi dalam membedakan struktur dengan tingkat kontras yang berbeda. Selanjutnya, data sinogram dibentuk sebagai $Q=Ax^*$ dengan menerapkan transformasi Radon pada sudut-sudut proyeksi yang terdistribusi merata pada interval $[0,180^{\circ})$. Tujuan simulasi ini adalah merekonstruksi citra $x^*$ berdasarkan data sinogram tersebut menggunakan skema iterasi Sabri. Untuk itu, digunakan pemetaan 
\begin{align}\label{eq:rekonoperator}
    T(x)=P_C(x-\gamma A^*(Ax-Q)),
\end{align}
dengan $A^*$ merupakan aproksimasi dari transformasi adjoin $A$ yang diperoleh menggunakan proyeksi mundur tanpa filter, serta $P_C$ menyatakan operator proyeksi ke himpunan batas $C=[0,1]^n$. 

Berikut ini diberikan diagram alir dari skema iterasi Sabri yang digunakan dalam simulasi rekonstruksi citra tomografi.
\begin{figure}[H]
    \centering
    \tikzstyle{startstop} = [ellipse, 
minimum width=5.5cm, 
minimum height=0.5cm,
text centered, 
draw=black]

\tikzstyle{io} = [trapezium, 
trapezium stretches=true, % A later addition
trapezium left angle=70, 
trapezium right angle=110, 
minimum width=6cm, 
minimum height=0.5cm, text centered, 
draw=black]

\tikzstyle{process} = [rectangle, 
minimum height=0.5cm, 
text centered, 
text width=8.5cm, 
draw=black]

\tikzstyle{decision} = [diamond, 
minimum width=2cm, 
minimum height=0.5cm, 
text centered, 
aspect=1.8,
inner sep=2pt,
draw=black]
\tikzstyle{arrow} = [thick,->,>=stealth]

\begin{tikzpicture}[node distance=1.4cm,scale=1, every node/.style={transform shape}]

% ===== Nodes =====
\node (start) [startstop] 
{Mulai};

\node (input) [io, below of=start,align=center,yshift=-0.3cm] 
    {Inisialisasi data Sinogram $Q$, gambar awal $x_0\in \mathbb{R}^n$,\\
     $k=0$, batas iterasi $K$, dan parameter $\gamma$, $\{a_k\},\{c_k\}\subseteq (0,1)$};

\node (hitung) [process, below of=input,yshift=-0.3cm] 
{Definisikan $T(x)$ sesuai persamaan \eqref{eq:rekonoperator}};

\node (qn) [process, below of=hitung] 
{$q_k = T\big((1-c_k)x_k + c_k T x_k\big)$};

\node (yn) [process, below of=qn] 
{$y_k = T(Tq_k)$};

\node (xn) [process, below of=yn] 
{$x_{k+1} = T\big((1-a_k)Tq_k + a_k Ty_k\big)$};

\node (decision) [decision, below of=xn, yshift=-0.8cm,align=center] 
{Apakah\\$k\geq K$?};

\node (stop) [startstop, below of=decision, yshift=-0.8cm] 
{Selesai};

% ===== Arrows =====
\draw [arrow] (start) -- (input);
\draw [arrow] (input) -- (hitung);
\draw [arrow] (hitung) -- (qn);
\draw [arrow] (qn) -- (yn);
\draw [arrow] (yn) -- (xn);
\draw [arrow] (xn) -- (decision);
\draw [arrow] (decision) -- node[anchor=east]{Ya} (stop);

\draw [arrow] (decision.east) -- ++(4,0) 
node[anchor=west]{Tidak}
|- (qn.east);

\end{tikzpicture}

    \caption{\centering Diagram alir skema iterasi Sabri untuk rekonstruksi citra tomografi}
    \label{fig:flowrekonstruksi}
\end{figure}

Selanjutnya, pada \ref{tab:parametersim} diberikan parameter-parameter yang digunakan dalam simulasi rekonstruksi citra tomografi tersebut.
\begin{table}[H]
    \centering
    \caption{Parameter simulasi rekonstruksi citra}
    \begin{tabular}{ll}
    \hline
    \textbf{Parameter} & \textbf{Nilai/Deskripsi}\\
    \hline
    Ukuran gambar     &  $512\times 512$ \\
    Sudut proyeksi     &  180 terdistribusi merata pada $[0^\circ, 180^\circ)$\\
    Transformasi maju $A$ & Transformasi Radon\\
    Transformasi adjoin $A^*$ & Proyeksi mundur tanpa filter\\
    Himpunan batas $C$ & $[0,1]^n$\\
    Nilai awal $x_0$ & $x_0=\mathbf{0}$\\
    Jumlah iterasi & 60\\
    Parameter $\gamma$ & Konstan: 0.00245\\
    Parameter $a_n,c_n$ & Konstan: $a_n=0.42, c_n=0.31$\\
    Implementasi & Google Colab dengan scikit-image, matplotlib\\
    \hline
    \end{tabular}
    \label{tab:parametersim}
\end{table}





Data awal dari simulasi tersebut disajikan pada \ref{fig:rekoncit}. Kemudian hasil simulasi dari rekonstruksi citra tersebut pada iterasi ke-5, 10, 30, dan 60 diberikan oleh \ref{fig:rekoncit2}. Pada iterasi ke-5, terlihat bahwa derau dari citra tersebut masih cukup tinggi, yang juga tertera pada nilai (\textit{peak signal noise ratio}) PSNR, yaitu 22.03 dB. Nilai PSNR dari citra tersebut juga makin membesar pada iterasi ke-10 dan 30, serta untuk iterasi ke-60, nilai PSNR citra tersebut adalah 35.61 dB, yang mengindikasikan semakin sedikit derau pada citra.  
\begin{figure}[H]
    \centering
    \includegraphics[width=\linewidth]{Bab_4/citrafix.png}
    \caption{Data awal sinogram dan citra asli}
    \label{fig:rekoncit}
\end{figure}
\begin{figure}[H]
    \centering
    \includegraphics[width=\linewidth]{Bab_4/citrafix2.png}
    \caption{Hasil rekonstruksi citra pada iterasi ke-5, 10, 30, dan 60}
    \label{fig:rekoncit2}
\end{figure}
Pada \ref{tab:msepsnr}, diberikan nilai metrik PSNR dan MSE dari citra hasil rekonstruksi pada tiap iterasinya. Terlihat bahwa nilai galat, yaitu (\textit{mean squared error}) dari citra tersebut relatif kecil pada iterasi ke-60, yaitu $2.7\times 10^{-4}$, dibandingkan dengan galat pada iterasi pertama, yaitu $2\times 10^{-2}$. Terlihat pula bahwa nilai metrik PSNR dan MSE tersebut turun pada tiap iterasinya, sesuai dengan peningkatan kualitas citra pada \ref{fig:rekoncit2}, yang menunjukkan bahwa skema iterasi Sabri dapat digunakan dalam rekonstruksi citra tomografi.
\begin{longtable}{|c|>{\centering\arraybackslash}p{4cm} | >{\centering\arraybackslash}p{4cm} |}
\caption{\centering MSE dan PSNR Rekonstruksi Citra}\label{tab:msepsnr}\\
\hline
{\textbf{Iterasi ke-$n$}} & {\textbf{MSE}} & {\textbf{PSNR}} \\
\hline
\endfirsthead 
\multicolumn{3}{c}{\centering \thetable{} -- Lanjutan dari halaman sebelumnya} \\
\hline
{\textbf{Iterasi ke-$n$}} & {\textbf{MSE}} & {\textbf{PSNR}} \\
\hline
\endhead 
% \multicolumn{3}{c}{{Lanjut pada halaman berikutnya}} \\
\hline
\endfoot 
\hline
\endlastfoot
\hline
1  & 0.020302 & 16.924673 \\
2  & 0.013587 & 18.668847 \\
3  & 0.010072 & 19.968693 \\
4  & 0.007823 & 21.066098 \\
5  & 0.006272 & 22.026258 \\
6  & 0.005149 & 22.882519 \\
7  & 0.004309 & 23.655780 \\
8  & 0.003664 & 24.360387 \\
9  & 0.003157 & 25.006683 \\
10 & 0.002753 & 25.602432 \\
11 & 0.002425 & 26.153733 \\
12 & 0.002155 & 26.665575 \\
13 & 0.001931 & 27.142128 \\
14 & 0.001743 & 27.587012 \\
15 & 0.001584 & 28.003332 \\
16 & 0.001448 & 28.393736 \\
17 & 0.001330 & 28.760530 \\
18 & 0.001229 & 29.105746 \\
19 & 0.001140 & 29.431187 \\
20 & 0.001062 & 29.738458 \\
21 & 0.000993 & 30.028997 \\
22 & 0.000932 & 30.304104 \\
23 & 0.000878 & 30.564968 \\
24 & 0.000829 & 30.812682 \\
25 & 0.000786 & 31.048254 \\
26 & 0.000746 & 31.272582 \\
27 & 0.000710 & 31.486496 \\
28 & 0.000678 & 31.690770 \\
29 & 0.000648 & 31.886142 \\
30 & 0.000620 & 32.073295 \\
31 & 0.000595 & 32.252847 \\
32 & 0.000572 & 32.425370 \\
33 & 0.000551 & 32.591386 \\
34 & 0.000531 & 32.751347 \\
35 & 0.000512 & 32.905658 \\
36 & 0.000495 & 33.054684 \\
37 & 0.000479 & 33.198752 \\
38 & 0.000464 & 33.338168 \\
39 & 0.000449 & 33.473223 \\
40 & 0.000436 & 33.604183 \\
41 & 0.000424 & 33.731285 \\
42 & 0.000412 & 33.854741 \\
43 & 0.000400 & 33.974740 \\
44 & 0.000390 & 34.091450 \\
45 & 0.000380 & 34.205025 \\
46 & 0.000370 & 34.315610 \\
47 & 0.000361 & 34.423339 \\
48 & 0.000353 & 34.528338 \\
49 & 0.000344 & 34.630725 \\
50 & 0.000336 & 34.730611 \\
51 & 0.000329 & 34.828102 \\
52 & 0.000322 & 34.923289 \\
53 & 0.000315 & 35.016256 \\
54 & 0.000309 & 35.107081 \\
55 & 0.000302 & 35.195841 \\
56 & 0.000296 & 35.282605 \\
57 & 0.000291 & 35.367439 \\
58 & 0.000285 & 35.450403 \\
59 & 0.000280 & 35.531556 \\
60 & 0.000275 & 35.610951 \\
\hline
\end{longtable}

\chapter{KESIMPULAN DAN SARAN}
Berdasarkan uraian dari bab-bab sebelumnya, berikut diberikan kesimpulan yang diperoleh dari tesis ini dan saran untuk penelitian selanjutnya.
\section{Kesimpulan}
Berdasarkan hasil dan pembahasan pada Bab \ref{chap:bab4}, didapatkan kesimpulan dari tesis ini sebagai berikut.
\begin{enumerate}
    \item Pemetaan $(\alpha,\beta,\gamma)$-nonekspansif yang awalnya didefinisikan pada ruang Banach dapat diperluas pada ruang $CAT_p(0)$. Dalam ruang ini, pemetaan tersebut juga memenuhi sifat nonekspansif kuasi dan \textit{demiclosedness}. Untuk mendapatkan nilai aproksimasi titik tetap dari pemetaan ini, dapat digunakan skema iterasi Sabri yang terbukti memiliki konvergensi-$\Delta$ dan kuat. Untuk syarat cukup dari konvergensi$-\Delta$ adalah sebagai berikut.
    \begin{enumerate}
        \item Ruang $(X,d)$ adalah ruang $CAT_p(0)$ yang lengkap.
        \item Himpunan bagian $W$ merupakan himpunan bagian tak kosong yang tertutup dan konveks dari ruang $CAT_p(0)$.
        \item Pemetaan $T:W\to W$ adalah pemetaan $(\alpha,\beta,\gamma)$-nonekspansif dengan $Fix(T)\neq \emptyset$.
        \item Barisan koefisien $\{a_n\}$ dan $\{b_n\}$ berada pada interval $(0,1)$.
    \end{enumerate}
    Untuk konvergensi kuat, diperlukan syarat cukup tambahan, yaitu himpunan bagian $W$ merupakan himpunan kompak.
    \item Berdasarkan hasil percobaan numerik, skema iterasi Sabri memiliki laju yang lebih cepat dibanding skema iterasi JK, Thakur, Abbas, dan Agarwal, dalam aproksimasi titik tetap dari pemetaan $(\alpha,\beta,\gamma)$-nonekspansif di ruang $CAT_p(0)$, yang ditunjukkan oleh banyaknya jumlah iterasi yang diperlukan untuk mencapai galat yang ditentukan. Hasil ini selaras dengan hasil yang diperoleh peneliti sebelumnya pada ruang Banach.
    \item Dalam masalah optimasi, skema iterasi Sabri dapat diterapkan untuk permasalahan minimalisasi fungsi dan rekonstruksi citra tomografi. 
\end{enumerate}
\section{Saran}
Berdasarkan hasil yang telah diperoleh dalam tesis ini, terdapat beberapa saran yang dapat dipertimbangkan untuk penelitian selanjutnya. 
\begin{enumerate}
    \item Kajian mengenai pemetaan $(\alpha,\beta,\gamma)$-nonekspansif pada ruang $CAT_p(0)$ masih dapat diperluas ke kelas ruang metrik tak linier lainnya.
    \item Perbandingan kelajuan konvergensi pada penelitian ini masih terbatas secara percobaan numerik, sehingga disarankan untuk mendapatkan perbandingan lajunya secara analitik. 
    \item Penerapan skema iterasi Sabri pada permasalahan rekonstruksi citra tomografi dapat diuji lebih lanjut dengan menggunakan data medis nyata yang mengandung derau tinggi atau sudut proyeksi yang terbatas. 
\end{enumerate}  
\DaftarPustaka{Bibitem}
\pagebreak
\BukaLampiran
\lampiran{Kode Progam Simulasi dalam Bahasa Pemrograman Python}
\begin{lstlisting}[caption={Kode Python untuk Percobaan Numerik}]
import numpy as np
import matplotlib.pyplot as plt
import pandas as pd
import random

#metrik
def metrik(x,y):
  x1,x2 = x[0], x[1]
  y1,y2 = y[0], y[1]

  suku1 = np.abs(x1-y1)**3
  suku2 = np.abs(x1**3-x2-y1**3+y2)**3
  return (suku1+suku2)**(1/3)

#jalur geodesik
def geodesik(t,w,z):
  w1,w2 = w[0],w[1]
  z1,z2 = z[0],z[1]
  sk1 = (1-t)*w1+t*z1
  sk2 = (1-t)*(w1**3-w2)+t*(z1**3-z2)
  if (t>=0 and t<=1):
    return [sk1,sk1**3-sk2]
  else:
     raise ValueError("di luar domain")

#Pemetaan (alpha, beta, gamma)-nonekspansif
def pemetaan(w):
    w1 = w[0]

    if 1 <= w1 < 3:
        return [
            (w1+3)/4,
            (w1+3)**3 / 64
        ]

    elif 3 <= w1 <= 5:
        return [
            (w1+2)/4,
            (w1+2)**3 / 64
        ]

    else:
        raise ValueError("di luar domain")

#skema iterasi Sabri
def iterSabri(x0, tol, max_iter):
  data_x = [x0]
  data_error = []
  for n in range(max_iter):
    x_n = data_x[-1].copy()
    a_n = a(n)
    b_n = b(n)
    c_n = c(n)
    q_n = pemetaan(geodesik(c_n,x_n,pemetaan(x_n)))
    y_n = pemetaan(pemetaan(q_n))
    x_next = pemetaan(geodesik(a_n,pemetaan(q_n),pemetaan(y_n)))
    data_x.append(x_next)
    error = metrik(x_n,x_next)
    data_error.append(error)
    if error < tol:
      break
  return data_x, data_error

#skema iterasi JK
def iterJK(x0, tol, max_iter):
  data_x = [x0]
  data_error = []
  for n in range(max_iter):
    x_n = data_x[-1].copy()
    a_n = a(n)
    b_n = b(n)
    c_n = c(n)
    q_n = (geodesik(c_n,x_n,pemetaan(x_n)))
    y_n = (pemetaan(q_n))
    x_next = pemetaan(geodesik(a_n,pemetaan(q_n),pemetaan(y_n)))
    data_x.append(x_next)
    error = metrik(x_n, x_next)
    data_error.append(error)
    if error < tol:
      break
  return data_x, data_error

#skema iterasi Thakur
def iterThakur(x0, tol, max_iter):
  data_x = [x0]
  data_error = []
  for n in range(max_iter):
    x_n = data_x[-1].copy()
    a_n = a(n)
    b_n = b(n)
    c_n = c(n)
    q_n = (geodesik(c_n,x_n,pemetaan(x_n)))
    y_n = (pemetaan(geodesik(a_n, x_n, q_n)))
    x_next = pemetaan(y_n)
    data_x.append(x_next)
    error = metrik(x_n, x_next)
    data_error.append(error)
    if error < tol:
      break
  return data_x, data_error

#skema iterasi Abbas
def iterAbbas(x0, tol, max_iter):
  data_x = [x0]
  data_error = []
  for n in range(max_iter):
    x_n = data_x[-1].copy()
    a_n = a(n)
    b_n = b(n)
    c_n = c(n)
    q_n = geodesik(c_n,x_n,pemetaan(x_n))
    y_n = geodesik(b_n, pemetaan(x_n), pemetaan(q_n))
    x_next = geodesik(a_n,pemetaan(y_n),pemetaan(q_n))
    data_x.append(x_next)
    error = metrik(x_n, x_next)
    data_error.append(error)
    if error < tol:
      break
  return data_x, data_error

#skema iterasi Agarwal
def iterAgarwal(x0, tol, max_iter):
  data_x = [x0]
  data_error = []
  for n in range(max_iter):
    x_n = data_x[-1].copy()
    a_n = a(n)
    b_n = b(n)
    c_n = c(n)
    y_n = (geodesik(c_n, x_n, pemetaan(x_n)))
    x_next = geodesik(a_n,pemetaan(x_n),pemetaan(y_n))
    data_x.append(x_next)
    error = metrik(x_n, x_next)
    data_error.append(error)
    if error < tol:
      break
  return data_x, data_error

#konstan
def a(n): return 0.92
def b(n): return 0.83
def c(n): return 0.81

#turun
def a(n): return (n**2/(n**3+1))
def b(n): return (2/(n+3))
def c(n): return (4*n+2)/(7*n+4)

#naik
def a(n): return (1-(4*n+9)**(1/2)/(2*n+13))
def b(n): return 0.8
def c(n): return (1-(n**2/(n**7+3)**(1/2)))

#naikturun
def a(n): return 1-(n**3)/(n**5+1)
def b(n): return (1/(n+1))
def c(n): return (3*n+4)/(5*n**2+4)

#turunnaik
def c(n): return 1-np.sqrt(n**3+8)/(5*n**5+9)
def b(n): return (1-3/(n+5))
def a(n): return (n**5+1)/(5*n**7+4)


x0=[2,3]
tol = 1e-6
max_iter = 100

dataSabri, errorSabri = iterSabri(x0, tol, max_iter)
dataJK, errorJK = iterJK(x0, tol, max_iter)
dataThakur, errorThakur = iterThakur(x0, tol, max_iter)
dataAbbas, errorAbbas = iterAbbas(x0, tol, max_iter)
dataAgarwal, errorAgarwal = iterAgarwal(x0, tol, max_iter)
max_len = max(len(dataSabri), len(dataJK), len(dataThakur), len(dataAbbas), len(dataAgarwal))
dataSabri += [np.nan] * (max_len - len(dataSabri))
dataJK += [np.nan] * (max_len - len(dataJK))
dataThakur += [np.nan] * (max_len - len(dataThakur))
dataAbbas += [np.nan] * (max_len - len(dataAbbas))
dataAgarwal += [np.nan] * (max_len - len(dataAgarwal))
errorSabri += [np.nan] * (max_len - len(errorSabri))
errorJK += [np.nan] * (max_len - len(errorJK))
errorThakur += [np.nan] * (max_len - len(errorThakur))
errorAbbas += [np.nan] * (max_len - len(errorAbbas))
errorAgarwal += [np.nan] * (max_len - len(errorAgarwal))

table = pd.DataFrame({
    'Iterasi n': np.arange(max_len),
    'Sabri': dataSabri,
    'JK': dataJK,
    'Thakur': dataThakur,
    'Abbas': dataAbbas,
    'Agarwal': dataAgarwal
})

def titikX(x, d=6):
    if isinstance(x, (list, np.ndarray)):
        return f"({x[0]:.{d}f}, {x[1]:.{d}f}, 0, 0, \\dots)"
    return x

table_fmt = table.copy()

for col in ["Sabri", "JK", "Thakur", "Abbas", "Agarwal"]:
    table_fmt[col] = table_fmt[col].apply(titikX)
def show_pair(table, scheme_name):
    sub = table_fmt[["Iterasi n", "Sabri", scheme_name]]
    print(f"Tabel Sabri-{scheme_name}")
    print(sub.to_latex(index=False))
    return sub

skema = ["JK", "Thakur", "Abbas", "Agarwal"]
for s in skema:
    show_pair(table_fmt, s)
table_error = pd.DataFrame({
    'Iterasi n': np.arange(max_len),
    'Sabri': errorSabri,
    'JK': errorJK,
    'Thakur': errorThakur,
    'Abbas': errorAbbas,
    'Agarwal': errorAgarwal
})

print(table_error.to_latex(index=False))
import matplotlib.pyplot as plt
import seaborn as sns

plt.figure(figsize=(12, 10))

for col in ['Sabri', 'JK', 'Thakur', 'Abbas', 'Agarwal']:
    plt.plot(table_error['Iterasi n'], table_error[col], marker='o', linestyle='-', label=col)

plt.xlabel('Iterasi n', fontsize=18)
plt.ylabel('Galat $d(x_n,x^*)$', fontsize=18)
plt.xticks(fontsize=14)
plt.yticks(fontsize=14)
plt.legend(fontsize=14)
plt.grid(True, which='both', ls='-', alpha=0.8)
plt.show()


x0_list = [
    [2,3], [3,4], [4,1], [1.5,4], [2.7,4.2], [5,1.8], [3.1,3.5]
]

#Perhitungan jumlah iterasi yang diperlukan
tol = 1e-16
max_iter = 100
data = []
for x0 in x0_list:
  row = {"x0" : tuple(x0)}
  row["Sabri"] = len(iterSabri(x0,tol,max_iter)[0])
  row["JK"] = len(iterJK(x0,tol,max_iter)[0])
  row["Thakur"] = len(iterThakur(x0,tol,max_iter)[0])
  row["Abbas"] = len(iterAbbas(x0,tol,max_iter)[0])
  row["Agarwal"] = len(iterAgarwal(x0,tol,max_iter)[0])
  data.append(row)
tabel = pd.DataFrame(data)
tabel.set_index("x0", inplace=True)

#Penyesuaian titik
def titikdiX(x):
    if isinstance(x, tuple) and len(x) == 2:
        return f"({x[0]:.1f}, {x[1]:.1f}, 0, 0, \\dots)"
    return str(x)
tabel_display = tabel.copy()
tabel_display.index = tabel_display.index.map(titikdiX)

print(tabel_display.to_string())
print(tabel_display.to_latex())
\end{lstlisting}
\begin{lstlisting}[caption={Kode Python untuk Simulasi Minimalisasi Fungsi}]
import numpy as np
from scipy.optimize import minimize


def d(x, y):
    x1,x2 = x[0], x[1]
    y1,y2 = y[0], y[1]

    suku1 = np.abs(x1-y1)**3
    suku2 = np.abs(x1**3-x2-y1**3+y2)**3
    return (suku1+suku2)**(1/3)

def f(x):
    x1, x2 = x
    term1 = 20 * np.abs(x2 - x1**3)**3
    term2 = np.abs(26 - x1)**3
    return term1 + term2

def res(x):
  lam = 20
  def objective(y):
    di = d(x,y)
    return f(y)+1.0/(2.0*lam)*di**2
  res = minimize(objective, x,method="Nelder-Mead", tol=1e-15)
  return res.x

def geodesik(t,w,z):
  w1,w2 = w[0],w[1]
  z1,z2 = z[0],z[1]
  sk1 = (1-t)*w1+t*z1
  sk2 = (1-t)*(w1**3-w2)+t*(z1**3-z2)
  if (t>=0 and t<=1):
    return np.array([sk1,sk1**3-sk2])
  else:
     raise ValueError("di luar domain")

def iterSabri(x0, tol, max_iter):
  data_x = [x0]
  data_error = [np.nan]
  for n in range(max_iter):
    x_n = data_x[-1].copy()
    a_n = 0.34
    b_n = 0
    c_n = 0.93
    q_n = res(geodesik(c_n,x_n,res(x_n)))
    y_n = res(res(q_n))
    x_next = res(geodesik(a_n,res(q_n),res(y_n)))
    data_x.append(x_next)
    error = d(x_next,x_n)
    data_error.append(error)
    if error < tol:
      break
  return data_x, data_error

def iterJK(x0, tol, max_iter):
  data_x = [x0]
  data_error = [np.nan]
  for n in range(max_iter):
    x_n = data_x[-1].copy()
    a_n = 0.34
    b_n = 0
    c_n = 0.93
    q_n = geodesik(c_n,x_n,res(x_n))
    y_n = res(q_n)
    x_next = res(geodesik(a_n,res(q_n),res(y_n)))
    data_x.append(x_next)
    error = d(x_next,x_n)
    data_error.append(error)
    if error < tol:
      break
  return data_x, data_error

dataSabri, errorSabri = iterSabri([10,10],1e-4,500)
dataJK, errorJK = iterJK([10,10],1e-4,500)

max_len = max(len(dataSabri), len(dataJK))

dataSabri += [np.nan] * (max_len - len(dataSabri))
dataJK += [np.nan] * (max_len - len(dataJK))
errorSabri += [np.nan] * (max_len - len(errorSabri))
errorJK += [np.nan] * (max_len - len(errorJK))

import pandas as pd
table = pd.DataFrame({
    'Iterasi n': np.arange(max_len),
    'Sabri': dataSabri,
    'JK': dataJK
})
def fmt_point(x, d=5):
    if isinstance(x, (list, np.ndarray)):
        return f"({x[0]:.{d}f}, {x[1]:.{d}f}, 0, 0, \\dots)"
    return x


table_fmt = table.copy()

for col in ["Sabri", "JK"]:
    table_fmt[col] = table_fmt[col].apply(fmt_point)

print(table_fmt.to_latex(index=False, escape=False))

table_error = pd.DataFrame({
    'Iterasi n': np.arange(max_len),
    'Galat Sabri': errorSabri,
    'Galat JK': errorJK
})

print(table_error.to_latex(index=False, escape=False))

import matplotlib.pyplot as plt

plt.figure(figsize=(14, 10))

for col in ['Galat Sabri', 'Galat JK']:
    plt.plot(table_error['Iterasi n'], table_error[col], marker='o', linestyle='-', label=col)

plt.yscale('log') 
plt.xlabel('Iterasi n',fontsize=18)
plt.ylabel('Galat $d(x_n,x^*)$',fontsize=18)
plt.xticks(fontsize=14)
plt.yticks(fontsize=14)
plt.legend(fontsize=14)
plt.grid(True, which='both', ls='--')
plt.show()
\end{lstlisting}
\begin{lstlisting}[caption={Kode Python untuk Rekonstruksi Gambar}]
import numpy as np
import matplotlib.pyplot as plt
import pandas as pd
import os
from google.colab import files 
from skimage import io, color, transform
from skimage.data import shepp_logan_phantom, grass, moon, coins, text, brain
from skimage.transform import radon, iradon
from skimage.metrics import peak_signal_noise_ratio, mean_squared_error

image_size = 512
theta = np.linspace(0., 180., 180, endpoint=False)
n_iterations = 60
alpha_n = 0.7
beta_n = 0.5
gamma_n = 0.4
checkpoints = [5, 10, 30, 60]

def load_and_prep_image(uploaded_filename, target_size=512):
    print(f"-> Memproses gambar: {uploaded_filename}")
    image = io.imread(uploaded_filename)

    # Jika gambar berwarna (RGB/RGBA), ubah ke Grayscale
    if image.ndim == 3:
        if image.shape[2] == 4: # RGBA
            image = color.rgba2rgb(image)
        image = color.rgb2gray(image)

    # Resize ke target_size (128x128)
    image_resized = transform.resize(image, (target_size, target_size), anti_aliasing=True)
    return image_resized

def gambar():
    print("=== UPLOAD GAMBAR ===")
    uploaded = files.upload()

    if not uploaded:
        print("Tidak ada file yang diupload. Menggunakan default Shepp-Logan Phantom.")
        original_image = shepp_logan_phantom()
        original_image = transform.resize(original_image, (image_size, image_size), anti_aliasing=True)
    else:
        filename = next(iter(uploaded))
        original_image = load_and_prep_image(filename, target_size=image_size)
    return original_image

original_image = gambar()

def A(x):
        return radon(x, theta=theta, circle=False)

def A_adjoint(sinogram):
        return iradon(sinogram, theta=theta, circle=False, output_size=image_size, filter_name=None)

def P_C(x):
        return np.clip(x, 0, 1)

def gammaoptm():
    print("-> Menghitung step size optimal...")
    # Menggunakan random noise untuk estimasi operator
    x_test = np.random.rand(image_size, image_size)
    for _ in range(10):
        Ax = A(x_test)
        AtAx = A_adjoint(Ax)
        norm_val = np.linalg.norm(AtAx)
        x_test = AtAx / norm_val

    Lx = A_adjoint(A(x_test))
    L = np.linalg.norm(Lx) / np.linalg.norm(x_test)
    gamma_step = 1.9 / L
    #gamma_step = 0.001
    print(f"   Gamma Step: {gamma_step:.5f}")
    return gamma_step
    
Q = A(original_image)
gamma_step = gammaoptm()
def T(x):
    Ax = A(x)
    residual = Ax - Q
    gradient = A_adjoint(residual)
    return P_C(x - gamma_step * gradient)

def sabriIter():
    x = np.zeros((image_size, image_size))

    saved_images = {}
    saved_metrics = {}
    data_mse = []
    data_psnr = []

    print(f"-> Memulai Rekonstruksi ({n_iterations} iterasi)...")

    for n in range(n_iterations):
        Tx = T(x)
        z = T((1 - gamma_n) * x + gamma_n * Tx)
        Tz = T(z)
        y = T(Tz)
        Ty = T(y)
        x_next = T((1 - alpha_n) * Tz + alpha_n * Ty)

        x = x_next

        current_psnr = peak_signal_noise_ratio(original_image, x, data_range=1)
        current_mse = mean_squared_error(original_image, x)
        data_mse.append(current_mse)
        data_psnr.append(current_psnr)

        current_iter = n + 1
        if current_iter in checkpoints:
            saved_images[current_iter] = x.copy()
            saved_metrics[current_iter] = (current_psnr, current_mse)
            print(f"   Iterasi {current_iter}:")
            print(f"     PSNR = {current_psnr:.2f} dB")
            print(f"     MSE  = {current_mse:.6f}")
    return saved_images, saved_metrics, data_mse, data_psnr

saved_images, saved_metrics, data_mse, data_psnr = sabriIter()

table = pd.DataFrame({
    'Iterasi n': np.arange(1, len(data_mse) + 1),
    'mse': data_mse,
    'psnr': data_psnr
})
print(table.to_latex(index=False))

def tampilgambar():
    fig, axes = plt.subplots(3, 2, figsize=(10,15))
    ax = axes.ravel()

    # Tampilkan Gambar Asli
    im0 = ax[0].imshow(original_image, cmap='gray', vmin=0, vmax=1)
    ax[0].set_title('Gambar Asli')

    # Tampilkan Sinogram
    ax[1].imshow(Q, cmap='gray', aspect='auto')
    ax[1].set_title('Sinogram')

    # Tampilkan Hasil Iterasi
    for i, iteration in enumerate(checkpoints):
        if iteration in saved_images:
            idx = i + 2
            if idx < 6:
                psnr_val, mse_val = saved_metrics[iteration]
                im = ax[idx].imshow(saved_images[iteration], cmap='gray', vmin=0, vmax=1)
                title_text = (f"Iterasi {iteration}\n"
                              f"PSNR: {psnr_val:.2f} dB\n"
                              f"MSE: {mse_val:.5f}")
                ax[idx].set_title(title_text)
                plt.colorbar(im, ax=ax[idx], fraction=0.046, pad=0.04)

    plt.tight_layout()
    plt.show()

tampilgambar()
\end{lstlisting}
\lampiran{Data MSE dan PSNR dari Rekonstruksi Citra}\label{lamp:mse}
\begin{longtable}{|c|>{\centering\arraybackslash}p{4cm} | >{\centering\arraybackslash}p{4cm} |}
% \caption{\centering MSE dan PSNR Rekonstruksi Citra}\label{tab:msepsnr}\\
\hline
{\textbf{Iterasi ke-$n$}} & {\textbf{MSE}} & {\textbf{PSNR}} \\
\hline
\endfirsthead 
\multicolumn{3}{c}{\centering \thetable{} -- Lanjutan dari halaman sebelumnya} \\
\hline
{\textbf{Iterasi ke-$n$}} & {\textbf{MSE}} & {\textbf{PSNR}} \\
\hline
\endhead 
% \multicolumn{3}{c}{{Lanjut pada halaman berikutnya}} \\
\hline
\endfoot 
\hline
\endlastfoot
\hline
1 & 0.020543 & 16.873256 \\
2 & 0.012665 & 18.973962 \\
3 & 0.009144 & 20.388844 \\
4 & 0.007003 & 21.547021 \\
5 & 0.005555 & 22.553198 \\
6 & 0.004521 & 23.447207 \\
7 & 0.003757 & 24.251257 \\
8 & 0.003176 & 24.980621 \\
9 & 0.002725 & 25.646383 \\
10 & 0.002368 & 26.257024 \\
11 & 0.002080 & 26.819342 \\
12 & 0.001845 & 27.338959 \\
13 & 0.001652 & 27.820674 \\
14 & 0.001490 & 28.268505 \\
15 & 0.001353 & 28.685854 \\
16 & 0.001237 & 29.075668 \\
17 & 0.001137 & 29.440519 \\
18 & 0.001051 & 29.782664 \\
19 & 0.000976 & 30.104105 \\
20 & 0.000911 & 30.406625 \\
21 & 0.000853 & 30.691849 \\
22 & 0.000801 & 30.961254 \\
23 & 0.000756 & 31.216167 \\
24 & 0.000715 & 31.457778 \\
25 & 0.000678 & 31.687186 \\
26 & 0.000645 & 31.905426 \\
27 & 0.000615 & 32.113456 \\
28 & 0.000587 & 32.312125 \\
29 & 0.000562 & 32.502223 \\
30 & 0.000539 & 32.684438 \\
31 & 0.000518 & 32.859369 \\
32 & 0.000498 & 33.027547 \\
33 & 0.000480 & 33.189442 \\
34 & 0.000463 & 33.345487 \\
35 & 0.000447 & 33.496087 \\
36 & 0.000432 & 33.641613 \\
37 & 0.000419 & 33.782387 \\
38 & 0.000406 & 33.918695 \\
39 & 0.000393 & 34.050788 \\
40 & 0.000382 & 34.178888 \\
41 & 0.000371 & 34.303203 \\
42 & 0.000361 & 34.423922 \\
43 & 0.000351 & 34.541223 \\
44 & 0.000342 & 34.655271 \\
45 & 0.000334 & 34.766220 \\
46 & 0.000326 & 34.874213 \\
47 & 0.000318 & 34.979372 \\
48 & 0.000310 & 35.081809 \\
49 & 0.000303 & 35.181635 \\
50 & 0.000297 & 35.278948 \\
51 & 0.000290 & 35.373839 \\
52 & 0.000284 & 35.466395 \\
53 & 0.000278 & 35.556697 \\
54 & 0.000273 & 35.644818 \\
55 & 0.000267 & 35.730825 \\
56 & 0.000262 & 35.814783 \\
57 & 0.000257 & 35.896750 \\
58 & 0.000253 & 35.976784 \\
59 & 0.000248 & 36.054942 \\
60 & 0.000244 & 36.131280 \\
61 & 0.000240 & 36.205855 \\
62 & 0.000236 & 36.278718 \\
63 & 0.000232 & 36.349917 \\
64 & 0.000228 & 36.419499 \\
65 & 0.000225 & 36.487511 \\
66 & 0.000221 & 36.553995 \\
67 & 0.000218 & 36.618991 \\
68 & 0.000215 & 36.682537 \\
69 & 0.000212 & 36.744680 \\
70 & 0.000209 & 36.805454 \\
71 & 0.000206 & 36.864897 \\
72 & 0.000203 & 36.923043 \\
73 & 0.000200 & 36.979926 \\
74 & 0.000198 & 37.035589 \\
75 & 0.000195 & 37.090063 \\
76 & 0.000193 & 37.143385 \\
77 & 0.000191 & 37.195576 \\
78 & 0.000189 & 37.246666 \\
79 & 0.000186 & 37.296681 \\
80 & 0.000184 & 37.345643 \\
81 & 0.000182 & 37.393584 \\
82 & 0.000180 & 37.440528 \\
83 & 0.000178 & 37.486502 \\
84 & 0.000177 & 37.531531 \\
85 & 0.000175 & 37.575634 \\
86 & 0.000173 & 37.618835 \\
87 & 0.000171 & 37.661157 \\
88 & 0.000170 & 37.702623 \\
89 & 0.000168 & 37.743261 \\
90 & 0.000167 & 37.783084 \\
91 & 0.000165 & 37.822111 \\
92 & 0.000164 & 37.860363 \\
93 & 0.000162 & 37.897858 \\
94 & 0.000161 & 37.934619 \\
95 & 0.000160 & 37.970657 \\
96 & 0.000158 & 38.005993 \\
97 & 0.000157 & 38.040643 \\
98 & 0.000156 & 38.074626 \\
99 & 0.000155 & 38.107957 \\
100 & 0.000153 & 38.140655 \\
101 & 0.000152 & 38.172740 \\
102 & 0.000151 & 38.204231 \\
103 & 0.000150 & 38.235136 \\
104 & 0.000149 & 38.265467 \\
105 & 0.000148 & 38.295240 \\
106 & 0.000147 & 38.324464 \\
107 & 0.000146 & 38.353154 \\
108 & 0.000145 & 38.381328 \\
109 & 0.000144 & 38.408997 \\
110 & 0.000143 & 38.436173 \\
111 & 0.000142 & 38.462868 \\
112 & 0.000142 & 38.489089 \\
113 & 0.000141 & 38.514855 \\
114 & 0.000140 & 38.540169 \\
115 & 0.000139 & 38.565041 \\
116 & 0.000138 & 38.589480 \\
117 & 0.000138 & 38.613496 \\
118 & 0.000137 & 38.637101 \\
119 & 0.000136 & 38.660303 \\
120 & 0.000135 & 38.683111 \\
121 & 0.000135 & 38.705535 \\
122 & 0.000134 & 38.727583 \\
123 & 0.000133 & 38.749263 \\
124 & 0.000133 & 38.770582 \\
125 & 0.000132 & 38.791548 \\
126 & 0.000131 & 38.812170 \\
127 & 0.000131 & 38.832458 \\
128 & 0.000130 & 38.852419 \\
129 & 0.000130 & 38.872059 \\
130 & 0.000129 & 38.891383 \\
131 & 0.000129 & 38.910396 \\
132 & 0.000128 & 38.929107 \\
133 & 0.000127 & 38.947522 \\
134 & 0.000127 & 38.965647 \\
135 & 0.000126 & 38.983489 \\
136 & 0.000126 & 39.001055 \\
137 & 0.000125 & 39.018349 \\
138 & 0.000125 & 39.035378 \\
139 & 0.000124 & 39.052147 \\
140 & 0.000124 & 39.068661 \\
141 & 0.000123 & 39.084928 \\
142 & 0.000123 & 39.100952 \\
143 & 0.000123 & 39.116738 \\
144 & 0.000122 & 39.132293 \\
145 & 0.000122 & 39.147617 \\
146 & 0.000121 & 39.162718 \\
147 & 0.000121 & 39.177600 \\
148 & 0.000120 & 39.192266 \\
149 & 0.000120 & 39.206725 \\
150 & 0.000120 & 39.220980 \\
151 & 0.000119 & 39.235034 \\
152 & 0.000119 & 39.248889 \\
153 & 0.000119 & 39.262551 \\
154 & 0.000118 & 39.276024 \\
155 & 0.000118 & 39.289312 \\
156 & 0.000117 & 39.302419 \\
157 & 0.000117 & 39.315350 \\
158 & 0.000117 & 39.328107 \\
159 & 0.000116 & 39.340691 \\
160 & 0.000116 & 39.353108 \\
161 & 0.000116 & 39.365361 \\
162 & 0.000115 & 39.377454 \\
163 & 0.000115 & 39.389390 \\
164 & 0.000115 & 39.401172 \\
165 & 0.000114 & 39.412801 \\
166 & 0.000114 & 39.424280 \\
167 & 0.000114 & 39.435614 \\
168 & 0.000114 & 39.446804 \\
169 & 0.000113 & 39.457855 \\
170 & 0.000113 & 39.468768 \\
171 & 0.000113 & 39.479546 \\
172 & 0.000112 & 39.490190 \\
173 & 0.000112 & 39.500703 \\
174 & 0.000112 & 39.511089 \\
175 & 0.000112 & 39.521348 \\
176 & 0.000111 & 39.531485 \\
177 & 0.000111 & 39.541500 \\
178 & 0.000111 & 39.551398 \\
179 & 0.000111 & 39.561181 \\
180 & 0.000110 & 39.570849 \\
181 & 0.000110 & 39.580405 \\
182 & 0.000110 & 39.589850 \\
183 & 0.000110 & 39.599190 \\
184 & 0.000109 & 39.608423 \\
185 & 0.000109 & 39.617552 \\
186 & 0.000109 & 39.626577 \\
187 & 0.000109 & 39.635502 \\
188 & 0.000109 & 39.644327 \\
189 & 0.000108 & 39.653055 \\
190 & 0.000108 & 39.661690 \\
191 & 0.000108 & 39.670233 \\
192 & 0.000108 & 39.678683 \\
193 & 0.000107 & 39.687043 \\
194 & 0.000107 & 39.695314 \\
195 & 0.000107 & 39.703497 \\
196 & 0.000107 & 39.711595 \\
197 & 0.000107 & 39.719609 \\
198 & 0.000106 & 39.727540 \\
199 & 0.000106 & 39.735392 \\
200 & 0.000106 & 39.743164 \\
201 & 0.000106 & 39.750857 \\
202 & 0.000106 & 39.758472 \\
203 & 0.000106 & 39.766012 \\
204 & 0.000105 & 39.773476 \\
205 & 0.000105 & 39.780866 \\
206 & 0.000105 & 39.788184 \\
207 & 0.000105 & 39.795432 \\
208 & 0.000105 & 39.802609 \\
209 & 0.000104 & 39.809718 \\
210 & 0.000104 & 39.816758 \\
211 & 0.000104 & 39.823732 \\
212 & 0.000104 & 39.830640 \\
213 & 0.000104 & 39.837483 \\
214 & 0.000104 & 39.844265 \\
215 & 0.000103 & 39.850986 \\
216 & 0.000103 & 39.857645 \\
217 & 0.000103 & 39.864243 \\
218 & 0.000103 & 39.870783 \\
219 & 0.000103 & 39.877265 \\
220 & 0.000103 & 39.883688 \\
221 & 0.000103 & 39.890055 \\
222 & 0.000102 & 39.896365 \\
223 & 0.000102 & 39.902620 \\
224 & 0.000102 & 39.908821 \\
225 & 0.000102 & 39.914968 \\
226 & 0.000102 & 39.921063 \\
227 & 0.000102 & 39.927107 \\
228 & 0.000102 & 39.933099 \\
229 & 0.000101 & 39.939040 \\
230 & 0.000101 & 39.944932 \\
231 & 0.000101 & 39.950775 \\
232 & 0.000101 & 39.956569 \\
233 & 0.000101 & 39.962317 \\
234 & 0.000101 & 39.968019 \\
235 & 0.000101 & 39.973674 \\
236 & 0.000100 & 39.979283 \\
237 & 0.000100 & 39.984848 \\
238 & 0.000100 & 39.990368 \\
239 & 0.000100 & 39.995845 \\
240 & 0.000100 & 40.001278 \\
241 & 0.000100 & 40.006670 \\
242 & 0.000100 & 40.012020 \\
243 & 0.000100 & 40.017330 \\
244 & 0.000099 & 40.022599 \\
245 & 0.000099 & 40.027828 \\
246 & 0.000099 & 40.033017 \\
247 & 0.000099 & 40.038168 \\
248 & 0.000099 & 40.043280 \\
249 & 0.000099 & 40.048353 \\
250 & 0.000099 & 40.053389 \\
251 & 0.000099 & 40.058388 \\
252 & 0.000099 & 40.063350 \\
253 & 0.000098 & 40.068277 \\
254 & 0.000098 & 40.073169 \\
255 & 0.000098 & 40.078026 \\
256 & 0.000098 & 40.082849 \\
257 & 0.000098 & 40.087638 \\
258 & 0.000098 & 40.092394 \\
259 & 0.000098 & 40.097117 \\
260 & 0.000098 & 40.101808 \\
261 & 0.000098 & 40.106465 \\
262 & 0.000097 & 40.111091 \\
263 & 0.000097 & 40.115686 \\
264 & 0.000097 & 40.120249 \\
265 & 0.000097 & 40.124783 \\
266 & 0.000097 & 40.129287 \\
267 & 0.000097 & 40.133761 \\
268 & 0.000097 & 40.138205 \\
269 & 0.000097 & 40.142620 \\
270 & 0.000097 & 40.147007 \\
271 & 0.000097 & 40.151364 \\
272 & 0.000096 & 40.155694 \\
273 & 0.000096 & 40.159996 \\
274 & 0.000096 & 40.164270 \\
275 & 0.000096 & 40.168517 \\
276 & 0.000096 & 40.172738 \\
277 & 0.000096 & 40.176931 \\
278 & 0.000096 & 40.181099 \\
279 & 0.000096 & 40.185241 \\
280 & 0.000096 & 40.189357 \\
281 & 0.000096 & 40.193447 \\
282 & 0.000096 & 40.197513 \\
283 & 0.000095 & 40.201554 \\
284 & 0.000095 & 40.205571 \\
285 & 0.000095 & 40.209563 \\
286 & 0.000095 & 40.213533 \\
287 & 0.000095 & 40.217479 \\
288 & 0.000095 & 40.221402 \\
289 & 0.000095 & 40.225302 \\
290 & 0.000095 & 40.229179 \\
291 & 0.000095 & 40.233033 \\
292 & 0.000095 & 40.236866 \\
293 & 0.000095 & 40.240676 \\
294 & 0.000095 & 40.244465 \\
295 & 0.000094 & 40.248232 \\
296 & 0.000094 & 40.251978 \\
297 & 0.000094 & 40.255703 \\
298 & 0.000094 & 40.259408 \\
299 & 0.000094 & 40.263091 \\
300 & 0.000094 & 40.266755 \\
301 & 0.000094 & 40.270399 \\
302 & 0.000094 & 40.274023 \\
303 & 0.000094 & 40.277628 \\
304 & 0.000094 & 40.281214 \\
305 & 0.000094 & 40.284781 \\
306 & 0.000094 & 40.288329 \\
307 & 0.000094 & 40.291858 \\
308 & 0.000093 & 40.295369 \\
309 & 0.000093 & 40.298862 \\
310 & 0.000093 & 40.302336 \\
311 & 0.000093 & 40.305792 \\
312 & 0.000093 & 40.309231 \\
313 & 0.000093 & 40.312652 \\
314 & 0.000093 & 40.316056 \\
315 & 0.000093 & 40.319442 \\
316 & 0.000093 & 40.322811 \\
317 & 0.000093 & 40.326163 \\
318 & 0.000093 & 40.329498 \\
319 & 0.000093 & 40.332817 \\
320 & 0.000093 & 40.336121 \\
321 & 0.000092 & 40.339408 \\
322 & 0.000092 & 40.342679 \\
323 & 0.000092 & 40.345934 \\
324 & 0.000092 & 40.349174 \\
325 & 0.000092 & 40.352397 \\
326 & 0.000092 & 40.355605 \\
327 & 0.000092 & 40.358798 \\
328 & 0.000092 & 40.361975 \\
329 & 0.000092 & 40.365138 \\
330 & 0.000092 & 40.368286 \\
331 & 0.000092 & 40.371419 \\
332 & 0.000092 & 40.374537 \\
333 & 0.000092 & 40.377641 \\
334 & 0.000092 & 40.380730 \\
335 & 0.000092 & 40.383806 \\
336 & 0.000091 & 40.386867 \\
337 & 0.000091 & 40.389916 \\
338 & 0.000091 & 40.392950 \\
339 & 0.000091 & 40.395971 \\
340 & 0.000091 & 40.398979 \\
341 & 0.000091 & 40.401973 \\
342 & 0.000091 & 40.404955 \\
343 & 0.000091 & 40.407923 \\
344 & 0.000091 & 40.410878 \\
345 & 0.000091 & 40.413819 \\
346 & 0.000091 & 40.416748 \\
347 & 0.000091 & 40.419665 \\
348 & 0.000091 & 40.422568 \\
349 & 0.000091 & 40.425459 \\
350 & 0.000091 & 40.428338 \\
351 & 0.000091 & 40.431205 \\
352 & 0.000090 & 40.434061 \\
353 & 0.000090 & 40.436905 \\
354 & 0.000090 & 40.439737 \\
355 & 0.000090 & 40.442557 \\
356 & 0.000090 & 40.445365 \\
357 & 0.000090 & 40.448162 \\
358 & 0.000090 & 40.450947 \\
359 & 0.000090 & 40.453720 \\
360 & 0.000090 & 40.456482 \\
361 & 0.000090 & 40.459232 \\
362 & 0.000090 & 40.461972 \\
363 & 0.000090 & 40.464700 \\
364 & 0.000090 & 40.467417 \\
365 & 0.000090 & 40.470123 \\
366 & 0.000090 & 40.472818 \\
367 & 0.000090 & 40.475502 \\
368 & 0.000090 & 40.478175 \\
369 & 0.000090 & 40.480838 \\
370 & 0.000089 & 40.483490 \\
371 & 0.000089 & 40.486132 \\
372 & 0.000089 & 40.488764 \\
373 & 0.000089 & 40.491385 \\
374 & 0.000089 & 40.493996 \\
375 & 0.000089 & 40.496597 \\
376 & 0.000089 & 40.499187 \\
377 & 0.000089 & 40.501768 \\
378 & 0.000089 & 40.504339 \\
379 & 0.000089 & 40.506900 \\
380 & 0.000089 & 40.509451 \\
381 & 0.000089 & 40.511992 \\
382 & 0.000089 & 40.514525 \\
383 & 0.000089 & 40.517047 \\
384 & 0.000089 & 40.519561 \\
385 & 0.000089 & 40.522065 \\
386 & 0.000089 & 40.524559 \\
387 & 0.000089 & 40.527045 \\
388 & 0.000089 & 40.529521 \\
389 & 0.000088 & 40.531988 \\
390 & 0.000088 & 40.534446 \\
391 & 0.000088 & 40.536895 \\
392 & 0.000088 & 40.539336 \\
393 & 0.000088 & 40.541767 \\
394 & 0.000088 & 40.544190 \\
395 & 0.000088 & 40.546603 \\
396 & 0.000088 & 40.549009 \\
397 & 0.000088 & 40.551405 \\
398 & 0.000088 & 40.553793 \\
399 & 0.000088 & 40.556173 \\
400 & 0.000088 & 40.558544 \\
401 & 0.000088 & 40.560908 \\
402 & 0.000088 & 40.563263 \\
403 & 0.000088 & 40.565611 \\
404 & 0.000088 & 40.567950 \\
405 & 0.000088 & 40.570282 \\
406 & 0.000088 & 40.572605 \\
407 & 0.000088 & 40.574920 \\
408 & 0.000088 & 40.577228 \\
409 & 0.000088 & 40.579527 \\
410 & 0.000087 & 40.581819 \\
411 & 0.000087 & 40.584104 \\
412 & 0.000087 & 40.586380 \\
413 & 0.000087 & 40.588649 \\
414 & 0.000087 & 40.590910 \\
415 & 0.000087 & 40.593164 \\
416 & 0.000087 & 40.595410 \\
417 & 0.000087 & 40.597649 \\
418 & 0.000087 & 40.599880 \\
419 & 0.000087 & 40.602104 \\
420 & 0.000087 & 40.604320 \\
421 & 0.000087 & 40.606529 \\
422 & 0.000087 & 40.608732 \\
423 & 0.000087 & 40.610926 \\
424 & 0.000087 & 40.613114 \\
425 & 0.000087 & 40.615295 \\
426 & 0.000087 & 40.617469 \\
427 & 0.000087 & 40.619635 \\
428 & 0.000087 & 40.621795 \\
429 & 0.000087 & 40.623948 \\
430 & 0.000087 & 40.626094 \\
431 & 0.000087 & 40.628233 \\
432 & 0.000086 & 40.630365 \\
433 & 0.000086 & 40.632491 \\
434 & 0.000086 & 40.634610 \\
435 & 0.000086 & 40.636722 \\
436 & 0.000086 & 40.638828 \\
437 & 0.000086 & 40.640927 \\
438 & 0.000086 & 40.643019 \\
439 & 0.000086 & 40.645106 \\
440 & 0.000086 & 40.647185 \\
441 & 0.000086 & 40.649259 \\
442 & 0.000086 & 40.651326 \\
443 & 0.000086 & 40.653387 \\
444 & 0.000086 & 40.655441 \\
445 & 0.000086 & 40.657489 \\
446 & 0.000086 & 40.659531 \\
447 & 0.000086 & 40.661567 \\
448 & 0.000086 & 40.663597 \\
449 & 0.000086 & 40.665620 \\
450 & 0.000086 & 40.667638 \\
451 & 0.000086 & 40.669650 \\
452 & 0.000086 & 40.671657 \\
453 & 0.000086 & 40.673658 \\
454 & 0.000086 & 40.675653 \\
455 & 0.000086 & 40.677642 \\
456 & 0.000086 & 40.679625 \\
457 & 0.000085 & 40.681603 \\
458 & 0.000085 & 40.683575 \\
459 & 0.000085 & 40.685541 \\
460 & 0.000085 & 40.687502 \\
461 & 0.000085 & 40.689457 \\
462 & 0.000085 & 40.691406 \\
463 & 0.000085 & 40.693350 \\
464 & 0.000085 & 40.695288 \\
465 & 0.000085 & 40.697221 \\
466 & 0.000085 & 40.699149 \\
467 & 0.000085 & 40.701071 \\
468 & 0.000085 & 40.702988 \\
469 & 0.000085 & 40.704899 \\
470 & 0.000085 & 40.706805 \\
471 & 0.000085 & 40.708705 \\
472 & 0.000085 & 40.710601 \\
473 & 0.000085 & 40.712491 \\
474 & 0.000085 & 40.714375 \\
475 & 0.000085 & 40.716255 \\
476 & 0.000085 & 40.718129 \\
477 & 0.000085 & 40.719999 \\
478 & 0.000085 & 40.721863 \\
479 & 0.000085 & 40.723722 \\
480 & 0.000085 & 40.725577 \\
\end{longtable}
\pagebreak
% \lampiran{Kode Python untuk Simulasi Minimalisasi Fungsi}
\begin{lstlisting}[caption={Kode Minimalisasi Fungsi}]
import numpy as np
from scipy.optimize import minimize


def d(x, y):
    x1,x2 = x[0], x[1]
    y1,y2 = y[0], y[1]

    suku1 = np.abs(x1-y1)**3
    suku2 = np.abs(x1**3-x2-y1**3+y2)**3
    return (suku1+suku2)**(1/3)

def f(x):
    x1, x2 = x
    term1 = 20 * np.abs(x2 - x1**3)**3
    term2 = np.abs(26 - x1)**3
    return term1 + term2

def res(x):
  lam = 20
  def objective(y):
    di = d(x,y)
    return f(y)+1.0/(2.0*lam)*di**2
  res = minimize(objective, x,method="Nelder-Mead", tol=1e-15)
  return res.x

def geodesik(t,w,z):
  w1,w2 = w[0],w[1]
  z1,z2 = z[0],z[1]
  sk1 = (1-t)*w1+t*z1
  sk2 = (1-t)*(w1**3-w2)+t*(z1**3-z2)
  if (t>=0 and t<=1):
    return np.array([sk1,sk1**3-sk2])
  else:
     raise ValueError("di luar domain")

def iterSabri(x0, tol, max_iter):
  data_x = [x0]
  data_error = [np.nan]
  for n in range(max_iter):
    x_n = data_x[-1].copy()
    a_n = 0.34
    b_n = 0
    c_n = 0.93
    q_n = res(geodesik(c_n,x_n,res(x_n)))
    y_n = res(res(q_n))
    x_next = res(geodesik(a_n,res(q_n),res(y_n)))
    data_x.append(x_next)
    error = d(x_next,x_n)
    data_error.append(error)
    if error < tol:
      break
  return data_x, data_error

def iterJK(x0, tol, max_iter):
  data_x = [x0]
  data_error = [np.nan]
  for n in range(max_iter):
    x_n = data_x[-1].copy()
    a_n = 0.34
    b_n = 0
    c_n = 0.93
    q_n = geodesik(c_n,x_n,res(x_n))
    y_n = res(q_n)
    x_next = res(geodesik(a_n,res(q_n),res(y_n)))
    data_x.append(x_next)
    error = d(x_next,x_n)
    data_error.append(error)
    if error < tol:
      break
  return data_x, data_error

dataSabri, errorSabri = iterSabri([10,10],1e-4,500)
dataJK, errorJK = iterJK([10,10],1e-4,500)

max_len = max(len(dataSabri), len(dataJK))

dataSabri += [np.nan] * (max_len - len(dataSabri))
dataJK += [np.nan] * (max_len - len(dataJK))
errorSabri += [np.nan] * (max_len - len(errorSabri))
errorJK += [np.nan] * (max_len - len(errorJK))

import pandas as pd
table = pd.DataFrame({
    'Iterasi n': np.arange(max_len),
    'Sabri': dataSabri,
    'JK': dataJK
})
def fmt_point(x, d=5):
    if isinstance(x, (list, np.ndarray)):
        return f"({x[0]:.{d}f}, {x[1]:.{d}f}, 0, 0, \\dots)"
    return x


table_fmt = table.copy()

for col in ["Sabri", "JK"]:
    table_fmt[col] = table_fmt[col].apply(fmt_point)

print(table_fmt.to_latex(index=False, escape=False))

table_error = pd.DataFrame({
    'Iterasi n': np.arange(max_len),
    'Galat Sabri': errorSabri,
    'Galat JK': errorJK
})

print(table_error.to_latex(index=False, escape=False))

import matplotlib.pyplot as plt

plt.figure(figsize=(14, 10))

for col in ['Galat Sabri', 'Galat JK']:
    plt.plot(table_error['Iterasi n'], table_error[col], marker='o', linestyle='-', label=col)

plt.yscale('log') 
plt.xlabel('Iterasi n',fontsize=18)
plt.ylabel('Galat $d(x_n,x^*)$',fontsize=18)
plt.xticks(fontsize=14)
plt.yticks(fontsize=14)
plt.legend(fontsize=14)
plt.grid(True, which='both', ls='--')
plt.show()
\end{lstlisting}
% \pagebreak
% \lampiran{Kode Python untuk Simulasi Rekonstruksi Gambar}
\begin{lstlisting}[caption={Kode Rekonstruksi Gambar}]
import numpy as np
import matplotlib.pyplot as plt
import pandas as pd
import os
from google.colab import files 
from skimage import io, color, transform
from skimage.data import shepp_logan_phantom, grass, moon, coins, text, brain
from skimage.transform import radon, iradon
from skimage.metrics import peak_signal_noise_ratio, mean_squared_error

image_size = 512
theta = np.linspace(0., 180., 180, endpoint=False)
n_iterations = 60
alpha_n = 0.7
beta_n = 0.5
gamma_n = 0.4
checkpoints = [5, 10, 30, 60]

def load_and_prep_image(uploaded_filename, target_size=512):
    print(f"-> Memproses gambar: {uploaded_filename}")
    image = io.imread(uploaded_filename)

    # Jika gambar berwarna (RGB/RGBA), ubah ke Grayscale
    if image.ndim == 3:
        if image.shape[2] == 4: # RGBA
            image = color.rgba2rgb(image)
        image = color.rgb2gray(image)

    # Resize ke target_size (128x128)
    image_resized = transform.resize(image, (target_size, target_size), anti_aliasing=True)
    return image_resized

def gambar():
    print("=== UPLOAD GAMBAR ===")
    uploaded = files.upload()

    if not uploaded:
        print("Tidak ada file yang diupload. Menggunakan default Shepp-Logan Phantom.")
        original_image = shepp_logan_phantom()
        original_image = transform.resize(original_image, (image_size, image_size), anti_aliasing=True)
    else:
        filename = next(iter(uploaded))
        original_image = load_and_prep_image(filename, target_size=image_size)
    return original_image

original_image = gambar()

def gammaoptm():
    print("-> Menghitung step size optimal...")
    # Menggunakan random noise untuk estimasi operator
    x_test = np.random.rand(image_size, image_size)
    for _ in range(10):
        Ax = A(x_test)
        AtAx = A_adjoint(Ax)
        norm_val = np.linalg.norm(AtAx)
        x_test = AtAx / norm_val

    Lx = A_adjoint(A(x_test))
    L = np.linalg.norm(Lx) / np.linalg.norm(x_test)
    gamma_step = 1.9 / L
    #gamma_step = 0.001
    print(f"   Gamma Step: {gamma_step:.5f}")
    return gamma_step

def A(x):
        return radon(x, theta=theta, circle=False)

def A_adjoint(sinogram):
        return iradon(sinogram, theta=theta, circle=False, output_size=image_size, filter_name=None)

def P_C(x):
        return np.clip(x, 0, 1)

Q = A(original_image)
gamma_step = gammaoptm()
def T(x):
    Ax = A(x)
    residual = Ax - Q
    gradient = A_adjoint(residual)
    return P_C(x - gamma_step * gradient)

def sabriIter():
    x = np.zeros((image_size, image_size))

    saved_images = {}
    saved_metrics = {}
    data_mse = []
    data_psnr = []

    print(f"-> Memulai Rekonstruksi ({n_iterations} iterasi)...")

    for n in range(n_iterations):
        Tx = T(x)
        z = T((1 - gamma_n) * x + gamma_n * Tx)
        Tz = T(z)
        y = T(Tz)
        Ty = T(y)
        x_next = T((1 - alpha_n) * Tz + alpha_n * Ty)

        x = x_next

        current_psnr = peak_signal_noise_ratio(original_image, x, data_range=1)
        current_mse = mean_squared_error(original_image, x)
        data_mse.append(current_mse)
        data_psnr.append(current_psnr)

        current_iter = n + 1
        if current_iter in checkpoints:
            saved_images[current_iter] = x.copy()
            saved_metrics[current_iter] = (current_psnr, current_mse)
            print(f"   Iterasi {current_iter}:")
            print(f"     PSNR = {current_psnr:.2f} dB")
            print(f"     MSE  = {current_mse:.6f}")
    return saved_images, saved_metrics, data_mse, data_psnr

saved_images, saved_metrics, data_mse, data_psnr = sabriIter()

table = pd.DataFrame({
    'Iterasi n': np.arange(1, len(data_mse) + 1),
    'mse': data_mse,
    'psnr': data_psnr
})
print(table.to_latex(index=False))

def tampilgambar():
    fig, axes = plt.subplots(3, 2, figsize=(10,15))
    ax = axes.ravel()

    # Tampilkan Gambar Asli
    im0 = ax[0].imshow(original_image, cmap='gray', vmin=0, vmax=1)
    ax[0].set_title('Gambar Asli')

    # Tampilkan Sinogram
    ax[1].imshow(Q, cmap='gray', aspect='auto')
    ax[1].set_title('Sinogram')

    # Tampilkan Hasil Iterasi
    for i, iteration in enumerate(checkpoints):
        if iteration in saved_images:
            idx = i + 2
            if idx < 6:
                psnr_val, mse_val = saved_metrics[iteration]
                im = ax[idx].imshow(saved_images[iteration], cmap='gray', vmin=0, vmax=1)
                title_text = (f"Iterasi {iteration}\n"
                              f"PSNR: {psnr_val:.2f} dB\n"
                              f"MSE: {mse_val:.5f}")
                ax[idx].set_title(title_text)
                plt.colorbar(im, ax=ax[idx], fraction=0.046, pad=0.04)

    plt.tight_layout()
    plt.show()

tampilgambar()
\end{lstlisting}
% \pagebreak
% % \UcapanTerimaKasih
Penyelesaian penulisan tugas akhir ini tidak lepas dari orang-orang terdekat Penulis yang telah mendukung, menemani, dan memotivasi penulis selama berkuliah di Departemen Matematika Institut Teknologi Sepuluh Nopember. Oleh sebab itu, Penulis mengucapkan terima kasih kepada:

Akhir kata, semoga tesis ini bermanfaat bagi semua pihak yang bersangkutan.\\
\begin{enumerate}
	\item Bintang Marvellianti yang senantiasa menemani keseharian penulis selama menjalani perkuliahan magister.
	\item Penghuni Laboratorium Aljabar Analisis, khususnya:
	\begin{enumerate}
		\item Mas Ilham, Mas Alvian, terima kasih ilmu dan apalah
		\item Teo, Junika, Huda
		\item Vincent, Jundi, Fajar sebagai teman main EAFC.
		\item Isa, Rasyidah, Isma, Feli
	\end{enumerate}
	\item Dzaky selaku rekan seperbimbingan penulis
	\item Krisna dan Kygung yang mengajak dan mengajari penulis untuk ngegym.
	\item Teman-teman main voli dan badminton, Mas Felza, Mas Ridho, Mas Zaki, Mas Ajrian.
\end{enumerate}

\begin{flushright}
\begin{tabular}{r}
	 Surabaya, 07 Februari 2026\vspace{1cm}\\\\\\
	 Ahmad Hisbu Zakiyudin
\end{tabular}
\end{flushright}
% \pagebreak
\Biodata{foto}
Lahir di Madiun, 15 Maret 2003, penulis memiliki nama lengkap Ahmad Hisbu Zakiyudin. Sebagian memanggil dengan Zaki, sebagian lagi dengan Hisbu, Bubu, Udin, atau Ahmad. Enam tahun penulis menempuh pendidikan dasar di MI Nurul Huda Sawahan, dua tahun menempuh pendidikan menengah pertama di MTsN Kota Madiun, dan tiga tahun menempuh pendidikan menengah atas di MAN 2 Kota Madiun. Kemudian pada saat pandemi mulai Covid-19 mulai menyebar, yaitu pada tahun 2020, penulis melanjutkan pendidikan tinggi di Departemen Matematika, Fakultas Sains dan Analitika Data (FSAD), Institut Teknologi Sepuluh Nopember (ITS) Surabaya. Penulis menyelesaikan program Sarjana pada Departemen Matematika ITS pada tahun 2020, dengan judul tugas akhir "Aplikasi Teorema Titik Tetap pada Ruang Metrik Kuasi untuk Masalah Eksistensi dan Ketunggalan Solusi dari Persamaan Cauchy Tidak Homogen". Hasil dari tugas akhir tersebut juga dipublikasikan dalam Jurnal Ilmiah terindeks Scopus dengan peringkat Q3, yaitu pada Jurnal Nonlinear Dynamics \& Systems Theory yang terbit pada Edisi ke-5 Tahun 2025. Penulis kemudian melanjutkan pendidikan pascasarjana pada program Magister by Riset di Departemen Matematika ITS. Selain itu, penulis juga sempat mengikuti dua konferensi ilmiah internasional sebagai \textit{presenter}, yaitu pada The International Conference in Mathematical Analysis and Applications (ICONMAA) di tahun 2024 dan International Conference on Mathematics: Pure, Applied, and Computation (ICoMPAC) di tahun 2025. 

Adapun kritik dan saran atau informasi lebih lanjut mengenai Tesis ini dapat ditujukan ke penulis melalui email penulis, yakni \href{mailto:azakiyudin1790@gmail.com}{azakiyudin1790@gmail.com} atau LinkedIn penulis, yaitu Ahmad Hisbu Zakiyudin atau bisa diakses melalui \url{www.linkedin.com/in/ahmadhisbuzakiyudin}.
\end{document}
