\chapter{KESIMPULAN DAN SARAN}
Berdasarkan uraian dari bab-bab sebelumnya, berikut diberikan kesimpulan yang diperoleh dari tesis ini dan saran untuk penelitian selanjutnya.
\section{Kesimpulan}
Berdasarkan hasil dan pembahasan pada Bab \ref{chap:bab4}, didapatkan kesimpulan dari tesis ini sebagai berikut.
\begin{enumerate}
    \item Pemetaan $(\alpha,\beta,\gamma)$-nonekspansif yang awalnya didefinisikan pada ruang Banach dapat diperluas pada ruang $CAT_p(0)$. Dalam ruang ini, pemetaan tersebut juga memenuhi sifat nonekspansif kuasi dan \textit{demiclosedness}. Untuk mendapatkan nilai aproksimasi titik tetap dari pemetaan ini, dapat digunakan skema iterasi Sabri yang terbukti memiliki konvergensi-$\Delta$ dan kuat. Syarat cukup konvergensi$-\Delta$ dari aproksimasi titik tetap pemetaan $(\alpha,\beta,\gamma)$-nonekspansif di ruang $CAT_p(0)$ adalah ruangnya lengkap, pemetaannya memiliki titik tetap dan merupakan pemetaan diri sendiri pada himpunan bagian tak kosong yang tertutup dan konveks, serta barisan koefisien pada skema iterasinya berada pada interval $(0,1)$. Syarat cukup konvergensi kuat sama seperti dengan konvergensi$-\Delta$, dengan syarat tambahan bahwa pemetaan tersebut pada himpunan kompak. Selain itu, didapatkan syarat perlu dari konvergensi$-\Delta$ dan kuat, yaitu titik tetapnya tunggal.
    % Untuk syarat cukup dari konvergensi$-\Delta$ adalah sebagai berikut.
    % \begin{enumerate}
    %     \item Ruang $(X,d)$ adalah ruang $CAT_p(0)$ yang lengkap.
    %     \item Himpunan bagian $W$ merupakan himpunan bagian tak kosong yang tertutup dan konveks dari ruang $CAT_p(0)$.
    %     \item Pemetaan $T:W\to W$ adalah pemetaan $(\alpha,\beta,\gamma)$-nonekspansif dengan $Fix(T)\neq \emptyset$.
    %     \item Barisan koefisien $\{a_n\}$ dan $\{b_n\}$ berada pada interval $(0,1)$.
    % \end{enumerate}
    % Untuk konvergensi kuat, diperlukan syarat cukup tambahan, yaitu himpunan bagian $W$ merupakan himpunan kompak.
    \item Berdasarkan hasil percobaan numerik, skema iterasi Sabri memiliki laju yang lebih cepat dibanding skema iterasi JK, Thakur, Abbas, dan Agarwal, dalam aproksimasi titik tetap dari pemetaan $(\alpha,\beta,\gamma)$-nonekspansif di ruang $CAT_p(0)$, yang ditunjukkan oleh banyaknya jumlah iterasi yang diperlukan untuk mencapai galat yang ditentukan. 
    % Hasil ini selaras dengan hasil yang diperoleh peneliti sebelumnya pada ruang Banach yaitu oleh \cite{Arif2025}.
    \item Dalam masalah optimasi, skema iterasi Sabri menghasilkan barisan yang konvergen menuju solusi dari permasalahan minimalisasi fungsi yang konveks dan \textit{proper lower semi-continuous}. Selain itu, dalam rekonstruksi citra tomografi, skema ini berhasil merekonstruksi citra Shepp-Logan Phantom dengan penurunan nilai galat dan peningkatan kualitas citra pada tiap iterasinya.
\end{enumerate}
\section{Saran}
Berdasarkan hasil yang telah diperoleh dalam tesis ini, terdapat beberapa saran yang dapat dipertimbangkan untuk penelitian selanjutnya. 
\begin{enumerate}
    \item Kajian mengenai pemetaan $(\alpha,\beta,\gamma)$-nonekspansif pada ruang $CAT_p(0)$ masih dapat diperluas ke kelas ruang metrik tak linier lainnya.
    \item Kajian syarat konvergensi kuat dari skema iterasi Sabri dapat diteliti lebih lanjut dengan menghilangkan syarat kompak pada himpunan bagian $W$.
    \item Perbandingan kelajuan konvergensi pada penelitian ini masih terbatas secara percobaan numerik, sehingga disarankan untuk mendapatkan perbandingan lajunya secara analitik. 
    \item Penerapan skema iterasi Sabri pada permasalahan rekonstruksi citra tomografi dapat diuji lebih lanjut dengan menggunakan data medis nyata yang mengandung derau tinggi atau sudut proyeksi yang terbatas. 
\end{enumerate}  