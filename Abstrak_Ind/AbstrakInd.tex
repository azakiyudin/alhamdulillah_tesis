
\begin{Abstrak}
Teori titik tetap, yang diawali oleh Teorema Titik Tetap Banach untuk pemetaan kontraktif, telah berkembang secara signifikan untuk mencakup kelas pemetaan yang lebih luas seperti pemetaan nonekspansif dan generalisasinya. Berbagai skema iterasi, seperti Mann, Ishikawa, dan Agarwal, Thakur, JK dan yang lainnya, telah dikembangkan untuk aproksimasi titik tetap dari kelas-kelas pemetaan tersebut di berbagai ruang, baik itu di ruang linear seperti ruang Banach maupun di ruang tak linear seperti ruang geodesik. Baru-baru ini, Sabri dkk. mengusulkan skema iterasi baru dengan tingkat konvergensi yang lebih cepat dibanding skema iterasi lainnya pada ruang Banach konveks seragam. Pada penelitian ini, didapatkan syarat cukup untuk konvergensi-$\Delta$ dan kuat dari skema iterasi Sabri untuk aproksimasi titik tetap dari pemetaan $(\alpha,\beta,\gamma)$-nonekspansif di ruang metrik geodesik $CAT_p(0)$. Berdasarkan percobaan numerik, skema iterasi ini memiliki laju konvergensi yang lebih cepat dibandingkan skema iterasi JK, Thakur, Agarwal, dan Abbas. Selain itu, didapatkan pula aplikasi dari skema ini untuk permasalahan optimasi, tepatnya untuk minimalisasi fungsi dan rekonstruksi citra tomografi. 
\katakunci{aproksimasi titik tetap, pemetaan nonekspansif, ruang geodesik, skema iterasi, masalah optimasi}

\end{Abstrak}
