\lampiran{Kode Progam Simulasi dalam Bahasa Pemrograman Python}
\begin{lstlisting}[caption={Kode Python untuk Percobaan Numerik}]
import numpy as np
import matplotlib.pyplot as plt
import pandas as pd
import random

#metrik
def metrik(x,y):
  x1,x2 = x[0], x[1]
  y1,y2 = y[0], y[1]

  suku1 = np.abs(x1-y1)**3
  suku2 = np.abs(x1**3-x2-y1**3+y2)**3
  return (suku1+suku2)**(1/3)

#jalur geodesik
def geodesik(t,w,z):
  w1,w2 = w[0],w[1]
  z1,z2 = z[0],z[1]
  sk1 = (1-t)*w1+t*z1
  sk2 = (1-t)*(w1**3-w2)+t*(z1**3-z2)
  if (t>=0 and t<=1):
    return [sk1,sk1**3-sk2]
  else:
     raise ValueError("di luar domain")

#Pemetaan (alpha, beta, gamma)-nonekspansif
def pemetaan(w):
    w1 = w[0]

    if 1 <= w1 < 3:
        return [
            (w1+3)/4,
            (w1+3)**3 / 64
        ]

    elif 3 <= w1 <= 5:
        return [
            (w1+2)/4,
            (w1+2)**3 / 64
        ]

    else:
        raise ValueError("di luar domain")

#skema iterasi Sabri
def iterSabri(x0, tol, max_iter):
  data_x = [x0]
  data_error = []
  for n in range(max_iter):
    x_n = data_x[-1].copy()
    a_n = a(n)
    b_n = b(n)
    c_n = c(n)
    q_n = pemetaan(geodesik(c_n,x_n,pemetaan(x_n)))
    y_n = pemetaan(pemetaan(q_n))
    x_next = pemetaan(geodesik(a_n,pemetaan(q_n),pemetaan(y_n)))
    data_x.append(x_next)
    error = metrik([1,1],x_next)
    data_error.append(error)
    if error < tol:
      break
  return data_x, data_error

#skema iterasi JK
def iterJK(x0, tol, max_iter):
  data_x = [x0]
  data_error = []
  for n in range(max_iter):
    x_n = data_x[-1].copy()
    a_n = a(n)
    b_n = b(n)
    c_n = c(n)
    q_n = (geodesik(c_n,x_n,pemetaan(x_n)))
    y_n = (pemetaan(q_n))
    x_next = pemetaan(geodesik(a_n,pemetaan(q_n),pemetaan(y_n)))
    data_x.append(x_next)
    error = metrik([1,1], x_next)
    data_error.append(error)
    if error < tol:
      break
  return data_x, data_error

#skema iterasi Thakur
def iterThakur(x0, tol, max_iter):
  data_x = [x0]
  data_error = []
  for n in range(max_iter):
    x_n = data_x[-1].copy()
    a_n = a(n)
    b_n = b(n)
    c_n = c(n)
    q_n = (geodesik(c_n,x_n,pemetaan(x_n)))
    y_n = (pemetaan(geodesik(a_n, x_n, q_n)))
    x_next = pemetaan(y_n)
    data_x.append(x_next)
    error = metrik([1,1], x_next)
    data_error.append(error)
    if error < tol:
      break
  return data_x, data_error

#skema iterasi Abbas
def iterAbbas(x0, tol, max_iter):
  data_x = [x0]
  data_error = []
  for n in range(max_iter):
    x_n = data_x[-1].copy()
    a_n = a(n)
    b_n = b(n)
    c_n = c(n)
    q_n = geodesik(c_n,x_n,pemetaan(x_n))
    y_n = geodesik(b_n, pemetaan(x_n), pemetaan(q_n))
    x_next = geodesik(a_n,pemetaan(y_n),pemetaan(q_n))
    data_x.append(x_next)
    error = metrik([1,1], x_next)
    data_error.append(error)
    if error < tol:
      break
  return data_x, data_error

#skema iterasi Agarwal
def iterAgarwal(x0, tol, max_iter):
  data_x = [x0]
  data_error = []
  for n in range(max_iter):
    x_n = data_x[-1].copy()
    a_n = a(n)
    b_n = b(n)
    c_n = c(n)
    y_n = (geodesik(c_n, x_n, pemetaan(x_n)))
    x_next = geodesik(a_n,pemetaan(x_n),pemetaan(y_n))
    data_x.append(x_next)
    error = metrik([1,1], x_next)
    data_error.append(error)
    if error < tol:
      break
  return data_x, data_error

#konstan
def a(n): return 0.42
def b(n): return 0.83
def c(n): return 0.31

#turun
def a(n): return (n**2/(n**3+1))
def b(n): return (2/(n+3))
def c(n): return (4*n+2)/(7*n+4)

#naik
def a(n): return (1-(4*n+9)**(1/2)/(2*n+13))
def b(n): return 0.8
def c(n): return (1-(n**2/(n**7+3)**(1/2)))

#naikturun
def a(n): return 1-(n**3)/(n**5+1)
def b(n): return (1/(n+1))
def c(n): return (3*n+4)/(5*n**2+4)

#turunnaik
def c(n): return 1-np.sqrt(n**3+8)/(5*n**5+9)
def b(n): return (1-3/(n+5))
def a(n): return (n**5+1)/(5*n**7+4)


x0=[2,3]
tol = 1e-6
max_iter = 100

dataSabri, errorSabri = iterSabri(x0, tol, max_iter)
dataJK, errorJK = iterJK(x0, tol, max_iter)
dataThakur, errorThakur = iterThakur(x0, tol, max_iter)
dataAbbas, errorAbbas = iterAbbas(x0, tol, max_iter)
dataAgarwal, errorAgarwal = iterAgarwal(x0, tol, max_iter)
max_len = max(len(dataSabri), len(dataJK), len(dataThakur), len(dataAbbas), len(dataAgarwal))
dataSabri += [np.nan] * (max_len - len(dataSabri))
dataJK += [np.nan] * (max_len - len(dataJK))
dataThakur += [np.nan] * (max_len - len(dataThakur))
dataAbbas += [np.nan] * (max_len - len(dataAbbas))
dataAgarwal += [np.nan] * (max_len - len(dataAgarwal))
errorSabri += [np.nan] * (max_len - len(errorSabri))
errorJK += [np.nan] * (max_len - len(errorJK))
errorThakur += [np.nan] * (max_len - len(errorThakur))
errorAbbas += [np.nan] * (max_len - len(errorAbbas))
errorAgarwal += [np.nan] * (max_len - len(errorAgarwal))

table = pd.DataFrame({
    'Iterasi n': np.arange(max_len),
    'Sabri': dataSabri,
    'JK': dataJK,
    'Thakur': dataThakur,
    'Abbas': dataAbbas,
    'Agarwal': dataAgarwal
})

def titikX(x, d=6):
    if isinstance(x, (list, np.ndarray)):
        return f"({x[0]:.{d}f}, {x[1]:.{d}f}, 0, 0, \\dots)"
    return x

table_fmt = table.copy()

for col in ["Sabri", "JK", "Thakur", "Abbas", "Agarwal"]:
    table_fmt[col] = table_fmt[col].apply(titikX)
def show_pair(table, scheme_name):
    sub = table_fmt[["Iterasi n", "Sabri", scheme_name]]
    print(f"Tabel Sabri-{scheme_name}")
    print(sub.to_latex(index=False))
    return sub

skema = ["JK", "Thakur", "Abbas", "Agarwal"]
for s in skema:
    show_pair(table_fmt, s)
table_error = pd.DataFrame({
    'Iterasi n': np.arange(max_len),
    'Sabri': errorSabri,
    'JK': errorJK,
    'Thakur': errorThakur,
    'Abbas': errorAbbas,
    'Agarwal': errorAgarwal
})

print(table_error.to_latex(index=False))
import matplotlib.pyplot as plt
import seaborn as sns

plt.figure(figsize=(12, 10))

for col in ['Sabri', 'JK', 'Thakur', 'Abbas', 'Agarwal']:
    plt.plot(table_error['Iterasi n'], table_error[col], marker='o', linestyle='-', label=col)

plt.xlabel('Iterasi n', fontsize=18)
plt.ylabel('Galat $d(x_n,x^*)$', fontsize=18)
plt.xticks(fontsize=14)
plt.yticks(fontsize=14)
plt.legend(fontsize=14)
plt.grid(True, which='both', ls='-', alpha=0.8)
plt.show()


x0_list = [
    [2,3], [3,4], [4,1], [1.5,4], [2.7,4.2], [5,1.8], [3.1,3.5]
]

#Perhitungan jumlah iterasi yang diperlukan
tol = 1e-16
max_iter = 100
data = []
for x0 in x0_list:
  row = {"x0" : tuple(x0)}
  row["Sabri"] = len(iterSabri(x0,tol,max_iter)[0])
  row["JK"] = len(iterJK(x0,tol,max_iter)[0])
  row["Thakur"] = len(iterThakur(x0,tol,max_iter)[0])
  row["Abbas"] = len(iterAbbas(x0,tol,max_iter)[0])
  row["Agarwal"] = len(iterAgarwal(x0,tol,max_iter)[0])
  data.append(row)
tabel = pd.DataFrame(data)
tabel.set_index("x0", inplace=True)

#Penyesuaian titik
def titikdiX(x):
    if isinstance(x, tuple) and len(x) == 2:
        return f"({x[0]:.1f}, {x[1]:.1f}, 0, 0, \\dots)"
    return str(x)
tabel_display = tabel.copy()
tabel_display.index = tabel_display.index.map(titikdiX)

print(tabel_display.to_string())
print(tabel_display.to_latex())
\end{lstlisting}
\begin{lstlisting}[caption={Kode Python untuk Simulasi Minimalisasi Fungsi}]
import numpy as np
from scipy.optimize import minimize


def d(x, y):
    x1,x2 = x[0], x[1]
    y1,y2 = y[0], y[1]

    suku1 = np.abs(x1-y1)**3
    suku2 = np.abs(x1**3-x2-y1**3+y2)**3
    return (suku1+suku2)**(1/3)

def f(x):
    x1, x2 = x
    term1 = 20 * np.abs(x2 - x1**3)**3
    term2 = np.abs(26 - x1)**3
    return term1 + term2

def res(x):
  lam = 20
  def objective(y):
    di = d(x,y)
    return f(y)+1.0/(2.0*lam)*di**2
  res = minimize(objective, x,method="Nelder-Mead", tol=1e-15)
  return res.x

def geodesik(t,w,z):
  w1,w2 = w[0],w[1]
  z1,z2 = z[0],z[1]
  sk1 = (1-t)*w1+t*z1
  sk2 = (1-t)*(w1**3-w2)+t*(z1**3-z2)
  if (t>=0 and t<=1):
    return np.array([sk1,sk1**3-sk2])
  else:
     raise ValueError("di luar domain")

def iterSabri(x0, tol, max_iter):
  data_x = [x0]
  data_error = [np.nan]
  for n in range(max_iter):
    x_n = data_x[-1].copy()
    a_n = 0.34
    b_n = 0
    c_n = 0.93
    q_n = res(geodesik(c_n,x_n,res(x_n)))
    y_n = res(res(q_n))
    x_next = res(geodesik(a_n,res(q_n),res(y_n)))
    data_x.append(x_next)
    error = d(x_next,[26,26**3])
    data_error.append(error)
    if error < tol:
      break
  return data_x, data_error

def iterJK(x0, tol, max_iter):
  data_x = [x0]
  data_error = [np.nan]
  for n in range(max_iter):
    x_n = data_x[-1].copy()
    a_n = 0.34
    b_n = 0
    c_n = 0.93
    q_n = geodesik(c_n,x_n,res(x_n))
    y_n = res(q_n)
    x_next = res(geodesik(a_n,res(q_n),res(y_n)))
    data_x.append(x_next)
    error = d(x_next,[26,26**3])
    data_error.append(error)
    if error < tol:
      break
  return data_x, data_error

dataSabri, errorSabri = iterSabri([10,10],1e-4,500)
dataJK, errorJK = iterJK([10,10],1e-4,500)

max_len = max(len(dataSabri), len(dataJK))

dataSabri += [np.nan] * (max_len - len(dataSabri))
dataJK += [np.nan] * (max_len - len(dataJK))
errorSabri += [np.nan] * (max_len - len(errorSabri))
errorJK += [np.nan] * (max_len - len(errorJK))

import pandas as pd
table = pd.DataFrame({
    'Iterasi n': np.arange(max_len),
    'Sabri': dataSabri,
    'JK': dataJK
})
def fmt_point(x, d=5):
    if isinstance(x, (list, np.ndarray)):
        return f"({x[0]:.{d}f}, {x[1]:.{d}f}, 0, 0, \\dots)"
    return x


table_fmt = table.copy()

for col in ["Sabri", "JK"]:
    table_fmt[col] = table_fmt[col].apply(fmt_point)

print(table_fmt.to_latex(index=False, escape=False))

table_error = pd.DataFrame({
    'Iterasi n': np.arange(max_len),
    'Galat Sabri': errorSabri,
    'Galat JK': errorJK
})

print(table_error.to_latex(index=False, escape=False))

import matplotlib.pyplot as plt

plt.figure(figsize=(14, 10))

for col in ['Galat Sabri', 'Galat JK']:
    plt.plot(table_error['Iterasi n'], table_error[col], marker='o', linestyle='-', label=col)

plt.yscale('log') 
plt.xlabel('Iterasi n',fontsize=18)
plt.ylabel('Galat $d(x_n,x^*)$',fontsize=18)
plt.xticks(fontsize=14)
plt.yticks(fontsize=14)
plt.legend(fontsize=14)
plt.grid(True, which='both', ls='--')
plt.show()
\end{lstlisting}
\begin{lstlisting}[caption={Kode Python untuk Rekonstruksi Gambar}]
import numpy as np
import matplotlib.pyplot as plt
import pandas as pd
import os
from google.colab import files 
from skimage import io, color, transform
from skimage.data import shepp_logan_phantom, grass, moon, coins, text, brain
from skimage.transform import radon, iradon
from skimage.metrics import peak_signal_noise_ratio, mean_squared_error

image_size = 512
theta = np.linspace(0., 180., 180, endpoint=False)
n_iterations = 60
alpha_n = 0.7
beta_n = 0.5
gamma_n = 0.4
checkpoints = [5, 10, 30, 60]

def load_and_prep_image(uploaded_filename, target_size=512):
    print(f"-> Memproses gambar: {uploaded_filename}")
    image = io.imread(uploaded_filename)

    # Jika gambar berwarna (RGB/RGBA), ubah ke Grayscale
    if image.ndim == 3:
        if image.shape[2] == 4: # RGBA
            image = color.rgba2rgb(image)
        image = color.rgb2gray(image)

    # Resize ke target_size (128x128)
    image_resized = transform.resize(image, (target_size, target_size), anti_aliasing=True)
    return image_resized

def gambar():
    print("=== UPLOAD GAMBAR ===")
    uploaded = files.upload()

    if not uploaded:
        print("Tidak ada file yang diupload. Menggunakan default Shepp-Logan Phantom.")
        original_image = shepp_logan_phantom()
        original_image = transform.resize(original_image, (image_size, image_size), anti_aliasing=True)
    else:
        filename = next(iter(uploaded))
        original_image = load_and_prep_image(filename, target_size=image_size)
    return original_image

original_image = gambar()

def A(x):
        return radon(x, theta=theta, circle=False)

def A_adjoint(sinogram):
        return iradon(sinogram, theta=theta, circle=False, output_size=image_size, filter_name=None)

def P_C(x):
        return np.clip(x, 0, 1)

def gammaoptm():
    print("-> Menghitung step size optimal...")
    # Menggunakan random noise untuk estimasi operator
    x_test = np.random.rand(image_size, image_size)
    for _ in range(10):
        Ax = A(x_test)
        AtAx = A_adjoint(Ax)
        norm_val = np.linalg.norm(AtAx)
        x_test = AtAx / norm_val

    Lx = A_adjoint(A(x_test))
    L = np.linalg.norm(Lx) / np.linalg.norm(x_test)
    gamma_step = 1.9 / L
    #gamma_step = 0.001
    print(f"   Gamma Step: {gamma_step:.5f}")
    return gamma_step
    
Q = A(original_image)
gamma_step = gammaoptm()
def T(x):
    Ax = A(x)
    residual = Ax - Q
    gradient = A_adjoint(residual)
    return P_C(x - gamma_step * gradient)

def sabriIter():
    x = np.zeros((image_size, image_size))

    saved_images = {}
    saved_metrics = {}
    data_mse = []
    data_psnr = []

    print(f"-> Memulai Rekonstruksi ({n_iterations} iterasi)...")

    for n in range(n_iterations):
        Tx = T(x)
        z = T((1 - gamma_n) * x + gamma_n * Tx)
        Tz = T(z)
        y = T(Tz)
        Ty = T(y)
        x_next = T((1 - alpha_n) * Tz + alpha_n * Ty)

        x = x_next

        current_psnr = peak_signal_noise_ratio(original_image, x, data_range=1)
        current_mse = mean_squared_error(original_image, x)
        data_mse.append(current_mse)
        data_psnr.append(current_psnr)

        current_iter = n + 1
        if current_iter in checkpoints:
            saved_images[current_iter] = x.copy()
            saved_metrics[current_iter] = (current_psnr, current_mse)
            print(f"   Iterasi {current_iter}:")
            print(f"     PSNR = {current_psnr:.2f} dB")
            print(f"     MSE  = {current_mse:.6f}")
    return saved_images, saved_metrics, data_mse, data_psnr

saved_images, saved_metrics, data_mse, data_psnr = sabriIter()

table = pd.DataFrame({
    'Iterasi n': np.arange(1, len(data_mse) + 1),
    'mse': data_mse,
    'psnr': data_psnr
})
print(table.to_latex(index=False))

def tampilgambar():
    fig, axes = plt.subplots(3, 2, figsize=(10,15))
    ax = axes.ravel()

    # Tampilkan Gambar Asli
    im0 = ax[0].imshow(original_image, cmap='gray', vmin=0, vmax=1)
    ax[0].set_title('Gambar Asli')

    # Tampilkan Sinogram
    ax[1].imshow(Q, cmap='gray', aspect='auto')
    ax[1].set_title('Sinogram')

    # Tampilkan Hasil Iterasi
    for i, iteration in enumerate(checkpoints):
        if iteration in saved_images:
            idx = i + 2
            if idx < 6:
                psnr_val, mse_val = saved_metrics[iteration]
                im = ax[idx].imshow(saved_images[iteration], cmap='gray', vmin=0, vmax=1)
                title_text = (f"Iterasi {iteration}\n"
                              f"PSNR: {psnr_val:.2f} dB\n"
                              f"MSE: {mse_val:.5f}")
                ax[idx].set_title(title_text)
                plt.colorbar(im, ax=ax[idx], fraction=0.046, pad=0.04)

    plt.tight_layout()
    plt.show()

tampilgambar()
\end{lstlisting}
\begin{longtable}{|c|>{\centering\arraybackslash}p{4cm} | >{\centering\arraybackslash}p{4cm} |}
\caption{\centering MSE dan PSNR Rekonstruksi Citra}\label{tab:msepsnr}\\
\hline
{\textbf{Iterasi ke-$n$}} & {\textbf{MSE}} & {\textbf{PSNR}} \\
\hline
\endfirsthead 
\multicolumn{3}{c}{\centering \thetable{} -- Lanjutan dari halaman sebelumnya} \\
\hline
{\textbf{Iterasi ke-$n$}} & {\textbf{MSE}} & {\textbf{PSNR}} \\
\hline
\endhead 
% \multicolumn{3}{c}{{Lanjut pada halaman berikutnya}} \\
\hline
\endfoot 
\hline
\endlastfoot
\hline
1  & 0.020302 & 16.924673 \\
2  & 0.013587 & 18.668847 \\
3  & 0.010072 & 19.968693 \\
4  & 0.007823 & 21.066098 \\
5  & 0.006272 & 22.026258 \\
6  & 0.005149 & 22.882519 \\
7  & 0.004309 & 23.655780 \\
8  & 0.003664 & 24.360387 \\
9  & 0.003157 & 25.006683 \\
10 & 0.002753 & 25.602432 \\
11 & 0.002425 & 26.153733 \\
12 & 0.002155 & 26.665575 \\
13 & 0.001931 & 27.142128 \\
14 & 0.001743 & 27.587012 \\
15 & 0.001584 & 28.003332 \\
16 & 0.001448 & 28.393736 \\
17 & 0.001330 & 28.760530 \\
18 & 0.001229 & 29.105746 \\
19 & 0.001140 & 29.431187 \\
20 & 0.001062 & 29.738458 \\
21 & 0.000993 & 30.028997 \\
22 & 0.000932 & 30.304104 \\
23 & 0.000878 & 30.564968 \\
24 & 0.000829 & 30.812682 \\
25 & 0.000786 & 31.048254 \\
26 & 0.000746 & 31.272582 \\
27 & 0.000710 & 31.486496 \\
28 & 0.000678 & 31.690770 \\
29 & 0.000648 & 31.886142 \\
30 & 0.000620 & 32.073295 \\
31 & 0.000595 & 32.252847 \\
32 & 0.000572 & 32.425370 \\
33 & 0.000551 & 32.591386 \\
34 & 0.000531 & 32.751347 \\
35 & 0.000512 & 32.905658 \\
36 & 0.000495 & 33.054684 \\
37 & 0.000479 & 33.198752 \\
38 & 0.000464 & 33.338168 \\
39 & 0.000449 & 33.473223 \\
40 & 0.000436 & 33.604183 \\
41 & 0.000424 & 33.731285 \\
42 & 0.000412 & 33.854741 \\
43 & 0.000400 & 33.974740 \\
44 & 0.000390 & 34.091450 \\
45 & 0.000380 & 34.205025 \\
46 & 0.000370 & 34.315610 \\
47 & 0.000361 & 34.423339 \\
48 & 0.000353 & 34.528338 \\
49 & 0.000344 & 34.630725 \\
50 & 0.000336 & 34.730611 \\
51 & 0.000329 & 34.828102 \\
52 & 0.000322 & 34.923289 \\
53 & 0.000315 & 35.016256 \\
54 & 0.000309 & 35.107081 \\
55 & 0.000302 & 35.195841 \\
56 & 0.000296 & 35.282605 \\
57 & 0.000291 & 35.367439 \\
58 & 0.000285 & 35.450403 \\
59 & 0.000280 & 35.531556 \\
60 & 0.000275 & 35.610951 \\
\hline
\caption{Lampiran A.4}
\end{longtable}