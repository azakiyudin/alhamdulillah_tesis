\UcapanTerimaKasih
\setstretch{1.25}
Penyelesaian penulisan tugas akhir ini tidak lepas dari orang-orang terdekat Penulis yang telah mendukung, menemani, dan memotivasi penulis selama berkuliah di Departemen Matematika Institut Teknologi Sepuluh Nopember. Oleh sebab itu, Penulis mengucapkan terima kasih kepada:

\begin{enumerate}[nosep]
	\item Bintang Marvellianti yang senantiasa menemani keseharian penulis selama menjalani perkuliahan magister.
	\item Penghuni Laboratorium Aljabar Analisis, khususnya:
	\begin{enumerate}[nosep]
		\item Mas Ilham, Mas Alvian, terima kasih ilmu dan diskusinya,
		\item Vincent, Huda, Jundi, Fajar sebagai teman main EAFC,
	\end{enumerate}
	serta penghuni yang lainnya, yaitu Teo, Junika, Dias, Isa, Rasyidah, Isma, Feli yang telah membantu dalam keberlangsungan kegiatan di Laboratorium.
	\item Junika yang membantu penulis sebagai koordinator Asisten Dosen Kalkulus.
	\item Dzaky selaku rekan seperbimbingan penulis.
	\item Krisna dan Kygung yang mengajak dan mengajari penulis terkait dengan gym.
	\item Teman-teman main voli dan badminton, yaitu Mas Felza, Mas Ridho, Mas Zaki, Mas Ajrian, Davit, Atma, dan yang lainnya yang tidak bisa penulis sebutkan satu persatu.
\end{enumerate}

Akhir kata, semoga tesis ini bermanfaat bagi semua pihak yang bersangkutan.\\
\begin{flushright}
\begin{tabular}{r}
	 Surabaya, 27 Januari 2026\vspace{1cm}\\\\\\
	 Ahmad Hisbu Zakiyudin
\end{tabular}
\end{flushright}