\tikzstyle{startstop} = [ellipse, 
minimum width=5.5cm, 
minimum height=0.5cm,
text centered, 
draw=black]

\tikzstyle{io} = [trapezium, 
trapezium stretches=true, % A later addition
trapezium left angle=70, 
trapezium right angle=110, 
minimum width=6cm, 
minimum height=0.5cm, text centered, 
draw=black]

\tikzstyle{process} = [rectangle, 
minimum height=0.5cm, 
text centered, 
text width=8.5cm, 
draw=black]

\tikzstyle{decision} = [diamond, 
minimum width=5cm, 
minimum height=1cm, 
text centered, 
draw=black]
\tikzstyle{arrow} = [thick,->,>=stealth]
\begin{tikzpicture}[node distance=1cm]

\node (start) [startstop] {Mulai};
\node (pro1) [process, below of=start,yshift=-0.3cm] {Studi Literatur};
\node (pro2) [process, below of=pro1, yshift=-0.6cm] {Mengkaji Pemetaan \\$(\alpha,\beta,\gamma)$-nonekspansif di ruang $CAT_p(0)$ };
\node (pro3) [process, below of=pro2, yshift=-1.6cm] {Mendapatkan Konvergensi dari Skema Iterasi Sabri di Ruang $CAT_p(0)$ untuk Aproksimasi Titik Tetap dari Pemetaan $(\alpha,\beta,\gamma)$-nonekspansif };
\node (pro4) [process, below of=pro3, yshift=-1.2cm] {Melakukan Percobaan Numerik };
\node (pro5) [process, below of=pro4,yshift=-0.6cm] {Mendapatkan Aplikasi dari Skema Iterasi Sabri untuk Masalah Optimasi};
\node (pro6) [process, below of=pro5,yshift=-0.6cm] {Diseminasi};
\node (pro7) [process, below of=pro6, yshift=-0.2cm] {Penyusunan Laporan Tesis};

\node (stop) [startstop, below of=pro7, yshift=-0.3cm] {Selesai};

\draw [arrow] (start) -- (pro1);
\draw [arrow] (pro1) -- (pro2);
\draw [arrow] (pro2) -- (pro3);
\draw [arrow] (pro3) -- (pro4);
\draw [arrow] (pro4) -- (pro5);
\draw [arrow] (pro5) -- (pro6);
\draw [arrow] (pro6) -- (pro7);
\draw [arrow] (pro7) -- (stop);

\end{tikzpicture}