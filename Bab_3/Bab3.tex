\chapter{METODE PENELITIAN}

\indent Pada bab ini diuraikan beberapa tahapan penelitian yang akan dikerjakan untuk mencapai tujuan penelitian.
\section{Tahapan Penelitian}
\begin{itemize}
    \item[(a)] Studi Literatur\\
    Pada tahap ini dilakukan tinjauan pustaka dengan fokus utama meliputi konsep mengenai titik tetap, ruang $CAT_p(0)$, pemetaan nonekspansif dan perumumannya, serta terkait skema iterasi. Akan dilakukan kajian terkait syarat-syarat untuk konvergensi dari suatu skema iterasi untuk mendapatkan nilai aproksimasi titik tetap dari pemetaan nonekspansif dan perumumannya. Kemudian dikaji pula konsep-konsep yang ada pada ruang $CAT_p(0)$, seperti dari definisi, sifat-sifat, dan hasil penting lainnya yang berkaitan dengan aproksimasi titik tetap dari pemetaan nonekspansif di ruang tersebut.
    \item[(b)] Mengkaji Pemetaan $(\alpha,\beta,\gamma)$-nonekspansif di ruang $CAT_p(0)$.\\
    Pada tahap ini pemetaan $(\alpha,\beta,\gamma)$-nonekspansif yang awalnya didefinisikan untuk ruang Banach akan dikaji di ruang $CAT_p(0)$. Beberapa lema dan teorema yang berlaku untuk pemetaan tersebut akan diperluas dalam konteks ruang $CAT_p(0)$. 
    \item[(c)] Mendapatkan Konvergensi dari Skema Iterasi Sabri di Ruang $CAT_p(0)$ untuk Aproksimasi Titik Tetap dari Pemetaan $(\alpha,\beta,\gamma)$-nonekspansif.\\
    Tahapan ini memiliki tujuan untuk membuktikan bahwa barisan yang diperoleh dari skema iterasi Sabri konvergen titik tetap dari pemetaan $(\alpha,\beta,\gamma)$-nonekspansif di ruang $CAT_p(0)$. Beberapa syarat untuk barisan tersebut konvergen, baik itu konvergen-$\Delta$ maupun konvergen kuat akan diselidiki pada tahap ini. 
    \item [(d)] Melakukan Percobaan Numerik.\\
    Untuk memvalidasi hasil teoritis pada tahap sebelumnya, dilakukan eksperimen numerik. Dalam tahap ini, dilakukan percobaan dengan mendefinisikan suatu pemetaan $(\alpha,\beta,\gamma)$-nonekspansif di ruang $CAT_p(0)$, kemudian titik tetap dari pemetaan tersebut akan didekati dengan menggunakan skema iterasi Sabri. Dalam percobaan ini, juga akan digunakan beberapa variasi parameter. Selanjutnya, jumlah iterasi dari skema iterasi Sabri akan dibandingkan dengan beberapa skema iterasi lainnya yang telah ada sebelumnya. Hasil yang diperoleh akan disajikan dalam bentuk tabel. 
    \item [(e)] Mendapatkan Aplikasi dari Skema Iterasi Sabri untuk Masalah Optimasi.\\
    Banyak masalah dalam bidang optimisasi, seperti \textit{convex feasibility problem} atau masalah minimalisasi, dapat diformulasikan ulang sebagai masalah pencarian titik tetap. Dalam tahap ini akan diidentifikasi masalah yang relevan dan menyelesaikannya dengan skema iterasi Sabri.
    \item [(f)] Diseminasi.\\
    Pada tahap diseminasi, dilakukan penulisan artikel ilmiah dan akan dipublikasikan pada seminar internasional atau jurnal internasional bereputasi. 
    
    %ini tulisan
    \item [(g)] Penyusunan Laporan Tesis.\\
   Pada tahap ini, dilakukan penulisan laporan tesis yang meliputi seluruh hasil penelitian, baik itu hasil teoritis maupun hasil numerik. Penjelasan dari hasil tersebut ditulis secara rinci, terstruktur, dan lengkap. Selain itu, kesimpulan dari penelitian dan saran untuk penelitian selanjutnya akan dituliskan dalam laporan tesis.
    
\end{itemize}
\newpage
\section{Diagram Alir Penelitian} 

Diagram alir untuk penelitian ini disajikan dalam  \ref{3_gambar_LangkahUmum} sebagai berikut:
\begin{figure}[H]
        \centering
	\tikzstyle{startstop} = [ellipse, 
minimum width=5.5cm, 
minimum height=0.5cm,
text centered, 
draw=black]

\tikzstyle{io} = [trapezium, 
trapezium stretches=true, % A later addition
trapezium left angle=70, 
trapezium right angle=110, 
minimum width=6cm, 
minimum height=0.5cm, text centered, 
draw=black]

\tikzstyle{process} = [rectangle, 
minimum height=0.5cm, 
text centered, 
text width=8.5cm, 
draw=black]

\tikzstyle{decision} = [diamond, 
minimum width=5cm, 
minimum height=1cm, 
text centered, 
draw=black]
\tikzstyle{arrow} = [thick,->,>=stealth]
\begin{tikzpicture}[node distance=1cm]

\node (start) [startstop] {Mulai};
\node (pro1) [process, below of=start,yshift=-0.3cm] {Studi Literatur};
\node (pro2) [process, below of=pro1, yshift=-0.6cm] {Mengkaji Pemetaan \\$(\alpha,\beta,\gamma)$-nonekspansif di ruang $CAT_p(0)$ };
\node (pro3) [process, below of=pro2, yshift=-1.6cm] {Mendapatkan Konvergensi dari Skema Iterasi Sabri di Ruang $CAT_p(0)$ untuk Aproksimasi Titik Tetap dari Pemetaan $(\alpha,\beta,\gamma)$-nonekspansif };
\node (pro4) [process, below of=pro3, yshift=-1.2cm] {Melakukan Percobaan Numerik };
\node (pro5) [process, below of=pro4,yshift=-0.6cm] {Mendapatkan Aplikasi dari Skema Iterasi Sabri untuk Masalah Optimasi};
\node (pro6) [process, below of=pro5,yshift=-0.6cm] {Diseminasi};
\node (pro7) [process, below of=pro6, yshift=-0.2cm] {Penyusunan Laporan Tesis};

\node (stop) [startstop, below of=pro7, yshift=-0.3cm] {Selesai};

\draw [arrow] (start) -- (pro1);
\draw [arrow] (pro1) -- (pro2);
\draw [arrow] (pro2) -- (pro3);
\draw [arrow] (pro3) -- (pro4);
\draw [arrow] (pro4) -- (pro5);
\draw [arrow] (pro5) -- (pro6);
\draw [arrow] (pro6) -- (pro7);
\draw [arrow] (pro7) -- (stop);

\end{tikzpicture}
	\caption{Blok Diagram Penelitian.}
        \label{3_gambar_LangkahUmum}
\end{figure}
% \newpage
% \section{Jadwal Pelaksanaan}
% Rencana dan jadwal kerja penelitian, serta penyusunan tesis disajikan dalam~\ref{Jadwal} sebagai berikut:

% \begin{table}[th!]
% \caption{Rencana Pelaksanaan Penelitian}
% \centering
% \begin{tabular}{|c|L{3cm}|c|c|c|c|c|c|c|c|c|c|c|c|c|c|c|c|}	
% \hline
% &&\multicolumn{16}{c|}{Bulan ke-}\\
% \cline{3-18}
% \multicolumn{1}{|c|}{No.}&\multicolumn{1}{c|}{Jenis Kegiatan}&\multicolumn{4}{c|}{1}&\multicolumn{4}{c|}{2}&\multicolumn{4}{c|}{3}&\multicolumn{4}{c|}{4}\\\cline{3-18}
% &&1&2&3&4&1&2&3&4&1&2&3&4&1&2&3&4\\\cline{1-18}
% 1&Studi literatur&\cellcolor{gray}&\cellcolor{gray}&\cellcolor{gray}&\cellcolor{gray}&&&&&&&&&&&&\\\hline
% 2& Mengkaji Pemetaan $(\alpha,\beta,\gamma)$-nonekspansif di ruang $CAT_p(0)$ &&&\cellcolor{gray}&\cellcolor{gray}&\cellcolor{gray}&&&&&&&&&&&\\\hline
% 3&  Mendapatkan Konvergensi dari Skema Iterasi Sabri di Ruang $CAT_p(0)$ untuk Aproksimasi Titik Tetap dari Pemetaan $(\alpha,\beta,\gamma)$-nonekspansif &&&&\cellcolor{gray}&\cellcolor{gray}&\cellcolor{gray}&\cellcolor{gray}&\cellcolor{gray}&&&&&&&&\\\hline
% 4&Melalukan Percobaan Numerik &&&&&&&\cellcolor{gray}&\cellcolor{gray}&\cellcolor{gray}&\cellcolor{gray}&\cellcolor{gray}&\cellcolor{gray}&&&&\\\hline
% 5& Mendapatkan Aplikasi dari Skema Iterasi Sabri untuk Masalah Optimasi &&&&&&&&&&\cellcolor{gray}&\cellcolor{gray}&\cellcolor{gray}&\cellcolor{gray}&&&\\\hline
% 6& Diseminasi.&&&&&&&&\cellcolor{gray}&\cellcolor{gray}&\cellcolor{gray}&\cellcolor{gray}&\cellcolor{gray}&\cellcolor{gray}&\cellcolor{gray}&&\\\hline
% 7&Penyusunan Laporan Tesis.&&&&&&&&\cellcolor{gray}&\cellcolor{gray}&\cellcolor{gray}&\cellcolor{gray}&\cellcolor{gray}&\cellcolor{gray}&\cellcolor{gray}&\cellcolor{gray}&\cellcolor{gray}\\\hline

% \end{tabular}

% \label{Jadwal}
% \end{table} 

