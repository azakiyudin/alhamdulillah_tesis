\section{Pemetaan Nonekspansif}
Pemetaan nonekspansif merupakan salah satu jenis pemetaan yang mempunyai banyak aplikasi di berbagai bidang. Pemetaan ini merupakan perluasan dari kontraktif yang dikenalkan oleh Banach. 
Berikut ini diberikan definisi dan contoh dari pemetaan nonekspansif. 
\begin{defn}\cite{Browder1965}
    Diberikan $(B,\|\cdot\|)$ adalah ruang Banach dan $W\subseteq B$ dengan $W\neq\emptyset$. Pemetaan $T:W\to W$ disebut pemetaan nonekspansif jika untuk setiap $x,y\in W$ berlaku 
    \begin{equation}
        \|Tx-Ty\|\leq \|x-y\|.
    \end{equation}
\end{defn}
\begin{exam}
    Diberikan $(\mathbb{R},|\cdot|)$ adalah ruang Banach dan $W=[0,\infty)$. Pemetaan $T:W\to W$ yang didefinisikan sebagai $Tx=|x|$ untuk setiap $x\in W$ adalah pemetaan nonekspansif karena untuk setiap $x,y\in W$ diperoleh
    \begin{align*}
        |Tx-Ty| &= \qty||x|-|y|| \leq |x-y|.
    \end{align*}
\end{exam}
Pemetaan nonekspansif ini juga mengalami perluasan, salah satu perluasannya adalah pemetaan nonekspansif kuasi. Berikut adalah definisi dan contoh dari pemetaan nonekspansif kuasi.
\begin{defn}
    Diberikan $(B,\|\cdot\|)$ adalah ruang Banach dan $W\subseteq B$ dengan $W\neq\emptyset$. Pemetaan $T:W\to W$ disebut pemetaan nonekspansif kuasi jika untuk setiap titik tetap dari $T$ yaitu $y$ dan $x\in W$ berlaku 
    \begin{equation}
        \|Tx-y\|\leq \|x-y\|.
    \end{equation}
\end{defn}
\begin{exam}
    Diberikan $(\mathbb{R},|\cdot|)$ adalah ruang Banach dan $W=[0,3]$. Pemetaan $T:W\to W$ yang didefinisikan sebagai $$Tx=\begin{cases}
    \frac{x}{3}, & \text{jika } x\in [0,1],\\
    1, & \text{jika } x\in (1,3],
    \end{cases}$$ merupakan pemetaan nonekspansif kuasi, karena titik tetap dari $T$ adalah $y=0$ dan untuk setiap $x\in [0,1]$ diperoleh
    \begin{align*}
        |Tx-y| &= \qty|\frac{x}{3}-0| = \frac{x}{3} \leq x = |x-0| = |x-y|, 
    \end{align*}
    serta untuk setiap $x\in (1,3]$ diperoleh
    \begin{align*}
        |Tx-y| &= |1-0| = 1 \leq x = |x-0| = |x-y|.
    \end{align*}
    Pemetaan ini bukan merupakan pemetaan nonekspansif karena untuk $x=0$ dan $y=2$ diperoleh
    \begin{align*}  
        |Tx-Ty| &= |0-1| = 1 > 2 = |0-2| = |x-y|.
    \end{align*}
\end{exam}

Pada tahun 2008, Suzuki juga mengenalkan bentuk lain pemetaan nonekspansif, yaitu pemetaan Suzuki nonekspansif atau pemetaan yang memenuhi kondisi $(C)$ yang didefinisikan sebagai berikut
\begin{defn}\cite{Suzuki2008}
    Diberikan $(B,\|\cdot\|)$ adalah ruang Banach dan $W\subseteq B$ dengan $W\neq\emptyset$. Pemetaan $T:W\to W$ disebut pemetaan yang memenuhi kondisi $(C)$ jika untuk setiap $x,y\in W$ berlaku 
    \begin{equation}
        \dfrac{1}{2}\|x-Tx\|\leq \|x-y\| \qquad \Longrightarrow \qquad \|Tx-Ty\|\leq \|x-y\|.
    \end{equation}
\end{defn}

Pemetaan ini merupakan kondisi lemah dari pemetaan nonekspansif karena ketaksamaan nonekspansif wajib terpenuhi hanya untuk dua titik yang memenuhi kondisi $\frac{1}{2}\|x-Tx\|\leq \|x-y\|$.
\begin{exam}\label{con:pemetaankondisi C}\cite{Suzuki2008}
    Diberikan $(\mathbb{R},|\cdot|)$ adalah ruang Banach dan $W=[0,3]$. Pemetaan $T:W\to W$ yang didefinisikan sebagai 
    \begin{align*}
        Tx=\begin{cases}
        0, & \text{jika } x\neq 3,\\
        1, & \text{jika } x=3,
        \end{cases}
    \end{align*}
    merupakan pemetaan yang memenuhi kondisi $(C)$, tetapi bukan pemetaan nonekspansif.
\end{exam}
Penjelasan dari Contoh \ref{con:pemetaankondisi C} adalah sebagai berikut. 

Diketahui bahwa untuk setiap $x,y\in W$ dengan $x,y\neq 3$ diperoleh
\begin{align*}
    |Tx-Ty| &= |0-0| = 0 \leq |x-y|,
\end{align*}
dan untuk $x=y=3$ diperoleh
\begin{align*}
    |Tx-Ty| &= |1-1| = 0 = |x-y|,
\end{align*}
sehingga ketaksamaan nonekspansif terpenuhi. Selanjutnya, tanpa mengurangi keumuman, untuk $x\neq 3$ dan $y=3$ dibagi dua kasus, yaitu
\begin{itemize} 
    \item untuk $x\in [0,2]$, diperoleh
    \begin{align*}
        |Tx-Ty| &= |0-1| = 1 \leq |x-3|, 
    \end{align*}
    sehingga ketaksamaan nonekspansif terpenuhi dan
    \item untuk $x\in (2,3)$, diperoleh
    \begin{align*}
        \frac{1}{2}|x-Tx| &= \frac{1}{2}|x-0| = \frac{x}{2} > 1> |x-3|=|x-y|,
    \end{align*}
    dan 
    \begin{align*}
        \frac{1}{2}|y-Ty| &= \frac{1}{2}|3-1| = 1 > |x-3|=|x-y|,
    \end{align*}
    sehingga premis pada kondisi $(C)$ tidak terpenuhi.
\end{itemize}
Hal ini menunjukkan bahwa pemetaan tersebut memenuhi kondisi $(C)$. Akan tetapi, pemetaan tersebut bukan merupakan pemetaan nonekspansif karena untuk $x=2.5$ dan $y=3$ diperoleh
\begin{align*}  
    |Tx-Ty| &= |0-1| = 1 > 0.5 = |2.5-3| = |x-y|.
\end{align*}

Kemudian, García-Falset dkk. memperumum pemetaan Suzuki nonekspansif dengan mengganti konstanta $\frac{1}{2}$ menjadi konstanta $\mu\in (0,1)$. Pemetaan tersebut didefinisikan sebagai berikut. 
\begin{defn}\cite{GARCIAFALSET2011185}
    Diberikan $(B,\|\cdot\|)$ adalah ruang Banach dan $W\subseteq B$ dengan $W\neq\emptyset$. Pemetaan $T:W\to W$ disebut pemetaan yang memenuhi kondisi $(C_{\mu})$ jika terdapat $\mu\in (0,1)$ sehingga untuk setiap $x,y\in W$ berlaku 
    \begin{equation}
        \mu \|x-Tx\|\leq \|x-y\| \qquad \Longrightarrow \qquad \|Tx-Ty\|\leq \|x-y\|
    \end{equation}
\end{defn}
\begin{exam}\cite{GARCIAFALSET2011185}\label{con:muC}
    Diberikan $(\mathbb{R},|\cdot|)$ adalah ruang Banach dan $W=[0,1]$. Pemetaan $T:W\to W$ yang didefinisikan sebagai 
    \begin{align*}
        Tx=\begin{cases}
        \frac{x}{2}, & \text{jika } x\neq 1,\\
        \frac{7}{11}, & \text{jika } x=1,
        \end{cases}
    \end{align*}
    merupakan pemetaan yang memenuhi kondisi $(C_{\mu})$ dengan $\mu=\frac{3}{4}$, tetapi bukan pemetaan nonekspansif.
\end{exam}
Penjelasan dari Contoh \ref{con:muC} adalah sebagai berikut. 

Diketahui bahwa untuk setiap $x,y\in W$ dengan $x,y\neq 1$ diperoleh
\begin{align*}
    |Tx-Ty| &= \qty|\frac{x}{2}-\frac{y}{2}| = \frac{|x-y|}{2} \leq |x-y|,
\end{align*}
dan untuk $x=y=1$ diperoleh
\begin{align*}
    |Tx-Ty| &= \qty|\frac{7}{11}-\frac{7}{11}| = 0 = |x-y|, 
\end{align*}
sehingga ketaksamaan nonekspansif terpenuhi. Selanjutnya, tanpa mengurangi keumuman, untuk $x\neq 1$ dan $y=1$ dibagi dua kasus, yaitu
\begin{itemize}
    \item untuk $x\in \qty[0,\frac{8}{11}]$, diperoleh
    \begin{align*}
        |Tx-Ty| &= \qty|\frac{x}{2}-\frac{7}{11}| = \qty|\frac{11x-14}{22}| \leq |x-1|=|x-y|,
    \end{align*}
    sehingga ketaksamaan nonekspansif terpenuhi dan 
    \item untuk $x\in \qty(\frac{8}{11},1)$, diperoleh
    \begin{align*}
        \mu |x-Tx| &= \frac{3}{4}\qty|x-\frac{x}{2}| = \frac{3x}{8} > \frac{3}{11} > |x-1|=|x-y|, 
    \end{align*}
    sehingga premis pada kondisi $(C_{\mu})$ tidak terpenuhi.
\end{itemize}
Hal ini menunjukkan bahwa pemetaan tersebut memenuhi kondisi $(C_{\mu})$ dengan $\mu=\frac{3}{4}$. Akan tetapi, pemetaan tersebut bukan merupakan pemetaan nonekspansif karena untuk $x=1$ dan $y=\frac{9}{11}$ diperoleh   
\begin{align*}  
    |Tx-Ty| &= \qty|\frac{7}{11}-\frac{9}{22}| = \frac{5}{22} > \frac{2}{11} = \qty|1-\frac{9}{11}| = |x-y|.
\end{align*}
Pemetaan tersebut juga bukan merupakan pemetaan yang memenuhi kondisi $(C)$ karena untuk $x=\frac{81}{110}$ dan $y=1$ diperoleh
\begin{align*}
    \frac{1}{2}|x-Tx| &= \frac{1}{2}\qty|\frac{81}{110}-\frac{81}{220}| = \frac{81}{440} < \frac{29}{110} = |x-1|=|x-y|,    
\end{align*}
tetapi
\begin{align*}
    |Tx-Ty| &= \qty|\frac{81}{220}-\frac{7}{11}| = \frac{59}{220} > \frac{29}{110} = |x-y|.
\end{align*}

Di sisi lain, Aoyama dan Kohsaka juga mengembangkan pemetaan nonekspansif menjadi pemetaan $\alpha$-nonekspansif. Berikut adalah definisi dan contoh dari pemetaan $\alpha$-nonekspansif.

\begin{defn}\cite{Aoyama2011}
    Diberikan $(B,\|\cdot\|)$ adalah ruang Banach dan $W\subseteq B$ dengan $W\neq\emptyset$. Pemetaan $T:W\to W$ disebut pemetaan $\alpha$-nonekspansif jika terdapat bilangan real $\alpha\in[0,1)$ sehingga untuk setiap $x,y\in W$  berlaku
    \begin{equation}\label{eq:anonban}
        \|Tx-Ty\|^2\leq \alpha \|x-Ty\|^2+\alpha \|y-Tx\|^2 +(1-2\alpha) \|x-y\|^2.
    \end{equation}
\end{defn}
\begin{exam}\label{con:alphanon}
    Diberikan $(\mathbb{R},|\cdot|)$ adalah ruang Banach dan $W=[0,2]$. Pemetaan $T:W\to W$ yang didefinisikan sebagai 
    \begin{align*}
        Tx=\begin{cases}
        0, & \text{jika } x\neq 2,\\
        1, & \text{jika } x=2,
        \end{cases}
    \end{align*}
    merupakan pemetaan $\alpha$-nonekspansif dengan $\alpha=\frac{1}{2}$, tetapi bukan pemetaan nonekspansif.
\end{exam}
Penjelasan dari Contoh \ref{con:alphanon} adalah sebagai berikut.

Diketahui bahwa untuk setiap $x,y\in W$ dengan $x,y\neq 2$ diperoleh
\begin{align*}
    |Tx-Ty|^2 &= |0-0|^2 = 0 \leq \frac{1}{2}|x-Ty|^2 + \frac{1}{2}|y-Tx|^2 + 0\cdot |x-y|^2,
\end{align*}
dan untuk $x=y=2$ diperoleh
\begin{align*}
    |Tx-Ty|^2 &= |1-1|^2 = 0 \leq \frac{1}{2}|x-Ty|^2 + \frac{1}{2}|y-Tx|^2 + 0\cdot |x-y|^2,
\end{align*}
sehingga Ketaksamaan \eqref{eq:anonban} terpenuhi. Selanjutnya, tanpa mengurangi keumuman, untuk $x\neq 2$ dan $y=2$ diperoleh
\begin{align*}
    |Tx-Ty|^2 = |0-1|^2 = 1 &\leq 2 + \frac{1}{2}|x-1|^2 \\
    &= \frac{1}{2}|x-1|^2 + \frac{1}{2}|2-0|^2 + 0\cdot |x-2|^2\\
    &= \frac{1}{2}|x-Ty|^2 + \frac{1}{2}|y-Tx|^2 + 0\cdot |x-y|^2,
\end{align*}
sehingga Ketaksamaan \eqref{eq:anonban} terpenuhi. Jadi, pemetaan tersebut merupakan pemetaan $\alpha$-nonekspansif. Akan tetapi, pemetaan tersebut bukan merupakan pemetaan nonekspansif karena untuk $x=0$ dan $y=2$ diperoleh   
\begin{align*}  
    |Tx-Ty| &= |0-1| = 1 > 2 = |0-2| = |x-y|.   
\end{align*}

Pant dan Pandey di tahun 2019 mengenalkan pemetaan tipe Reich-Suzuki nonekspansif yang didefinisikan sebagai berikut. 

\begin{defn}\cite{Pant2019}
    Diberikan $(B,\|\cdot\|)$ adalah ruang Banach dan $W\subseteq B$ dengan $W\neq\emptyset$. Pemetaan $T:W\to W$ disebut pemetaan tipe Reich-Suzuki nonekspansif jika terdapat bilangan real $\alpha\in[0,1)$ sehingga untuk setiap $x,y\in W$ berlaku
    \begin{equation}\label{eq:rsn}
        \|Tx-Ty\|\leq \alpha \|x-Tx\|+\alpha \|y-Ty\| +(1-2\alpha) \|x-y\|.
    \end{equation}
\end{defn}
\begin{exam}\label{con:rsn}
    Diberikan $(\mathbb{R},|\cdot|)$ adalah ruang Banach dan $W=[0,4]$. Pemetaan $T:W\to W$ yang didefinisikan sebagai 
    \begin{align*}
        Tx=\begin{cases}
        0, & \text{jika } x\neq 4,\\
        1, & \text{jika } x=4,
        \end{cases}
    \end{align*}
    merupakan pemetaan tipe Reich-Suzuki nonekspansif dengan $\alpha=\frac{1}{2}$, tetapi bukan pemetaan nonekspansif.
\end{exam}
Penjelasan dari Contoh \ref{con:rsn} adalah sebagai berikut.

Diketahui bahwa untuk setiap $x,y\in W$ dengan $x,y\neq 4$ diperoleh
\begin{align*}
    |Tx-Ty| &= |0-0| = 0 \leq \frac{1}{2}|x-Tx| + \frac{1}{2}|y-Ty| + 0\cdot |x-y|,  
\end{align*}
dan untuk $x=y=4$ diperoleh
\begin{align*}
    |Tx-Ty| &= |1-1| = 0 \leq \frac{1}{2}|x-Tx| + \frac{1}{2}|y-Ty| + 0\cdot |x-y|,
\end{align*}
sehingga Ketaksamaan \eqref{eq:rsn} terpenuhi. Selanjutnya, tanpa mengurangi keumuman, untuk $x\neq 4$ dan $y=4$ diperoleh
\begin{align*}
    |Tx-Ty| = |0-1| = 1 &\leq \frac{3}{2} + \frac{1}{2}|x-0| \\
    &= \frac{1}{2}|x-0| + \frac{1}{2}|4-1| + 0\cdot |x-4|\\
    &= \frac{1}{2}|x-Tx| + \frac{1}{2}|y-Ty| + 0\cdot |x-y|,
\end{align*}
sehingga Ketaksamaan \eqref{eq:rsn} terpenuhi. Jadi, pemetaan tersebut merupakan pemetaan tipe Reich-Suzuki nonekspansif. Akan tetapi, pemetaan tersebut bukan merupakan pemetaan nonekspansif karena untuk $x=0$ dan $y=4$ diperoleh   
\begin{align*}  
    |Tx-Ty| &= |0-1| = 1 > 4 = |0-4| = |x-y|.
\end{align*}

Hasil terbaru, tahun 2023 Ullah dkk. mengenalkan pemetaan $(\alpha,\beta,\gamma)$-nonekspansif. 

\begin{defn}\cite{Ullah2023}
    Diberikan $(B,\|\cdot\|)$ adalah ruang Banach dan $W\subseteq B$ dengan $W\neq\emptyset$. Pemetaan $T:W\to W$ disebut pemetaan $(\alpha,\beta,\gamma)$-nonekspansif jika terdapat bilangan real $\alpha,\beta,\gamma\in \mathbb{R}^+\cup\{0\}$ dengan $\gamma\in[0,1)$ dan $\alpha+\gamma\leq 1$ sehingga untuk setiap $x,y\in W$ berlaku 
    \begin{equation}\label{eq:abcnonban}
        \|Tx-Ty\|\leq \alpha \|x-y\|+\beta \|x-Tx\| +\gamma \|x-Ty\|.
    \end{equation}
\end{defn}
Berikut ini diberikan contoh dari pemetaan tersebut. 
\begin{exam}\label{con:abcnonban}
    Diberikan $(\mathbb{R},|\cdot|)$ adalah ruang Banach dan $W=[0,2]$. Pemetaan $T:W\to W$ yang didefinisikan sebagai 
    \begin{align*}
        Tx=\begin{cases}
        0, & \text{jika } x\neq 2,\\
        1, & \text{jika } x=2,
        \end{cases}
    \end{align*}
    merupakan pemetaan $(\alpha,\beta,\gamma)$-nonekspansif dengan $\alpha=\beta=\gamma=\frac{1}{2}$, tetapi bukan pemetaan nonekspansif.
\end{exam}
Penjelasan dari Contoh \ref{con:abcnonban} adalah sebagai berikut.

Diketahui bahwa untuk setiap $x,y\in W$ dengan $x,y\neq 2$ diperoleh
\begin{align*}
    |Tx-Ty| &= |0-0| = 0 \leq \frac{1}{2}|x-y| + \frac{1}{2}|x-Tx| + \frac{1}{2}|x-Ty|,
\end{align*}
dan untuk $x=y=2$ diperoleh
\begin{align*}
    |Tx-Ty| &= |1-1| = 0 \leq \frac{1}{2}|x-y| + \frac{1}{2}|x-Tx| + \frac{1}{2}|x-Ty|,
\end{align*}
sehingga Ketaksamaan \eqref{eq:abcnonban} terpenuhi. Selanjutnya, untuk $x\neq 2$ dan $y=2$ diperoleh
\begin{align*}
    |Tx-Ty| = |0-1| = 1 &\leq \qty|\frac{y}{2}| \\
    &\leq \qty|\frac{x-y}{2}|+\qty|\frac{x}{2}|\\
    &= \frac{1}{2}|x-y| + \frac{1}{2}|x-Tx| + \frac{1}{2}|x-Ty|,
\end{align*}
serta untuk $x=2$ dan $y\neq 2$ diperoleh
\begin{align*}
    |Tx-Ty| = |1-0| = 1 \leq \frac{3}{2} &= \qty|x-\frac{1}{2}| \\
    &\leq \qty|\frac{x-1}{2}|+\qty|\frac{x}{2}|\\
    &= \frac{1}{2}|x-y| + \frac{1}{2}|x-Tx| + \frac{1}{2}|x-Ty|,
\end{align*}
sehingga Ketaksamaan \eqref{eq:abcnonban} terpenuhi. Jadi, pemetaan tersebut merupakan pemetaan $(\alpha,\beta,\gamma)$-nonekspansif. Akan tetapi, pemetaan tersebut bukan merupakan pemetaan nonekspansif karena untuk $x=0$ dan $y=2$ diperoleh   
\begin{align*}
    |Tx-Ty| &= |0-1| = 1 > 2 = |0-2| = |x-y|.
\end{align*}

