\section{Pemetaan Nonekspansif}
Pemetaan nonekspansif merupakan salah satu jenis pemetaan yang memiliki banyak aplikasi di berbagai bidang. Pemetaan ini merupakan perluasan dari kontraktif yang dikenalkan oleh Banach. 
Berikut ini diberikan definisi dari pemetaan nonekspansif. 
\begin{defn}\cite{Browder1965}
    Diberikan $(B,\|\cdot\|)$ adalah ruang Banach dan $W$ adalah himpunan bagian tak kosong dari $B$. Pemetaan $T:W\to W$ disebut pemetaan nonekspansif jika untuk setiap $x,y\in W$ berlaku 
    \begin{equation}
        \|Tx-Ty\|\leq \|x-y\|.
    \end{equation}
\end{defn}
Pemetaan nonekspansif ini juga mengalami perluasan, salah satu perluasannya adalah pemetaan nonekspansif kuasi. Berikut adalah definisi dari pemetaan nonekspansif kuasi.
\begin{defn}
    Diberikan $(B,\|\cdot\|)$ adalah ruang Banach dan $W$ adalah himpunan bagian tak kosong dari $B$. Pemetaan $T:W\to W$ disebut pemetaan nonekspansif kuasi jika untuk setiap titik tetap dari $T$ yaitu $y$ dan $x\in W$ berlaku 
    \begin{equation}
        \|Tx-y\|\leq \|x-y\|.
    \end{equation}
\end{defn}

Pada tahun 2008, Suzuki juga mengenalkan bentuk lain pemetaan nonekspansif, yaitu pemetaan Suzuki nonekspansif atau pemetaan yang memenuhi kondisi $(C)$ yang didefinisikan sebagai berikut
\begin{defn}\cite{Suzuki2008}
    Diberikan $(B,\|\cdot\|)$ adalah ruang Banach dan $W$ adalah himpunan bagian tak kosong dari $B$. Pemetaan $T:W\to W$ disebut pemetaan yang memenuhi kondisi $(C)$ jika untuk setiap $x,y\in W$ berlaku 
    \begin{equation}
        \dfrac{1}{2}\|x-Tx\|\leq \|x-y\| \qquad \Longrightarrow \qquad \|Tx-Ty\|\leq \|x-y\|.
    \end{equation}
\end{defn}

Pemetaan ini merupakan kondisi lemah dari pemetaan nonekspansif karena ketaksamaan nonekspansif hanya berlaku untuk dua titik yang memenuhi kondisi $\frac{1}{2}\|x-Tx\|\leq \|x-y\|$.

García-Falset dkk. memperumum pemetaan Suzuki nonekspansif dengan mengganti konstanta $\frac{1}{2}$ menjadi konstanta $\mu\in (0,1)$. Pemetaan tersebut didefinisikan sebagai berikut. 
\begin{defn}\cite{GARCIAFALSET2011185}
    Diberikan $(B,\|\cdot\|)$ adalah ruang Banach dan $W$ adalah himpunan bagian tak kosong dari $B$. Pemetaan $T:W\to W$ disebut pemetaan yang memenuhi kondisi $(C_{\mu})$ jika untuk setiap $x,y\in W$ berlaku 
    \begin{equation}
        \mu \|x-Tx\|\leq \|x-y\| \qquad \Longrightarrow \qquad \|Tx-Ty\|\leq \|x-y\|
    \end{equation}
\end{defn}

Di sisi lain, Aoyama dan Kohsaka juga mengembangkan pemetaan nonekspansif menjadi pemetaan $\alpha$-nonekspansif. Berikut adalah definisi dari pemetaan $\alpha$-nonekspansif.

\begin{defn}\cite{Aoyama2011}
    Diberikan $(B,\|\cdot\|)$ adalah ruang Banach dan $W$ adalah himpunan bagian tak kosong dari $B$. Pemetaan $T:W\to W$ disebut pemetaan $\alpha$-nonekspansif jika untuk setiap $x,y\in W$ terdapat bilangan real $\alpha<1$ sedemikian hingga
    \begin{equation}
        \|Tx-Ty\|^2\leq \alpha \|x-Ty\|^2+\alpha \|y-Tx\|^2 +(1-2\alpha) \|x-y\|^2.
    \end{equation}
\end{defn}

Pant dan Pandey di tahun 2019 mengenalkan pemetaan pemetaan tipe Reich-Suzuki nonekspansif yang didefinisikan sebagai berikut. 

\begin{defn}\cite{Pant2019}
    Diberikan $(B,\|\cdot\|)$ adalah ruang Banach dan $W$ adalah himpunan bagian tak kosong dari $B$. Pemetaan $T:W\to W$ disebut pemetaan tipe Reich-Suzuki nonekspansif jika untuk setiap $x,y\in W$ terdapat bilangan real $\alpha\in[0,1)$ sedemikian hingga
    \begin{equation}
        \|Tx-Ty\|\leq \alpha \|x-Tx\|+\alpha \|y-Ty\| +(1-2\alpha) \|x-y\|.
    \end{equation}
\end{defn}

Yang terbaru, pada tahun 2023 Ullah dkk. mengenalkan pemetaan $(\alpha,\beta,\gamma)$-nonekspansif. 

\begin{defn}\cite{Ullah2023}
    Diberikan $(B,\|\cdot\|)$ adalah ruang Banach dan $W$ adalah himpunan bagian tak kosong dari $B$. Pemetaan $T:W\to W$ disebut pemetaan $(\alpha,\beta,\gamma)$-nonekspansif jika untuk setiap $x,y\in W$ terdapat bilangan real $\alpha,\beta,\gamma\in \mathbb{R}^+$ dengan $\gamma\in[0,1)$ dan $\alpha+\gamma\leq 1$ sedemikian hingga
    \begin{equation}
        \|Tx-Ty\|\leq \alpha \|x-y\|+\beta \|x-Tx\| +\gamma \|y-Tx\|.
    \end{equation}
\end{defn}
% Berikut ini diberikan contoh dari pemetaan tersebut. 
% \begin{exam}
    
% \end{exam}
