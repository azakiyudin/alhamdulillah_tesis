\section{Penelitian Terdahulu}
Bermula dari teorema titik tetap Banach dengan skema iterasi Picard, penelitian mengenai titik tetap dan aproksimasinya terus berkembang hingga saat ini. Dari yang awalnya pemetaan kontraksi, kemudian dikembangkan menjadi pemetaan nonekspansif dan berbagai perumumannya. Tidak seperti pemetaan kontraktif yang titik tetapnya terjamin ada dan tunggal jika di ruang Banach lengkap, pemetaan nonekspansif mensyaratkan beberapa kondisi lain, yaitu titik tetapnya ada jika pemetaannya adalah pemetaan diri sendiri dari himpunan bagian tertutup dan terbatas dari ruang Banach konveks seragam. Walaupun sudah dijamin ada, titik tetapnya juga tidak selalu tunggal \cite{Browder1965}. 


Untuk mendapatkan titik tetap dari pemetaan nonekspansif yang sulit diperoleh secara langsung, diperlukan berbagai macam skema iterasi yang konvergen ke titik tetapnya dengan mendapatkan nilai aproksimasinya. Skema iterasi pertama yang konvergen untuk pemetaan nonekspansif diperoleh oleh Mann pada tahun 1953 \cite{mann1953}. Kemudian Ishikawa mengembangkannya menjadi skema iterasi dua tahap karena skema iterasi Mann gagal konvergen pada pemetaan pseudo-kontraktif \cite{Ishikawa1974}. Dari hasil yang diperoleh Mann dan Ishikawa, Noor kemudian mengembangkan skema iterasi untuk aproksimasi titik tetap menjadi skema iterasi tiga tahap \cite{Noor2000}. Selanjutnya, pada tahun 2007 Agarwal mengenalkan skema iterasinya dengan memodifikasi skema iterasi yang dikenalkan oleh Mann \cite{agarwal}. Pada tahun berikutnya, Suzuki memperkenalkan pemetaan nonekspansif yang baru dengan melemahkan kondisi pemetaannya. Pemetaan baru tersebut ia namakan sebagai pemetaan yang memenuhi kondisi $(C)$ \cite{Suzuki2008}. García-Falset dkk. kemudian juga mengembangkan pemetaan tersebut dengan memperumum kondisinya \cite{GARCIAFALSET2011185}. 

Penelitian lain terkait skema iterasi untuk pemetaan nonekspansif diperoleh Abbas dan Nazir. Mereka mengenalkan skema iterasi baru yang konvergen untuk pemetaan nonekspansif biasa dan mendapatkan aplikasinya untuk masalah minimalisasi dengan konstrain dan masalah \textit{feasibility} \cite{abbas}. Berikutnya, Thakur juga mengenalkan skema iterasi baru yang konvergen lebih cepat daripada skema iterasi sebelum-sebelumnya untuk pemetaan nonekspansif Suzuki di ruang Banach \cite{Thakur2016}.  

Pemetaan nonekspansif beserta aproksimasi titik tetapnya tidak hanya dipelajari dalam ruang Banach, tetapi juga dalam ruang metrik lain, khususnya ruang metrik geodesik. Ruang ini merupakan ruang yang memiliki struktur geodesik, yaitu lintasan terpendek di antara dua titik. Ruang ini memiliki keunggulan dalam hal struktur nonlinier, sehingga dapat memodelkan suatu permasalahan nonlinier secara lebih akurat. Salah satu contoh dari ruang ini adalah ruang $CAT(0)$. Kemudian ruang tersebut diperumum oleh Khamsi dkk. menjadi ruang $CAT_p(0)$ \cite{Khamsi2017}. Salah satu hasil aproksimasi titik tetap di ruang tersebut didapatkan oleh Calder dkk. dengan mengenalkan skema iterasi Agarwal pertubasi di ruang $CAT_p(0)$. Mereka menggunakan tiga pemetaan nonekspansif biasa dan mendapatkan hasil konvergensi-$\Delta$ dengan skema iterasi tersebut \cite{CALDERN2021}. 

Dalam ruang Banach, Pant dan Pandey mengusulkan perumuman baru dari pemetaan nonekspansif yang dikenal sebagai pemetaan nonekspansif tipe Reich–Suzuki. Skema iterasi yang mereka gunakan adalah skema iterasi Thakur dan ruang yang digunakan adalah ruang hiperbolik \cite{Pant2019}. Kemudian, Ahmad dkk. mengenalkan skema iterasi baru yang dinamakan skema iterasi JK yang konvergen untuk pemetaan nonekspansif yang memenuhi kondisi $(C)$ \cite{Ahmad2021}. Hasil menarik juga diperoleh Ullah dkk. dengan mengenalkan pemetaan $(\alpha,\beta,\gamma)$-nonekspansif yang juga perumuman dari pemetaan nonekspansif \cite{Ullah2023}. Arif dkk. mendapatkan hasil konvergensi dari skema iterasi JK untuk pemetaan jenis tersebut \cite{Arif2025}. 

Selain mengalami berbagai pengembangan dalam bentuk perumuman kelas pemetaan, pemetaan nonekspansif beserta skema iterasi yang menyertainya juga memiliki beragam aplikasi di berbagai bidang. Perumuman tersebut mencakup perluasan dari pemetaan nonekspansif klasik ke berbagai kelas pemetaan yang lebih umum dengan kondisi kontraktivitas yang lebih lemah. Beberapa aplikasi yang diperoleh antara lain pada persamaan diferensial, persamaan integral, masalah optimasi, rekonstruksi citra, dan bidang terkait lainnya \cite{Byrne2003,Ramage2023,Salisu2022}.
\todo{sitasi} 

Hasil terbaru di ruang Banach diperoleh Sabri dkk. dengan mengenalkan skema iterasinya yang konvergen untuk pemetaan nonekspansif tipe Reich-Suzuki. Skema iterasi yang mereka peroleh secara numerik juga lebih cepat konvergen dibanding skema iterasi sebelumnya \cite{sabri2025}. Hasil terbaru juga, Salisu dkk. meneliti terkait pemetaan nonekspansif yang memenuhi kondisi $(C_\lambda)$ di ruang $CAT_p(0)$. Mereka mendapatkan bahwa pemetaan tersebut memiliki sifat \textit{demiclosedness}. Kemudian, skema iterasi yang mereka gunakan adalah skema iterasi JK. Dengan skema iterasi tersebut, mereka mendapatkan beberapa teorema terkait dengan konvergen-$\Delta$ maupun konvergen kuat untuk pemetaan nonekspansif yang memenuhi kondisi $(C_\lambda)$. Lebih lanjut lagi, mereka juga mendapatkan beberapa aplikasi dari hasil yang mereka peroleh sekaligus memberikan hasil numeriknya \cite{Salisu2022}. 

